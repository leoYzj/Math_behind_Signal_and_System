%================================
% note-setup.tex
% fenglielie@qq.com 2025-09-12
%================================
\documentclass{ctexbook}
%================================
% note-setup.tex
% fenglielie@qq.com 2025-09-12
%================================

\usepackage{amsmath,amsthm,amsfonts,amssymb}
\usepackage{mathtools}
\usepackage{mathrsfs}
\usepackage{bm}
\usepackage{extarrows}
\usepackage[a4paper, margin=1in]{geometry}
\usepackage{float}
\usepackage{indentfirst}
\usepackage{anyfontsize}
\usepackage{booktabs,multirow,multicol}
\usepackage[shortlabels,inline]{enumitem}
\usepackage{appendix}

\usepackage[dvipsnames]{xcolor}
%\usepackage{graphicx}
%\graphicspath{
%    {./figure/}{./figures/}{./image/}{./images/}{./graphic/}{./graphics/}{./picture/}{./pictures/}
%}
\usepackage{subcaption}

\usepackage[ruled,linesnumbered,noline]{algorithm2e}
\usepackage{listings}
\lstdefinestyle{simpleStyle}{
    basicstyle=\ttfamily\small,
    breaklines=true,
    keywordstyle=\color{blue},
    identifierstyle=\color{black},
    stringstyle=\color{violet},
    commentstyle=\color[RGB]{34,139,34},
    showstringspaces=false,
    numbers=left,
    numbersep=2em,
    numberstyle=\footnotesize,
    frame=single,
    framesep=1em,
}
\lstset{style=simpleStyle}

\usepackage{hyperref}
\hypersetup{
    colorlinks=true,linkcolor=,urlcolor=cyan
}

\renewcommand*{\proofname}{\normalfont\bfseries Proof}

\usepackage{thmtools}

%% define environments

\declaretheorem[style=plain, name=Theorem, numbered=yes, numberwithin=section]{theorem}
\declaretheorem[style=plain, name=Theorem, numbered=no]{theorem*}

\declaretheorem[style=plain, name=Proposition, numbered=yes, sibling=theorem]{proposition}
\declaretheorem[style=plain, name=Proposition, numbered=no]{proposition*}

\declaretheorem[style=plain, name=Corollary, numbered=yes, sibling=theorem]{corollary}
\declaretheorem[style=plain, name=Corollary, numbered=no]{corollary*}

\declaretheorem[style=plain, name=Lemma, numbered=yes, sibling=theorem]{lemma}
\declaretheorem[style=plain, name=Lemma, numbered=no]{lemma*}

\declaretheorem[style=plain, name=Claim, numbered=yes, sibling=theorem]{claim}
\declaretheorem[style=plain, name=Claim, numbered=no]{claim*}

\declaretheorem[style=definition, name=Definition, numbered=yes, numberwithin=section]{definition}
\declaretheorem[style=definition, name=Definition, numbered=no]{definition*}

\declaretheorem[style=definition, name=Example, numbered=yes, numberwithin=section]{example}
\declaretheorem[style=definition, name=Example, numbered=no]{example*}

\declaretheorem[style=definition, name=Problem, numbered=yes, numberwithin=section]{problem}
\declaretheorem[style=definition, name=Problem, numbered=no]{problem*}

\declaretheorem[style=remark, name=Remark, numbered=yes, numberwithin=section]{remark}
\declaretheorem[style=remark, name=Remark, numbered=no]{remark*}

\declaretheoremstyle[headfont=\color{orange!80}\bfseries, bodyfont=\normalfont, spaceabove=3pt, spacebelow=3pt]{notestyle}

\declaretheorem[style=notestyle, name=Note, numbered=yes, numberwithin=section]{note}
\declaretheorem[style=notestyle, name=Note, numbered=no]{note*}

\declaretheoremstyle[headfont=\bfseries, bodyfont=\normalfont, spaceabove=3pt, spacebelow=3pt, qed=\ensuremath{\square}]{solutionstyle}

\declaretheorem[style=solutionstyle, name=Solution, numbered=yes, numberwithin=section]{solution}
\declaretheorem[style=solutionstyle, name=Solution, numbered=no]{solution*}

\usepackage[most]{tcolorbox}

\newcommand{\newtcbenvironment}[2]{
    \tcolorboxenvironment{#1}{#2, enhanced, breakable, sharp corners, boxrule=1pt}
    \tcolorboxenvironment{#1*}{#2, enhanced, breakable, rounded corners, boxrule=1pt}
}

%% define styles

\newtcbenvironment{theorem}{colframe=RoyalPurple, colback=RoyalPurple!8}
\newtcbenvironment{proposition}{colframe=RoyalPurple, colback=RoyalPurple!8}
\newtcbenvironment{corollary}{colframe=NavyBlue, colback=SkyBlue!8}
\newtcbenvironment{lemma}{colframe=NavyBlue, colback=SkyBlue!8}
\newtcbenvironment{claim}{colframe=NavyBlue, colback=SkyBlue!8}

\newtcbenvironment{definition}{colframe=ForestGreen, colback=ForestGreen!5}
\newtcbenvironment{example}{colframe=RawSienna, colback=RawSienna!5}
\newtcbenvironment{problem}{colframe=WildStrawberry!30, colback=WildStrawberry!5}

%% cbox
\newtcolorbox{cbox}[1][]{%
    enhanced,
    breakable,
    sharp corners,
    leftrule=2pt, rightrule=0pt, toprule=0pt, bottomrule=0pt,
    colframe=SkyBlue,
    colback=SkyBlue!8,
    #1
}

%% cover
\usepackage{titling}
\newcommand{\extrainfo}{}
\renewcommand{\extrainfo}[1]{\renewcommand{\extrainfocontent}{#1}}
\newcommand{\extrainfocontent}{}
\newcommand{\makecover}[1]{%
    \begin{titlepage}
        \newgeometry{margin=0in}
        \parindent=0pt
        \includegraphics[width=\linewidth]{#1} % size = 1280*1024
        \vfill
        \begin{center}
            \parbox{0.618\textwidth}{%
                \raggedleft{\bfseries \Huge \thetitle} \\[0.6pt]
                \rule{0.618\textwidth}{4pt} \\
            }
        \end{center}
        \vfill
        \begin{center}
            \parbox{0.618\textwidth}{%
                \raggedleft\Large
                \begin{tabular}{r}
                    \theauthor \\
                    \thedate   \\
                \end{tabular}%
            }
        \end{center}
        \vfill
        \begin{center}
            \parbox[t]{0.7\textwidth}{\centering \itshape \extrainfocontent}
        \end{center}
        \vfill
    \end{titlepage}
    \restoregeometry
    \thispagestyle{empty}
}
% USAGE
% \extrainfo{xxx}
% \makecover{/path/to/cover.png}

%自行添加的包
\usepackage{multicol}
\usepackage{varwidth}
\usepackage{dsfont}
\usepackage{fix-cm}
\usepackage{array}
\usepackage{booktabs}
\usepackage{float}
\usepackage{pifont}
\usepackage{enumitem}
\usepackage{circuitikz}
\usepackage{standalone}
\usepackage{silence}
\usetikzlibrary{calc}
\allowdisplaybreaks
\raggedbottom
\usetikzlibrary{shapes,arrows,positioning,calc}
\usepackage{graphicx}
\usepackage{grffile}
\graphicspath{{D:/code_clone/Math_behind_Signal_and_System/Figures}}
\let\cleardoublepage\clearpage
%结束

%自定义命令
\newcommand{\shah}{\operatorname{III}}
% 便捷命令:在文中书写原函数在上下限的取值,例如 \evalat{F(x)}{a}{b} 输出为 \left.F(x)\right|_{a}^{b}
\newcommand{\evalat}[3]{\left.#1\right|_{#2}^{#3}}
\newcommand{\lr}[2]{%
    \begin{center}
    \begin{minipage}[t]{0.45\textwidth}
        \centering
        \allowdisplaybreaks
        \textcolor{blue}{%
            \begin{varwidth}{\linewidth}
            $\begin{aligned}
            #1
            \end{aligned}$
            \end{varwidth}
        }
    \end{minipage}
    \hfill
    \begin{minipage}[t]{0.45\textwidth}
        \centering
        \allowdisplaybreaks
        \textcolor{red}{%
            \begin{varwidth}{\linewidth}
            $\begin{aligned}
            #2
            \end{aligned}$
            \end{varwidth}
        }
    \end{minipage}
\end{center}
}
\newlist{circlist}{enumerate}{1}
\setlist[circlist]{
    label=\protect\ding{\numexpr171+\arabic*\relax},
    left=0pt
}

 \usepackage{biblatex}
 \addbibresource{reference.bib}

\title{信号与系统的数学原理}
\author{严子竣}
\date{\today}

\begin{document}
\makecover{bnds2}
\frontmatter%前言
信号与系统是电子信息等相关专业本科生的重要基础课,是后续课程“数字信号处理”“通信原理”“自动控制”“随机信号处理”“数字图像处理”“现代信号处理”等课程的基础,以确定信号通过线性时不变系统的过程为核心,综合运用多门数学课程的知识,尤其是傅里叶分析。

然而,国内外的信号与系统课程对于傅里叶变换的定义有所不同,国内的课程中所谓“频域分析”实际上指的是角频率域,而这种定义上的不同会导致诸多公式的内核相同但形式不同,使学习者感到困惑。我自学过斯坦福大学公开课EE261,同样面临着这个问题,所以希望尝试这种新的笔记形式,梳理信号与系统课程的理论部分,同时也使本书能够作为同学们学习信号与系统时的参考书。我们将看到,同时给出两种傅里叶变换,不仅方便我们对比、查找公式,理解傅里叶变换的本质,还能在必要时转向其中一种傅里叶变换,使得讨论更加简洁明快。

同时,我注意到现在的信号与系统教材中对很多数学上的内容的处理过于简单化,以至于同学们对一些公式感到不解,我希望结合《信号与系统》教材和实变函数、复变函数和傅里叶分析等教材,对这样的内容做一些补充和深化。对于其中比较复杂的内容,我会用“*”进行标注,以提示读者可以选择性地阅读。

本书将不会涉及信号与系统实践部分,对一些语文性质的概念从简处理,希望了解这些内容的读者可以自行查看信号与系统教材。对于两种不同的傅里叶变换,将并排给出两种公式,左侧为频率版本,右侧为角频率版本。为了逻辑的完整性和内容的精确性,本书会引入一些远超工科教学范围的内容,力求朴素平和却又不失深度,其中所涉及的技术细节不在我们的关注范围,读者也可以自行选择是否阅读这些内容。本书默认读者已经学习过一到两学期的数学分析(或高等数学/微积分)以及高等代数(或线性代数)课程,不过只要理解一元微积分,阅读前四章及附录就不会有问题;只要掌握复数运算、理解曲线积分,阅读后两章就不会有问题。

限于作者水平,书中难免存在不足之处,希望读者可以提出宝贵的意见或建议,可以反馈至我的电子邮箱:2024212622@bupt.cn。

{
\raggedleft{}
北京市十一学校2024届毕业生\\
严子竣\\
2025年秋\\
}
\tableofcontents%目录
\mainmatter%正文
\chapter{预备知识}%第一章
\section{信号与系统的基本概念}\label{sec:basic_concept}
信号 (signal)是信息 (information)的表现形式与传送载体,信息是蕴含在信
号中的具体内容。

在电学领域中,通过电压或电流对真是信号进行连续的记录,模拟其变化过程得到
的信号称为模拟信号 (analog signal)或连续信号 (continuous signal),但由于
计算机中信号只能以有限位数的形式储存,往往需要将模拟信号转化为数字信号
(digital siganal)。这个过程往往需要利用模数转换器 (analog-to-digital converter,ADC)
,使用时再利用数模转换器 (digital-to-analog converter,DAC)将数字信号转化
为模拟信号。

在信息科学与技术领域,系统指对信号产生影响的装置或算法,例如滤波系统、调制系
统、发射系统,将这些系统组合起来就组成了无线电广播系统;例如通信系统由信源、
发射机、信道、接收机和信宿五个部分组成。

电子信息系统的基本构件是传感器、处理器和激励器。传感器 (sensor)将物理量转化为电信号,处理器 (processor)对信号进行处理,激励器 (actuator)将处理后的电信号转化为物理量。其中,电路层面的处理器处理的是电信号,完成电信号处理的功能电路包括:\begin{itemize}
    \item 放大器 (amplifier):对信号进行放大处理,与之对应的是衰减器 (attenuator)。
    \item 滤波器 (filter):传感器输出的电信号多是包含信息的低频电信号,又称为基带信号 (baseband signal);在信号传输过程中,信号往往被置于某个高频频段内,这个频带称为通带 (passband);在传输前后,都需要滤波器来排除所需频带之外信号的干扰。见\ref{sec:convolution}、\ref{sec:freq_response}。
    \item 调制器 (modulator)与解调器 (demodulator):将基带信号转化为通带信号的过程称为调制 (modulation),将通带信号还原为基带信号的过程称为解调 (demodulation)。见\ref{sec:modulation}。
    \item 振荡器 (oscillator):产生周期性信号的装置,常见的有正弦波振荡器、方波振荡器等,也用于产生高频载波信号进行调制。
    \item 模数转换器 (ADC)与数模转换器 (DAC):将模拟信号转化为数字信号的装置称为模数转换器,将数字信号转化为模拟信号的装置称为数模转换器。所谓数字信号,是指在时间和幅值上均为离散的信号;而模拟信号则是在时间和幅值上均为连续的信号,用于模拟真实世界中的信号。
    \item 存储器 (memory):用于存储数字信号的装置,例如随机存取存储器 (RAM)、只读存储器 (ROM)等。
    \item 处理器 (processor):对数字信号进行处理的装置,例如数字信号处理器 (DSP)、微控制器 (MCU)等。
\end{itemize}
此外,还有整流器、稳压器、变压器等。其中,滤波器、调制解调器属于信号与系统课程的内容。

\section{连续信号与离散信号}\label{sec:signal}
信号的本质是函数,\textbf{连续信号}就是定义域“连续”的函数,例如一个时间的函数$f(t)$,
一般要求定义域具有连续统的势。相应的,\textbf{离散信号}的定义域往往是至多可数集,于是
我们常认为其定义域是整数集,并记离散信号为$x[n]$,用研究数列的方法研究离散信
号。

从电学的经验来看,电压、电流信号的功率总为其幅值的平方乘以某个常数,于是对于
一般的信号我们也采取此定义。真实的物理世界中的信号的能量总是有限的,但是,为
了更好地研究它们,很多时候使用真实信号的一部分做适当延拓作为研究对象,因此课
程中将遇到一些不满足“方均可积”性质$\int_{-\infty}^{\infty}| f(x)| ^2\,dx<\infty$的信号,于是定义:
\begin{definition}[功率信号与能量信号]
能量无限而功率有限的信号称为\textbf{功率信号}:\[\int_{-\infty}^{\infty}| f(x)| ^2\,dx\to\infty,\frac{1}{T}\int_{-\infty}^{\infty}| f(x)|^2  \,dx<\infty\]
而能量有限的信号称为\textbf{能量信号}:\[\int_{-\infty}^{\infty}| f(x)| ^2\,dx<\infty\]
\end{definition}

为了区分某段时间内和全时域上的能量和功率,又定义:\begin{definition}
\quad\newline
    \textbf{归一化能量}:$E_T=\int_{-T/2}^{T/2}| f(x)| ^2\,dx$;\\
    \textbf{归一化功率}:$P_T=\frac{1}{T}\int_{-T/2}^{T/2}| f(x)| ^2\,dx$\\
\textbf{全时域能量}:$E=\int_{-\infty}^{\infty}| f(x)| ^2\,dx$;\\
\textbf{全时域功率}:$P=\lim_{T \to \infty}  \frac{1}{T}\int_{-T/2}^{T/2}| f(x)| ^2\,dx$
\end{definition}

类似地,对于离散信号,定义:
\begin{definition}
    \quad\newline
    \textbf{归一化能量}:$E_N=\sum_{n = -N}^{N}  |x[n]| ^2$;\\
    \textbf{归一化功率}:$P_N=\frac{1}{2N+1}\sum_{n = -N}^{N}  |x[n]| ^2$\\
\textbf{全时域能量}:$E=\sum_{n = -\infty}^{\infty}  |x[n]| ^2$;\\
\textbf{全时域功率}:$P=\lim_{N \to \infty}  \frac{1}{2N+1}\sum_{n = -N}^{N}  |x[n]| ^2$
\end{definition}

读者可能已经注意到,并不是所有的函数都是可积的,例如$e^x$即使视为功率信号,其
能量和功率也不具有意义,因此它既不属于能量信号,也不属于功率信号;另一种情况是:
\begin{example}[狄利克雷函数]
    \[D(x)=
    \begin{cases}
        1 & x \in \mathbb{Q}    \\
        0 & x \in \mathbb{Q} ^C
    \end{cases}\]
(其中$\mathbb{Q}$ 表示有理数集,上标C在不引起歧义的前提下用来表示取补集)
\end{example}
它在黎曼积分(也就是数学分析或高等数学课程,以及多数工科课程中用的积分)的意义下
是不可积的,因为它在所有点不连续,但是这样的函数在\textbf{勒贝格积分}的意义下可积,读
者可以这样理解:勒贝格积分考察函数在某个值处的“区间长度”(严格来讲应为测度)
,狄利克雷函数在一个有限区间上取值1的长度为0,因为有理数是可数集,取值0的长度
就是区间长度,所以其积分值为0。勒贝格积分是黎曼积分的推广,对于非反常积分,黎曼可积的函数一定
是勒贝格可积的,并且在同样的区间上积分值相等。对于勒贝格不可积的函数,一般不
在本课程的讨论范围。尽管我们很少接触只在勒贝格意义下才可积的函数,但后面我们将
逐步认识到勒贝格积分在傅里叶分析中的重要地位,读者应当对其有一个初步的认识。

\section{典型的连续与离散信号}
\begin{definition}[取样信号]
    \[Sa(t)=\frac{\sin(t)}{t}\]
\end{definition}
它在0处连续延拓为1,在$\pi$的整数倍处为0,是偶函数,在正半轴上的积分值为$\pi /2$(这个结果称为狄利克雷积分)。
国外教材中取样信号一般为一个类似的信号\[sinc(t)=\frac{\sin(\pi t)}{\pi t}=Sa(\pi t)\]
它们的参数不同但性质类似、本质相同,其作用在后续学习采样和插值时将体现出来。需
要注意的是,在绘图时python和MATLAB中只有sinc信号,而且实际上指的是Sa信号。
\begin{definition}[钟形信号/高斯信号]
    \[f(t)=Ee^{-{\left(\frac{t}{\tau }\right)}^2}\]
    $\tau$为衰减速率或时间常数,$\tau$越大,函数值衰减越慢。
\end{definition}
\begin{definition}[单位脉冲函数/狄拉克函数]
    $\delta (x)=\begin{cases}
        0      & x\neq 0 \\
        \infty & x=0
    \end{cases}$
\end{definition}
这个函数可视为一些性质较好的函数如高斯函数、取样函数集中到x=0附近时
的极限情况,从而具备很多优美的性质,见\ref{sec:convolution}。实际上,它还是无限
阶可导的,我们将在\ref{sec:distributions}
中详细讨论这一点。一些简单的性质列举如下。\\
\ding{172} 取样性质:$f(x)\delta (x)=f(0)\delta (x)$\\
\ding{173} 积分性质:$\int_{-\infty}^{\infty}\delta (x)\,dx=1$ (由此我们
说这个函数的强度为1 ),从而
$\int_{-\infty}^{\infty}f(x)\delta (x)\,dx=f(0)$。
\ding{174} $\delta$ 函数是偶函数:$\delta (-x)=\delta (x)$
\ding{175} 尺度变换性质:用函数逼近的观点来理解$\delta $函数,为保证强
度为1,自然应当要求$\delta (ax)=\frac{1}{|a|}\delta (x)$\\
这个函数在做卷积运算时有很好的性质,见\ref{sec:convolution}卷积;关于它的逼近方式,见\\
附录\ref{sec:approach}。我们规定,$\delta_a(x)=\delta (x-a)$,即在$x=a$处取样的狄拉克函数。
\begin{definition}[单位阶跃函数]
    \[u(x)=
    \begin{cases}
        0 & x<0 \\
        1 & x>0
    \end{cases}\]
\end{definition}
这个函数在0处的值可以任意定义,一般取0、1或1/2,不会有影响,所以一般也不会讨论。由单
位阶跃函数衍生出许多其他函数:\\
\textbf{符号函数}\[sgn(x)=\begin{cases}
        -1 & x<0 \\
        1  & x>0
    \end{cases}=u(t)-u(-t)\]
\textbf{矩形脉冲、门函数(矩形函数、$\Pi $函数)}\[R_T(t)=u(t)-u(t-T),G_T(t)=\Pi_T(t)=u(t+T/2)-u(t-T/2)\]
这里T为脉冲宽度,不加角标时,默认为1。

单位阶跃函数的积分为单位斜变信号 (the unit ramp),导数为狄拉克函数,从直观上
这不难理解,将在\ref{sec:distributions}中讨论。
\begin{figure}[H]
    \centering
    \includegraphics[width=0.7\textwidth]{Figure_1}
    \caption{取样函数、高斯函数、狄拉克函数、单位阶跃函数}
\end{figure}

有时,使用\textbf{特征函数}(indicator function)更为方便:
\begin{definition}
    设集合$A\subset\mathbb{R}$,其特征函数为
    \[\chi_A(x)=\begin{cases}
        1 & x\in A \\
        0 & x\notin A
    \end{cases}\]
\end{definition}
我们将在实变函数的理论中用到它。例如,狄利克雷函数$D(x)$可表示为$\chi_{\mathbb{Q}}(x)$。

下面介绍离散信号。
\begin{definition}[单位脉冲序列/克罗内克$\delta$函数]
    \[\delta[n]=\begin{cases}
        1, & \text{if }n=0     \\
        0, & \text{if }n\neq 0
    \end{cases}\]
\end{definition}
将它进行时移,可得\[\delta_k[n]=\delta[n-k]=\begin{cases}
        1, & \text{if }n=0     \\
        0, & \text{if }n\neq 0
    \end{cases}\]
它与狄拉克$\delta$函数一样,都具有取样性质:$x[n]\delta_k[n]=x[k]\delta_k[n],\forall n$.
\begin{definition}[单位阶跃序列]
    \[u[n]=\begin{cases}
        1, & \text{if }n\geq 0 \\
        0, & \text{if }n< 0
    \end{cases}\]
\end{definition}
类似连续信号,由它可以衍生出矩形窗序列:
\[R_N[n]=n[n]-u[n-N]=\begin{cases}
        1, & \text{if }0\leq n\leq N-1 \\
        0, & \text{otherwise}
    \end{cases}\]
\begin{definition}[正弦序列]
    $x[n]=\sin(\Omega_0n)$,其中$\Omega_0$称为数字角频率。
\end{definition}
\begin{figure}[H]
    \centering
    \includegraphics[width=0.4\textwidth]{disct_sin}
\end{figure}

\section{特殊函数}\label{sec:special_function}
%伽马,贝塔
%正弦积分,指数积分,误差函数,补误差函数
%狄利克雷积分
%贝塞尔函数(?)

\section{实变函数概要}\label{sec:real_analysis}

\chapter{傅里叶变换}%第二章
在数学分析课程中,我们都学习过傅里叶级数的计算,所以本书将直接回顾几种情况下的
傅里叶级数展开公式,然后从正交函数系的观点出发构建傅里叶级数的理论,再从傅里
叶级数过渡到傅里叶变换,从对周期现象的研究转向对非周期现象的研究,介绍其运算性
质、卷积性质,最后介绍分布(也即广义函数)理论,从而研究常函数、狄拉克函数、正
余弦函数等在常规意义下无法进行傅里叶变换,却又十分重要的函数。

\section{线性空间,正交基}\label{sec:Linear_Space}
首先介绍一些本章需可能用到的概念:度量公理、范数公理、内积公理、正交基、无穷维线
性空间和$L^p$空间,对其不感兴趣的读者,可以等需要时再阅读此小节的内
容,只要知道函数空间上的正交基是怎么回事即可。
\begin{definition}[度量]
    在集合$V$中,如果定义了运算
$d(\cdot,\cdot):V\times V\rightarrow \mathbb{R}  $,满足\textbf{度量公理}:
\begin{enumerate}
    \item \textbf{非负性}:$d(x, y) \geq 0$
    \item \textbf{同一性}:$d(x, y) = 0$ 当且仅当 $x = y$
    \item \textbf{对称性}:$d(x, y) = d(y, x)$
    \item \textbf{三角不等式}:$d(x, z) \leq d(x, y) + d(y, z)$
\end{enumerate}
则称在$V$上定义了一种度量。
\end{definition}
特别地,可以在$\mathbb{C}$ -线性空间$V$上定义度量。
\begin{definition}[范数]
    如果定义了运算
$\|\cdot \| :V\rightarrow \mathbb{R} $,满足\textbf{范数公理}
\begin{enumerate}
    \item \textbf{非负性}:$\|\mathbf{x}\| \geq 0$
    \item \textbf{同一性}:$\|\mathbf{x}\| = 0$ 当且仅当 $\mathbf{x} = \mathbf{0}$
    \item \textbf{齐次性}:$\|\alpha \mathbf{x}\| = |\alpha| \, \|\mathbf{x}\|$
    \item \textbf{三角不等式/次可加性}:$\|\mathbf{x} + \mathbf{y}\| \leq \|\mathbf{x}\| + \|\mathbf{y}\|$
\end{enumerate}
则称在$V$上定义了一种范数,称$V$为线性赋范空间。
\end{definition}
特别地,在$\mathbb{C}$ -线性空间$V$上定义范数后,可以自然地诱导出一种度量:
$$d(\mathbf x,\mathbf y)=||\mathbf x-\mathbf y||$$
在线性赋范空间上可以定义极限:
\[\lim_{x \to x_0} f(x)=A:=\forall \epsilon>0\exists \delta>0(||x-x_0||<\delta\Rightarrow ||f(x)-f(x_0)||<\epsilon) \]
这就是数学分析中的极限定义,只是将绝对值改成了范数,其余类似的极限定义不再赘述。
如果线性赋范空间中所有柯西列都有极限,那么这个空间称为完备的线性赋范空间,或称
为巴拿赫空间(有限维)、希尔伯特空间(无限维)。
\begin{example}
    连续函数空间$C[a,b]$是不完备的线性赋范空间:\\
    在1-范数
$\|f\|_1 =\int_{a}^{b}|f(t)|\,\text dt$下,其中的柯西列\begin{align*}
    f_n(t)=\begin{cases}
        0,&\text{if }-1\leq t <0\\
        nt,&\text{if }0\leq t <\frac{1}{n}\\
        1,&\text{if }\frac{1}{n}\leq t \leq 1
    \end{cases}
\end{align*}
的极限为单位阶跃函数$u(t)$(在0处取值为0),不属于$C[-1,1]$.
\end{example}
\begin{definition}[内积]
    如果定义了运算
$\langle \cdot,\cdot\rangle:V\times V\rightarrow \mathbb{C} $,满足\textbf{内积公理}
\begin{enumerate}
    \item \textbf{正定性}:$\langle \mathbf{v}, \mathbf{v} \rangle \geq 0$ 且 $\langle \mathbf{v}, \mathbf{v} \rangle = 0$ 当且仅当 $\mathbf{v} = \mathbf{0}$
    \item \textbf{共轭对称性}:$\langle \mathbf{v}, \mathbf{w} \rangle = \overline{\langle \mathbf{w}, \mathbf{v} \rangle}$
    \item \textbf{第一变元的线性性}:
          \begin{itemize}
              \item \textbf{齐性}:$\langle \alpha \mathbf{v}, \mathbf{w} \rangle = \alpha \langle \mathbf{v}, \mathbf{w} \rangle$
              \item \textbf{可加性}:$\langle \mathbf{v} + \mathbf{w}, \mathbf{u} \rangle = \langle \mathbf{v}, \mathbf{u} \rangle + \langle \mathbf{w}, \mathbf{u} \rangle$
          \end{itemize}
\end{enumerate}
则称在V上定义了一种内积 (inner product),称V为内积空间。
\end{definition}
在内积空间上有著名的
柯西-施瓦兹不等式 (Cauchy-Shwartz inequality):
\begin{theorem}[柯西-施瓦兹不等式]
    \[|\langle \mathbf{a,b}\rangle| \leq \| \mathbf{a}\| \| \mathbf{b}\| \]
\end{theorem}
\begin{proof}
不妨设$\mathbf{b}$不是零向量,任取$t\in \mathbb{R}$,有
\[0\leq \|\mathbf{a}+t\mathbf{b}\|^2=\|\mathbf{a}\|^2+2t\langle\mathbf{a,b}\rangle +t^2\|\mathbf{b}\|^2\]
令$t=-\frac{\langle\mathbf{a,b}\rangle}{\|\mathbf{b}\|^2}$,即得
\[0\leq \|\mathbf{a}\|^2-2\frac{\langle\mathbf{a,b}\rangle^2}{\|\mathbf{b}\|^2} +\frac{\langle\mathbf{a,b}\rangle^2}{\|\mathbf{b}\|^4}\|\mathbf{b}\|^2=\|\mathbf{a}\|^2-\frac{\langle\mathbf{a,b}\rangle^2}{\|\mathbf{b}\|^2}\]
这与要证明的不等式是等价的。
\end{proof} 
有了柯西-施瓦兹不等式,三角不等式就是显然的了,这里仅给出其表述,读者可以自行证明。
\begin{proposition}[三角不等式]
    \[\forall \mathbf{a,b}\in V,\|\mathbf{a+b}\|\leq\|\mathbf{a}\|+\|\mathbf{b}\|\]
\end{proposition}
不难发现,只要取
$\|\mathbf{v}\|^2=\langle \mathbf{v}, \mathbf{v} \rangle$,
就由内积导出了一种范数,并且这种范数具有比一般的范数更强的性质;
只要取$d(\mathbf{x},\mathbf{y})=\|\mathbf{x-y}\|$,就由范数导出了一种度量,
并且这种度量具有比一般的度量更强的性质。
\begin{definition}[正交基]
    如果有内积空间$V$的一组基$\{\mathbf{v_1},\mathbf{v_2},\dots ,\mathbf{v_n}\}$
,满足$\langle \mathbf{v_i},\mathbf{v_j}\rangle =0,i \neq j$,则称这组基是一组\textbf{正交基}(orthognal bases)。\\
如果对$V$中任一向量$\mathbf{w}$,均有以下分解式:
\[\mathbf{w}=\sum_{i = 1}^{n}  c_i \mathbf{v_i}\]
则这组基是\textbf{完备正交基}(complete orthognal bases)。\\
如果$\{\mathbf{v_1},\mathbf{v_2},\dots ,\mathbf{v_n}\}$还满足
$\langle \mathbf{v_i},\mathbf{v_i}\rangle =1,i \in \{1,2,\dots ,n\}$,称这组基是\textbf{标准完备正交基} (complete othornormal bases),此时空间中任意向量均有分解式
\begin{align*}
    \mathbf{w}=\sum_{i=1}^{n}\langle \mathbf{w},\mathbf{v_i}\rangle \mathbf{v_i}
\end{align*}
\end{definition}
我们对向量在标准正交基下的分解式做一些解释。在$\{\mathbf{v_1},\mathbf{v_2},\dots ,\mathbf{v_n}\}$是完备正交基的情况下,就可以将$\mathbf{w}=\sum_{i = 1}^{n}  c_i \mathbf{v_i}$两边同时对$\mathbf{v_j}$做内积,得到
\begin{align*}
    \langle \mathbf{w},\mathbf{v_j} \rangle=\langle \sum_{i = 1}^{n}  c_i \mathbf{v_i},\mathbf{v_j} \rangle =c_j\langle \mathbf{v_j},\mathbf{v_j} \rangle \\
    c_j=\frac{\langle \mathbf{w},\mathbf{v_j} \rangle}{\langle \mathbf{v_j},\mathbf{v_j} \rangle},
    \mathbf{w}=\sum_{j = 1}^{n}  \frac{\langle \mathbf{w},\mathbf{v_j} \rangle}{\langle \mathbf{v_j},\mathbf{v_j} \rangle} \mathbf{v_j}
\end{align*}
而标准正交基是$\langle \mathbf{v_j},\mathbf{v_j} \rangle=1$的情况。

给出了标准完备正交基后,就可以得到以下命题,它是勾股定理的高维推广。
\begin{proposition}[勾股定理/毕达哥拉斯恒等式]
    \[|\mathbf{w}|^2=\sum_{i=1}^{n}|\langle\mathbf{w,v_i}\rangle|^2|\mathbf{v_i}|^2=\sum_{i=1}^{n}|\langle\mathbf{w,v_i}\rangle|^2\]
\end{proposition}
直接将$\mathbf{w}$与自己做内积并将之前得到的分解式带入,很容易证明这个命题。
\begin{proof}\begin{align*}
    |\mathbf{w}|^2 & =\langle \sum_{i=1}^{n}\langle \mathbf{w},\mathbf{v_i}\rangle \mathbf{v_i},\sum_{i=1}^{n}\langle \mathbf{w},\mathbf{v_i}\rangle \mathbf{v_i}\rangle                                           \\
                   & =\sum_{i=1}^{n}|\langle\mathbf{w,v_i}\rangle|^2\langle\mathbf{v_i,v_i}\rangle+\sum_{1\leq i<j\leq n}\langle \mathbf{w,v_i}\rangle\langle \mathbf{w,v_j}\rangle\langle \mathbf{v_i,v_j}\rangle \\
                   & =\sum_{i=1}^{n}|\langle\mathbf{w,v_i}\rangle|^2|\mathbf{v_i}|^2=\sum_{i=1}^{n}|\langle\mathbf{w,v_i}\rangle|^2
\end{align*}\end{proof}
可见做正交基分解能够极大地简化对线性赋范空间的研究。
\begin{definition}[无限维线性空间和基]
    如果一个线性空间V的维数是无限的,就称之为\textbf{无限维线性空间}。\\
对于V,我们称向量列
$\lbrace\mathbf{v_i}\rbrace_{i=1}^{\infty}$是V的一组基,如果
\begin{itemize}
    \item \raggedright{} 线性无关性:任取基中的有限个向量,它们是线性无关的\\
    \item 有限生成性:任取向量$\mathbf{w} \in V$,存在有限个向量
          $V'=\lbrace\mathbf{v_1,v_2,\dots,v_r}\rbrace\subset V $,$\mathbf{w}$可以用$V'$线性表出
\end{itemize}
\end{definition}
这里要求“有限”是为了避免敛散性的问题:例如收敛的级数构成线性空间,如果我们声称取定了一组基(当然是无限的),并考察其中无限个基张成的空间,那么对于构成级数的每一项,均需要考察其敛散性。然而,级数的敛散性自然可以对前有限项不做要求,它们求和很可能不会收敛;从另一个角度来讲,一些更加抽象的线性空间中,也说不清楚基的无限和是否收敛。

函数空间是一种典型的无限维线性空间(因为多项式空间已经是无限维的),根据学习线性代数的经验,我们希望能找到一组单位正交基,使得函数在这组正交基下的分解能够体现函数的某些性质并便于后续的运算,然而,若只考虑有限和,这种想法所能研究的函数十分有限,例如我们马上就会见到的三角函数系和指数函数系,它们作为无限阶可微函数,有限和也是无限阶可微的。所以,我们应考虑将函数$f(t)$分解为一组相互正交的函数系$\{f_i(t)\}_{i=1}^{\infty}$组成的函数项级数。

我们面临的另一个问题是如何在函数空间上定义内积,从而定义正交性。一种比较自然的想法是利用(勒贝格)积分,积分区间有限,对于周期函数,自然地取为一个周期。
\begin{definition}[函数空间上的内积]
    在空间\[L^2([0,T]):=\{f:[0,T]\rightarrow \mathbb{C} \big| \int_{T}|f(t)|^2\,\text dt<\infty\}\]
上定义内积(为了区别于分布的符号,这里内积用圆括号表示,$^*$表示取共轭):
\[(f,g):=\int_{T}f(t)g^*(t)\,dt \]
\end{definition}
我们对这个定义做一些说明,但不给出证明,因为证明需要首先建立勒贝格积分的体系,
读者可借助黎曼积分直观地理解它们:\\
1.$L^p([0,T])(0<p\leq \infty)$空间表示在区间$[0,T]$上$p$次勒贝格可积的函数组成的函数空间,即
\[L^p([0,T]):=\{f:[0,T]\rightarrow \mathbb{C} \big| \int_{T}f^p(t)\,dt<\infty\}\]
$L^p([0,T])$具有性质:
\begin{itemize}
    \item \raggedright{} $L^p([0,T])$是线性空间\\
    \item 当$1\leq p \leq \infty$时,$L^p([0,T])$是线性赋范空间,
          $\| f \|_{p} := \bigl( \int_{T} |f(t)|^p \, dt \bigr)^{1/p}$,
          称之为$L^p$范数,次可加性由闵可夫斯基不等式保证
\end{itemize}
2.要求$f(t)$平方可积是为了保证$(f,f)=\int_{T}|f(t)|^2\,dt<\infty$,$f(t)$平
方可积能够推出$f(t)$是绝对可积的,从而是可积的(有限区间I上有$L^p(I)\supset  L^q(I),p<q$,无限区间上它们互不包含)\\
3.尽管对函数空间做了一些限制,我们研究的范围依旧是足够大的,闭区间上的平方可积
是一个比较弱的条件\\
4.柯西-施瓦兹不等式和三角不等式(它是闵可夫斯基不等式的特例)自然成立,它们证
明的过程不涉及空间的维数是否有限。

有了内积就可以定义范数,从而可以给出$L^2([0,T])$空间上的函数项级数的(依范数)
收敛的定义:
\begin{definition}[依范数收敛]
    如果
\[\lim_{n \to \infty} \| f(t)-\sum_{i = 1}^{n}  a_i f_i(t)\|=0\]
就认为级数$\sum_{i = 1}^{n}  a_i f_i(t)$是$f(t)$在这个正交函数系下的分解,
此时记\[f\sim\sum_{n=1}^{\infty}a_n f_n\]
\end{definition}
它并不意味着等式右侧的函数项级数在某一点收敛于$f$.在$L^2([0,T])$空间中,我们不
区分仅在零测集(“区间长度”的总和总能取到任意小正数,例如至多可数集)上不相等的
函数,换言之,$L^2([0,T])$空间不是常规意义下的函数的集合,而是\textbf{几乎处处}
(almost every,a.e.)相等的函数构成的等价类,这里的$\sim$表示的是两侧的函数同属一
个等价类,至于逐点收敛、一致收敛性,需要另作讨论。

可以想象,依范数收敛要求极限内的函数相当接近于0,但如果在一个点处产生了误差,不论
误差多大,都不会影响积分的值。事实上,只要存在误差的点构成零测集,就不会影响积分的值,这时我们称
$\sum_{i = 1}^{n}  a_i f_i(t)$几乎处处收敛
于$f(t)$,只是这样弱的要求有时会导致积分在黎曼积分的意义下不存在,但勒贝格积分
可以处理这种情况,读者可以参考\ref{sec:signal}连续信号与离散信号中对勒贝格积分
的讨论。
\begin{definition}[完备正交函数系]
    对于一个正交函数系$\{f_i(t)\}_{i=1}^{\infty}$,如果不存在非零的函数$g(t)\notin\{f_i(t)\}_{i=1}^{\infty}$使得$g(t)$与
$\{f_i(t)\}_{i=1}^{\infty}$中的所有函数正交,就称$\{f_i(t)\}_{i=1}^{\infty}$为\textbf{完备正交函数系}
\end{definition}
完备性意味着$L^2([0,T])$空间中的任一函数$f(t)$均可分解为这个函数系的级数$\sum_{i = 1}^{\infty}  a_i f_i(t)$,我们沿用前文中给出的公式计算这些系数:
\[a_i=\frac{(g,f_i)}{(f_i,f_i)}=\frac{\int_{T}g(t)f_i^*(t)\,dt}{\int_{T}|f_i(t)|^2\,dt}\]
细心的读者可能已经发现,这里得到的公式用到了有限维线性空间中的结论,但要推广到
无限维线性空间并不是显然的。我们将在下一节给出帕塞瓦尔定理之后一并讨论这个问题。

典型的标准完备正交函数集有贝塞尔函数(Bessel functions)、勒让德多项式 (Legendre polynomials)、小
波变换基函数 (wavelet basis functions)等,下面仅讨论三角函数系和指数函数系。

\section{傅里叶级数}\label{sec:Fourier_Series}

十九世纪初,傅里叶为了求解热传导问题中的偏微分方程,断言:任何周期函数都可以通过无穷多个正弦和余弦函数的线性组合来表示,即傅里叶级数。这一猜想引发了许多数学家的质疑,促进了分析学的发展,直到1829年,狄利克雷提出了傅里叶级数收敛的充分条件和严格证明。今天,傅里叶级数已经成为数学、物理中的重要工具。

我们首先回顾数学分析中学过的几个计算傅里叶级数的公式。将周期为T的函数f展开为三角函数形式的傅里叶级数:
\begin{align*}
    f(t) & =\frac{a_0}{2}+\sum_{k = 1}^{\infty} a_k \cos(k\omega t)+b_k\sin(k\omega t) \\
         & =\frac{c_0}{2}+\sum_{k = 1}^{\infty} c_k\cos(k\omega t+\varphi _k)
\end{align*}
(其中$\omega =\frac{2\pi }{T}$为\textbf{基波角频率},$k\omega (k>1,k\in \mathbb{Z} )$
为k次\textbf{谐波角频率})
\begin{claim}[三角函数形式的傅里叶系数公式]
\[a_k=\frac{2}{T}\int_T f(t)\cos(k\omega t)\,dt\]
\[b_k=\frac{2}{T}\int_T f(t)\sin(k\omega t)\,dt\]
\[c_k=\sqrt{a_k^2+b_k^2}\]
\end{claim}
下面用完备标准正交函数系的观点来得到以上公式。在学习数学分析时,我们已经看到三
角函数系$1,\sin(\omega t),\cos(\omega t),\sin(2\omega t),\cos(2\omega t),\dots(\omega =\frac{2\pi}{T})$
是正交的(读者可以自行验证),但不是单位正交的,因为
\[(\sin(k\omega t),\sin(k\omega t))=\int_{T}\sin^2(k\omega t)\,dt=\int_{T}\frac{1-\cos(2k\omega t)}{2}=\frac{T}{2}\]
\[(\cos(k\omega t),\cos(k\omega t))=\int_{T}\cos^2(k\omega t)\,dt=\int_{T}\frac{1+\cos(2k\omega t)}{2}=\frac{T}{2}\]
可以将它们单位化,也可以直接采用完备正交基分解的公式,
\[a_k=\frac{(f(t),\cos(k\omega t))}{(\cos(k\omega t),\cos(k\omega t))}=\frac{\int_{T}f(t)\cos(k\omega t)^*(t)\,dt}{\int_{T}|\cos(k\omega t)|^2\,dt}
    =\frac{2}{T}\int_{T}f(t)\cos(k\omega t)(t)\,dt\]
\[b_k=\frac{(f(t),\sin(k\omega t))}{(\sin(k\omega t),\sin(k\omega t))}=\frac{\int_{T}f(t)\sin(k\omega t)^*(t)\,dt}{\int_{T}|\sin(k\omega t)|^2\,dt}
    =\frac{2}{T}\int_{T}f(t)\sin(k\omega t)(t)\,dt\]
\begin{claim}[偶函数的傅里叶展开]
当$f(t)$为偶函数,或者由$f(t)$做偶延拓时,展开式为
\[f(t)=\frac{a_0}{2}+\sum_{k = 1}^{\infty} a_k\cos(k\omega t)\]
其中
\[a_k=\frac{4}{T}\int_{0}^{\frac{T}{2}} f(t)\cos(k\omega t)\,dt\]
\[b_k=0\]
\end{claim}\begin{claim}[奇函数的傅里叶展开]
当$f(t)$为奇函数,或者由$f(t)$做奇延拓时,展开式为
\[f(t)=\sum_{k = 1}^{\infty} b_k \sin(k\omega t)\]
其中
\[a_k=0\]
\[b_k=\frac{4}{T}\int_{0}^{\frac{T}{2}} f(t)\sin(k\omega t)\,dt\]
\end{claim}
容易利用对称性得到这些公式。

如果将傅里叶级数展开
式$f(t) =\frac{a_0}{2}+\sum_{k = 1}^{\infty} a_k \cos(k\omega t)+b_k\sin(k\omega t)$
写为
\begin{align*}
    f(t) = & \frac{a_0}{2}+\sum_{k = 1}^{\infty} a_k \cos(k\omega t) \\
           & +\sum_{k = 1}^{\infty} b_k \sin(k\omega t)
\end{align*}
则前半部分为偶函数,称之为$f(t)$的\textbf{偶分量}$f_e(t)$;后半部分为奇函数,称之为
$f(t)$的\textbf{奇分量}$f_o(t)$。高中数学中我们知道,函数的偶分量和奇分量都是唯一的,
并且\begin{align*}
    f_e(t)=\frac{f(t)+f(-t)}{2} \\
    f_o(t)=\frac{f(t)-f(-t)}{2}
\end{align*}

除了奇偶性,还可以从奇次谐波、偶次谐波的角度来理解函数。
\begin{definition}[奇谐函数]
    函数$f(t)$称为
\textbf{奇谐函数},如果后半个周期的函数是前半个周期的负镜像,即\begin{equation*}
    f\left(t+\frac{T}{2}\right)=-f(t)
\end{equation*}
\end{definition}
奇谐函数的傅里叶级数展开只有奇次谐波分量:
\begin{align*}
    a_k =&\frac{2}{T}\int_{T}f(t)\cos(k\omega t)\,dt\\
    &=\frac{2}{T}\int_{-T/2}^{0}f(t)\cos(k\omega t)\,dt+\frac{2}{T}\int_{0}^{T/2}f(t)\cos(k\omega t)\,dt\\
    &=\frac{2}{T}\int_{0}^{T/2}\left(f(t)\cos(k\omega t)+f(t-T/2)\cos(k\omega(t-T/2))\right)\,dt\\
    &=(1-\cos(k\pi))\frac{2}{T}\int_{0}^{T/2}f(t)\cos(k\omega t)\,dt\\
    &=\begin{cases}
        0,&\text{if }k\text{为偶数}\\
        \frac{4}{T}\int_{0}^{T/2}f(t)\cos(k\omega t)\,dt,&\text{if }k\text{为奇数}
    \end{cases}
\end{align*}
同理有\begin{align*}
    b_k=\begin{cases}
        0,&\text{if }k\text{为偶数}\\
        \frac{4}{T}\int_{0}^{T/2}f(t)\sin(k\omega t)\,dt,&\text{if }k\text{为奇数}
    \end{cases}
\end{align*}
有了奇谐函数自然也能够定义偶谐函数:$f\left(t+\frac{T}{2}\right)=f(t)$,容
易验证它只有偶次谐波分量,但从定义式可以看出这只是周期减半的函数。

由欧拉公式$e^{ik\omega t}=\cos(k\omega t)+i\sin(k\omega t)$,得到
\[\cos(k\omega t)=\frac{e^{ik\omega t}+e^{-ik\omega t}}{2},\sin(k\omega t)=\frac{e^{ik\omega t}-e^{-ik\omega t}}{2i}\]
故函数$f(t)$也可在指数函数系下展开:\begin{definition}[指数形式傅里叶展开]
    \[f(t)=\sum_{k = 0}^{\infty}  c_k e^{ik\omega t} ,c_k=\frac{a_k-ib_k}{2},c_{-k}=\frac{a_k+ib_k}{2}=c_k^*,k\in \mathbb{N}\]
(特别地,$c_0=c_0^*\Rightarrow c_0\in\mathbb{R} $)
\end{definition}
容易验证$\{e^{ik\omega t}\}_{k=0}^{\infty}$是完备正交函数系:
\begin{align*}
    (e^{ik_1\omega t},e^{ik_2\omega t}) & =\int_{T}e^{ik_1\omega t}(e^{ik_2\omega t})^*\,dt                 \\
                                        & =\int_{T}e^{i(k_1-k_2)\omega t}                                   \\
                                        & =\frac{2}{i\omega (k_1-k_2)}\evalat{e^{i(k_1-k_2)\omega t}}{0}{T} \\
                                        & =0(k_1,k_2\in \mathbb{Z},k_1\neq k_2)                             \\
    (e^{ik\omega t},e^{ik\omega t})     & =\int_{T}e^{ik\omega t}(e^{ik\omega t})^*\,dt                     \\
                                        & =\int_{T}\,dt=T(k\in \mathbb{Z} )
\end{align*}
和三角函数系的情况一样,我们得到$c_k$的计算公式:
\begin{align*}
    c_k & =\frac{(f(t),e^{ik\omega t})}{(e^{ik\omega t},e^{ik\omega t})} \\
        & =\frac{1}{T}\int_{T}f(t)(e^{ik\omega t})^*\,dt                 \\
        & =\frac{1}{T}\int_{T}f(t)e^{-ik\omega t}\,dt
\end{align*}
有时也将$c_k$记作$\hat{f}(k\omega)$或$\hat{F}(k)$,表示f在频域中的点$k\omega$处的值。一般而
言,我们只将最小正周期称为一个函数的周期,但周期为T的函数可以有多个频率
$k\omega(k \in \mathbb{Z})$,绘制频谱时,由于难以画出复数,常用\textbf{幅度谱}
$|\hat{f}(k\omega)|-\omega$和\textbf{相位谱}$\phi_k-\omega$来表征函数,其
中$\phi_k=\arg\hat{f}(k\omega)$。对于实信号,
\[c_k=(c_{-k})^*,|\hat{f}(k\omega)|=|\hat{f}(-k\omega)|,\phi_{-k}=-\phi_k\]
即幅度谱为偶函数,相位谱为奇函数,所以实信号的频谱中有一半是冗余的,按照展开式
\[\frac{c_0}{2}+\sum_{k = 1}^{\infty} c_k\cos(k\omega t+\varphi _k)\]绘制
的频谱$c_k-\omega$(注意不是指数函数形式的傅里叶系数)和$\phi_k-\omega$称为\textbf{单边频谱}
,而完整的频谱称为\textbf{双边频谱},从
\[\cos(k\omega t+\varphi _k)=\frac{e^{i(k\omega t+\varphi_k)}+e^{-i(k\omega t+\varphi_k)}}{2}\]
可知单边频谱相比双边频谱,在给定正频率处的幅值加倍,相位不变。这里的频率实际上是角频率$\omega$,用频率
f画频谱只涉及图像的横向伸缩,此处不再赘述。

需要指出的是,本小节中研究的函数均在$L^2([0,T])$空间中,但这并不能保证傅里叶
级数存在且收敛,保证这一点需要\textbf{狄利克雷条件}:
\begin{definition}[狄利克雷条件]
    \quad
    \begin{itemize}[nosep, left=0pt]
    \item $\int_{T}|f(t)|\,dt<\infty$
    \item 在一个周期内f连续或有有限个第一类间断点,即\textbf{分段连续} (piecewise continuous)
    \item 在一个周期内,f的极值点个数有限
\end{itemize}
\end{definition}
\begin{theorem}[狄利克雷定理]
    满足狄利克雷条件时,f的傅里叶级数展开在在任意点收敛到其左右极限的平均值:
    \[\forall t\in \mathbb{R},\sum_{k=\infty}^{\infty}c_k e^{2\pi ikt}=\frac{f(t+)+f(t-)}{2}\]
\end{theorem}
最后一个条件实际上相当于要求f是有界变差函数(Bounded Variatioin Function),
感兴趣的读者可以在实变函数的教材中了解这种函数。在附录\ref{sec:Asymptotic_Behaviour}中
我们将讨论另外的更易理解的条件。

下面考虑函数空间中的“勾股定理”。类比有限维线性空间中的勾股定理,有\begin{claim}[帕塞瓦尔定理/瑞利恒等式]
    \[\|f\|_2=\sum_{k=1}^{\infty}c_k^2|e^{ik\omega t}|^2=T\sum_{k=1}^{\infty}c_k^2\]
即\[P=\frac{1}{T}\int_{T}|f(t)|^2\,dt=\sum_{k=1}^{\infty}c_k^2\]
其中P为平均功率。
\end{claim}

至此,我们得到了傅里叶系数的公式和帕塞瓦尔定理,但其实证明用到的结论是基于有限
维线性空间的,现在就来填补这个逻辑漏洞,对此不感兴趣的读者可以忽略这部分内容。
以下设$\{\phi_n\}_{n=1}^{\infty}$是$L^2(a,b)$的标准正交基,$f\in L^2(a,b)$
(注意这里已经不局限于讨论傅里叶级数,并且与前文未标准化的正交基略有形式上的差别)。

\begin{lemma}[*贝塞尔不等式]
\[\sum_{n=1}^{\infty}|(f,\phi_n)|^2\leq\|f\|_2^2\]
\end{lemma}
\begin{proof}
    \begin{flalign*}
     & \text{由勾股定理,}\left\|\sum_{n=1}^{N}(f,\phi_n)\phi_n\right\|_2^2= \sum_{n=1}^{N}\left(f,(f,\phi_n)\phi_n\right)=\sum_{n=1}^{N}\overline{(f,\phi_n)}(f,\phi_n)=\sum_{n=1}^{N}|(f,\phi_n)|^2         \\
     & \text{因此,对任意正整数N,}0\leq                            \left\|f-\sum_{n=1}^{N}(f,\phi_n)\phi_n\right\|_2                                                                                 \\
     & \hspace{4cm}=                                                  \|f\|_2^2-2\text{Re}\ \left(f,\sum_{n=1}^{N}(f,\phi_n)\phi_n\right)+\left\|\sum_{n=1}^{N}(f,\phi_n)\phi_n\right\|_2^2                        \\
     & \hspace{4cm}=                                                    \|f\|_2^2-2\sum_{n=1}^{N}|(f,\phi_n)|^2+\sum_{n=1}^{N}|(f,\phi_n)|^2=\|f\|_2^2-\sum_{n=1}^{N}|(f,\phi_n)|^2
\end{flalign*}
令$N\to\infty$即证。
\end{proof}
从第二行到第三行用到了恒等式$\|\mathbf{a+b}\|^2=\|\mathbf{a}\|^2+2\text{Re}\ \langle\mathbf{a,b}\rangle+\|\mathbf{b}\|^2$,
Re表示取实部,这个结论十分简单,留予读者自证。在最终的结论帕塞瓦尔定理中这个不
等号将变成等号,但它是不可或缺的,并且我们还将在附录\ref{sec:Asymptotic_Behaviour}中见到它。

\begin{lemma}*
级数$\sum_{n=1}^{N}(f,\phi_n)\phi_n$依范数收敛,并且$\left\|\sum_{n=1}^{\infty}(f,\phi_n)\phi_n\right\|_2\leq\|f\|_2$
\end{lemma}
\begin{proof}
\begin{flalign*}
     & \text{由贝塞尔不等式,}\sum_{n=1}^{\infty}|(f,\phi_n)|^2                    \leq\|f\|_2^2<\infty,n\to\infty\text{时}|(f,\phi_n)|\to 0                         \\
     & \text{任取}m_1,m_2\in\mathbb{N},m_1<m_2,\text{由勾股定理,}                 \left\|\sum_{n=m_1}^{m_2}(f,\phi_n)\phi_n\right\|_2^2=\sum_{n=m_1}^{m_2}\left|(f,\phi_n)\right|^2\to 0 \\
     & \text{因此}\sum_{n=1}^{\infty}(f,\phi_n)\phi_n\text{构成柯西列.}                                                                                          \\
     & \text{令}m_1=1,m_2\to\infty,\left\|\sum_{n=1}^{\infty}(f,\phi_n)\phi_n\right\|_2 =\sum_{n=1}^{\infty}\left|(f,\phi_n)\phi_n\right|^2\leq\|f\|_2^2
\end{flalign*}
\end{proof}
柯西列能够推出收敛是因为$L^2(a,b)$是无限维的完备度量空间,即\textbf{希尔伯特空间} (Hilbert space)
,见\ref{sec:Linear_Space}。构建这个引理是为了使用希尔伯特空间中内积的连续性,其表述
见下一个命题。

\begin{proposition}[*希尔伯特空间H中的内积的连续性]
如果级数$\sum_{n=1}^{\infty}\phi_n$
的部分和$S_N$依范数收敛到S,则任给$y\in H$,总有
\[\lim_{N\to\infty}\left\langle S_n,y\right\rangle=\langle S,y\rangle\]
\end{proposition}
\begin{proof}
\begin{align*}
     & \langle S,y\rangle-\lim_{N\to\infty}\left\langle S_n,y\right\rangle=\lim_{N\to\infty}\langle S-S_n,y\rangle                                     \\
     & \lim_{N\to\infty}\|S-S_N\|=0\implies \lim_{N\to\infty}|\left\langle S-S_n,y\right\rangle|\leq\lim_{N\to\infty}\left\|S-S_N\right\|\|y\|=0               \\
     & \hspace{3cm}\implies \lim_{N\to\infty}\langle S-S_n,y\rangle=0\implies\lim_{N\to\infty}\left\langle S_n,y\right\rangle=\langle S,y\rangle
\end{align*}
\end{proof}

\begin{theorem}*
以下三个命题是等价的:(对于符号$\sim$,参考\ref{sec:Linear_Space})
\begin{enumerate}
    \item $\forall n,(f,\phi_n)=0\implies f\sim 0$,即$\{\phi_n\}_{n=1}^{\infty}$是完备的标准正交基
    \item $\forall f\in L^2(a,b)$,有$f\sim\sum_{n=1}^{\infty}(f,\phi_n)\phi_n$
    \item $\forall f\in L^2(a,b)$,有\textbf{帕塞瓦尔恒等式}:
          \[\|f\|_2^2=\sum_{n=1}^{\infty}|(f,\phi_n)|^2\]
\end{enumerate}
\end{theorem}
\begin{proof}
我们将证明$1\implies 2\implies 3\implies 1$.\\
$1\implies 2$:\begin{align*}
     & \text{令}g\sim \left(f-\sum_{n=1}^{\infty}(f,\phi_n)\phi_n.\right)                                                              \\
     & \forall m\in\mathbb{N},\left(g,\phi_m\right)=\left(f,\phi_m\right)-\sum_{n=1}^{\infty}\left(f,\phi_n\right)\left(\phi_n,\phi_m\right)=\left(f,\phi_m\right)-\left(f,\phi_m\right)=0 \\
\end{align*}
根据1知g=0,即2.这里内积与求和的换序是由命题2.3保证的。\\
$2\implies 3$:由勾股定理,
\[\|f\|_2^2=\lim_{N\to\infty}\left\|\sum_{n=1}^{N}(f,\phi_n)\phi_n\right\|_2^2=\lim_{N\to\infty}\sum_{n=1}^{N}|(f,\phi_n)|^2=\sum_{n=1}^{\infty}|(f,\phi_n)|^2\]
$3\implies 1$:$(f,\phi_n)=0\implies\|f\|_2=0\implies f\sim 0$.
\end{proof}
\begin{example}
\textbf{周期矩形脉冲信号}的傅里叶级数展开和频谱图\\
脉冲宽度为$\tau$,脉冲幅度为E,周期为$T (\tau<T)$的周期矩形脉冲信号,基波角频率
$\omega=\frac{2\pi}{T}$,傅里叶级数展开为
\[f(t)=\sum_{k = 1}^{\infty}  \frac{E\tau}{T}Sa\left(\frac{k\omega \tau}{2}\right)e^{ik\omega t}\]
\end{example}
\begin{figure}[H]
    \centering
    \includegraphics[width=0.6\textwidth]{Figure_2}
    \caption{周期矩形脉冲信号及其频谱}
\end{figure}
如图,可以看到,这个频谱与取样函数$Sa(\omega)$非常相似(为了体现这一点,绘制频谱时将
基波角频率大幅减小,并不是第一张图直接做傅里叶级数展开的结果),原因将在\ref{sec:Fourier}
中给出。

\textbf{带宽} (bandwidth)指最高频率与最低频率之差,表征信号频率的集中程度。对
于实信号,有时仅考虑正频率,带宽减半。周期矩形脉冲信号的频谱是无限的,但能量基
本集中在最靠近y轴的两个零点之间,此时可以将带宽定义为\textbf{第一过零点带宽}
$B=\frac{2\pi}{\tau}$(仅考虑正频率)。

\section{傅里叶变换}\label{sec:Fourier}
在构建傅里叶级数时,使用频率和角频率只涉及书写问题,因为傅里叶级数不会涉及尺度
变换、逆变换和卷积,但在傅里叶变换的理论中,这将导致许多公式在形式上有一些差别。
这时,将同时给出两种傅里叶变换的公式,左侧为频率版本,用蓝色标注,右侧为角频率
版本,用红色标注。对于频率的符号,物理上一般使用f或$\nu $,而一些傅里叶分析的
书上则使用s或$\xi$,但鉴于f常常用来表示信号或函数,s用于表示复频率,我们将使用$\xi$作为频率的符号。

定义傅里叶变换的一种动机是从傅里叶级数出发。要从傅里叶级数研究的周期现象转向傅
里叶变换研究的非周期现象,自然能够想到在傅里叶级数相关的理论中,令T趋于无穷;
另一方面,复指数函数比三角函数更适合作为描述振荡(周期)行为的基本函数。

我们做一个简单的尝试,令$f\in L^2(\mathbb{R})$(关于$L^2$空间的讨论,见\ref{sec:Linear_Space}),$T\to \infty$
,则\[c_n=\frac{1}{T}\int_{T}f(t)e^{-ik\omega t}\,dt=\frac{1}{T}\int_{T}f(t)e^{-2\pi ik\xi t}\,dt\to 0\]
这样做变换将丢失f的所有信息,不是我们希望看到的,但很明显,只要给以上公式乘上T,并认为$k\omega$
或$k\xi$是自变量,问题就迎刃而解,得到一个很有意思的积分变换,它正是\textbf{傅里叶变换} (Fourier Tansform,FT)。傅里叶本人同样意识到了傅里叶级数向傅里叶变换的过渡,他用同样的方法从形式上得到了傅里叶变换的公式,但我们使用的指数函数其实是后来柯西、泊松等数学家开始使用的,它使得公式具有更加简洁的形式。
\begin{definition}[傅里叶变换]
设$f\in L^1(\mathbb{R})$,其傅里叶变换定义为
\lr{
    \mathcal{F} f(\xi)=\int_{-\infty}^{\infty}f(t)e^{-2\pi i\xi t}\,dt
}{
    \mathcal{F} f(\omega)=\int_{-\infty}^{\infty}f(t)e^{-i\omega t}\,dt
}
有时也用$\hat{f}$或F表示f的傅里叶变换,记作
\[f\overset{\mathcal{F} }{\longleftrightarrow}F\]
并称之为\textbf{傅里叶变换对}。
\end{definition}

傅里叶变换没有最好的符号,在不引起歧义时采用最简洁和便于理解的即可。注意在傅里叶变换的理论中,要求$f\in L^1(\mathbb{R})$
而不是$L^2(\mathbb{R})$,$\mathbb{R}$是无穷区间,$L^1(\mathbb{R})$与
$L^2(\mathbb{R})$之间不存在包含关系。

可以验证,$f\in L^1(\mathbb{R})$时,它的
傅里叶变换存在并且是连续的:\begin{align*}
    |\mathcal{F} f(\xi)|                      & =\left|\int_{-\infty}^{\infty}f(t)e^{-2\pi i\xi t}\,dt\right|                       \\
                                              & \leq\int_{-\infty}^{\infty}|f(t)||e^{-2\pi i\xi t}|\,dt\\
                                              &=\int_{-\infty}^{\infty}|f(t)|\,dt<\infty                      \\
    |\mathcal{F} f(\xi+h)-\mathcal{F} f(\xi)| & =\left|\int_{-\infty}^{\infty}f(t)(e^{-2\pi i(\xi+h)t}-e^{-2\pi i\xi t})\,dt\right| \\
                                              & \leq\int_{-\infty}^{\infty}|f(t)||e^{-2\pi iht}-1|\,dt\to 0(h\to 0)
\end{align*}
另外,$\mathcal{F} f(x)\to 0(x\to\infty)$,这个结果称为\textbf{黎曼-勒贝格引理} (Riemann-Lebesgue Lemma),
将在附录\ref{sec:approach}给出证明。

下面解释“乘T,认为$k\omega$或$k\xi$为自变量”的本质。我们来看上一节\ref{sec:Fourier_Series}
末尾的例子,为了体现双边频谱与取样函数的相似性,我们取了一个较为特殊的周期矩形
脉冲信号,它的基波角频率应为图中两相邻竖直线间的间隔,即\textbf{谱线间隔},可见其频率极
小、周期极大,与我们研究非周期现象所用到的极限情况$T\to\infty$是一致的,换言之,
令$T\to\infty$自动地使“傅里叶系数”在频谱中的间隔变小,\textbf{周期信号趋向非
    周期信号的过程自动地使离散频谱趋向连续频谱}。这样,乘T就不难理解了,它的作
用是“除以$\frac{1}{T}$”,$\frac{1}{T}$是所在频率成分处小矩形的宽(类似于黎曼
积分),换言之,以频率$\xi$为横坐标,\textbf{谱系数}$c_n$是$\frac{n}{T}=n\xi$处的小矩形面积,
$Tc_n$是f中对应频率成分的含量,可以理解为单位频段内的谱系数,即频谱密度;以角频
率$\omega$为横坐标,$c_n$是$\frac{2\pi n}{T}=n\omega$处的小矩形面积,
$Tc_n$是f中对应角频率成分的含量。因此,$F(\xi)$或$F(\omega)$也称为频谱密度函
数。

实际上,从这个角度出发,可以立即得到\textbf{傅里叶逆变换} (Inverse Fourier Tansform,IFT)
的公式,因为我们已经将f展开为傅里叶级数,这对应着由f的傅里叶变换$\mathcal{F} f$
还原出f。我们知道
\lr{
f(t)&=\sum_{n=-\infty}^{\infty}c_n e^{2\pi i\xi t}\\
&=\frac{1}{T}\sum_{n=-\infty}^{\infty}(\int_{T}f(t)e^{-2\pi i\xi t}\,dt)e^{2\pi i\xi t}
}{
f(t)&=\sum_{n=-\infty}^{\infty}c_n e^{i\omega t}\\
&=\frac{1}{T}\sum_{n=-\infty}^{\infty}(\int_{T}f(t)e^{-i\omega t}\,dt)e^{i\omega t}
}
根据前文所述的对应关系,做以下替换(注意$T\to\infty$):
\lr{
    &\frac{1}{T}\rightarrow d\xi,\int_{T}f(t)e^{-2\pi i\xi t}\,dt\rightarrow \mathcal{F} f(\xi)\\
    &f(t)=\mathcal{F} ^{-1}\mathcal{F} (t)=\int_{-\infty}^{\infty}\mathcal{F} f(\xi)e^{2\pi i\xi t}\,d\xi
}{
    &\frac{2\pi}{T}\rightarrow d\omega,\int_{T}f(t)e^{i\omega t}\,dt\rightarrow \mathcal{F} f(\omega)\\
    &f(t)=\mathcal{F} ^{-1}\mathcal{F} (t)=\frac{1}{2\pi}\int_{-\infty}^{\infty}\mathcal{F} f(\omega)e^{i\omega t}\,d\omega
}
这里$\mathcal{F} ^{-1}$表示取傅里叶逆变换,和$\mathcal{F} $一样,是一种从函数空间到函数空间的映射,$\mathcal{F} $
是从时域函数到频域(角频域)函数的映射,$\mathcal{F} ^{-1}$是从频域(角频域)
到时域函数的映射。因此,严格来说我们总应该写上自变量,但在不引起歧义的情况
下允许略去,例如我们同时承认$F(\xi)=\mathcal{F} [f(t)](\xi)$和
$F(\xi)=\mathcal{F} [f(t)]$的写法。

至此我们得到了傅里叶逆变换\footnote{在一些文献中,傅里叶变换为$\frac{1}{\sqrt{2\pi}}\int_{-\infty}^{\infty}f(t)e^{-i\omega t}\,dt$,这样逆变换为$\frac{1}{\sqrt{2\pi}}\mathcal{F} ^{-1}f(t)=\frac{1}{2\pi}\int_{-\infty}^{\infty}f(\omega)e^{i\omega t}\,d\omega$,公式具有对称性,但不具有直接的物理意义,并且没有本质区别,故本书不加以讨论。}的定义:
\begin{definition}[傅里叶逆变换]
    设$f\in L^1(\mathbb{R})$,其傅里叶逆变换为\lr{
        \mathcal{F} ^{-1}f(t)=\int_{-\infty}^{\infty}f(\xi)e^{2\pi i\xi t}\,d\xi
    }{
        \mathcal{F} ^{-1}f(t)=\frac{1}{2\pi}\int_{-\infty}^{\infty}f(\omega)e^{i\omega t}\,d\omega
    }
\end{definition}
并且从形式上得到了\textbf{傅里叶反演公式} (the Fourier Inversion Thoerem):
\begin{claim}[傅里叶反演公式]
    \[f=\mathcal{F} \mathcal{F} ^{-1}f=\mathcal{F} ^{-1}\mathcal{F} f\]
\end{claim}
\begin{remark}
    简洁起见,这里给出的是$f(t)$连续时的结果,更一般的结果将在\ref{sec:approach}中给出,而不会导致循环论证。类似傅里叶级数,我们实际上也应该要求$f\in PC$,傅里叶逆变换收敛到左右极限的平均值,但更多时候我们不关心不连续点的值。
\end{remark}
下面介绍一些傅里叶变换的运算性质。
\begin{theorem}[傅里叶变换的运算性质]
    \begin{itemize}
    \quad    
    \item \textbf{对偶性}:记反转信号 (the reversed siganl)为$f^-(t)=f(-t)$,则
          \lr{(\mathcal{F}f)^-=\mathcal{F} (f^-)=\mathcal{F} ^{-1}f\\
              \mathcal{F} \mathcal{F} f=f^-\\
              f\text{是实信号}\Rightarrow \mathcal{F} f^-=\overline{\mathcal{F} f}
          }{(\mathcal{F}f)^-=\mathcal{F} (f^-)=2\pi\mathcal{F} ^{-1}f\\
              \mathcal{F} \mathcal{F} f=2\pi f^-\\
              f\text{是实信号}\Rightarrow \mathcal{F} f^-=\overline{\mathcal{F} f}}
          \begin{remark}
            时域反转,频域也反转,因此我们可以不区分$(\mathcal{F}f)^-=\mathcal{F} (f^-)$,将它们全部写作$\mathcal{F} f^-$。\\
          不涉及收敛性的问题时,的确可以做多次傅里叶变换,只是此时不再具有明显的物理意义,因而也不纠结所选用的符号。
          \end{remark}
    \item \textbf{对称性}:$\mathcal{F} f$与f奇偶性相同;f是实函数时,如果f还是偶函数,则$\mathcal{F} f$也是实函数,
          如果f还是奇函数,则$\mathcal{F} $是纯虚函数
    \item \textbf{线性性}:$\forall f,g\in L^1(\mathbb{R}),\mathcal{F} (af+bg)=a\mathcal{F} f+b\mathcal{F} g$,即$\mathcal{F} $是线性算子
    \item \textbf{平移定理}:\lr{
          &\mathcal{F} [f(t-b)](\xi)=e^{-2\pi i \xi b}\mathcal{F} f(\xi)\\
          &\mathcal{F} [f(t)e^{2\pi i \xi t}]=\mathcal{F} f(\xi-b)
          }{
          &\mathcal{F} [f(t-b)](\omega)=e^{-i\omega t}\mathcal{F} f(\omega)\\
          &\mathcal{F} [f(t)e^{ibt}](\omega)=\mathcal{F} f(\omega-b)
          }
          \begin{remark}
          可见信号时移$|\mathcal{F} f|$,而仅改变$\mathcal{F} f$的相位。
          \end{remark}
    \item \textbf{伸缩定理}:\lr{
              &\mathcal{F} [f(at)](\xi)=\frac{1}{|a|}\mathcal{F} f(\frac{\xi}{a})\\
              &\mathcal{F} f(a\xi)=\frac{1}{|a|}\mathcal{F} [f(\frac{t}{a})]
          }{
              &\mathcal{F} [f(at)](\omega)=\frac{1}{|a|}\mathcal{F} f(\frac{\omega}{a})\\
              &\mathcal{F} f(a\xi)=\frac{1}{|a|}\mathcal{F} [f(\frac{t}{a})]
          }
          \begin{remark}
            信号反转可看作a=-1的特例。可以认为两种频率下的傅里叶变换是通过伸缩得到的,即
          \[2\pi\xi=\omega,\textcolor{blue}{\mathcal{F}}(2\pi\xi)=\textcolor{red}{\mathcal{F}}(\omega)\]
          $a>1$,时域收缩,频域舒张、变矮;$0<a<1$,时域舒张,频域收缩、变高;$a<0$,时域和频域都额外做一次反转。
          \end{remark}
    \item \textbf{微分性质}:\lr{
              &\mathcal{F} (f')(\xi)=2\pi i\xi\mathcal{F} f(\xi)\\
              &\mathcal{F}(2\pi itf)(\xi)=-(\mathcal{F} f)'(\xi)\\
              &\text{即}\mathcal{F}(tf)(\xi)=\frac{i}{2\pi}(\mathcal{F} f)'(\xi)
          }{
              &\mathcal{F} (f')(\omega)=i\omega\mathcal{F} f(\omega)\\
              &\mathcal{F} [itf(t)](\omega)=-(\mathcal{F} f)'(\omega)\\
              &\text{即}\mathcal{F}(tf)(\omega)=i(\mathcal{F} f)'(\omega)
          }
          最后一步使用了线性性;容易将此性质推广至任意阶导数。
\end{itemize}
\end{theorem}
\begin{proof}
1.对偶性
\lr{
    (\mathcal{F} f)^-(\xi)&=\int_{-\infty}^{\infty}f(t)e^{2\pi i \xi t}\,dt\\
    \mathcal{F} (f^-)(\xi)&=\int_{-\infty}^{\infty}f(-t)e^{-2\pi i \xi t}\,dt\\
    &=\int_{-\infty}^{\infty}f(t)e^{2\pi i \xi t}\,dt\\
    \mathcal{F} ^{-1}f(x)&=\int_{-\infty}^{\infty}f(t)e^{2\pi ixt}\,dt
}{
    (\mathcal{F} f)^-(\omega)&=\int_{-\infty}^{\infty}f(t)e^{i \omega t}\,dt\\
    \mathcal{F} (f^-)(\omega)&=\int_{-\infty}^{\infty}f(-t)e^{-i \omega t}\,dt\\
    &=\int_{-\infty}^{\infty}f(t)e^{i \omega t}\,dt\\
    \mathcal{F} ^{-1}f(x)&=\frac{1}{2\pi}\int_{-\infty}^{\infty}f(t)e^{i\omega t}\,dt
}
因此\lr{(\mathcal{F}f)^-=\mathcal{F} (f^-)=\mathcal{F} ^{-1}f}{(\mathcal{F}f)^-=\mathcal{F} (f^-)=2\pi\mathcal{F} ^{-1}f}
同时取傅里叶变换,即得
\lr{
    &\mathcal{F} \mathcal{F} f=f^-\\
    &\text{f是实信号时,}f=\overline{f},\\
    &\mathcal{F} f^-(\xi)=\int_{-\infty}^{\infty}f(t)e^{2\pi i \xi t}\,dt\\
    &\ =\overline{\int_{-\infty}^{\infty}f(t)e^{-2\pi i \xi t}\,dt}=\overline{\mathcal{F} f(\xi)}
}{
    &\mathcal{F} \mathcal{F} f=2\pi f^-\\
    &\text{f是实信号时,}f=\overline{f},\\
    &\mathcal{F} f^-(\omega)=\int_{-\infty}^{\infty}f(t)e^{i \omega t}\,dt\\
    &\ =\overline{\int_{-\infty}^{\infty}f(t)e^{-i \omega t}\,dt}=\overline{\mathcal{F} f(\omega)}
}
\noindent 2.对称性:根据对偶性立即得到。\\
3.线性性:得自积分的线性性。\\
4.平移定理
\lr{
\mathcal{F} [f(t-b)](\xi)&=\int_{-\infty}^{\infty}f(t-b)e^{-2\pi i \xi t}\,dt\\
&=\int_{-\infty}^{\infty}f(t)e^{-2\pi i \xi (t+b)}\,dt\\
&=e^{-2\pi i \xi b}\int_{-\infty}^{\infty}f(t)e^{-2\pi i \xi t}\,dt\\
&=e^{-2\pi i \xi b}\mathcal{F} f(\xi)\\
\mathcal{F} f(\xi-b)&=\int_{-\infty}^{\infty}f(t)e^{-2\pi i(\xi-b)t}\,dt\\
&=\int_{-\infty}^{\infty}\left(f(t)e^{2\pi ibt}\right)e^{-2\pi i\xi t}\,dt\\
&=\mathcal{F} [f(t)e^{2\pi ibt}](\xi)
}{
\mathcal{F} [f(t-b)](\omega)&=\int_{-\infty}^{\infty}f(t-b)e^{-i \omega t}\,dt\\
&=\int_{-\infty}^{\infty}f(t)e^{-i \omega (t+b)}\,dt\\
&=e^{-i \omega b}\int_{-\infty}^{\infty}f(t)e^{-i \omega t}\,dt\\
&=e^{-i \omega b}\mathcal{F} f(\omega)\\
\mathcal{F} f(\omega-b)&=\int_{-\infty}^{\infty}f(t)e^{-i(\omega-b)t}\,dt\\
&=\int_{-\infty}^{\infty}\left(f(t)e^{ibt}\right)e^{-i\omega t}\,dt\\
&=\mathcal{F} [f(t)e^{ibt}](\xi)
}
\noindent 5.伸缩定理\lr{
    \mathcal{F} [f(at)](\xi)&=\int_{-\infty}^{\infty}f(at)e^{2\pi i \xi t}\,dt\\
    &=\frac{1}{|a|}\int_{-\infty}^{\infty}f(t)e^{\frac{2\pi i \xi t}{a}}\,dt&(at\to t)\\
    &=\frac{1}{|a|}\mathcal{F} f(\frac{\xi}{a})
}{
    \mathcal{F} [f(at)](\omega)&=\int_{-\infty}^{\infty}f(at)e^{i \omega t}\,dt\\
    &=\frac{1}{|a|}\int_{-\infty}^{\infty}f(t)e^{\frac{i \omega t}{a}}\,dt&(at\to t)\\
    &=\frac{1}{|a|}\mathcal{F} f(\frac{\omega}{a})
}
注意变量代换时,如果$a<0$,积分上下限也会改变,这正是绝对值的来源,对此有疑惑的读者
可以自行分情况验算。对于频域的伸缩定理,仅仅是时域伸缩定理的直接推论。\\
6.微分性质\\
\ding{172}时域微分
\lr{
\mathcal{F} (f')(\xi)&=\int_{-\infty}^{\infty}f'(t)e^{-2\pi i\xi t}\,dt\\
&=\int_{-\infty}^{\infty}e^{-2\pi i\xi t}\,df(t)\\
&=\evalat{e^{-2\pi i\xi t}f(t)}{-\infty}{\infty}+2\pi i\xi\int_{-\infty}^{\infty}f(t)e^{-2\pi i\xi t}\,dt\\
&=2\pi i\xi\mathcal{F} f(\xi)
}{
\mathcal{F} (f')(\omega)&=\int_{-\infty}^{\infty}f'(t)e^{-i\omega t}\,dt\\
&=\int_{-\infty}^{\infty}e^{-i\omega t}\,df(t)\\
&=\evalat{e^{-i\omega t}f(t)}{-\infty}{\infty}+i\omega\int_{-\infty}^{\infty}f(t)e^{-i\omega t}\,dt\\
&=i\omega\mathcal{F} f(\omega)
}
\ding{173}频域微分
\lr{
    (\mathcal{F} f)'(\xi)&=\frac{d}{d\xi}\int_{-\infty}^{\infty}f(t)e^{-2\pi i\xi t}\,dt\\
&=\int_{-\infty}^{\infty}f(t)\frac{\partial e^{-2\pi i\xi t}}{\partial \xi}\,dt\\
&=-\int_{-\infty}^{\infty}2\pi itf(t)e^{-2\pi i\xi t}\,dt\\
&=-\mathcal{F} (2\pi itf)(\xi)
}{
    (\mathcal{F} f)'(\omega)&=\frac{d}{d\omega}\int_{-\infty}^{\infty}f(t)e^{-i\omega t}\,dt\\
&=\int_{-\infty}^{\infty}f(t)\frac{\partial e^{-i\omega t}}{\partial \omega}\,dt\\
&=-\int_{-\infty}^{\infty}itf(t)e^{-i\omega t}\,dt\\
&=-\mathcal{F} (itf)(\omega)
}
也可以由第一个微分性质取傅里叶逆变换,再令$f'=\mathcal{F} g$证明。
\end{proof}
$\evalat{e^{-2\pi i\xi t}f(t)}{-\infty}{\infty},\evalat{e^{-i\omega t}f(t)}{-\infty}{\infty}=0$
是因为$f\in L^1(\mathbb{R})$要求反常积分$\int_{\mathbb{R}}|f(t)|\,dt<\infty$
,其必要条件为$f(t)\to 0,t\to\infty$,而复指数函数部分模值恒为1。对于后一等式
中将求导与积分交换使用了\textbf{莱布尼兹积分法则},也即含参变量积分
的求导,通常需要条件(以角频率形式为例)\begin{circlist}
    \item $\int_{-\infty}^{\infty}f(t)e^{-i\omega t}\,dt$对每个$\omega$可积
    \item $\partial f(t)e^{-i\omega t}/\partial \omega=-itf(t)e^{-i\omega t}$存在
    \item 存在可积函数$g(t)$使得$|-itf(t)e^{-i\omega t}|\leq g(t)$
\end{circlist}
最后一条成立是因为定理假设$f(t),tf(t)\in L^1(\mathbb(R)),g(t)=tf(t)$,否则无法做傅里叶变换。

下面介绍一些常用信号的傅里叶变换,并使用傅里叶反演公式和对偶性得到一些难以直接
计算的常用傅里叶变换。

\begin{example}[矩形函数]
在\ref{sec:Fourier_Series}中讨论了矩形函数$f(t)=E\cdot\Pi_T(t)$的傅里
叶变换,现在可以验证它的频谱与取样函数相似:
\lr{
\mathcal{F} f(\xi)&=\int_{-\infty}^{\infty}f(t)e^{-2\pi i\xi t}\,dt\\
&=E\int_{-\infty}^{\infty}\Pi_T(t)e^{-2\pi i\xi t}\,dt\\
&=E\int_{-\frac{T}{2}}^{\frac{T}{2}}e^{-2\pi i\xi t}\,dt\\
&=-\frac{E}{2\pi i\xi}\evalat{e^{-2\pi i\xi t}}{-\frac{T}{2}}{\frac{T}{2}}\\
&=\frac{e^{\pi i\xi T}-e^{-\pi i\xi T}}{2i}\frac{E}{\pi\xi}\\
&=\frac{E}{\pi\xi}\sin(\pi T\xi)=ETsinc(T\xi)
}{
\mathcal{F} f(\omega)&=\int_{-\infty}^{\infty}f(t)e^{-i\omega t}\,dt\\
&=E\int_{-\infty}^{\infty}\Pi_T(t)e^{-i\omega t}\,dt\\
&=E\int_{-\frac{T}{2}}^{\frac{T}{2}}e^{-i\omega t}\,dt\\
&=-\frac{E}{i\omega}\evalat{e^{-i\omega t}}{-\frac{T}{2}}{\frac{T}{2}}\\
&=\frac{e^{\frac{i\omega T}{2}}-e^{-\frac{i\omega T}{2}}}{2i}\frac{2E}{\pi\omega}\\
&=\frac{2E}{\omega}\sin(\frac{T\omega}{2})=ETSa\left(\frac{T\omega}{2}\right)
}
因此\lr{
    \mathcal{F} \Pi_T(\xi)&=Tsinc(T\xi)\\
    \mathcal{F} \left[sinc(Tt)\right](\xi)&=\frac{1}{T}\Pi\left(\frac{\xi}{T}\right)
}{
    \mathcal{F} \Pi_T(\omega)&=TSa\left(\frac{T\omega}{2}\right)\\
    \mathcal{F}\left[Sa(Tt)\right](\omega)&=\frac{\pi}{T}\Pi_{2T}(\omega)
}
\end{example}
\begin{example}[狄拉克函数]
在\ref{sec:signal}中介绍了狄拉克$\delta$函数,实际上它应该作为一个分布来理解,
见\ref{sec:distributions},不过我们可以从形式上求出它的傅里叶变换。
\lr{
\mathcal{F} \delta(\xi)&=\int_{-\infty}^{\infty}\delta(t)e^{-2\pi i\xi t}\,dt\\
&=\int_{-\infty}^{\infty}\delta(t)\,dt=1
}{
\mathcal{F} \delta(\omega)&=\int_{-\infty}^{\infty}\delta(t)e^{-i\omega t}\,dt\\
&=\int_{-\infty}^{\infty}\delta(t)\,dt=1
}
\end{example}
根据傅里叶变换的对偶性,我们当然希望恒为1的函数的傅里叶变换是$\delta$或$2\pi\delta$
(取决于是用频率做变换还是用角频率做变换),然而,1在无限区间上必定是不可积的,
在常规意义下它不能够做傅里叶变换。这个问题将在\ref{sec:distributions}中讨论,
那时就可以对相当大范围内的函数做傅里叶变换,还将在\ref{sec:Dirac_Comb}中
看到周期函数的傅里叶变换与傅里叶级数的深刻关系。现在我们暂且承认公式:
\begin{claim}
    用$\mathds{1}$表示恒为1的函数。
\lr{
    &\mathcal{F} \delta(\xi)=1\\
    &\mathcal{F} \mathds{1}=\delta(\xi)
}{
    &\mathcal{F} \delta(\omega)=1\\
    &\mathcal{F} \mathds{1}=2\pi\delta(\omega)
}
\end{claim}
\begin{corollary}
    根据傅里叶变换的平移定理,立即得到:
\lr{
\mathcal{F} [\delta_a](\xi)&=e^{-2\pi i\xi a}\\
\mathcal{F} \left[e^{2\pi ia t}\right]&=\delta_a
}{
\mathcal{F} [\delta_a](\omega)&=e^{-i\omega a}\\
\mathcal{F} \left[e^{ia t}\right]&=2\pi\delta_a
}
根据傅里叶变换的微分性质得到:\lr{
    \mathcal{F} [t^n](\xi)&=(\frac{i}{2\pi})^n \delta^{(n)}(\xi)\\
}{
    \mathcal{F} [t^n](\omega)&=2\pi i^n\delta^{(n)}(\omega)\\
}\end{corollary}
我们还希望从$u'(t)=\delta(t),sgn'(t)=2\delta(t)$
得到单位阶跃函数u(t)和符号函数sgn(t)的傅里叶变换,但在考虑$\delta$的不定积分时,必须
处理“C”,它将导致频域中出现$C\delta$或$2\pi C\delta$项。注意到$sgn(t)$是奇函数,其
傅里叶变换应为纯虚的奇函数,我们可以由此确定它和单位阶跃函数的傅里叶变换中C的值,从而
得到正确的结果:
\begin{corollary}
    \quad
    \lr{
\mathcal{F} sgn(\xi)&=\frac{1}{\pi i\xi}\\
\mathcal{F} u(\xi)&=\mathcal{F} \left[\frac{1}{2}(sgn(t)+1)\right]\\
&=\frac{1}{2}\left(\delta+\frac{1}{\pi i\xi}\right)\\
}{
\mathcal{F} sgn(\omega)&=\frac{2}{i\omega}\\
\mathcal{F} u(\omega)&=\mathcal{F} \left[\frac{1}{2}(sgn(t)+1)\right]\\
&=\pi\delta+\frac{1}{i\omega}\\
}
\end{corollary}

用傅里叶反演公式,得到:
\begin{corollary}
    \quad
\lr{
    \mathcal{F} \left[\frac{1}{t}\right](\xi)&=-\pi i sgn(\xi)\\
}{
    \mathcal{F} \left[\frac{1}{t}\right](\omega)&=-\pi i sgn(\omega)\\
}
\end{corollary}

\begin{example}[$\Lambda$函数]
    \[\Lambda(t)=\begin{cases}
        1-|t| & \text{if }|t|\leq 1 \\
        0     & \text{if }|t|>1
    \end{cases}\]
它在卷积的章节中是一个很好的例子。由于其计算过于繁琐,这里不给出具体证明,感兴趣的读者可以自行验证。\lr{
    \mathcal{F} sinc^2(\xi)&=\Lambda(t)\\
}{
    \mathcal{F} Sa^2\left(\frac{\omega}{2}\right)&=\Lambda(t)
}
\end{example}

\begin{figure}[htbp]
    \centering
    \includegraphics[width=0.4\textwidth]{lambda}
    \caption{$\Lambda$函数图像}
\end{figure}

\begin{example}[高斯函数]
 $G(t)=\frac{1}{\sqrt{2\pi}\sigma}e^{-\frac{t^2}{2\sigma^2}}$,求它的傅里叶变换的方法较为特殊:
\lr{
\mathcal{F} G(\xi)&=\int_{-\infty}^{\infty}\frac{1}{\sqrt{2\pi}\sigma}e^{-\frac{t^2}{2\sigma^2}}e^{-2\pi i\xi t}\,dt\\
\frac{d}{d\xi}\mathcal{F} G(\xi)&=\int_{-\infty}^{\infty}\frac{1}{\sqrt{2\pi}\sigma}e^{-\frac{t^2}{2\sigma^2}}(-2\pi it)e^{-2\pi i\xi t}\,dt\\
&=2\pi i\sigma^2\int_{-\infty}^{\infty}e^{-2\pi i\xi t}\,d\frac{1}{\sqrt{2\pi}\sigma}e^{-\frac{t^2}{2\sigma^2}}\\
&=-4\pi^2\sigma^2\xi\int_{-\infty}^{\infty}\frac{1}{\sqrt{2\pi}\sigma}e^{-\frac{t^2}{2\sigma^2}}e^{-2\pi i\xi t}\,dt\\
&=-4\pi^2\sigma^2\xi\mathcal{F} G
}{
\mathcal{F} G(\omega)&=\int_{-\infty}^{\infty}\frac{1}{\sqrt{2\pi}\sigma}e^{-\frac{t^2}{2\sigma^2}}e^{-i\omega t}\,dt\\
\frac{d}{d\xi}\mathcal{F} G(\xi)&=\int_{-\infty}^{\infty}\frac{1}{\sqrt{2\pi}\sigma}e^{-\frac{t^2}{2\sigma^2}}(-it)e^{-i\omega t}\,dt\\
&=i\sigma^2\int_{-\infty}^{\infty}e^{-i\omega t}\,d\frac{1}{\sqrt{2\pi}\sigma}e^{-\frac{t^2}{2\sigma^2}}\\
&=-\sigma^2\omega\int_{-\infty}^{\infty}\frac{1}{\sqrt{2\pi}\sigma}e^{-\frac{t^2}{2\sigma^2}}e^{-i\omega t}\,dt\\
&=-\sigma^2\omega\mathcal{F} G
}
这是一个可分离变量的微分方程,
\lr{
\mathcal{F} G(\xi)&=\mathcal{F} G(0)e^{-2\pi^2\sigma^2\xi^2}\\
\mathcal{F} G(0)&=\int_{-\infty}^{\infty}\frac{1}{\sqrt{2\pi}\sigma}e^{-\frac{t^2}{2\sigma^2}}\,dt=1\\
\mathcal{F} G(\xi)&=e^{-2\pi^2\sigma^2\xi^2}
}{
\mathcal{F} G(\omega)&=\mathcal{F} G(0)e^{-\frac{\sigma^2\omega^2}{2}}\\
\mathcal{F} G(0)&=\int_{-\infty}^{\infty}\frac{1}{\sqrt{2\pi}\sigma}e^{-\frac{t^2}{2\sigma^2}}\,dt=1\\
\mathcal{F} G(\xi)&=e^{-\frac{\sigma^2\omega^2}{2}}
}
\end{example}
\begin{figure}[htbp]
    \centering
    \includegraphics[width=0.4\textwidth]{Gauss}
    \caption{高斯函数图像}
\end{figure}

\begin{example}[单边指数函数和双边指数函数]
$f(t)=e^{-at}u(t)=\begin{cases}
        e^{-at}, & \text{if }t>0 \\
        0,       & \text{if }t<0
    \end{cases}$称为单边按指数函数,
$g(t)=f(t)+f(-t)=\begin{cases}
        e^{-at}, & \text{if }t\geq 0 \\
        e^{at},  & \text{if }t<0
    \end{cases}$称为双边指数函数\\
\lr{
\mathcal{F} f(\xi)&=\int_{-\infty}^{\infty}f(t)e^{-2\pi i\xi t}\,dt\\
&=\int_{0}^{\infty}e^{-at}e^{-2\pi i\xi t}\,dt\\
&=-\frac{1}{a+2\pi i\xi}\left.e^{-(a+2\pi i\xi)t}\right|_{0}^{\infty}\\
&=\frac{1}{a+2\pi i\xi}
}{
\mathcal{F} f(\omega)&=\int_{-\infty}^{\infty}f(t)e^{-i\omega t}\,dt\\
&=\int_{0}^{\infty}e^{-at}e^{-i\omega t}\,dt\\
&=-\frac{1}{a+i\omega}\left.e^{-(a+i\omega)t}\right|_{0}^{\infty}\\
&=\frac{1}{a+i\omega}
}
运用对偶性,立即得到
\lr{
\mathcal{F} g(\xi)&=\mathcal{F} f(\xi)+\overline{\mathcal{F} f(\xi)}\\
&=2Re{\mathcal{F} f(\xi)}=\frac{2a}{a^2 + 4\pi^2 \xi^2}
}{
\mathcal{F} g(\omega)&=\mathcal{F} f(\omega)+\overline{\mathcal{F} f(\omega)}\\
&=2Re{\mathcal{F} f(\omega)}=\frac{2a}{a^2 +\omega^2}
}
\end{example} 

最后,我们给出\textbf{帕塞瓦尔恒等式}(Parseval's identity),帕塞瓦尔在18世纪末提出了它的三角级数形式,但没有给出证明,如果将傅里叶变换视作傅里叶级数的推广,那么“能量守恒”也自然地推广到傅里叶变换。有时我们也称之为\textbf{普朗歇尔定理}(Plancherel theorem),因为它于1910年普朗歇尔的博士论文中被严格证明,进一步,傅里叶变换是$L^1(\mathbb{R})\cap L^2(\mathbb{R})$上的等距变换,保持$L^2$范数不变。
\begin{theorem}[帕塞瓦尔恒等式/普朗歇尔定理]
    设$f\in L^1(\mathbb{R})\cap L^2(\mathbb{R})$,则
    \lr{
    \int_{-\infty}^{\infty}|f(t)|^2\,dt=\int_{-\infty}^{\infty}|\mathcal{F} f(\xi)|^2\,d\xi
}{
    \int_{-\infty}^{\infty}|f(t)|^2\,dt=\frac{1}{2\pi}\int_{-\infty}^{\infty}|\mathcal{F} f(\omega)|^2\,d\omega
}
\end{theorem}
\begin{proof}
设$f,g\in L^1(\mathbb{R})\cap L^2(\mathbb{R})$,\lr{
&\quad\int_{-\infty}^{\infty}\mathcal{F} f(\xi)\overline{\mathcal{F} g(\xi)}\,d\xi\\
&=\int_{-\infty}^{\infty}\mathcal{F} f(\xi)\mathcal{F}^{-1} \overline{g}(\xi)\,d\xi\\
&=\int_{-\infty}^{\infty}\left(\int_{-\infty}^{\infty}f(x)e^{-2\pi i\xi x}\,dx\right)\mathcal{F}^{-1} \overline{g}(\xi)\,d\xi\\
&=\int_{-\infty}^{\infty}f(x)\,dx\int_{-\infty}^{\infty}\mathcal{F}^{-1} \overline{g}(\xi)e^{-2\pi i\xi x}\,d\xi\\
&=\int_{-\infty}^{\infty}f(x)\mathcal{F} \mathcal{F} ^{-1}\overline{g}(x)\,dx\\
&=\int_{-\infty}^{\infty}f(x)\overline{g(x)}\,dx
}{
&\quad\int_{-\infty}^{\infty}\mathcal{F} f(\omega)\overline{\mathcal{F} g(\omega)}\,d\omega\\
&=2\pi\int_{-\infty}^{\infty}\mathcal{F} f(\omega)\mathcal{F}^{-1} \overline{g}(\omega)\,d\omega\\
&=2\pi\int_{-\infty}^{\infty}\left(\int_{-\infty}^{\infty}f(x)e^{-i\omega x}\,dx\right)\mathcal{F}^{-1} \overline{g}(\omega)\,d\omega\\
&=2\pi\int_{-\infty}^{\infty}f(x)\,dx\int_{-\infty}^{\infty}\mathcal{F}^{-1} \overline{g}(\omega)e^{-i\omega x}\,d\omega\\
&=2\pi\int_{-\infty}^{\infty}f(x)\mathcal{F} \mathcal{F} ^{-1}\overline{g}(x)\,dx\\
&=2\pi\int_{-\infty}^{\infty}f(x)\overline{g(x)}\,dx
}
取$g=f$即证。注意我们并不要求$g$是实信号,$\overline{\mathcal{F} g}\neq\mathcal{F} ^{-1}g$.
\end{proof}

\section{卷积}\label{sec:convolution}

信号处理讨论的一个基本问题是\textbf{滤波},即希望把一个信号输入滤波系统后,输
出的信号的一些频率成分被剔除或大幅减少,以低通滤波器为例,从数学上讲,就是把信
号的频域形式乘以一个矩形函数或一个在给定的频率值之外快速下降到接近于0的函数,
这就引出了一个问题:在频域乘一个函数,在时域上的表现是什么?我们知道,一般而言
没有$\mathcal{F} (fg)=\mathcal{F} f\mathcal{F} g$。一个自然的想法是,看能
否定义一种运算,使得在频域乘一个函数,相当于在时域与这个函数的时域形式做该种运
算。实际上,这种运算是存在的,它正是\textbf{卷积}(convolution)

下面就来找出这个运算。设$f\overset{\mathcal{F} }{\longleftrightarrow}F,g\overset{\mathcal{F} }{\longleftrightarrow}G$,
\lr{
    F(\xi)G(\xi)=&\int_{-\infty}^{\infty}f(x)e^{-2\pi i\xi x}\,dx\int_{-\infty}^{\infty}g(y)e^{-2\pi i\xi y}\,dy\\
    &=\iint\limits_{\mathbb{R}^2}f(x)g(y)e^{-2\pi i\xi(x+y)}\,dx\,dy
}{
    F(\omega)G(\omega)=&\int_{-\infty}^{\infty}f(x)e^{-i\omega x}\,dx\int_{-\infty}^{\infty}g(y)e^{-i\omega y}\,dy\\
    &=\iint\limits_{\mathbb{R}^2}f(x)g(y)e^{-i\omega(x+y)}\,dx\,dy
}
令$z=x+y$,则积分区域仍为$\mathbb{R}^2$,
\[dxdz=\left|\frac{\partial(x,z)}{\partial(x,y)}\right|dxdy=\left|\begin{vmatrix}
        1 & 0 \\
        1 & 1
    \end{vmatrix}\right| dxdy=dxdy\]
\lr{
F(\xi)G(\xi)=&\iint\limits_{\mathbb{R}^2}f(x)g(z-x)e^{-2\pi i\xi z}\,dx\,dz\\
&=\int_{-\infty}^{\infty}e^{-2\pi i\xi z}\,dz\int_{-\infty}^{\infty}f(x)g(z-x)dx\\
&=\mathcal{F} \left[\int_{-\infty}^{\infty}f(x)g(z-x)dx\right](\xi)
}{
F(\omega)G(\omega)=&\iint\limits_{\mathbb{R}^2}f(x)g(z-x)e^{-i\omega z}\,dx\,dz\\
&=\int_{-\infty}^{\infty}e^{-i\omega z}\,dz\int_{-\infty}^{\infty}f(x)g(z-x)dx\\
&=\mathcal{F} \left[\int_{-\infty}^{\infty}f(x)g(z-x)dx\right](\omega)
}

因此我们定义\begin{definition}[卷积]
函数f,g的\textbf{卷积}为
\begin{equation*}
    (f*g)(x)=\int_{-\infty}^{\infty}f(y)g(x-y)\,dy
\end{equation*}
\end{definition}
根据前文中的结果,有$\mathcal{F} (f*g)=\mathcal{F} f\mathcal{F} g$,这是时域卷积的性质,由
傅里叶反演公式,不难想到频域卷积也有类似的性质。令$f=\mathcal{F} \mathfrak{f},g=\mathcal{F} \mathfrak{g}$
,对以上公式两边同时取傅里叶逆变换:
\lr{
f*g&=\mathcal{F} ^{-1}\left(\mathcal{F} f\mathcal{F} g\right)\\
\iff \mathcal{F} \mathfrak{f}*\mathcal{F} \mathfrak{g}&=\mathcal{F} ^{-1}\left[\mathcal{F}\mathcal{F} \mathfrak{f}\cdot\mathcal{F} \mathcal{F} \mathfrak{g}\right]\\
&=\mathcal{F} ^{-1}\left(\mathfrak{f}^- \mathfrak{g}^- \right)\\
&=\mathcal{F} (\mathfrak{fg})\\
\iff\mathcal{F} (fg)(\xi)&=\mathcal{F} f*\mathcal{F} g(\xi)
}{
f*g&=\mathcal{F} ^{-1}\left(\mathcal{F} f\mathcal{F} g\right)\\
\iff \mathcal{F} \mathfrak{f}*\mathcal{F} \mathfrak{g}&=\mathcal{F} ^{-1}\left[\mathcal{F}\mathcal{F} \mathfrak{f}\cdot\mathcal{F} \mathcal{F} \mathfrak{g}\right]\\
&=\mathcal{F} ^{-1}\left(4\pi^2 \mathfrak{f}^- \mathfrak{g}^- \right)\\
&= 2\pi\mathcal{F} (\mathfrak{fg})\\
\iff\mathcal{F} (fg)(\omega )&=\frac{1}{2\pi}\mathcal{F} f*\mathcal{F} g(\omega)
}
综上得到\textbf{卷积定理}(the convolution thoerem):\begin{theorem}[卷积定理]
    \quad
\lr{
    \mathcal{F} (f*g)(\xi)=\mathcal{F} f(\xi)\mathcal{F} g(\xi)\\
    \mathcal{F} (fg)(\xi)=(\mathcal{F} f*\mathcal{F} g)(\xi)
}{
    \mathcal{F} (f*g)(\omega)=\mathcal{F} f(\omega)\mathcal{F} g(\omega)\\
    \mathcal{F} (fg)(\omega)=\frac{1}{2\pi}(\mathcal{F} f*\mathcal{F} g)(\omega)
}
\end{theorem}

上一节中我们曾花费大量的篇幅寻找$\Lambda$的傅里叶变换,现在可以用卷积定理得到
它,因为$\Lambda=\Pi_{1/2}*\Pi_{1/2}$(读者可以自行用代数方法验证)。

在不引起歧义时,我们也承认$f(t)*g(t)$和$f(t)*e^{t+1}$这样的写法。需要注意,$f(2t)*g(t)$
对应着两种理解:$\int_{-\infty}^{\infty}f(2t-x)g(x)\,dx$和$\int_{-\infty}^{\infty}f(2(t-x))g(x)\,dx$
,第二种才是对的,因为我们认为$f(2t)$作为一个新的函数$F(t)=f(2t)$与$g(t)$进行卷积。

作为一种新的函数空间上的运算,我们自然要讨论它是否满足线性性、结合律、
交换律。事实上,它们都是成立的:
\begin{proposition}[卷积的线性性、结合律、交换律]
    \begin{align*}
    f*(ag_1+bg_2) & =af*g_1+bf*g_2 \\
    (f*g)*h       & =f*(g*h)       \\
    f*g           & =g*f
\end{align*}
\end{proposition}
线性性得自积分的线性性,交换律通过变量替换即可证明,下面仅证明结合律。
\begin{proof}
\begin{align*}
    (f*g)*h(x) & =\int_{-\infty}^{\infty}(f*g)(x-y)h(y)\,dy=\int_{-\infty}^{\infty}h(y)\,dy\int_{-\infty}^{\infty}f(z)g(x-y-z)\,dz \\
               & =\int_{-\infty}^{\infty}\int_{-\infty}^{\infty}f(z)g(x-y-z)h(y)\,dy\,dz=\int_{-\infty}^{\infty}f(z)(g*h)(x-z)\,dz
\end{align*}
\end{proof}

也可以通过取傅里叶变换的方式证明它们,但这样会缩减证明有效的范围,因为
卷积存在只要求积分$(f*g)(x)=\int_{-\infty}^{\infty}f(y)g(x-y)\,dy$
存在(它有许多种充分条件,不再一一讨论),而取傅里叶变换则要求$f,g,f*g$
的傅里叶变换存在。

接着讨论卷积是否具有“幺元”,即与任一函数卷积,总得到它本身。事实上,只有$\delta$可以起到这个作用,但它应该作为一个分布来理解,见\ref{sec:distributions}和附录\ref{sec:approach}。
\begin{proposition}[卷积幺元]
    设$f\in L^1(\mathbb{R})$,则$\delta$是“卷积幺元”,即
    \lr{
    f(t)=(f*\delta)(t)=\int_{-\infty}^{\infty}f(x)\delta(t-x)\,dx
}{
    f(t)=(f*\delta)(t)=\int_{-\infty}^{\infty}f(x)\delta(t-x)\,dx
}
\end{proposition}
\begin{proof}
设$f\in L^1(\mathbb{R})$,则\lr{
\mathcal{F} (f*\delta)(\xi)=\mathcal{F} f(\xi)\mathcal{F} \delta(\xi)=\mathcal{F} f(\xi)
}{
    \mathcal{F} (f*\delta)(\omega)=\mathcal{F} f(\omega)\mathcal{F} \delta(\omega)=\mathcal{F} f(\omega)
}
由傅里叶反演公式得证。
\end{proof}

初次接触卷积时,往往会对它的定义感到疑惑,因为在数学分析的课程中我们很少见到这
种“翻转、平移、相乘、积分”的结构。需要指出,卷积并不只有“时域相乘,频域卷积;
时域卷积,频域相乘”的物理意义,例如概率论中两独立的连续型随机变量X,Y之和作为一种新的
随机变量Z,其概率密度函数$f_Z(z)$正是两个独立的随机变量的概率密度函数的卷积$f_X*f_Y(z)$.
类似于用“求曲线下方的面积”或“已知速度求位移”引入积分,尽管我们用一种较为自然的方
式引入了卷积,但不应该认为它只有单一的意义。不过,还是可以建立一些卷积的性质来
辅助我们理解卷积。

\noindent 1.\textbf{卷积是一种起“平均化”作用的运算}
\begin{definition}[加权均值]
    给定区间$[a,b]$和权函数$w(x)$,$f(x)$的加权均值为\[\frac{\int_{a}^{b}f(x)w(x)\,dx}{\int_{a}^{b}w(x)\,dx}\]
\end{definition}
给定$x$时,$w(y)=g(x-y)$就是$f*g$中f的加权均值的倍数。进一步,卷积的光滑性高于用来卷积的两
个函数,并且在f可导时有$(f*g)'=f'*g$,因为
\[(f*g)'(x)=\frac{d}{dx}\int_{-\infty}^{\infty}f(x-y)g(y)\,dy=\int_{-\infty}^{\infty}f'(x-y)g(y)\,dy=f'*g\]
例如,$\Pi_{1/2}*\Pi_{1/2}=\Lambda$,等式左侧是两个不连续的函数,右侧是连续并且分段
光滑的函数。

特别地,如果$g(x)=\frac{1}{2a}\chi_{[-a,a]}$,则$f*g(x)$是$f(x)$在区间$[x-a,x+a]$上的平均值:
\[(f*g)(x)=\int_{-\infty}^{\infty}f(y)g(x-y)\,dy=\frac{1}{2a}\int_{x-a}^{x+a}f(y)\,dy\]

\noindent 2.\textbf{卷积函数的支集}

我们首先引入\textbf{支集}(support)的概念,读者只需理解其直观,真正理解它需要一些拓扑学的基础。
\begin{definition}[支集]
    设$f:\mathbb{R}\to \mathbb{R}$,f的\textbf{支集}
$\text{supp}\ f=\overline{\left\{x\in\mathbb{R}\left|f(x)\neq 0\right.\right\}}$
\end{definition}
这里上划线不是取共轭,而是对集合取闭包,闭包包括集合本身的和它的极限点,$\mathbb{R}$中集合的闭包
是闭集,例如,$(a,b)$的闭包是$[a,b]$.集合$\{x\in\mathbb{R}:f(x)\neq 0\}$的提
出是自然的,取闭包则不那么容易理解。实际上,在度量空间(采用度量导出的拓扑)中,
一个集合是紧集(任给一个开区间组成的覆盖,总能从中取出有限覆盖)就等价于它是有
界闭集,描述有界区间的一种方式是说它是闭包紧的。因此,我们可以说在无穷远处为0的
函数有\textbf{紧支集}(compact support),对于p次连续可导的函数$f\in C^p(\mathbb{R})$,我们记其中具有紧支集的函数空间为$C_0^p(\mathbb{R})$。

卷积能够将两个函数的支集“相加”。设$\text{supp}\ f\in [a,b],\text{supp}\ g\in [c,d]$,则
$\text{supp}\ (f*g)\in [a+c,b+d]$,因为:
\begin{align*}
     & (f*g)(x) =\int_{-\infty}^{\infty}f(y)g(x-y)\,dy                               \\
     & \text{积分值非0}\implies y\in [a,b],x-y\in [c,d]\iff x\in [a+c,b+d]
\end{align*}

\noindent 3.\textbf{函数变换下的卷积}
\begin{proposition}[信号反转下的卷积]
    \[(f^-)*(g^-)=(f*g)^-\]
\end{proposition}
\begin{proof}
    \begin{align*}
    (f^-)*(g^-)(x) & =\int_{-\infty}^{\infty}f^-(y)g^-(x-y)\,dy          \\
                   & =\int_{-\infty}^{\infty}f(-y)g(-(x-y))\,dy          \\
                   & =\int_{-\infty}^{\infty}f(z)g(-(x+z))(-dz) & (z=-y) \\
                   & =\int_{-\infty}^{\infty}f(z)g(-x-z)\,dz             \\
                   & =(f*g)^-(x)
\end{align*}
\end{proof}

如果只有一个信号反转,则结果不再是卷积,而是f与g的互相关(在f,g都是实信号时),下面很快将讨论它。

在建立卷积在信号时延或伸缩下的性质前,我们先来明确所使用的符号。我们将
用$\tau$表示时延$\tau_b f(t)=f(t-b)$,用$\sigma$表示伸缩$\sigma_a f(t)=f(at)$
,并避免使用$\tau,\sigma$作为变量的符号。这样的做法在研究分布的性质时是必要的,
因为严格来讲不能给出分布的“自变量”,但例如$\delta$函数这样的分布又具有明显的
尺度变换的性质,见\ref{sec:distributions}.

\begin{proposition}[信号时延、伸缩时的卷积]
    \begin{align*}
    (\tau_b f)*g=\tau_b (f*g)=f*(\tau_b g) \\
    (\sigma_a f)*(\sigma_a g)=\frac{1}{|a|}\sigma_a(f*g)
\end{align*}
\end{proposition}
\begin{proof}
    \begin{align*}
    (\tau_b f)*g(x) & =\int_{-\infty}^{\infty}\tau_b f(y)g(x-y)\,dy           \\
                    & =\int_{-\infty}^{\infty}f(y-b)g(x-y)\,dy                \\
                    & =\int_{-\infty}^{\infty}f(z)g(x-(z+b))\,dz    & (z=y-b) \\
                    & =\int_{-\infty}^{\infty}f(z)g((x-b)-z)\,dz              \\
                    & =(f*g)(x-b)=\tau_b (f*g)(x)
\end{align*}
类似地,可以证明$f*(\tau_b g)=\tau_b (f*g)$.
\begin{align*}
    (\sigma_a f)*(\sigma_a g)(x) & =\int_{-\infty}^{\infty}\sigma_a f(y)\sigma_a g(x-y)\,dy                          \\
                                 & =\int_{-\infty}^{\infty}f(ay)g(a(x-y))\,dy                                        \\
                                 & =\frac{1}{|a|}\int_{-\infty}^{\infty}f(z)g(a x - z)\,dz  & \left(z=ay,dy=\frac{dz}{a}\right) \\
                                 & =\frac{1}{|a|}(f*g)(a x)=\frac{1}{|a|}\sigma_a(f*g)(x)
\end{align*}
和傅里叶变换的伸缩定理类似,这里也需要讨论a的正负,但不再赘述。
\end{proof}

\noindent 4.\textbf{“面积”关系}

我们已经提到,概率论中两个独立随机变量之和的概率密度函数是它们各自概率密度函数的卷积,这当
然要求两个概率密度函数在$\mathbb{R}$上的积分值为1时,它们的卷积在$\mathbb{R}$上的积
分值也为1。更一般地:
\begin{proposition}[卷积的“面积”关系]
\[\int_{-\infty}^{\infty}(f*g)(x)\,dx=\int_{-\infty}^{\infty}f(y)\,dy\int_{-\infty}^{\infty}g(x)\,dx\]
\end{proposition}
\begin{proof}
\begin{align*}
    \int_{-\infty}^{\infty}(f*g)(x)\,dx & =\int_{-\infty}^{\infty}\,dx\int_{-\infty}^{\infty}f(y)g(x-y)\,dy                                   \\
                                        & =\int_{-\infty}^{\infty}f(y)\,dy\int_{-\infty}^{\infty}g(x-y)\,dx                                   \\
                                        & =\int_{-\infty}^{\infty}f(y)\,dy\int_{-\infty}^{\infty}g(u)\,du                           & (u=x-y) \\
                                        & =\left(\int_{-\infty}^{\infty}f(x)\,dx\right)\left(\int_{-\infty}^{\infty}g(x)\,dx\right)
\end{align*}
\end{proof}

在统计学中,我们引入相关系数来描述两个随机变量之间的相关程度,类似地,在信号
处理中,我们引入\textbf{相关系数}(correlation coefficient)来描述两个信号之间的相关
程度。
\begin{definition}[相关系数]
设$f,g\in L^2(\mathbb{R})$,它们的相关系数定义为
\begin{equation*}
    \rho(f,g)=\frac{\langle f,g\rangle}{\|f\|\cdot\|g\|}
\end{equation*}
\end{definition}
其中$\langle f,g\rangle=\int_{-\infty}^{\infty}f(x)\overline{g(x)}dx$是$f$和
$g$的内积,$\|f\|=\sqrt{\langle f,f\rangle}$是$f$的范数。根据柯西-施瓦茨不等式,
相关系数的绝对值不超过1,且当且仅当$f$与$g$几乎处处成比例时取等,因此我们说,
$\rho(f,g)=\pm 1$时两信号\textbf{线性相关};$\rho(f,g)=0$时两信号\textbf{线性无关},
这相当于两函数正交。

有时信号之间存在时延差,这时我们可以定义\textbf{互相关}(cross-correlation)来描述它
们之间的相关程度。
\begin{definition}[互相关]
设$f,g\in L^2(\mathbb{R})$,它们的互相关定义为
\begin{equation}
    (f\star g)(x)=\int_{-\infty}^{\infty}f(y)\overline{g(x+y)}\,dy
\end{equation}
同样可以定义f的自相关$(f\star f)(x)$
\end{definition}
互相关具有以下性质:
\begin{proposition}[互相关的性质]
\begin{enumerate}
    \quad
    \item $(f\star g) =f^-* \overline{g}=\overline{(g\star f)^-}$,如果$f,g$都是实信号,则$f\star g = (g\star f)^- =f^- *g=(f*g^-)^-$.
    \item $\mathcal{F} (f\star g) =\mathcal{F} f\overline{\mathcal{F} g}$,特别地,
          $\mathcal{F} (f\star f) =|\mathcal{F} f|^2$,这个结果称为\textbf{维纳-辛钦定理}(Wiener-Khinchin theorem)
    \item $f\star (\tau_b g)=\tau_{b} (f\star g)=(\tau_{-b}f)\star g$
    \item $(f\star g)\leq\|f\|_2\|g\|_2$,特别地,$(f\star f)(x)\leq (f\star f)(0)=\|f\|_2^2$
\end{enumerate}
\end{proposition}
\begin{proof}
\begin{align*}
    1.(f\star g)(x)               & =\int_{-\infty}^{\infty}f(y)\overline{g(x+y)}\,dy                                                                   \\
                                  & =\int_{-\infty}^{\infty}f(-y)\overline{g(x-y)}\,dy                                                                  \\
                                  & =\int_{-\infty}^{\infty}f^-(y)\overline{g(x-y)}\,dy                                                                  \\
                                  & = (f^- * \overline{g})(x)                                                                                           \\
    (f\star g)(x)                 & =\int_{-\infty}^{\infty}f(y)\overline{g(x+y)}\,dy                                                                   \\
                                  & =\overline{\int_{-\infty}^{\infty}g(x+y)\overline{f(y)}\,dy}                                                        \\
                                  & =\overline{(g\star f)(-x)}=\overline{(g\star f)^-(x)}                                                               \\
    2.\mathcal{F} (f\star g)(\xi) & =\int_{-\infty}^{\infty}(f\star g)(x)e^{-2\pi i\xi x}\,dx                                                           \\
                                  & =\int_{-\infty}^{\infty}e^{-2\pi i\xi x}\,dx\int_{-\infty}^{\infty}f(y)\overline{g(x+y)}\,dy                        \\
                                  & =\iint\limits_{\mathbb{R}^2}f(y)\overline{g(x+y)}e^{-2\pi i\xi x}\,dy\,dx                                           \\
                                  & =\int_{-\infty}^{\infty}f(y)\,dy\int_{-\infty}^{\infty}\overline{g(x+y)}e^{-2\pi i\xi x}\,dx                        \\
                                  & =\int_{-\infty}^{\infty}f(y)e^{2\pi i\xi y}\,dy\int_{-\infty}^{\infty}\overline{g(u)}e^{-2\pi i\xi u}\,du & (u=x+y) \\
                                  & =\mathcal{F} f(\xi)\overline{\mathcal{F} g(\xi)}
\end{align*}这里不涉及频率、角频率的问题,读者可以自行验证。
\begin{align*}
    3.\left(f\star (\tau_b g)\right)(x)  & =\int_{-\infty}^{\infty}f(y)\overline{(\tau_b g)(x+y)}\,dy                                                  \\
                              & =\int_{-\infty}^{\infty}f(y)\overline{g(x+y-b)}\,dy                                                         \\
                              & =(f\star g)(x-b)=\tau_{b} (f\star g)(x)                                                                     \\
    ( (\tau_{-b}f)\star g)(x) & =\int_{-\infty}^{\infty}(\tau_{-b}f)(y)\overline{g(x+y)}\,dy                                                \\
                              & =\int_{-\infty}^{\infty}f(y+b)\overline{g(x+y)}\,dy                                                         \\
                              & =\int_{-\infty}^{\infty}f(z)\overline{g(x+z-b)}\,dz                                               & (z=y+b) \\
                              & =(f\star g)(x-b)=\tau_{b} (f\star g)(x)                                                                     \\
    4. (f\star g)(x)          & =\int_{-\infty}^{\infty}f(y)\overline{g(x+y)}\,dy                                                           \\
                              & \leq \sqrt{\int_{-\infty}^{\infty}|f(y)|^2\,dy\int_{-\infty}^{\infty}\left|\overline{g(x+y)}\right|^2\,dy  }           \\
                              & =\sqrt{\|f\|_2^2\|g\|_2^2}=\|f\|_2\cdot\|g\|_2                                              
\end{align*}
\end{proof}

对于功率信号,我们需要对以上互相关的定义略作修改,以避免趋于无穷的情况发生。
\begin{definition}[功率信号的互相关]
    对于功率信号$f,g$,定义互相关函数为
\begin{equation}
    (f\star g)(x)=\lim_{T\to\infty}\frac{1}{2T}\int_{-T}^{T}f(y)\overline{g(x+y)}\,dy
\end{equation}
\end{definition}
其性质不再单独讨论。

一些文献也将互相关函数定义为$R_{fg}(x)=\int_{-\infty}^{\infty}f(y)\overline{g(x-y)}\,dy=\int_{-\infty}^{\infty}f(x+y)g(y)\,dy$
,功率信号同理,这与上面讨论的互相关没有本质区别。

最后我们给出一个互相关的应用。
\begin{example}[雷达测距]
    设雷达发射了一个信号$f(t)$,经过时间T信号接触到
物体并发生反射,再经过时间T信号回到雷达,雷达接收到的信号$f_r(t)=\alpha (\tau_{2T}f(t))+n(t)$,
其中$\alpha\in(0,1)$,表示信号在传播过程中衰减;$n(t)$为噪声信号。我们
希望确定时间T,以计算雷达到物体间的距离。对于噪声信号$n(t)$,它满足
\[(f\star n)(t)=C\]
C为常数,因此可以考虑求发射信号与接收信号的互相关函数:\begin{align*}
    (f\star f_r)(t)&=\alpha(f\star \tau_{2T}f)(t)+(f\star n)(t)\\
    &=\alpha\tau_{2T}(f\star f)(t)+\alpha C
\end{align*}
根据前面得到的性质4,$(f\star f)(t)$在0处取最大值,因此$(f\star f_r)(t)$在$2T$
处取最大值,于是我们只需要观察这个互相关函数的最大值点,就可以得到T。
\end{example}

\section{分布及其傅里叶变换}\label{sec:distributions}

经典的数学分析理论难以处理单位阶跃函数的导数,也无法对一些比较比基本的函数如正弦
、余弦函数做傅里叶变换,甚至因此傅里叶反演公式不总是成立(注意我们之前仅在形式上
使用傅里叶反演公式),要扩充这个理论,标准的做法是引入\textbf{广义函数}(gerneralized function)
,又称\textbf{分布}(distribution)。分布
这个名称一开始是由物理学家引入的,例如在描述点电荷的分布时,经典函数是失效的,于
是在20世纪20年代末到30年代初,狄拉克及一众物理学家开始用分布进行运算,到30年代
中,索伯列夫首先明确提出了广义函数的思想,后于40年代末由施瓦兹发展,他因这一工作
获得1950年的菲尔兹奖。因此,下面将提到的广义函数空间$\mathcal{D} $也称为
\textbf{索伯列夫-施瓦兹广义函数空间}。

下面首先对一般的分布理论做一些讨论,再转回傅里叶分析中对于分布的应用,对于只希望掌握分布理论中的形式计算的读者,可以直接跳过前半部分,从定义2.5.6.施瓦兹空间与施瓦兹函数处开始阅读。

我们先介绍\textbf{泛函} (functional)和\textbf{对偶空间}(dual space)的概念。
\begin{definition}[线性映射与线性泛函]
    设X和Y是同一数域上的线性空间(这里不妨设为$\mathbb{R}$)如果
$\forall x_1,x_2\in X$,映射$A:X\to Y$满足
\begin{align*}
    A(x_1)+A(x_2) & =A(x_1+x_2)                        \\
    A(\lambda x)  & =\lambda A(x),\lambda\in\mathbb{R}
\end{align*}
则称A是\textbf{线性映射}。特别地,如果Y是一个数域(例如$\mathbb{R,C}$),则
将A称为\textbf{线性函数};如果X还是某种函数空间,则将A称为\textbf{线性泛函}。
例如,$A:C([a,b],\mathbb{R})\to\mathbb{R},A(f):=f(x_0)$和$A:C([a,b],\mathbb{R})\to\mathbb{R},A(f):=\int_{a}^{b}f(x)\,dx$
都是线性泛函。有时,我们不区分X究竟是不是函数空间,而统一地把线性函数称为线性泛
函。
\end{definition}
\begin{definition}[对偶空间]
    给定一个实线性空间V,它的\textbf{对偶空间}是V上所有线性函数$A:V\to\mathbb{R}$
构成的线性空间(读者可以自行定义线性函数的加法和数乘,并验证它是线性空间),记
为$\mathcal{L} (V;\mathbb{R})$或$V^*$
\end{definition}
对有限维线性空间,它的对偶空间与它本身的维数相同,因为定义V上的线性函数就等价于对V的一组基定义线性函数;对于无限维线性空间,它的对偶空间也是无限维的。

我们已经看到,分布(例如狄拉克$\delta$)难以用经典的“函数”来描述,这时我们可以
考察它们与一系列\textbf{检验函数}(test function)$\varphi$的作用,具体来说:
\begin{definition}[分布]
记$\mathbb{R}$
上的复值光滑紧支函数集为$\mathcal{C} $,如果将检验函数集取为$\mathcal{C}$,我们
将其对偶空间$\mathcal{D} $中的元素称为分布,并规定函数$f\in \mathcal{C} $所产生的分布$T_f$
为作用在$\mathcal{C}$上的以下\textbf{泛函}:
\begin{equation*}
    \left\langle T_f,\varphi\right\rangle:=\int_{-\infty}^{\infty}f(x)\varphi(x)\,dx,\varphi\in\mathcal{C}
\end{equation*}
将这样的分布称为\textbf{正则分布},而将无法用紧支函数描述的分布称为\textbf{奇异分布}。
\end{definition}
\begin{remark}
   实际上这里不需要要求$f\in\mathcal{C} $,只需要f在$\mathbb{R}$上局部可积(在任意闭区间上可积)即可,但为了简化讨论,我们仅考虑光滑紧支函数。 
\end{remark}
\begin{example}
    尽管我们从形式上给出了$\delta$函数的定义,但它实际上应该采用定义
\[\langle \delta,\varphi\rangle:=\delta(\varphi):=\varphi(0)\]
由于“狄拉克函数”严格来说并不能算作一种函数,$\delta$是一个奇异分布。容易验证这与我们一
开始给出的$\delta$作为“函数”的性质是相符的:
\[\langle \delta,\varphi\rangle=\int_{-\infty}^{\infty}\delta(x)\varphi(x)\,dx=\int_{-\infty}^{\infty}\delta(x)\varphi(0)\,dx=\varphi(0)\]
\end{example}

下面来定义分布与函数的乘法和分布的导数。这一部分中,我们的原则是:\textbf{奇异分布与正
    则分布具有相同的性质},换言之,只要能够对正则分布定义的算子,就能够对奇异分布
做相同的定义。在讨论分布的傅里叶变换时,还将定义更多的算子,例如分布卷积、尺度变换等。

设$f,g,\varphi\in\mathcal{C} $,有
\begin{align*}
    \left\langle (f\cdot g),\varphi\right\rangle=\int_{-\infty}^{\infty}(f\cdot g)(x)\varphi(x)\,dx=\int_{-\infty}^{\infty}f(x)(g\cdot\varphi)(x)\,dx=\langle f,(g\cdot\varphi)\rangle
\end{align*}
因此对于任意的分布$T\in\mathcal{D} $,定义它与$g\in\mathcal{C} $的乘积$gT$
由以下等式给出:
\begin{definition}[分布与函数的乘法]
    \begin{equation*}
    \left\langle gT,\varphi\right\rangle=\langle T,g\varphi\rangle
\end{equation*}
\end{definition}
$g\varphi$就是普通的函数乘法。现在就可以说,分布集$\mathcal{D} $构成函数环
$\mathcal{C} $上的模(module),并且可以验证$\delta$的取样性质:
\begin{align*}
     & \left\langle g\delta,\varphi\right\rangle=\langle \delta,g\varphi\rangle=g(0)\varphi(0)=g(0)\langle\delta,\varphi\rangle \Rightarrow  g\delta=g(0)\delta
\end{align*}

用同样的思路定义分布的微分:设$f,g,\varphi\in\mathcal{C} $,有
\begin{align*}
    \left\langle f',\varphi\right\rangle=\int_{-\infty}^{\infty}f'(x)\varphi(x)\,dx=-\int_{-\infty}^{\infty}f(x)\varphi'(x)\,dx=\left\langle f,\varphi'\right\rangle
\end{align*}
因此\begin{definition}[分布的微分]
    \begin{equation*}
    \left\langle T',\varphi\right\rangle=-\left\langle T,\varphi'\right\rangle
\end{equation*}
\end{definition}
注意$\varphi\in\mathcal{C} $是无限阶可导的,我们可以由此定义分布的任意阶导数。
\begin{example}[$\delta$的导数]
\ref{sec:signal}中提到的单位阶跃函数$u(t)$和$\delta$,现在就可以将它们
视为分布并求各阶导数:
\begin{align*}
     & \left\langle u,\varphi\right\rangle:=\int_{0}^{\infty}\varphi(x)\,dx                                                                      \\
     & \left\langle u',\varphi\right\rangle=-\left\langle u,\varphi'\right\rangle=-\int_{0}^{\infty}\varphi'(x)\,dx=\varphi(0)=\left\langle\delta,\varphi\right\rangle \\
     & \left\langle \delta',\varphi\right\rangle=-\left\langle \delta,\varphi'\right\rangle=-\varphi'(0)
\end{align*}
$\delta$的高阶导数可以依此类推,它们已经难以用类似$\delta(x)$的“函数”描述,但
可以看到$\delta^{(n)}$作为一个泛函(分布)作用是取测试函数的$(-1)^n$倍的n阶导。
现在也就不难理解$\delta^{(n)}$与函数相乘的公式,例如,
\begin{align*}
    \left\langle g\delta',\varphi\right\rangle & =\left\langle \delta',g\varphi\right\rangle                \\
                                    & =-\left\langle\delta,(g\varphi)'\right\rangle              \\
                                    & =-\left\langle\delta,g'\varphi+g\varphi'\right\rangle      \\
                                    & =-(g(0)\varphi'(0)+g'(0)\varphi(0))             \\
                                    & =\left\langle g(0)\delta'-g'(0)\delta,\varphi\right\rangle
    \Rightarrow g\delta'=g(0)\delta'-g'(0)\delta
\end{align*}
\end{example}

现在指出分布的微分运算的某些性质。
\begin{theorem}[分布的微分运算的性质]
    \begin{enumerate}
        \quad
    \item 任何分布$T\in\mathcal{D} $都是无穷次可微的
    \item 微分算子$D:\mathcal{D} \to\mathcal{D} $是线性的
    \item 微分算子D满足莱布尼兹法则(Leibniz rule):
          \[\left(gT\right)'=g'T+gT'\]从而数学分析中的莱布尼兹公式在分布理论中仍成立:
          \[\left(gT\right)^{(m)}=\sum_{k=0}^{m}C_m^k T^{(k)}g^{(m-k)}\]
    \item 微分算子D是连续的(表述见证明)
\end{enumerate}
\end{theorem}
\begin{proof}
\begin{enumerate}
    \item 得自$\mathcal{C} $中函数的无限可微性:$\left\langle T^{(m)},\varphi\right\rangle=(-1)^m\left\langle T,\varphi^{(m)}\right\rangle$.
    \item 显然。
    \item 只需验证莱布尼兹法则。
          \begin{align*}
              \left\langle (gT)',\varphi\right\rangle & =-\left\langle gT,\varphi'\right\rangle=-\left\langle T,g\varphi'\right\rangle=-\left\langle T,(g\varphi)'-g'\varphi\right\rangle                                            \\
                                           & =\left\langle T',g\varphi\right\rangle+\left\langle T,g'\varphi\right\rangle=\left\langle gT',\varphi\right\rangle+\left\langle g'T,\varphi\right\rangle=\left\langle gT'+g'T,\varphi\right\rangle
          \end{align*}
    \item 设当$m\to\infty$时,$T_m\to T$,即$\forall\varphi\in\mathcal{C} ,\left\langle T_m,\varphi\right\rangle\to\left\langle T,\varphi\right\rangle$,
          则\[\left\langle T_m',\varphi\right\rangle=-\left\langle T_m,\varphi'\right\rangle\to-\left\langle T,\varphi'\right\rangle=\left\langle T',\varphi\right\rangle\]
\end{enumerate}
可以看到,分布理论中极限的概念是通过测试函数来定义的,如果分布序列$\{T_m\}_{m=1}^{\infty}$
作用在任何测试函数上都是趋于某个分布T作用于这个测试函数的值,就说序列$\{T_m\}$
\textbf{弱收敛}(converge weakly)于T,并记为$T_m\to T$。
\end{proof}

接下来讨论分布理论在傅里叶分析中的应用。我们的目标是,在这个新的理论体系下:
\begin{itemize}
    \item 允许$\delta$信号,单位阶跃信号,多项式,正弦、余弦函数等信号(作为分布)做傅里叶变换
    \item 分布的傅里叶变换和其反变换同时有定义;傅里叶反演公式成立
    \item 对于分布,帕塞瓦尔恒等式成立(之前我们只能在$L^1(\mathbb{R})\cap L^2(\mathbb{R})$中使用它)
\end{itemize}
我们将看到,分布$T$的傅里叶变换$\mathcal{F} T$定义为
\begin{equation}
    \left\langle\mathcal{F}T,\varphi\right\rangle=\left\langle T,\mathcal{F}\varphi\right\rangle
\end{equation}
然而,测试函数$\varphi\in\mathcal{C} $的傅里叶变换$\mathcal{F}\varphi$并不属于
$\mathcal{C} $
(关于这一点的说明,以及施瓦兹函数类的引出,见附录\ref{sec:Schwartz_Functions})
,这说明测试函数集$\mathcal{C} $在傅里叶分析中的表现不够好,我们需要引入新的测试函数空间,以保证
$\mathcal{F}\varphi$仍然是测试函数。这个测试函数空间正是\textbf{施瓦兹空间}(Schwartz space)$\mathcal{S} $
,它是$\mathbb{R}$上所有无限可微函数的集合,这些函数及其各阶导数都以比任何负幂更快的速度趋于0,即
\begin{definition}[施瓦兹空间与施瓦兹函数]
    \[\mathcal{S} =\left\{\varphi\in C^{\infty}(\mathbb{R} )\left|\lim_{|x|\to\infty}|x|^m\varphi^{(n)}(x)=0,\forall m,n\in\mathbb{N} \right.\right\}\]
施瓦兹空间中的函数称为\textbf{施瓦兹函数}(Schwartz function),它们是非常光滑且
衰减很快的函数,因此又称为\textbf{速降函数}(rapidly decreasing function)
\end{definition}
\begin{example}
高斯函数$e^{-x^2}$及其各阶导数都属于施瓦兹空间。
\end{example}
\begin{definition}[施瓦兹分布]
    施瓦兹空间的对偶空间$\mathcal{T} :=\mathcal{S}^*$中的元素称为\textbf{施瓦兹分布}(Schwartz distribution)
或\textbf{缓增分布}(tempered distribution)。称之为“缓增分布”,是因为一些能够用函数描述的施瓦兹分布在自变量趋于无穷时增长速度较缓,以至于它们与施瓦兹函数的乘积可积。
\end{definition}
我们仍定义
\begin{align}
    T(\varphi)                 & =\left\langle T,\varphi\right\rangle                                        \\
    \left\langle T_f,\varphi\right\rangle & =\int_{-\infty}^{\infty}f(x)\varphi(x)\,dx,\varphi\in\mathcal{S}
\end{align}
易见$\mathcal{C} \subset \mathcal{S} ,\mathcal{T} \subset\mathcal{D}. $

有了施瓦兹函数类$\mathcal{S} $和缓增分布$\mathcal{T} $,我们就可以定义分布的傅里
叶变换,还可以定义有关分布的一系列算子。前文中曾定义分布与函数的乘法和分布的导数,
现在认为正则分布是由施瓦兹函数导出的,则显然能够推广到缓增分布,即\begin{align}
    \left\langle gT,\varphi\right\rangle=\left\langle T,g\varphi\right\rangle,\left\langle T',\varphi\right\rangle=\left\langle T,\varphi'\right\rangle,g\in\mathcal{S} ,T\in\mathcal{T}
\end{align}
用同样的方式,我们依次讨论作用于缓增分布的各种算子。

\noindent 1.\textbf{傅里叶变换}

设$f,\varphi\in\mathcal{S} ,T_f\in\mathcal{T} $,则
\begin{align*}
    \left\langle\mathcal{F}T_f,\varphi\right\rangle & =\int_{-\infty}^{\infty}\mathcal{F}f(x)\varphi(x)\,dx                                       \\
                                         & =\int_{-\infty}^{\infty}\varphi(x)\,dx\int_{-\infty}^{\infty}f(y)e^{-2\pi ixy}\,dy          \\
                                         & =\int_{-\infty}^{\infty}f(y)\,dy\int_{-\infty}^{\infty}\varphi(x)e^{-2\pi ixy}\,dx          \\
                                         & =\int_{-\infty}^{\infty}f(y)\mathcal{F}\varphi(y)\,dy=\left\langle T_f,\mathcal{F}\varphi\right\rangle
\end{align*}
因此我们定义\begin{definition}[分布的傅里叶变换]
    \begin{align*}
        \left\langle \mathcal{F} T,\varphi\right\rangle & =\left\langle T,\mathcal{F} \varphi\right\rangle,T\in\mathcal{T} ,\varphi\in\mathcal{S}
    \end{align*}
\end{definition}

傅里叶逆变换同理:
\begin{definition}[分布的傅里叶反变换]
    \begin{align*}
        \left\langle \mathcal{F} ^{-1}T,\varphi\right\rangle & =\left\langle T,\mathcal{F} ^{-1}\varphi\right\rangle,T\in\mathcal{T} ,\varphi\in\mathcal{S}
    \end{align*}
\end{definition}

只要承认函数的傅里叶反演公式,分布的傅里叶反演公式就自然成立:\begin{theorem}[分布的傅里叶反演公式]
    \[\mathcal{F} \mathcal{F} ^{-1}T=\mathcal{F} ^{-1}\mathcal{F} T=T,\forall T\in\mathcal{T}\]
\end{theorem}
\begin{proof}
    \begin{align*}
    \left\langle \mathcal{F} ^{-1}\mathcal{F} T,\varphi\right\rangle & =\left\langle \mathcal{F} T,\mathcal{F} ^{-1}\varphi\right\rangle \\
                                                          & =\left\langle T,\mathcal{F} \mathcal{F} ^{-1}\varphi\right\rangle \\
                                                          & =\left\langle T,\varphi\right\rangle                              \\
    \left\langle \mathcal{F} \mathcal{F} ^{-1}T,\varphi\right\rangle & =\left\langle T,\mathcal{F} ^{-1}\mathcal{F} \varphi\right\rangle \\
                                                          & =\left\langle T,\varphi\right\rangle
\end{align*}
\end{proof}

傅里叶变换的线性性依然成立:
\begin{theorem}[分布的傅里叶变换的线性]
    \[\mathcal{F} (aT+bS)=a\mathcal{F} T+b\mathcal{F} S,a,b\in\mathbb{C},S,T\in\mathcal{T}\]
\end{theorem}
\begin{proof}
    尽管没有明确指出,我们也会很自然的想到定义
\begin{equation}
    \left\langle aT+bS,\varphi\right\rangle=a\left\langle T,\varphi\right\rangle+b\left\langle S,\varphi\right\rangle
\end{equation}
于是\begin{align*}
    \left\langle \mathcal{F} (aT+bS),\varphi\right\rangle & =a\left\langle T,\mathcal{F} \varphi\right\rangle+b\left\langle S,\mathcal{F} \varphi\right\rangle                     \\
                                               & =a\left\langle \mathcal{F} T,\varphi\right\rangle+b\left\langle \mathcal{F} S,\varphi\right\rangle                     \\
                                               & =\left\langle a\mathcal{F} T+b\mathcal{F} S,\varphi\right\rangle                                            \\
                                               & \Rightarrow \mathcal{F} (aT+bS)=a\mathcal{F} T+b\mathcal{F} S,a,b\in\mathbb{C},S,T\in\mathcal{T}
\end{align*}
\end{proof}
\begin{example}[$\delta$的傅里叶变换]
    现在可以严谨地求出$\delta$的傅里叶变换:
\begin{align*}
    \left\langle\mathcal{F}\delta,\varphi\right\rangle=\left\langle \delta,\mathcal{F}\varphi\right\rangle=\mathcal{F}\varphi(0)=\int_{-\infty}^{\infty}\varphi(x)\,dx=\left\langle \mathds{1},\varphi\right\rangle \\
    \Rightarrow\mathcal{F}\delta =\mathds{1}
\end{align*}
\end{example}
\begin{example}[$\delta$的平移的傅里叶变换]
    $\delta$的平移$\delta_a$作为一种分布,定义为
\begin{equation}
    \delta_a(\varphi)=\left\langle \delta_a,\varphi\right\rangle=\varphi(a)
\end{equation}
它的傅里叶变换为
\lr{
\left\langle\mathcal{F}\delta_a,\varphi\right\rangle=\left\langle \delta_a,\mathcal{F}\varphi\right\rangle\\
&=\mathcal{F}\varphi(a)\\
&=\int_{-\infty}^{\infty}\varphi(x)e^{-2\pi iax}\,dx\\
&=\left\langle e^{-2\pi iax},\varphi\right\rangle \\
&\Rightarrow\mathcal{F}\delta_a =e^{-2\pi iax}
}{
\left\langle\mathcal{F}\delta_a,\varphi\right\rangle=\left\langle \delta_a,\mathcal{F}\varphi\right\rangle\\
&=\mathcal{F}\varphi(a)\\
&=\int_{-\infty}^{\infty}\varphi(x)e^{-i a x}\,dx\\
&=\left\langle e^{-i a x},\varphi\right\rangle \\
&\Rightarrow\mathcal{F}\delta_a =e^{-i a x}
}
根据阿贝尔-狄利克雷判别法(A-D判别法,请自行查看数学分析的教材),$\langle e^{-i a x},\varphi\rangle$
作为一个反常积分是有意义的。可以看出这与函数的傅里叶变换的平移定理很相似,我们将在后面
给出分布的平移、伸缩,并由此得到一些分布的傅里叶变换的性质。
\end{example}

\begin{example}[分布$\mathds{1}$的傅里叶变换]
尽管$\mathds{1}\notin L^1(\mathbb{R})$,但可以
认为它是缓增分布,因为对于任意的$\varphi\in\mathcal{S} $,都有
\[\left\langle \mathds{1},\varphi\right\rangle=\int_{-\infty}^{\infty}\varphi(x)\,dx<\infty\]
因此可以求它的傅里叶变换:
\lr{
    \left\langle \mathcal{F} \mathds{1},\varphi\right\rangle&=\left\langle \mathds{1},\mathcal{F} \varphi\right\rangle\\
    &=\int_{-\infty}^{\infty}\mathcal{F} \varphi(\xi)\,d\xi\\
    &=\mathcal{F} \mathcal{F} \varphi(0)\\
    &=\varphi(0)=\left\langle \delta,\varphi\right\rangle\\
    \Rightarrow \mathcal{F} \mathds{1}&=\delta
}{
    \left\langle \mathcal{F} \mathds{1},\varphi\right\rangle&=\left\langle \mathds{1},\mathcal{F} \varphi\right\rangle\\
    &=\int_{-\infty}^{\infty}\mathcal{F} \varphi(\xi)\,d\xi\\
    &=\mathcal{F} \mathcal{F} \varphi(0)\\
    &=2\pi\varphi(0)=2\pi\left\langle \delta,\varphi\right\rangle\\
    \Rightarrow \mathcal{F} \mathds{1}&=2\pi\delta
}
\end{example}

\noindent 2.\textbf{分布的反转}

设$f,\varphi\in\mathcal{S} ,T_f\in\mathcal{T} $,自然可以定义$T_f^-=T_{f^-}$,
\begin{align*}
    \left\langle T_f^-,\varphi\right\rangle & =\int_{-\infty}^{\infty}f(-x)\varphi(x)\,dx \\
                                 & =\int_{-\infty}^{\infty}f(y)\varphi(-y)\,dy \\
                                 & =\left\langle T_f,\varphi^-\right\rangle
\end{align*}
因此我们定义\begin{definition}[分布的反转]
    \begin{equation*}
    \left\langle T^-,\varphi\right\rangle=\left\langle T,\varphi^-\right\rangle,T\in\mathcal{T} ,\varphi\in\mathcal{S}
\end{equation*}
\end{definition}

有了反转就可以定义分布的奇偶性:
\begin{definition}[分布的奇偶性]
    如果$T^-=T$,则称T为\textbf{偶分布};如果$T^-=-T$,则称T为\textbf{奇分布}。
\end{definition}

从施瓦兹函数的傅里叶变换的对偶性,就能得到缓增分布的傅里叶变换的对偶性:
\begin{theorem}[分布傅里叶变换的对偶性]
    \quad
    \lr{
    \mathcal{F}(T^-)= (\mathcal{F}T)^-=\mathcal{F} ^{-1}T
    }{
    \mathcal{F}(T^-)= (\mathcal{F}T)^-=2\pi\mathcal{F} ^{-1}T
    }
\end{theorem}
和常规的函数一样,现在也可以不区分反转与傅里叶变换的先后,而统一地记作$\mathcal{F} T^-$.
\begin{proof}
    \begin{align*}
    \left\langle\mathcal{F}(T^-),\varphi\right\rangle=\left\langle T^-,\mathcal{F}\varphi\right\rangle=\left\langle T,\mathcal{F} \varphi^-\right\rangle=\left\langle\mathcal{F}T,\varphi^-\right\rangle=\left\langle(\mathcal{F}T)^-,\varphi\right\rangle \\
    \Rightarrow\mathcal{F}(T^-)= (\mathcal{F}T)^-
\end{align*}
考察反转与傅里叶逆变换的关系:
\lr{
    \left\langle\mathcal{F}T^-,\varphi\right\rangle&=\left\langle T,\mathcal{F}\varphi^-\right\rangle\\
    &=\left\langle T,\mathcal{F} ^{-1}\varphi\right\rangle\\
    &=\left\langle\mathcal{F}^{-1}T,\varphi\right\rangle\\
    &\Rightarrow\mathcal{F}T^-= \mathcal{F}^{-1}T
}{
    \left\langle\mathcal{F}T^-,\varphi\right\rangle=\left\langle T,\mathcal{F}\varphi^-\right\rangle\\
    &=\left\langle T,2\pi\mathcal{F} ^{-1}\varphi\right\rangle\\
    &=\left\langle2\pi\mathcal{F}^{-1}T,\varphi\right\rangle\\
    &\Rightarrow\mathcal{F}T^-= 2\pi\mathcal{F}^{-1}T
}
\end{proof}
这与前文中函数的傅里叶变换对偶性完全一样。有了对偶性,就可以得到:
\begin{theorem}[帕塞瓦尔恒等式/普朗歇尔定理]
    \quad
    \lr{
    \int_{-\infty}^{\infty}|f(t)|^2\,dt=\int_{-\infty}^{\infty}|\mathcal{F} f(\xi)|^2\,d\xi
}{
    \int_{-\infty}^{\infty}|f(t)|^2\,dt=\frac{1}{2\pi}\int_{-\infty}^{\infty}|\mathcal{F} f(\omega)|^2\,d\omega
}
\end{theorem}
\begin{proof}
设$f,g\in\mathcal{S} $,\lr{
    \left\langle \mathcal{F} f,\overline{\mathcal{F} g}\right\rangle&=\left\langle \mathcal{F} f,\mathcal{F}^{-1}\overline{g}\right\rangle\\
    &=\left\langle f,\mathcal{F} \mathcal{F} ^{-1}\overline{g}\right\rangle\\
    &=\left\langle f,\overline{g}\right\rangle
}{
    \left\langle \mathcal{F} f,\overline{\mathcal{F} g}\right\rangle=\left\langle f,2\pi\mathcal{F}^{-1}\overline{g}\right\rangle\\
    &=\left\langle f,2\pi\mathcal{F} \mathcal{F} ^{-1}\overline{g}\right\rangle\\
    &=2\pi\left\langle f,\overline{g}\right\rangle
}
取$g=f$即证。
\end{proof}
看上去,这里只是将\ref{sec:Fourier}中的证明换了一中符号,并没有用到刚才建立的分布的傅里叶变换的对偶性,然而,现在我们可以得到正则分布的帕塞瓦尔恒等式。我们不仅扩大了证明适用的范围,还得到了比积分换序的方法简单得多的证明。

\begin{example}[$\delta$是偶分布]
    \begin{align*}
    \left\langle \delta^-,\varphi\right\rangle & =\left\langle \delta,\varphi^-\right\rangle=\varphi^-(0)=\varphi(0)=\left\langle \delta,\varphi\right\rangle \\
    \Rightarrow \delta^-            & =\delta
\end{align*}
\end{example}

\begin{example}[分布$\mathds{1}$的傅里叶逆变换]
应用对偶性求$\mathds{1}$的傅里叶逆变换:
\lr{
    \mathcal{F} \mathds{1}=\mathcal{F} ^{-1}\mathds{1}^-=\delta^-=\delta
}{
    \mathcal{F} \mathds{1}=2\pi\mathcal{F} ^{-1}\mathds{1}^- =2\pi\delta^- =2\pi\delta
}
\end{example}
\begin{example}[复指数函数$e^{iat}$的傅里叶变换]
作为函数当然不能求$e^{iat}$的傅里叶变换,我们甚至无法确定它在0处的傅里叶变换:$\int_{-\infty}^{\infty}e^{iat}\,dt$不存在。
但它可以视为一个缓增分布,前文中已经提到$\langle e^{iat},\varphi\rangle$是收敛
的。下面应用对偶性求它的傅里叶变换:
\lr{
\mathcal{F} [e^{2\pi iat}]=\mathcal{F}^{-1}[e^{2\pi iat}]^-=\delta_{a}^-=\delta_{-a}
}{
\mathcal{F} [e^{iat}]=2\pi\mathcal{F}^{-1}[e^{iat}]^- =2\pi\delta_{a}^- =2\pi\delta_{-a}
}
请读者自行验证$\delta_{a}^-= \delta_{-a}$.
\end{example}

\begin{example}[正弦、余弦函数的傅里叶变换]
根据欧拉公式$e^{ix}=\cos(x)+i\sin(x)$,
有\[\cos(x)=\frac{e^{ix}+e^{-ix}}{2},\sin(x)=\frac{e^{ix}-e^{-ix}}{2i}\]因此
\lr{
    \mathcal{F} [\cos(2\pi ax)]&=\frac{1}{2}(\mathcal{F} [e^{2\pi iax}]+\mathcal{F} [e^{-2\pi iax}])\\
    &=\frac{1}{2}(\delta_{-a}+\delta_{a}) \\
    \mathcal{F} [\sin(2\pi ax)]&=\frac{1}{2i}(\mathcal{F} [e^{2\pi iax}]-\mathcal{F} [e^{-2\pi iax}])\\
    &=\frac{1}{2i}(\delta_{-a}-\delta_{a})
}{
    \mathcal{F} [\cos(ax)]&=\frac{1}{2}(\mathcal{F} [e^{iax}]+\mathcal{F} [e^{-iax}])\\
    &=\pi(\delta_{-a}+\delta_{a}) \\
    \mathcal{F} [\sin(ax)]&=\frac{1}{2i}(\mathcal{F} [e^{iax}]-\mathcal{F} [e^{-iax}])\\
    &=-i\pi(\delta_{-a}-\delta_{a})
}
\end{example}

\noindent 3.\textbf{分布的共轭}

设$f,\varphi\in\mathcal{S} ,T_f\in\mathcal{T} $,则
\begin{align*}
    \left\langle \overline{T_f},\varphi\right\rangle=\int_{-\infty}^{\infty}\overline{f(x)}\varphi(x)\,dx=\overline{\int_{-\infty}^{\infty}f(x)\overline{\varphi(x)}\,dx}=\overline{\left\langle T_f,\overline{\varphi}\right\rangle} \\
\end{align*}
因此我们定义\begin{definition}[分布的共轭]
    \begin{equation*}
    \left\langle \overline{T},\varphi\right\rangle=\overline{\left\langle T,\overline{\varphi}\right\rangle},T\in\mathcal{T} ,\varphi\in\mathcal{S}
\end{equation*}
\end{definition}

有了共轭就可以仿照复数定义分布的实部、虚部以及实分布、纯虚分布:
\begin{definition}[分布的实部、虚部]
    设$T\in\mathcal{T} $,则
    \[Re T=\frac{T+\overline{T}}{2}\]
    称为T的\textbf{实部},
    \[Im T=\frac{T-\overline{T}}{2}\]
    称为T的\textbf{虚部}。
\end{definition}
\begin{definition}[实分布与纯虚分布]
如果$\overline{T}=T$,则称T为\textbf{实分布};如果
$\overline{T}=-T$,则称T为\textbf{纯虚分布}。
\end{definition}
现在就来考察实分布的最后一条对偶性。在函数的傅里叶变换理论中,我们知道:
    \[f=\overline{f}\Rightarrow \mathcal{F}f^- =\overline{\mathcal{F} f}\]

对于分布,
\begin{align*}
    \left\langle\mathcal{F}T^-,\varphi\right\rangle =\left\langle T,\mathcal{F}\varphi^-\right\rangle
\end{align*}
但是,我们并不能保证$\varphi$是实值函数\footnote{尽管在原始定义中没有提到这一点,但和前
    面引入施瓦兹函数类一样,要保证施瓦兹函数的傅里叶变换仍然是施瓦兹函数,而实值函数的
    傅里叶变换往往是复值函数。},因此并不能推出实分布的最后一条对偶性。不过,我们还是可
以得到分布的傅里叶变换的对称性:
\begin{theorem}[分布的傅里叶变换的对称性]
    \begin{align*}
    T^-=T\Rightarrow\mathcal{F} T^-=\mathcal{F} T \\
    T^-=-T\Rightarrow\mathcal{F} T^-=-\mathcal{F} T
\end{align*}
\end{theorem}
换言之,分布的傅里叶变换的奇偶性与分布本身相同。

\noindent 4.\textbf{分布的平移和伸缩}

设$f,\varphi\in\mathcal{S} ,T_f\in\mathcal{T} $,自然可以定义$\tau_b T_f=T_{\tau_b f}$,
\begin{align*}
    \left\langle \tau_b T_f,\varphi\right\rangle & =\int_{-\infty}^{\infty}f(x-b)\varphi(x)\,dx \\
                                      & =\int_{-\infty}^{\infty}f(y)\varphi(y+b)\,dy \\
                                      & =\left\langle T_f,\tau_{-b}\varphi\right\rangle
\end{align*}
因此我们定义\begin{definition}[分布的平移]
    \begin{equation*}
    \left\langle \tau_b T,\varphi\right\rangle=\left\langle T,\tau_{-b}\varphi\right\rangle,T\in\mathcal{T} ,\varphi\in\mathcal{S}
\end{equation*}
\end{definition}

设$f,\varphi\in\mathcal{S} ,T_f\in\mathcal{T} $,自然可以定义$\sigma_a T_f=T_{\sigma_a f}$,
\begin{align*}
    \left\langle \sigma_a T_f,\varphi\right\rangle & =\int_{-\infty}^{\infty}f(ax)\varphi(x)\,dx                                  \\
                                        & =\frac{1}{|a|}\int_{-\infty}^{\infty}f(y)\varphi\left(\frac{y}{a}\right)\,dy \\
                                        & =\frac{1}{|a|}\left\langle T_f,\sigma_{1/a}\varphi\right\rangle
\end{align*}
因此我们定义\begin{definition}[分布的伸缩]
    \begin{equation*}
    \left\langle\sigma_a T,\varphi\right\rangle=\left\langle T,\frac{1}{|a|}\sigma_{1/a}\varphi\right\rangle,T\in\mathcal{T} ,\varphi\in\mathcal{S}
\end{equation*}
\end{definition}

现在就可以建立分布的傅里叶变换的平移和尺度变换定理:\begin{theorem}[分布的傅里叶变换的平移和尺度变换]
\begin{align*}
    &\mathcal{F}(\tau_b T)=\mathcal{F}(e^{2\pi ibx})T\\
    &\mathcal{F}(\sigma_a T)=\frac{1}{|a|}\mathcal{F}(\sigma_a T)
\end{align*}
\end{theorem}
\begin{proof}
        \begin{align*}
    \left\langle\mathcal{F}(\tau_b T),\varphi\right\rangle =\left\langle \tau_b T,\mathcal{F}\varphi\right\rangle=\left\langle T,\tau_{-b}\mathcal{F}\varphi\right\rangle=\left\langle T,e^{2\pi ibx}\mathcal{F}\varphi\right\rangle=\left\langle \mathcal{F}(e^{2\pi ibx}T),\varphi\right\rangle \\
    \Rightarrow\mathcal{F}(\tau_b T)           =\mathcal{F}(e^{2\pi ibx}T)
\end{align*}
\begin{align*}
    \left\langle\mathcal{F}(\sigma_a T),\varphi\right\rangle =\left\langle \sigma_a T,\mathcal{F}\varphi\right\rangle=\left\langle T,\frac{1}{|a|}\sigma_{\frac{1}{a}}\mathcal{F}\varphi\right\rangle=\left\langle T,\frac{1}{|a|}\mathcal{F}\sigma_a \varphi\right\rangle=\left\langle \frac{1}{|a|}\mathcal{F}(\sigma_a T),\varphi\right\rangle \\
    \Rightarrow\mathcal{F}(\sigma_a T)           =\frac{1}{|a|}\mathcal{F}(\sigma_a T)
\end{align*}
\end{proof}

\noindent 5.\textbf{分布的傅里叶变换的微分性质}

我们已经得到$\langle T',\varphi\rangle=-\langle T,\varphi'\rangle$,现在结合
分布与函数的乘法,仿照函数的情形得到分布的傅里叶变换的微分性质:
\lr{
    \left\langle\mathcal{F} (T'),\varphi\right\rangle=-\left\langle T,(\mathcal{F} \varphi)'\right\rangle\\
    =\left\langle T,\mathcal{F} (2\pi i t\varphi)\right\rangle\\
    =\left\langle 2\pi i\xi\mathcal{F}T,\varphi\right\rangle\\
    \Rightarrow\mathcal{F} (T')=2\pi i\xi\mathcal{F}T
}{
    \left\langle\mathcal{F} (T'),\varphi\right\rangle=-\left\langle T,(\mathcal{F} \varphi)'\right\rangle\\
    &=\left\langle T,\mathcal{F} (i t\varphi)\right\rangle\\
    &=\left\langle i\omega\mathcal{F}T,\varphi\right\rangle\\
    &\Rightarrow\mathcal{F} (T')=i\omega\mathcal{F}T
}
这与函数的傅里叶变换的微分性质一致,注意将$t$换成$\xi$或$\omega$,只是符号上的改
变,用以区分所讨论的场景。同样地,
\lr{
    \left\langle (\mathcal{F} T)',\varphi\right\rangle=\left\langle T,-\mathcal{F} (\varphi')\right\rangle\\
    =\left\langle T,-2\pi i\xi\mathcal{F} \varphi\right\rangle\\
    =\left\langle -\mathcal{F} (2\pi i tT),\varphi\right\rangle\\
    \Rightarrow (\mathcal{F} T)'=-\mathcal{F} (2\pi i tT)
}{
    \left\langle(\mathcal{F} T)',\varphi\right\rangle=\left\langle T,-\mathcal{F} (\varphi)'\right\rangle\\
    =\left\langle T,-i\omega\mathcal{F} \varphi\right\rangle\\
    =\left\langle -\mathcal{F} (itT),\varphi\right\rangle\\
    \Rightarrow (\mathcal{F} T)'=-\mathcal{F} (itT)
}
也即
\lr{
    \mathcal{F} (tT)=\frac{i}{2\pi}(\mathcal{F} T)'
}{
    \mathcal{F} (tT)=i(\mathcal{F} T)'
}

综上我们得到:\begin{theorem}[分布的傅里叶变换的微分性质]
    \quad
    \lr{
    \mathcal{F} (T')=2\pi i\xi\mathcal{F}T \\
    (\mathcal{F} T)'=-\mathcal{F} (2\pi i tT) \\
    \mathcal{F} (tT)=\frac{i}{2\pi}(\mathcal{F} T)'
}{
    \mathcal{F} (T')=i\omega\mathcal{F}T \\
    (\mathcal{F} T)'=-\mathcal{F} (itT) \\
    \mathcal{F} (tT)=i(\mathcal{F} T)'
}
    
\end{theorem}

\noindent 6.\textbf{分布的卷积}

设$f,g,\varphi\in\mathcal{S} ,T_f\in\mathcal{T} $,自然可以定义$(T_f)*g=T_{(f*g)}$,
\begin{align*}
    \left\langle T_f*g,\varphi\right\rangle & =\int_{-\infty}^{\infty}(f*g)(x)\varphi(x)\,dx                                           \\
                                 & =\int_{-\infty}^{\infty}\left(\int_{-\infty}^{\infty}f(y)g(x-y)\,dy\right)\varphi(x)\,dx \\
                                 & =\int_{-\infty}^{\infty}f(y)\,dy\int_{-\infty}^{\infty}g(x-y)\varphi(x)\,dx              \\
                                 & =\int_{-\infty}^{\infty}f(y)\,dy\int_{-\infty}^{\infty}g^- (y-x)\varphi(x)\,dx           \\
                                 & =\int_{-\infty}^{\infty}f(y)[(g^-) *\varphi](y)\,dy                                      \\
                                 & =\left\langle T_f,(g^-)*\varphi\right\rangle
\end{align*}
因此我们定义\begin{definition}[分布与函数的卷积]
    \[\left\langle T*g,\varphi\right\rangle=\left\langle T,(g^-)*\varphi\right\rangle\]
\end{definition}

从定义分布与函数的卷积的过程可以看出,应当要求卷积的交换律仍然成立;“结合律”在目前
只限于讨论$(f*g)*T$是否等于$f*(g*T)$,我们来验证这一性质:
\begin{align*}
    \left\langle (f*g)*T,\varphi\right\rangle & =\left\langle T,(f*g)^- *\varphi\right\rangle                             \\
                                   & =\left\langle T,(f^-)*(g^-)*\varphi\right\rangle                          \\
                                   & =\left\langle g*T,(f^-)*\varphi\right\rangle                              \\
                                   & =\left\langle f*(g*T),\varphi\right\rangle                                \\
                                   & \Rightarrow (f*g)*T=f*(g*T),f,g\in\mathcal{S} ,T\in\mathcal{T}
\end{align*}
而线性性则和前面傅里叶变换的线性性没有多少区别:\\
\ding{172}对函数的线性性
\begin{align*}
    \left\langle (af+bg)*T,\varphi\right\rangle & =\left\langle T,[(af+bg)^-]*\varphi\right\rangle                                               \\
                                     & =a\left\langle T,(f^-)*\varphi\right\rangle+b\left\langle T,(g^-)*\varphi\right\rangle                    \\
                                     & =a\left\langle f*T,\varphi\right\rangle+b\left\langle g*T,\varphi\right\rangle                            \\
                                     & =\left\langle af*T+bg*T,\varphi\right\rangle                                                   \\
                                     & \Rightarrow (af+bg)*T=af*T+bg*T,a,b\in\mathbb{C},f,g\in\mathcal{S} ,T\in\mathcal{T} 
\end{align*}
{\parskip=0pt \ding{173} 对分布的线性性}
\begin{align*}
    \left\langle f*(aS+bT),\varphi\right\rangle & =\left\langle aS+bT,(f^-)*\varphi\right\rangle                                                 \\
                                     & =a\left\langle S,(f^-)*\varphi\right\rangle+b\left\langle T,(f^-)*\varphi\right\rangle                    \\
                                     & =a\left\langle f*S,\varphi\right\rangle+b\left\langle f*T,\varphi\right\rangle                            \\
                                     & =\left\langle af*S+bf*T,\varphi\right\rangle                                                   \\
                                     & \Rightarrow f*(aS+bT)=af*S+bf*T,a,b\in\mathbb{C},f\in\mathcal{S} ,S,T\in\mathcal{T}
\end{align*}

下面研究卷积定理是否仍成立:\\\ding{172}时域卷积
\lr{
    \left\langle \mathcal{F} (g*T),\varphi\right\rangle&=\left\langle T,(g^-)*\mathcal{F} \varphi\right\rangle\\
    &=\left\langle T,(\mathcal{F} \mathcal{F} g)*\mathcal{F} \varphi\right\rangle\\
    &=\left\langle T,\mathcal{F} [(\mathcal{F} g) \varphi]\right\rangle\\
    &=\left\langle \mathcal{F} T,(\mathcal{F} g)\varphi\right\rangle\\
    &=\left\langle \mathcal{F} g\mathcal{F} T,\varphi\right\rangle\\
    &\Rightarrow \mathcal{F} (g*T)=\mathcal{F} g\mathcal{F} T
}{
    \left\langle \mathcal{F} (g*T),\varphi\right\rangle&=\left\langle T,(g^-)*\mathcal{F} \varphi\right\rangle\\
    &=\left\langle T,\frac{1}{2\pi}(\mathcal{F} \mathcal{F} g)*\mathcal{F} \varphi\right\rangle\\
    &=\left\langle T,\mathcal{F} [(\mathcal{F} g) \varphi]\right\rangle\\
    &=\left\langle \mathcal{F} T,(\mathcal{F} g)\varphi\right\rangle\\
    &=\left\langle \mathcal{F} g\mathcal{F} T,\varphi\right\rangle\\
    &\Rightarrow \mathcal{F} (g*T)=\mathcal{F} g\mathcal{F} T
}
\ding{173}频域卷积
\lr{
    \left\langle \mathcal{F} (gT),\varphi\right\rangle&=\left\langle T,g\mathcal{F} \varphi\right\rangle\\
    &=\left\langle T,\mathcal{F} \mathcal{F} (g^-) \mathcal{F} \varphi\right\rangle\\
    &=\left\langle T,\mathcal{F} [\mathcal{F} g^- *\varphi]\right\rangle\\
    &=\left\langle\mathcal{F} T*\mathcal{F} g,\varphi\right\rangle\\
    &\Rightarrow \mathcal{F} (gT)=\mathcal{F} g*\mathcal{F} T
}{
    \left\langle\mathcal{F} (gT),\varphi\right\rangle&=\left\langle T,g\mathcal{F} \varphi\right\rangle\\
    &=\left\langle T,\frac{1}{2\pi}\mathcal{F} \mathcal{F} (g^-) \mathcal{F} \varphi\right\rangle\\
    &=\left\langle T,\frac{1}{2\pi}\mathcal{F} [\mathcal{F} g^- *\varphi]\right\rangle\\
    &=\frac{1}{2\pi}\left\langle\mathcal{F} T*\mathcal{F} g,\varphi\right\rangle\\
    &\Rightarrow \mathcal{F} (gT)=\frac{1}{2\pi}\mathcal{F} g*\mathcal{F} T
}
综上我们得到:\begin{theorem}[分布与函数的卷积定理]
    \quad
    \lr{
    \mathcal{F} (g*T)=\mathcal{F} g\mathcal{F} T \\
    \mathcal{F} (gT)=\mathcal{F} g*\mathcal{F} T
}{
    \mathcal{F} (g*T)=\mathcal{F} g\mathcal{F} T \\
    \mathcal{F} (gT)=\frac{1}{2\pi}\mathcal{F} g*\mathcal{F} T
}
\end{theorem}
这与函数之间卷积的情况是完全一样的。

实际上,我们同样可以定义分布与分布的卷积,只是这时事情会麻烦的多。首先还是研究正则
分布的卷积,如果想将前文中定义分布与函数卷积时的g也换成分布,则需要另一种方式来处理
这个积分。设$f,g,\varphi\in\mathcal{S} ,S_f,T_g\in\mathcal{T} $,根据前面已经
得到的结果,
\begin{align*}
    \left\langle S_f *T_g,\varphi\right\rangle & =\int_{-\infty}^{\infty}\int_{-\infty}^{\infty}f(y)g(x-y)\varphi(x)\,dx\,dy           \\
                                    & =\int_{-\infty}^{\infty}\int_{-\infty}^{\infty}f(y)g(u)\varphi(u+y)\,du\,dy & (x-y=u) \\
                                    & =\int_{-\infty}^{\infty}f(y)\,dy\int_{-\infty}^{\infty}g(u)\varphi(u+y)\,du           \\
                                    & =\left\langle f(y),\langle g(u),\varphi(u+y)\rangle\right\rangle
\end{align*}
如果允许给分布标上自变量(对于能用函数表示的分布,即便不属于$\mathcal{S} $,这样做
也的确是有意义的),那么我们可以定义:\begin{definition}[分布与分布的卷积]
    \begin{equation*}
    \left\langle S*T,\varphi\right\rangle=\left\langle S(y),\left\langle T(x),\varphi(x+y)\right\rangle\right\rangle,S,T\in\mathcal{T} ,\varphi\in\mathcal{S} ,\left\langle T(x),\varphi(x+y)\right\rangle\in\mathcal{S}
\end{equation*}
\end{definition}

注意我们必须要求$\left\langle T(x),\varphi(x+y)\right\rangle\in\mathcal{S}$,如果不满足这
个条件,我们就只好说$S*T$无定义。对于分布之间的卷积,交换律和结合律就不总是成立了,
交换律在$\left\langle S,\varphi\right\rangle\in\mathcal{S} ,\left\langle T,\varphi\right\rangle\notin\mathcal{S}$
时被破坏,对于结合律,从形式上可以两次套用分布与分布卷积的定义,得到
\begin{equation*}
    \left\langle R*(S*T),\varphi\right\rangle=\left\langle T(x),\left\langle S(y),\left\langle R(z),\varphi(x+y+z)\right\rangle\right\rangle\right\rangle=\left\langle (R*S)*T,\varphi\right\rangle
\end{equation*}
然而当证明过程中任何一步遇到卷积无定义的情况,这个证明就会失效。
\begin{example}[结合律的失效]
\begin{align*}
    \left\langle\mathds{1}*\delta',\varphi\right\rangle     & =\left\langle \mathds{1}(y),\left\langle \delta'(x),\varphi(x+y)\right\rangle\right\rangle                                                                   \\
                                                 & =\left\langle \mathds{1},-\varphi'\right\rangle                                                                                                   \\
                                                 & =-\int_{-\infty}^{\infty}\varphi'(y)\,dy                                                                                                \\
                                                 & =-\evalat{\varphi(y)}{-\infty}{\infty}=0                              & \Rightarrow(\mathds{1}*\delta')*u=0                             \\
    \left\langle \delta'*u,\varphi\right\rangle             & =\left\langle \delta'(y),\left\langle u(x),\varphi(x+y)\right\rangle\right\rangle                                                                            \\
                                                 & =\left\langle \delta'(y),\int_{y}^{\infty}\varphi(x)\,dx\right\rangle                                                                             \\
                                                 & =\varphi(0)=\left\langle \delta,\varphi\right\rangle                            & \Rightarrow \delta'*u=\delta                                    \\
    \left\langle\mathds{1}*(\delta'*u),\varphi\right\rangle & =\left\langle \mathds{1}(y),\left\langle \delta(x),\varphi(x+y)\right\rangle\right\rangle                                                                    \\
                                                 & =\left\langle \mathds{1},\varphi\right\rangle                                                                                                     \\
                                                 & =\int_{-\infty}^{\infty}\varphi(y)\,dy                                                                                                 \\
                                                 & =\left\langle \mathds{1},\varphi\right\rangle                                   & \Rightarrow \mathds{1}*(\delta'*u)=\mathds{1}*\delta=\mathds{1}
\end{align*}
可见\begin{equation*}
    (\mathds{1}*\delta')*u                       \neq\mathds{1}*(\delta'*u)
\end{equation*}
\end{example}

\begin{example}[与$\delta$卷积]
    分布$T*\delta$必然有定义,并且$T*\delta=T$.
\begin{align*}
    \left\langle T*\delta,\varphi\right\rangle & =\left\langle T(y),\left\langle \delta(x),\varphi(x+y)\right\rangle\right\rangle \\
                                    & =\left\langle T,\varphi\right\rangle
\end{align*}
\end{example}

\begin{example}[$\delta_a$的平移作用]
    分布$\delta_a*\varphi$必然有定义,并且$\delta_a*\varphi=\tau_a \varphi$.
    \begin{align*}
    \left\langle \delta_a*\delta_b,\varphi\right\rangle & =\left\langle \delta_a(y),\left\langle\delta_b(x),\varphi(x+y)\right\rangle\right\rangle \\
                                             & =\left\langle \delta_a(y),\varphi(b+y)\right\rangle                    \\
                                             & =\varphi(a+b)                                               \\
                                             & =\left\langle\delta_{(a+b)},\varphi\right\rangle                       \\
                                             & \Rightarrow \delta_a*\delta_b=\delta_{(a+b)}
\end{align*}
这与函数的情况是一致的:\begin{align*}
    \left\langle f*\delta_a,\varphi\right\rangle & =\left\langle \delta_a,(f^-)*\varphi\right\rangle                  \\
                                      & =(f^-)*\varphi(a)                                       \\
                                      & =\int_{-\infty}^{\infty}f(-x)\varphi(a-x)\,dx & (a-x=y) \\
                                      & =\int_{-\infty}^{\infty}f(y-a)\varphi(y)\,dy            \\
                                      & =\left\langle \tau_a f,\varphi\right\rangle                        \\
                                      & \Rightarrow f*\delta_a=\tau_a f
\end{align*}
\end{example}
在\ref{sec:Fourier}中,我们已经从形式上得到了$sgn,u,1/t$,的傅里叶变换,现在有了
分布的工具,就可以借助分布的傅里叶变换的对偶性说明之前的结论是严谨的。

最后,我们指出,分布的傅里叶变换具有连续性,即:\begin{theorem}[分布的傅里叶变换的连续性]
    设$\{T_n\}\subset\mathcal{T} $,$T\in\mathcal{T} $,如果$T_n\to T$,则$\mathcal{F} T_n\to\mathcal{F} T$.
\end{theorem}
\begin{proof}
    设$\varphi\in\mathcal{S} $,则
    \begin{align*}
    \left\langle\mathcal{F} T_n-\mathcal{F} T,\varphi\right\rangle & =\left\langle T_n-T,\mathcal{F} \varphi\right\rangle \\
    \lim_{n\to\infty}\left\langle\mathcal{F} T_n-\mathcal{F} T,\varphi\right\rangle & =\lim_{n\to\infty}\left\langle T_n-T,\mathcal{F} \varphi\right\rangle=0
\end{align*}
\end{proof}

然而,要证明施瓦兹函数的傅里叶变换的连续性,要困难得多,我们将在附录\ref{sec:Schwartz_Functions}中给出其证明。

\section{小结}\label{sec:sammary}
本章所涉及的公式及性质汇总如下。鉴于我们已经获得了很多对于傅里叶分析的认识,本
小节将以不同的顺序或方式来表述它们。

\noindent 1.\underline{傅里叶级数}

周期为T的函数f,三角函数形式的展开式为
\begin{align*}
    f(t) & =\frac{a_0}{2}+\sum_{k = 1}^{\infty} a_k \cos(k\omega t)+b_k\sin(k\omega t) \\
         & =\frac{c_0}{2}+\sum_{k = 1}^{\infty} c_k\cos(k\omega t+\varphi _k)
\end{align*}
其中
\[a_k=\frac{2}{T}\int_T f(t)\cos(k\omega t)\,dt\]
\[b_k=\frac{2}{T}\int_T f(t)\sin(k\omega t)\,dt\]
\[c_k=\sqrt{a_k^2+b_k^2}\]

指数函数形式的展开式为
\[f(t)=\sum_{k = 0}^{\infty}  c_k e^{ik\omega t} ,c_k=\frac{a_k-ib_k}{2},c_{-k}=\frac{a_k+ib_k}{2}=c_k^*,k\in \mathbb{N}\]

\noindent 2.\underline{特殊情况下的傅里叶级数}

对于偶函数,
\[f(t)=\frac{a_0}{2}+\sum_{k = 1}^{\infty} a_k\cos(k\omega t)\]
其中
\[a_k=\frac{4}{T}\int_{0}^{\frac{T}{2}} f(t)\cos(k\omega t)\,dt\]
\[b_k=0\]
对于奇函数,
\[a_k=0\]
\[b_k=\frac{4}{T}\int_{0}^{\frac{T}{2}} f(t)\sin(k\omega t)\,dt\]
对于奇谐函数,$f\left(t+\frac{T}{2}\right)=-f(t)$,
\begin{align*}
    a_k=\begin{cases}
        0,&\text{if }k\text{为偶数}\\
        \frac{4}{T}\int_{0}^{T/2}f(t)\cos(k\omega t)\,dt,&\text{if }k\text{为奇数}
    \end{cases}
\end{align*}
\begin{align*}
    b_k=\begin{cases}
        0,&\text{if }k\text{为偶数}\\
        \frac{4}{T}\int_{0}^{T/2}f(t)\sin(k\omega t)\,dt,&\text{if }k\text{为奇数}
    \end{cases}
\end{align*}

\noindent 3.\underline{帕塞瓦尔定理/瑞利恒等式}
\[P=\frac{1}{T}\int_{T}|f(t)|^2\,dt=\sum_{k=1}^{\infty}c_k^2=a_0^2+\frac{1}{2}\sum_{k=1}^{\infty}(a_k^2+b_k^2)\]

\noindent 4. \underline{傅里叶变换}:
\lr{
    \mathcal{F} f(\xi)=\int_{-\infty}^{\infty}f(t)e^{-2\pi i\xi t}\,dt
}{
    \mathcal{F} f(\omega)=\int_{-\infty}^{\infty}f(t)e^{-i\omega t}\,dt
}
傅里叶逆变换:\lr{
    f(t)=\int_{-\infty}^{\infty}\mathcal{F} f(\xi)e^{2\pi i\xi t}\,d\xi
}{
    f(t)=\frac{1}{2\pi}\int_{-\infty}^{\infty}\mathcal{F} f(\omega)e^{i\omega t}\,d\omega
}
傅里叶反演公式:\lr{
    f=\mathcal{F} ^{-1}\mathcal{F} f=\mathcal{F} \mathcal{F} ^{-1}f
}{
    f=\mathcal{F} ^{-1}\mathcal{F} f=\mathcal{F}\mathcal{F} ^{-1} f
}

\noindent 5.\underline{常用信号及其傅里叶变换}
\begin{itemize}
    \item 矩形函数及取样函数:
    \lr{
        &\Pi_T(t)\overset{\mathcal{F} }{\longleftrightarrow}Tsinc(T\xi)\\
        &sinc(Tt)\overset{\mathcal{F} }{\longleftrightarrow}\frac{1}{T}\Pi_T(\xi)
    }{
        &\Pi_T(t)\overset{\mathcal{F} }{\longleftrightarrow}TSa\left(\frac{T\omega}{2}\right)\\
        &Sa(Tt)\overset{\mathcal{F} }{\longleftrightarrow} \frac{\pi}{T}\Pi_{2T}(\omega)
    }
    \item 高斯函数 $G(t)=\frac{1}{\sqrt{2\pi}\sigma}e^{-\frac{t^2}{2\sigma^2}}$
    \lr{
        G(t)\overset{\mathcal{F} }{\longleftrightarrow}e^{-2\pi\sigma^2\xi^2}
    }{
        G(t)\overset{\mathcal{F} }{\longleftrightarrow}e^{-\frac{\sigma^2\omega^2}{2}}
    }
    \item 单边指数函数$f(t)=e^{-at}u(t)$和双边指数函数$g(t)=f(t)+f(-t)$
    \lr{
        &f(t)\overset{\mathcal{F} }{\longleftrightarrow}\frac{1}{a+2\pi i\xi}\\
        &g(t)\overset{\mathcal{F} }{\longleftrightarrow}\frac{2a}{a^2+4\pi^2\xi^2}
    }{
        &f(t)\overset{\mathcal{F} }{\longleftrightarrow}\frac{1}{a+i\omega}\\
        &g(t)\overset{\mathcal{F} }{\longleftrightarrow}\frac{2a}{a^2+\omega^2}
    }
    
\end{itemize}

\noindent 6.\underline{常用分布及其傅里叶变换}
\begin{itemize}
    \item $\delta$分布和$\mathds{1}$分布
    \lr{
        &\delta\overset{\mathcal{F} }{\longleftrightarrow}\mathds{1}\\
        &\mathds{1}\overset{\mathcal{F} }{\longleftrightarrow}\delta\\
        &\delta_a\overset{\mathcal{F} }{\longleftrightarrow}e^{-2\pi ia\xi}\\
        &e^{2\pi iat}\overset{\mathcal{F} }{\longleftrightarrow}\delta_{a}
    }{
        &\delta\overset{\mathcal{F} }{\longleftrightarrow}\mathds{1}\\
        &\mathds{1}\overset{\mathcal{F} }{\longleftrightarrow}2\pi\delta\\
        &\delta_a\overset{\mathcal{F} }{\longleftrightarrow}e^{-ia\omega}\\
        &e^{iat}\overset{\mathcal{F} }{\longleftrightarrow}2\pi\delta_{a}
    }
    \item 正弦函数和余弦函数
    \lr{
        &\cos(2\pi at)\overset{\mathcal{F} }{\longleftrightarrow}\frac{1}{2}(\delta_a+\delta_{-a})\\
        &\sin(2\pi at)\overset{\mathcal{F} }{\longleftrightarrow}\frac{1}{2i}(\delta_a-\delta_{-a})
    }{
        &\cos(at)\overset{\mathcal{F} }{\longleftrightarrow}\pi(\delta_a+\delta_{-a})\\
        &\sin(at)\overset{\mathcal{F} }{\longleftrightarrow}-i\pi(\delta_a-\delta_{-a})
    }
    \item 符号函数、单位阶跃函数和多项式
    \lr{
        &sgn(t)\overset{\mathcal{F} }{\longleftrightarrow}\frac{1}{\pi i\xi}\\
        &u(t)\overset{\mathcal{F} }{\longleftrightarrow}\frac{1}{2}\left(\delta+\frac{1}{\pi i\xi}\right)\\
        &\frac{1}{t}\overset{\mathcal{F} }{\longleftrightarrow}-\pi isgn(\xi)\\
        &t^n\overset{\mathcal{F} }{\longleftrightarrow}\left(\frac{i}{2\pi}\right)^n\delta^{(n)}
    }{
        &sgn(t)\overset{\mathcal{F} }{\longleftrightarrow}\frac{2}{i\omega}\\
        &u(t)\overset{\mathcal{F} }{\longleftrightarrow}\pi\delta+\frac{1}{i\omega}\\
        &\frac{1}{t}\overset{\mathcal{F} }{\longleftrightarrow}-\pi isgn(\omega)\\
        &t^n\overset{\mathcal{F} }{\longleftrightarrow}2\pi i^n\delta^{(n)}
    }
\end{itemize}

\noindent 7.\underline{傅里叶变换的性质}:设\[f\overset{\mathcal{F} }{\longleftrightarrow}F\]
\begin{itemize}
    \item \textbf{对偶性}:\lr{
        &F^- =\mathcal{F} ^{-1}f\\
              &\mathcal{F}F=f^-\\
              &f\text{是实信号}\Rightarrow F^-=\overline{F}
          }{
            &F^- =2\pi\mathcal{F} ^{-1}f\\
              &\mathcal{F}F=2\pi f^-\\
              &f\text{是实信号}\Rightarrow F^-=\overline{F}
            }
    \item \textbf{对称性}:F与f奇偶性相同;f是实函数时,如果f还是偶函数,则F也是实函数,
          如果f还是奇函数,则F是纯虚函数
    \item \textbf{线性性}:$\forall f,g\in L^1(\mathbb{R}),\mathcal{F} (af+bg)=a\mathcal{F} f+b\mathcal{F} g$,即$\mathcal{F} $是线性算子
    \item \textbf{平移定理}:\lr{
          &f(t-b)\overset{\mathcal{F} }{\longleftrightarrow}e^{-2\pi i \xi b}F(\xi)\\
          &f(t)e^{2\pi i \xi t}\overset{\mathcal{F} }{\longleftrightarrow}F(\xi-b)
          }{
          &f(t-b)\overset{\mathcal{F} }{\longleftrightarrow}e^{-i\omega b}F(\omega)\\
          &f(t)e^{ibt}\overset{\mathcal{F} }{\longleftrightarrow}F(\omega-b)
          }
    \item \textbf{伸缩定理}:\lr{
              f(at)\overset{\mathcal{F} }{\longleftrightarrow}\frac{1}{|a|}F(\frac{\xi}{a})
          }{
              f(at)\overset{\mathcal{F} }{\longleftrightarrow}\frac{1}{|a|}F(\frac{\omega}{a})
          }
    \item \textbf{微分性质}:\lr{
              &f'\overset{\mathcal{F} }{\longleftrightarrow}2\pi i\xi F(\xi)\\
              &2\pi itf\overset{\mathcal{F} }{\longleftrightarrow}-F'(\xi)\\
              &\text{即}tf\overset{\mathcal{F} }{\longleftrightarrow}\frac{i}{2\pi}F'(\xi)
          }{
              &f'\overset{\mathcal{F} }{\longleftrightarrow}i\omega F(\omega)\\
              &itf(t)\overset{\mathcal{F} }{\longleftrightarrow}-F'(\omega)\\
              &\text{即}tf\overset{\mathcal{F} }{\longleftrightarrow}iF'(\omega)
          }
    \item \textbf{帕塞瓦尔恒等式}(Parseval's identity):\lr{
    \int_{-\infty}^{\infty}|f(t)|^2\,dt=\int_{-\infty}^{\infty}|\mathcal{F} f(\xi)|^2\,d\xi
}{
    \int_{-\infty}^{\infty}|f(t)|^2\,dt=\frac{1}{2\pi}\int_{-\infty}^{\infty}|\mathcal{F} f(\omega)|^2\,d\omega
}
\end{itemize}

\noindent 8.\underline{卷积}

函数f,g的卷积为
\begin{equation*}
    (f*g)(x)=\int_{-\infty}^{\infty}f(y)g(x-y)\,dy
\end{equation*}
设$f\overset{\mathcal{F} }{\longleftrightarrow}F,g\overset{\mathcal{F} }{\longleftrightarrow}G$,
\textbf{卷积定理}(the convolution thoerem):
\lr{
    \mathcal{F} (f*g)(\xi)=F(\xi)G(\xi)\\
    \mathcal{F} (fg)(\xi)=(F*G)(\xi)
}{
    \mathcal{F} (f*g)(\omega)=F(\omega)G(\omega)\\
    \mathcal{F} (fg)(\omega)=\frac{1}{2\pi}(F*G)(\omega)
}
$\delta$的卷积性质:\lr{
    &\mathcal{F} (f*\delta)(\xi)=F(\xi)\\
    &f(t)=(f*\delta)(t)=\int_{-\infty}^{\infty}f(x)\delta(t-x)\,dx\\
    &f*\delta_a=\tau_a f
}{
    &\mathcal{F} (f*\delta)(\omega)=F(\omega)\\
    &f(t)=(f*\delta)(t)=\int_{-\infty}^{\infty}f(x)\delta(t-x)\,dx\\
    &f*\delta_a=\tau_a f
}
卷积的性质:\begin{align*}
    &(f*g)'=f'*g=f*g'\\
    &(f^-)*(g^-)=(f*g)^-\\
    &(\tau_b f)*g=\tau_b(f*g)=f*(\tau_b g)\\
    &(\sigma_a f)*(\sigma_a g)=\frac{1}{|a|}\sigma_a(f*g)\\
    &\int_{\mathbb{R}}f*g=\int_{\mathbb{R}}f\cdot\int_{\mathbb{R}}g
\end{align*}

\noindent 9.\underline{相关函数}

$f,g\in L^2(\mathbb{R})$的\textbf{互相关}定义为
\[(f\star g)(x)=\int_{-\infty}^{\infty}f(y)\overline{g(x+y)}\,dy\]
互相关的性质:\begin{itemize}
    \item $(f\star g) =f^-* \overline{g}=\overline{(g\star f)^-}$,如果$f,g$都是实信号,则$f\star g = (g\star f)^- =f^- *g=(f*g^-)^-$.
    \item $\mathcal{F} (f\star g) =\mathcal{F} f\overline{\mathcal{F} g}$,特别地,
          $\mathcal{F} (f\star f) =|\mathcal{F} f|^2$,这是\textbf{维纳-辛钦定理}(Wiener-Khinchin theorem)
    \item $f\star (\tau_b g)=\tau_{b} (f\star g)=(\tau_{-b}f)\star g$
    \item $(f\star g)\leq\|f\|\|g\|$,特别地,$(f\star f)(x)\leq (f\star f)(0)=\|f\|^2$
\end{itemize}

\noindent 10.\underline{作用于分布的算子及其性质}
\begin{align*}
    &\langle \mathcal{F} T,\varphi\rangle=\langle T,\mathcal{F} \varphi\rangle,\langle \mathcal{F} ^{-1}T,\varphi\rangle=\langle T,\mathcal{F} ^{-1}\varphi\rangle\\
    &\langle T',\varphi\rangle=-\langle T,\varphi'\rangle,\langle gT,\varphi\rangle=\langle T,g\varphi\rangle\\
    &\langle \overline{T},\varphi\rangle=\overline{\langle T,\overline{\varphi}\rangle},\langle T^-,\varphi\rangle=\langle T,\varphi^-\rangle\\
    &\langle \tau_b T,\varphi\rangle=\langle T,\tau_{-b}\varphi\rangle,\langle \sigma_a T,\varphi\rangle=\langle T,\frac{1}{|a|}\sigma_{1/a}\varphi\rangle\\
    &\langle g*T,\varphi\rangle=\langle T,g^- *\varphi\rangle,\langle S*T,\varphi\rangle=\langle S(y),\langle T(x),\varphi(x+y)\rangle\rangle
\end{align*}
分布的傅里叶变换与经典情况的区别与联系:T是实分布\textbf{不能}推出
$\mathcal{F} T^-=\overline{\mathcal{F} T}$,分布与分布的卷积的结合律未必成
立,此外的性质都与经典情况一致。

\chapter{关于傅里叶变换的进一步讨论}

\section{取样与插值}\label{sec:Sampling_Interpolation}

对于连续信号,伸缩变换是很自然的,但对于离散信号,例如考虑$x[n]$的伸缩变换$x[\alpha n],\alpha\in\mathbb{R}$,如果$\alpha\in\mathbb{N}_+$,则伸缩后的信号仍为离散信号,只是遗漏了原信号的一些信息;如果$\alpha=\frac{1}{N},N\in\mathbb{N}_+$,则伸缩后的信号在$N$的倍数以外的点没有定义,我们必须人为规定它们的值,这就是\text{插值}。

最简单的处理方式是将这些点全部赋值为0。我们定义:
\begin{definition}[离散信号的伸缩]
    离散信号的伸缩分为两种:$$y[n]=x[Mn],M\in\mathbb{N}_+$$
    称为\textbf{降采样/下采样},对应\textbf{抽取};$$y[n]=\begin{cases}
    x\left[\frac{n}{N}\right],&n=Nk,N\in\mathbb{Z}\\
    0,&\text{otherwise}
    \end{cases}$$
    称为\textbf{升采样/上采样},对应\textbf{插值}。
\end{definition}

如果离散信号本身是由连续信号取样得到的,例如$x[n]=f(T_s n)$那么插值就对应于还原连续信号的过程,因此离散信号的伸缩相当于处理连续信号时增加或减少样值点,这也是升采样、降采样等名称的由来。\textbf{插值}在数学上需要我们给出一个连续信号来逼近甚至还原原信号,工程上则需要我们插入若干新的样值点使得离散信号看起来更接近原信号。下面介绍一些术语:
\begin{definition}
设有确定信号$f(t)$,通过取样的方式找出它在若干点处的\textbf{样值}$f(t_0),f(t_1),f(t_2),\cdots$,\textbf{取样与插值}问题即讨论在怎样的情况下可以借助这组值还原出$f(t)$。一般地,相邻取样点的间隔一定,记为$T_s$,称为\textbf{抽样间隔},将$f_s=1/T_s$称为\textbf{抽样频率},将$\omega_s=2\pi f_s=2\pi/T_s$称为\textbf{抽样角频率}。
\end{definition}

类似于用像素表示图片,利用抽样所得的离散信号$f_d[n]=f(nT_s)$表示信息,相当于用离散信号表示连续信号,能够大大降低传输和存储的成本。一个数字处理系统或数字传输系统的第一个环节就是对信号进行抽样,随后进行量化编码、数字传输或处理,到信号接收方或需要使用信号时再进行信号恢复。

插值的雏形来自于古代天文学,最简单的插值就是通过观察星体在几个时刻的位置推测中间时刻的位置。历史上,最早对插值问题进行系统研究的是牛顿和拉格朗日。牛顿提出的牛顿插值公式是组合数学讨论的话题,他通过引入离散信号的差商的概念,提出了一种使用三角形差商表构造过若干指定数据点的多项式的方法;相比之下拉格朗日的拉格朗日插值法要简单得多,其想法的关键在于构造在特定数据点处非0,而在另外所有数据点处均为0的多项式:
\begin{definition}[拉格朗日插值法]
    对于一组互异的数据点
 \[(x_0,y_0),(x_1,y_1),\dots,(x_n,y_n)\]
 其拉格朗日插值多项式为
\[L(x) = \sum_{i=0}^{n} y_i \ell_i(x)\]
其中 $\ell_i(x)$ 是拉格朗日基函数,满足$\ell_i(x_i)=1,\ell_i(x_j)=0(j\neq i)$:
\[\ell_i(x) = \prod_{\substack{j=0 \\ j \neq i}}^{n} \frac{x - x_j}{x_i - x_j}\]
该多项式满足 $L(x_i) = y_i,i=0,1,\dots,n$。
\end{definition}
这个方法后来被埃尔米特改进,不仅能保证函数值相等,还能保证若干阶导数的值相等,称为\textbf{埃尔米特插值法}。

然而,利用多项式进行插值对原信号的逼近效果十分有限,一些数学家们开始转向研究如何重建整个信号。1948年,克劳德·香农发表了他的巨著《通信的数学原理》,明确提出了他的香农采样定理,并给出了上一节中还原带限信号的公式。这个定理在我们下一节中将介绍的插值流程下是正确的,但要还原大带宽信号需要很高的抽样频率,会对硬件设备提出很高的要求。我们可以通过另外的插值流程还原信号,例如对于一些信号可以利用其稀疏性和相关性,用远低于$f_N$的采样频率还原信号。

另外,还有一种插值的思路将取样时产生的误差也考虑了进来,以这种视角进行最优线性拟合,就是\textbf{最小二乘法},在测量误差可能比较大时,这一类拟合方法的表现会更好。

%这一节就来讨论真实系统中的插值方法,包括自然采样、零阶抽样保持、一阶抽样保持以及降低采样率的一些办法。最后,还将给出有限采样公式,它从数学上给出了用有限的采样点还原带限周期信号的方法。

%自然采样PS6.2(4也有点意思)
%降低采样率的想法
%final2006.2,或许跟一阶抽样保持相通?
%零阶抽样保持,一阶抽样保持

\section{狄拉克梳状分布与理想采样}\label{sec:Dirac_Comb}

在研究周期现象时,经常会将一个函数周期化(periodize),例如,给定函数$f(t)$,要将它变成周期为T的函数,一种标准的做法是“平移、相加”,即在不涉及收敛性的问题时,令
\[f_T(t)=\sum_{k=-\infty}^{\infty}f(t-kT)\]
在\ref{sec:distributions}中,我们曾建立了$\delta_a$的平移性质:$\delta_a*f=\tau_a f$,
所以我们可以将以上周期化的操作抽象出来:
\[f_T(t)=\sum_{k=-\infty}^{\infty}f(t-kT)=\sum_{k=-\infty}^{\infty}\delta_{kT}*f(t)=\left(\sum_{k=-\infty}^{\infty}\delta_{kT}\right)*f(t)\]
由此定义:
\begin{definition}[狄拉克梳状分布]
    对任意$T>0$,定义分布
    \begin{equation*}
        \shah_T=\sum_{k=-\infty}^{\infty}\delta_{kT}
    \end{equation*}
    符号“$\shah$”读作“shah”,在T=1时可以将角标略去,直接写作$\shah$。由于其图像是等间距排列的一个个$\delta$,又称
之为\textbf{狄拉克梳状分布}(dirac comb)。
\end{definition}
容易看出,这个分布具有两条基本性质,其一是前面提到的周期化:$\shah_T*f(t)=f_T(t)$,其二是取样性质,它直接得自$\delta_a$的取样性质:$\shah_T f(t)=\sum_{k=-\infty}^{\infty}f(kT)\delta_{kT}$。另外,作为一个分布,有
\[\langle \shah_T,\varphi\rangle=\left\langle \sum_{k=-\infty}^{\infty}\delta_{kT},\varphi\right\rangle=\sum_{k=-\infty}^{\infty}\varphi(kT)\]
\begin{figure}[H]
    \centering
    \includegraphics[width=0.6\textwidth]{shah.jpeg}
\end{figure}
对于一个施瓦兹函数,以上的式子中均不会出现收敛性的问题。回顾施瓦兹函数类的定义:
\[\mathcal{S} =\left\{\varphi\in C^{\infty}(\mathbb{R} )\left|\lim_{|x|\to\infty}|x|^m\varphi^{(n)}(x)=0,\forall m,n\in\mathbb{N} \right.\right\}\]
对任意的t,考察级数$\sum_{k=-\infty}^{\infty}f(t-kT)$的收敛性时,只需要将n取为0,m取为不小于2的整数,则在$|k|$充分大时,级数比$C/n^2$更小,从而绝对收敛。注意这种估计方式并不依赖于自变量t的选取,因此这个函数项级数一致收敛,其和函数会保留$f(t)$的许多分析性质,例如我们可以逐项求导(从而和函数也无限阶可导)、逐项积分,还可以交换求和符号与其他极限过程的顺序(见数学分析教材)。

我们来考察函数项级数$f_T(t)=\sum_{k=-\infty}^{\infty}f(t-kT)$退化为有限时的情况。
\begin{definition}[时限函数与带限函数]
    时域上紧支(即在$|t|$充分大时函数恒为0)的函数称为\textbf{时限函数},此时级数
$\sum_{k=-\infty}^{\infty}f(t-kT)$实际上是有限和,当然收敛。相应的,在频域上紧支的函数称为\textbf{带限函数}。
\end{definition}
在这两种情况下,对充分大的k,$f(t-kT)=0$,$f_T(t)=\sum_{k=-\infty}^{\infty}f(t-kT)$实际上是有限和,当然是一致收敛的。设函数$f(t)$支于$[-T/2,T/2]$,则有以下两个恒等式:
\[f(x)=\Pi_{T}(f*\shah_T)(x)\]
\[\mathcal{F} f=\Pi_T\mathcal{F} f\]
前者在傅里叶变换下能够带来一些新的东西,见本节后续内容。这里我们先讨论后一恒等式:
\begin{theorem}[采样函数的“卷积幺元”作用]
    设带限函数$f(t)$满足$\text{supp}\ \mathcal{F} f\subset [-T/2,T/2]$,则\lr{
    f(t)=Tsinc(Tt)*f(t)
}{
    f(t)=\frac{T}{2\pi}Sa(Tt)*f(t)
}
可见对于带限函数,抽样信号能够起到类似$\delta$的卷积幺元的作用。
\end{theorem}
\begin{proof}
    由恒等式$\mathcal{F} f=\Pi_T\mathcal{F} f$,对两边同时取傅里叶逆变换,得到
    \lr{
        f(t)&=\mathcal{F} ^{-1}(\Pi_T\mathcal{F} f)(t)\\
               &=\left(\mathcal{F} ^{-1}\Pi_T\right)*\left(\mathcal{F} ^{-1}\mathcal{F} f\right)(t)\\
               &=\left(\mathcal{F} ^{-1}\Pi_T\right)*f(t)\\
               &=Tsinc(Tt)*f(t)
    }{
        f(t)&=\mathcal{F} ^{-1}(\Pi_T\mathcal{F} f)(t)\\
               &=\left(\mathcal{F} ^{-1}\Pi_T\right)*\left(\mathcal{F} ^{-1}\mathcal{F} f\right)(t)\\
               &=\left(\mathcal{F} ^{-1}\Pi_T\right)*f(t)\\
               &=\frac{T}{2\pi}Sa(Tt)*f(t)
    }
\end{proof}

在建立分布$\shah$的傅里叶变换之前,我们先来讨论另外一个话题,它揭示了傅里叶级数与傅里叶变换的深刻关系。考虑函数$f\in\mathcal{S} $,记其周期化函数为$\Phi=\shah*\varphi$,并将$\Phi$的傅里叶系数记为$\hat{\Phi}(k)$,由$\Phi$的无限阶可导性可知其傅里叶级数绝对一致收敛于它本身(这里只需要逐点收敛性,至于更强的收敛性,见\ref{sec:Asymptotic_Behaviour}),即
\[\Phi(t)=\sum_{k=-\infty}^{\infty}\hat{\Phi}(k)e^{2\pi ikt}\]
其中
\begin{align*}
    \hat{\Phi}(k)&=\int_{0}^{1}\Phi(t)e^{-2\pi ikt}\,dt\\
    &=\int_{0}^{1}\left(\sum_{n=-\infty}^{\infty}\varphi(t-n)\right)e^{-2\pi ikt}\,dt\\
    &=\sum_{n=-\infty}^{\infty}\left(\int_{0}^{1}\varphi(t-n)e^{-2\pi ikt}\,dt\right)\\
    &=\sum_{n=-\infty}^{\infty}\left(\int_{n}^{n+1}\varphi(t)e^{-2\pi ik(t+n)}\,dt\right)   &(t\to t+n)\\
    &=\sum_{n=-\infty}^{\infty}\int_{n}^{n+1}\varphi(t) e^{-2\pi ikt}\,dt\\
    &=\int_{-\infty}^{\infty}\varphi(t)e^{-2\pi ikt}\,dt
\end{align*}
即\lr{
    \hat{\Phi}(k)=\mathcal{F} \varphi(k)
}{
    \hat{\Phi}(k)=\mathcal{F} \varphi(2\pi k)
}
证明中,第二行到第三行的积分与求和的换序用到了周期化函数作为函数项级数的一致收敛性。这个等式意味着对时限函数的关系$\mathcal{F} f(k)=Tc_k$对施瓦兹函数都成立(以上是周期化为$T=1$的情况,读者可以自行推广至周期为其他值的情况),进一步,在恒等式
\lr{
    \Phi(t)&=\sum_{k=-\infty}^{\infty}\hat{\Phi}(k)e^{2\pi ikt}=\sum_{k=-\infty}^{\infty}\mathcal{F} \varphi(k)e^{2\pi ikt}
}{
    \Phi(t)&=\sum_{k=-\infty}^{\infty}\hat{\Phi}(k)e^{2\pi ikt}=\sum_{k=-\infty}^{\infty}\mathcal{F} \varphi(2\pi k)e^{2\pi ikt}
}
中取$t=0$,则得到
\lr{\Phi(0)=\sum_{k=-\infty}^{\infty}\mathcal{F} \varphi(k)}{\Phi(0)=\sum_{k=-\infty}^{\infty}\mathcal{F} \varphi(2\pi k)}
又由\[\Phi=\shah*\varphi,\Phi(0)=\sum_{k=-\infty}^{\infty}\varphi(k)\]
得到\textbf{泊松求和公式}(the Poisson summation formula):
\begin{theorem}[泊松求和公式]
    \quad
    \lr{
    \sum_{k=-\infty}^{\infty}\varphi(k)=\sum_{k=-\infty}^{\infty}\mathcal{F} \varphi(k)
}{
    \sum_{k=-\infty}^{\infty}\varphi(k)=\sum_{k=-\infty}^{\infty}\mathcal{F} \varphi(2\pi k)
}
\end{theorem}
我们当然也可以对$\shah_T*\varphi$做类似的推导,但其结论远不如以上两式简洁,这实际上是整数集$\mathbb{Z}$的自对偶性的体现。对偶性的概念,见\ref{sec:Multi_Fourier}。

让我们回到$\shah$分布的傅里叶变换的话题,可以借助泊松求和公式得到:
\lr{
    \left\langle \mathcal{F} \shah,\varphi \right\rangle&=\left\langle \shah,\mathcal{F} \varphi\right\rangle\\
    &=\sum_{k=-\infty}^{\infty}\mathcal{F} \varphi(k)\\
    &=\sum_{k=-\infty}^{\infty}\varphi(k)\\
    &=\left\langle \shah,\varphi\right\rangle
}{
    \left\langle \mathcal{F} \shah_{2\pi},\varphi \right\rangle&=\left\langle \shah_{2\pi},\mathcal{F} \varphi\right\rangle\\
    &=\sum_{k=-\infty}^{\infty}\mathcal{F} \varphi(2\pi k)\\
    &=\sum_{k=-\infty}^{\infty}\varphi(k)\\
    &=\left\langle \shah,\varphi\right\rangle
}
即
\begin{proposition}[$\shah$的傅里叶变换]
    \quad\lr{
    \mathcal{F} \shah=\shah
}{
    \mathcal{F} \shah_{2\pi}=\shah
}
\end{proposition}
可以看到,频率形式下$\shah$分布的傅里叶变换具有自对偶性,即$\shah\overset{\mathcal{F} }{\longleftrightarrow}\shah$,
而角频率形式下需要对$\shah$分布做伸缩变换,这启发我们讨论$\shah$以及更基本的$\delta$
在伸缩变换下的关系。可以将$\delta$作为形式上的函数进行推导,但这里采用更直接的分布伸缩的定义:($a>0$)
\begin{align*}
    \left\langle \sigma_a\delta,\varphi\right\rangle&=\left\langle \delta,\frac{1}{a}\sigma_{1/a}\varphi\right\rangle=\frac{1}{a}\varphi(0)=\left\langle \frac{1}{a}\delta,\varphi\right\rangle
\end{align*}
因此有\begin{proposition}[$\delta$的伸缩]
    \[\sigma_a\delta=\frac{1}{a}\delta\]
    $a<0$的情况可直接从$\delta$是偶分布得出。
\end{proposition}
对于$\shah$分布的伸缩,也可同理得到:\begin{align*}
    \left\langle \sigma_a\shah,\varphi\right\rangle&=\left\langle \shah,\frac{1}{a}\sigma_{1/a}\varphi\right\rangle=\frac{1}{a}\sum_{k=-\infty}^{\infty}\varphi\left(x-\frac{k}{a}\right)=\frac{1}{a}\left\langle \shah_{1/a},\varphi\right\rangle
\end{align*}
因此有\begin{proposition}[$\shah$的伸缩]
    \[\sigma_a\shah=\frac{1}{a}\shah_{1/a},\shah_T=\frac{1}{T}\sigma_{1/T}\shah \quad (a\to\frac{1}{T})\]
\end{proposition}

根据分布的傅里叶变换的连续性(见\ref{sec:distributions}最后的内容),我们还可以直接对$\shah$分布的傅里叶变换做另一种推导:
\lr{
     \mathcal{F} \shah&= \mathcal{F} \left(\lim_{N\to\infty}\sum_{k=-N}^{N}\delta_k\right)\\
    &=\lim_{N\to\infty}\sum_{k=-N}^{N}\mathcal{F} \delta_k\\
    &=\lim_{N\to\infty}\sum_{k=-N}^{N}e^{-2\pi ikx}\\
    &=\sum_{k=-\infty}^{\infty}e^{-2\pi ikx}
}{
    \mathcal{F} \shah_{2\pi}&= \mathcal{F} \left(\lim_{N\to\infty}\sum_{k=-N}^{N}\delta_{2\pi k}\right)\\
    &=\lim_{N\to\infty}\sum_{k=-N}^{N}\mathcal{F} \delta_{2\pi k}\\
    &=\lim_{N\to\infty}\sum_{k=-N}^{N}e^{-2\pi ikx}\\
    &=\sum_{k=-\infty}^{\infty}e^{-2\pi ikx}
}

结合$\shah$的傅里叶变换公式可以得到$\shah$分布作为形式上的函数的另一种定义:
\begin{definition}[$\shah$的等价定义]
    \[\shah(x)=\sum_{k=-\infty}^{\infty}e^{-2\pi ikx}\]
\end{definition}
右边正是附录\ref{sec:Asymptotic_Behaviour}中提到的狄利克雷核在$N\to\infty$时的式子。需要注意,这个式子仅在分布的意义下成立,例如将这个分布作用在某一区间的特征函数$\chi_{[a,b]}\in\mathcal{S}$上,则可验证函数项级数$$\sum_{k=-\infty}^{\infty}e^{-2\pi ikx}$$在整数点处趋于正无穷,在任意不包含整数点的区间上积分值为0。然而,作为函数项级数,我们只能得到整数点处的广义极限存在,这是一个几乎处处不收敛的函数项级数,例如在$x=1/2$处,级数项为$(-1)^k$,其和函数在多数点没有定义,只能作为奇异分布来定义。

根据角频率形式下的$\shah$分布的傅里叶变换,可以想到$\shah$的伸缩的傅里叶变换仍具有
$\shah$的形式,下面就来找出具体的公式:\lr{
    \mathcal{F} \shah_T&=\mathcal{F} \left[\frac{1}{T}\sigma_{1/T}\shah\right]=\sigma_T\shah=\frac{1}{T}\shah_{1/T}
}{
    \mathcal{F} \shah_T&=\mathcal{F} \left[\frac{2\pi}{T}\sigma_{2\pi/T}\shah_{2\pi}\right]=\sigma_{T/2\pi}\shah=\frac{2\pi}{T}\shah_{2\pi/T}
}
即
\begin{proposition}[$\shah_T$的傅里叶变换]
    \quad\lr{
    \mathcal{F} \shah_T=\frac{1}{T}\shah_{1/T}
}{
    \mathcal{F} \shah_T=\frac{2\pi}{T}\shah_{2\pi/T}
}
\end{proposition}

可以看到,傅里叶变换前后$\delta$的位置具有倒数关系,以频率为自变量的傅里叶变换,
$T\cdot\frac{1}{T}=1$,以角频率为自变量做傅里叶变换,$T\cdot\frac{2\pi}{T}=2\pi$,
其联系还是在于$\omega=2\pi\xi$。事实上,在多维傅里叶变换的理论中,我们还将看到更为深刻的关系:$\shah$分布的$\delta$的位置称为“格”,而其傅里叶变换的$\delta$的位置为其“对偶格”,特别地,一维的“对偶格”就是以上所述的“倒数”关系。直观上,我们曾提出对函数和分布的一维傅里叶变换的伸缩定理:
\[f(at)\overset{\mathcal{F} }{\longleftrightarrow}\frac{1}{|a|}F\left(\frac{\xi}{a}\right),\sigma_a T\overset{\mathcal{F} }{\longleftrightarrow}\frac{1}{|a|}\sigma_{1/a}\mathcal{F} T\]
它同样揭示了时域伸而频域缩,时域缩而频域伸的关系,而$\shah_T$为我们提供了一种新的视角。对多维的$\shah$分布的讨论,见\ref{sec:Multi_Fourier}。二维$\shah$分布在$\mathbb{R}^3$中的图像就像一个“钉床”,因此又可以将$\shah$分布称为\textbf{钉床函数}(bed of nails)。

讨论完$\shah$的傅里叶变换,就可以回到一开始的话题。对于时限函数$f(t),\text{supp}\ f\subset[-T/2,T/2]$,研究恒等式
\[f(x)=\Pi_{T}(f*\shah_T)(x)\]
通过傅里叶变换告诉我们的内容。对$f*\shah_T$做傅里叶变换,有
\lr{
    \mathcal{F} (f*\shah_T)(\xi)&=\mathcal{F} f(\xi)\cdot\mathcal{F} \shah_T(\xi)\\
    &=\mathcal{F} f(\xi)\cdot\frac{1}{T}\shah_{1/T}(\xi)\\
    &=\frac{1}{T}\sum_{k=-\infty}^{\infty}\mathcal{F} f(k/T)\delta_{k/T}(\xi)
}{
    \mathcal{F} (f*\shah_T)(\omega)&=\mathcal{F} f(\omega)\cdot\mathcal{F} \shah_T(\omega)\\
    &=\mathcal{F} f(\omega)\cdot\frac{2\pi}{T}\shah_{2\pi/T}(\omega)\\
    &=\frac{2\pi}{T}\sum_{k=-\infty}^{\infty}\mathcal{F} f(2\pi k/T)\delta_{2\pi k/T}(\omega)
}
其物理意义是,将f以周期T进行周期化时,其频谱会剩下频率为$k/T$处,也即角频率为$2\pi k/T$处的值,并且在这些频率处以$\delta$分布的形式“趋于无穷”。

对恒等式两边同时做傅里叶变换,有\lr{
    \mathcal{F} f(\xi)&=\mathcal{F} [\Pi_T(f*\shah_T)](\xi)\\
    &=\mathcal{F} [\Pi_T](\xi)*\mathcal{F} (f*\shah_T)(\xi)\\
    &=sinc(T\xi)*\left(\sum_{k=-\infty}^{\infty}\mathcal{F} f(k/T)\delta_{k/T}(\xi)\right)\\
    &=\sum_{k=-\infty}^{\infty}\mathcal{F} f\left(\frac{k}{T}\right)sinc\left(T(\xi-\frac{k}{T})\right)\\
    &=\sum_{k=-\infty}^{\infty}\mathcal{F} f\left(\frac{k}{T}\right)sinc(T\xi-k)
}{
    \mathcal{F} f(\omega)&=\mathcal{F} [\Pi_T(f*\shah_T)](\omega)\\
    &=\mathcal{F} [\Pi_T](\omega)*\mathcal{F} (f*\shah_T)(\omega)\\
    &=2\pi Sa\left(\frac{T\omega}{2}\right)*\left(\sum_{k=-\infty}^{\infty}\mathcal{F} f\left(\frac{2\pi k}{T}\right)\delta_{2\pi k/T}(\omega)\right)\\
    &=2\pi\sum_{k=-\infty}^{\infty}\mathcal{F} f\left(\frac{2\pi k}{T}\right)Sa\left(T\left(\omega-\frac{2\pi k}{T}\right)\right)\\
    &=2\pi\sum_{k=-\infty}^{\infty}\mathcal{F} f\left(\frac{2\pi k}{T}\right)Sa(T\omega-2\pi k)
}
可以看到,我们用$\mathcal{F} f(k/T)sinc(T\xi-k)$或$\mathcal{F} f(2\pi k/T)Sa(T\omega-2\pi k)$表示出了函数$\mathcal{F} f$,而表达式中的抽样函数完全依赖于所进行的周期化的周期T,也就是说我们只需要用$\mathcal{F} f(k/T)$或$\mathcal{F} f(2\pi k/T)$处的值,就能还原处完整的$\mathcal{F} f$,进一步,对于带限函数$f(t),\text{supp}\ \mathcal{F} f\subset[-\nu_m,\nu_m]$,可以提出完全类似的恒等式,称为\textbf{抽样定理}(the sampling thoerem):
\begin{theorem}[抽样定理]
    \[\mathcal{F} f(x)=\Pi_T(\mathcal{F} f*\shah_T)(x)\]
    其中$T=1/2\nu_m$,对应频率$f_N=1/T$被称为\textbf{奈奎斯特频率}(Nyquist frequency)。
\end{theorem}
对其做傅里叶反变换,有\lr{
    f(t)&=\mathcal{F} ^{-1} [\Pi_T(\mathcal{F} f*\shah_T)](t)\\
    &=Tsinc(Tt)*\mathcal{F} ^{-1} \left(\mathcal{F} f*\shah_T\right)\\
    &=sinc(Tt)*\left(\sum_{k=-\infty}^{\infty}f\left(\frac{k}{T}\right)\delta_{k/T}(t)\right)\\
    &=\sum_{k=-\infty}^{\infty}f\left(\frac{k}{T}\right)sinc\left(T\left(t-\frac{k}{T}\right)\right)\\
    &=\sum_{k=-\infty}^{\infty}f\left(\frac{k}{T}\right)sinc(Tt-k)
}{
    f(t)&=\mathcal{F} ^{-1} [\Pi_T(\mathcal{F} f*\shah_T)](t)\\
    &=\frac{T}{2\pi}Sa\left(\frac{Tt}{2}\right)*\mathcal{F} ^{-1} \left(\mathcal{F} f*\shah_T\right)\\
    &=Sa\left(\frac{Tt}{2}\right)*\left(\sum_{k=-\infty}^{\infty}f\left(\frac{2\pi k}{T}\right)\delta_{2\pi k/T}(t)\right)\\
    &=\sum_{k=-\infty}^{\infty}f\left(\frac{2\pi k}{T}\right)Sa\left(\frac{T}{2}\left(t-\frac{2\pi k}{T}\right)\right)\\
    &=\sum_{k=-\infty}^{\infty}f\left(\frac{2\pi k}{T}\right)Sa\left(\frac{Tt}{2}-\pi k\right)
}
因此有\begin{theorem}
    \quad\lr{
    f(t)=\sum_{k=-\infty}^{\infty}f\left(\frac{k}{T}\right)sinc(Tt-k)
}{
    f(t)=\sum_{k=-\infty}^{\infty}f\left(\frac{2\pi k}{T}\right)Sa\left(\frac{Tt}{2}-\pi k\right)
}
\end{theorem}
这正是\textbf{理想抽样}下,由抽样函数恢复出原函数的方法,它与上一节中讨论的拉格朗日插值有一定的相似之处。$sinc$函数在非零整数点处的值为0,在0处的值为1,于是可以直接利用每个数据点处的值限定所还原出信号在这一点的值,换言之,在以下这种公式所代表的插值方式下,数据点处的值是不会产生偏移的,$Sa$函数同理。不同的是,拉格朗日插值法在数据点以外的位置往往会有误差,而以上插值方法在没有产生混叠现象时是精确的。

利用$\shah$分布的抽样性质进行抽样,就好比将函数转化为一组无穷基底$\delta_k(k\in\mathbb{Z})$上的向量:
\[f(t)\shah(t)=\sum_{k=-\infty}^{\infty}f(k)\delta_k\text{相当于}x[k]=f[k]\text{或}\left(\cdots,f(-2),f(-1),f(0),f(1),f(2),\cdots\right)^T\]
这一过程称为理想抽样。实际问题中我们不可能获得$\delta$分布和$\shah$分布,也不可能进行无限多次抽样,我们将在本节最后建立有限采样公式。

我们刚才由抽样信号恢复出原信号的过程,正是\textbf{香农采样定理}(Shannon sampling theorem)中采样率取奈奎斯特频率时的极限情况。
\begin{theorem}[香农采样定理]
    如果已知信号的频谱支于$[-\nu_m,\nu_m]$,则采样频率必须大于$2\nu_m$,即奈奎斯特频率$f_N$,才能还原出原信号。
\end{theorem}
奈奎斯特频率同时也是信号的\textbf{带宽}(bandwidth)。

直观上,采样频率越高,所需要的成本就越高,提供的信息就越多,而如果采样频率低于$f_N$,
将发生\textbf{混叠}(alias)现象,还原所得的信号将产生失真。例如一个余弦信号,较低的采样频率可能使我们的样本点全部位于余弦信号的零点,此时显然无法还原余弦信号。即便样本点不在零点,也可能还原出其他的信号:
\begin{figure}[H]
    \centering
    \includegraphics[width=0.4\textwidth]{cos.jpeg}
\end{figure}
这正是混叠现象。从频域上不难理解它,在建立抽样定理时我们曾要求
$f(t),\text{supp}\ \mathcal{F} f\subset[-\nu_m,\nu_m]$,利用由此得到的公式
\lr{
    f(t)=\sum_{k=-\infty}^{\infty}f\left(\frac{k}{T}\right)sinc(Tt-k)
}{
    \sum_{k=-\infty}^{\infty}f\left(\frac{2\pi k}{T}\right)Sa\left(\frac{Tt}{2}-\pi k\right)
}
还原函数当然是有误差的,因为在将频域函数周期化时,两个平移后的频域函数会有一部分混在一起,再使用$\Pi$函数进行加窗时,会丢掉一部分原有频谱,并叠加一部分平移后的频谱,如图所示。
\begin{figure}[H]
    \centering
    \includegraphics[width=0.8\textwidth]{alias.jpeg}
\end{figure}

最后,我们建立有限采样公式,以处理实际问题中只能进行有限次采样的情况。

\section{多维傅里叶变换与汉克尔变换}\label{sec:Multi_Fourier}

\section{希尔伯特变换}\label{sec:Hilbert}

\section{回到有限区间}\label{sec:Finite_Interval}

%PS3.6卷积的周期化
%PS4.3
%加窗傅里叶变换,小波变换

\chapter{连续系统的时频分析}

\section{系统概述}\label{sec:System}
对于单输入单输出系统,将输入信号称为激励 (excitation),输出信号称为响应 (response)
,并将时域信号分别用$e(t),r(t)$表示,如果系统用字母H表示,可以记$r(t)=H[e(t)]$
,或更简洁地,$r(t)=He(t)$.H是函数空间上的映射。下面先引入几个基本的定义。
\begin{definition}[线性系统]
    如果算子H是线性的,即
\[\forall a,b\in\mathbb{C},\forall e_1(t),e_2(t),H[ae_1(t)+be_2(t)]=aH[e_1(t)]+bH[e_2(t)]\]
就说系统是\textbf{线性系统}(linear system),且满足叠加法则 (principle of superposition)。
\end{definition}
\begin{definition}[零输入响应与零状态响应]
    由于系统原有的能量,初始状态可能不为零,这时即便激励$e(t)=0$,响应$r(t)$也可能不为零。根据系统的线性性,可以将r(t)分解为两部分:\textbf{零输入响应}(zero input response,$r_{zi}(t)$)和\textbf{零状态响应}(zero state response,$r_{zs}(t)$),零输入响应是指$e(t)=0$,由初始状态产生的响应,零状态响应则是将初始状态置零时的响应。\textbf{全响应}$r(t)=r_{zi}(t)+r_{zs}(t)$。
\end{definition}
对于零输入响应不为0的线性系统,直接将两个响应相加将得到两倍的零输入响应,这不是我们希望看到的,因此我们其实应该定义零状态响应有线性性的系统为线性系统。
\begin{definition}[单位脉冲响应与单位阶跃响应]
    当$e(t)=\delta(t)$时,$r_{zs}(t)$为\textbf{单位脉冲响应},在不涉及零输入
响应时就说r(t)为单位脉冲响应。当$e(t)=u(t)$时,响应r(t)称为\textbf{单位阶跃响应}。
习惯上,将单位脉冲响应记为h(t),单位阶跃响应记为g(t)。需要指出,对于线性系统,脉冲响应应该定义为双变量函数$h(x,y)=H[\delta(x-y)]$,见后文的讨论。
\end{definition}

除了线性性,系统还有一些其他的特性,下面一一进行说明。
\begin{definition}[时不变性和移不变性]
    如果一个系统的输出不依赖于输入信号施加于系统的时间,输入信号发生时移,输出信号也发生相同的时移,即$\forall b\in\mathbb{R},H[e(t-b)]=r(t-b)$,则称该系统是\textbf{时不变系统}
用第二章中定义的时移算子的符号,时不变性可以记为$H\tau_b=\tau_b H$,换言之,时移和经过系统可交换,这与我们的直观相符。

如果激励和响应的自变量是空间,我们同样可以定义系统的\textbf{移不变性}(shift invariance),这时$H\tau_b=\tau_b H$中b是空间中的向量。后面的讨论中,不特意区分时域和空间域。
\end{definition}
\begin{definition}[线性时不变系统]
    如果一个系统既是线性的又是时不变的,就说它是\textbf{线性时不变系统}(linear time-invariant system,LTI)。
\end{definition}
线性时不变系统满足$r(t)=(h*e)(t)$,h为单位脉冲响应,将在本小节的后文介绍。在考虑系统的初始状态时,只要零状态响应具有线性
和时不变性,就说系统是线性时不变系统。

\begin{definition}[因果性]
一个系统在有激励时,才会出现响应,或者说
$r(t_0)$仅依赖于e(t)在$t<t_0$时的值(这里不等号不取等是标准的定义方式,我
们马上会看到它的作用),即
\begin{equation}
    \forall t_0,\left(e_1(t)=e_2(t)\ for\ t<t_0\right)\implies \left(r_1(t)=r_2(t)\ for\ t<t_0\right)\label{eq:4.1}
\end{equation}
这个条件称为\textbf{因果性条件}(casuality condition)。
\end{definition}
对于线性系统,容易看
出其因果性条件等价于
\begin{equation}
    \forall t_0,\left(e(t)=0\ for\ t<t_0\right)\implies \left(r(t)=0\ for\ t<t_0\right)\label{eq:4.2}
\end{equation}
如果系统还有时不变性,则因果性条件等价于
\begin{equation}
    \left(e(t)=0\ for\ t<0\right)\implies \left(r(t)=0\ for\ t<0\right)\label{eq:4.3}
\end{equation}
或者更简便的
\begin{equation}
    h(t)=0\ for\ t<0\label{eq:4.4}
\end{equation}
其中h为单位脉冲响应。只要取$t_0=0$,就从条件\ref{eq:4.2}推出条件\ref{eq:4.3};用
时不变性,条件\ref{eq:4.3}可以推出条件\ref{eq:4.2}。对于单位脉冲响应,从
$\delta=0\ for\ t<0$可以推出$h(t)=0\ for\ t<0$;如果$h(t)=0,e(t)=0\ for\ t<0$
,则根据\ref{sec:convolution}中介绍的卷积的“支集相加”性质,有$r(t)=(h*e)(t)=0\ for\ t<0$.

\begin{definition}[稳定性]
如果一个系统在激励信号有界时,响应也是有界的,则称之为\textbf{稳定系统}(bounded-input bounded-output,BIBO)。
\end{definition}
工程上,一个实用系统在所有可能条件下都保持稳定时至关重要的。我们将在\ref{sec:Complex_Freq_Analysis}中给出线性时不变系统具有稳定性的充分必要条件。

\begin{definition}[记忆性]
如果一个系统在$t_0$时刻的响应不仅与该时刻的输入有关,还与其他时刻的输入有关,则称之为\textbf{记忆系统/动态系统},与之相对的是\textbf{即时系统},在$t_0$时刻的响应仅与$t_0$时刻的激励有关,这使得我们可以直接用函数$r(t)=f(e(t))$对系统进行建模。
\end{definition}

\begin{definition}[可逆性]
    如果H(作为映射)如果是单射,则也是双射(因为我们不关注其值域),于是H是可逆的。换句话说,一种响应仅可能对应唯一的激励。
\end{definition}

以下列举一些常见的线性系统。

\begin{example}[时域乘积]
\begin{equation*}
    r(t)=f(t)e(t)
\end{equation*}
例如描述开关开闭的系统,$f(t)=\Pi_T(t)$或$u(t)$,描述取样的系统,$f(t)=\shah(t)$.
\end{example}

\begin{example}[矩阵乘法]
对于n维的离散激励信号$\mathbf{v}$,\begin{equation}
    \mathbf{w}=A\mathbf{v},A\in M_{n\times n}(\mathbb{C})
\end{equation}
例如,线性动力系统中,初值问题$\mathbf{\dot{x}}(t)=A\mathbf{x}(t)$的解
为$H[\mathbf{v}]=e^{At}\mathbf{v}$。
\end{example}
线性代数中,如果一个线性算子L的在某一组基下的矩阵表示是A,则将L的转置定义为用
$A^T$表示的线性算子,其中T表示矩阵的转置,如果矩阵A是对称的,则算子L称为对称
算子;L的共轭转置定义为用$A^H$表示的线性算子,其中H表示矩阵的共轭转置,如果A是埃尔
米特的(Hermitian),即$A^H=A$,则称算子L是埃尔米特的,并说L是\textbf{自伴算子}
(self-adjoint operator).对于矩阵乘法系统,可以根据A的性质定义
\textbf{对称系统、自伴系统}。

\begin{example}[积分]
\begin{equation*}
    r(t)=\int_{a}^{b}k(x,y)e(y)\,dy
\end{equation*}
其中,$k(x,y)$称为核函数(kernel),$\int_{a}^{b}k(x,y)e(y)\,dy$称为对核积分
(integrating against a kernel).
\end{example}
对核积分非常类似于连续版本的矩阵乘积,我们同
样可以定义:\begin{itemize}
    \item 上述系统的转置系统描述为$e(t)\mapsto\int_{a}^{b}k(y,x)e(y)\,dy$
    \item 如果$k(x,y)=k(y,x)$,称k是对称的并将系统称为\textbf{对称系统}
    \item 如果$k(x,y)=\overline{k(y,x)}$,称k是埃尔米特的并将系统称为\textbf{自伴系统}。
\end{itemize}

\begin{example}[卷积]
    取定连续信号$g,r(t)=g*e(t)$;取定周期离散信号$\mathbf{h},r[n]=\mathbf{h*e}[n]$.
\end{example}
在标准正交基底下,卷积系统的矩阵是一个循环矩阵(circulant matrix)或托普利兹矩
阵(Toepliz matrix):\[\begin{pmatrix}
    h[0] & h[N-1] & \cdots & h[1]\\
    h[1] & h[0] & \cdots &  h[2]\\
    \vdots & \vdots & \ddots & \vdots\\
    h[N-1] & h[N-2] & \cdots & h[0]
\end{pmatrix}\]

\begin{example}[信号平移]
    取定时延b,$r(t)=e(t-b)$;取定整数m,$r[n]=e[n-m]$.
\end{example}

\begin{example}[傅里叶变换]
$r(t)=\mathcal{F} e(t)$.

在一些光学仪器中可以实
现空间上的傅里叶变换,这时变量t是多维的空间变量,例如矩孔夫琅禾费衍射装置。作
为径向对称函数的傅里叶变换的特例,零阶汉克尔变换可以用圆孔夫琅禾费衍射装置实现。
对于这两个变换,见\ref{sec:Multi_Fourier}和\ref{sec:Hilbert}.
\end{example}

我们知道,函数与$\delta$的卷积还是它本身,这就好比信号在进入任何一个系统前,先经过了一个与$\delta$卷积的系统。这样的系统连接方式称为系统的\textbf{级联}(cascade),下面就来研究这个话题。
\begin{theorem}[线性系统的级联]
    设有两线性系统K,L,将两系统级联之后,所得的新系统仍是线性系统:\begin{align*}
    LK[ae_1+be_2]=L[aK[e_1]+bK[e_2]]=aLK[e_1]+bLK[e_2]
\end{align*}
\end{theorem}
对于周期离散信号,两个系统都是矩阵乘法,则级联系统也可以用矩阵乘法来描述,并且其矩阵是两子系统的矩阵之乘积,这
正是定义矩阵乘法的原因之一;对于连续信号,如果两个系统都是对核积分系统,我们也有类似的结论:
\begin{theorem}[对核积分系统的级联]
    设有两对核积分系统K,L:
    \[K[e(t)](x)=\int_{a}^{b}k(x,y)e(y)\,dy,L[e(t)](x)=\int_{a}^{b}l(x,y)e(y)\,dy\]
则级联系统$H=LK$也是对核积分系统,并且核函数为\begin{equation*}
    h(x,z)=L_y(k(y,z))(x)=\int_{a}^{b}l(x,y)k(y,z)\,dy
\end{equation*}
\end{theorem}
这里$L_y$表示$k(x,y)$作为y的函数输入系统L,输出为x,y的双变量函数。
\begin{proof}\begin{align*}
    H[e(z)](x)&=LK[e(z)](x)\\
    &=\int_{a}^{b}l(x,y)\left(\int_{a}^{b}k(y,z)e(z)\,dz\right)\,dy\\
    &=\int_{a}^{b}e(z)\,dz\int_{a}^{b}l(x,y)k(y,z)\,dy\\
    &=\int_{a}^{b}h(x,z)e(z)\,dz
\end{align*}
可以看到,只要我们使用的函数满足富比尼定理(Fubini's thoerem)的条件,就可以交
换积分次序,并得到前文中断言的证明。
\end{proof}

如果系统H是对核的无穷限积分,就有\textbf{叠加定理}(superposition thoerem),由于它是泛函分析中对应定理的特殊形式,有时也将它称为\textbf{施瓦兹核定理}(Schwartz kernel thoerem):
\begin{theorem}[叠加定理/施瓦兹核定理]
    \begin{equation*}
    r(x)=\int_{-\infty}^{\infty}h(x,y)e(y)\,dy
\end{equation*}
其中$h(x,y)$是线性系统H的脉冲响应。
\end{theorem}
\begin{proof}\begin{align*}
    H[e(y)]&=H[e*\delta(y)]\\
    &=H\left[\int_{-\infty}^{\infty}\delta(x-y)e(y)\,dy\right]\\
    &=\int_{-\infty}^{\infty}H[\delta(x-y)]e(y)\,dy\\
    &=\int_{-\infty}^{\infty}h(x,y)e(y)\,dy
\end{align*}
\end{proof}
反过来,如果一个系统是激励信号与脉冲响应的含参积分,那么这个系统也是线性的,
这得自积分的线性性。至此我们有了充分的动机去定义:
\begin{definition}[线性系统的脉冲响应]
    线性系统H的\textbf{脉冲响应}定义为$h(x,y)=H_x[\delta(x-y)]$,其中$H_x$表示$\delta(x-y)$作为x的函数输入系统,输出为x,y的双变量函数。
\end{definition}
\begin{proposition}[线性系统的等价定义]
    “一个系统是线性系统”等价于“响应是激励对脉冲响应积分”。
\end{proposition}

对于时不变系统,这个定理具有更加优美的形式。将系统视为线性系统,其脉冲响应为
\[h(x,y)=H_x[\delta(x-y)]=H[\tau_y\delta(x)]=\tau_y H[\delta(x)]\]
按照单变量的脉冲响应的定义,这就是$\tau_y h(x)=h(x-y)$,它只依赖于x、y的差值,
带入叠加定理:\begin{align*}
    r(x)&=\int_{-\infty}^{\infty}h(x-y)e(y)\,dy\\
    &=(h*e)(t)
\end{align*}
也就是说,信号经过线性时不变系统相当于与这个系统的单位脉冲响应做卷积。下面来验证,
如果一个系统是激励信号与脉冲响应的卷积,那么这个系统也是线性时不变系统。线性性是
显然的,我们来验证$H\tau_b=\tau_b H$:\begin{align*}
    H[\tau_b e(y)](x)=(h*\tau_b e)(x)=\tau_b(h*e)(x)=\tau_b H[e(y)]
\end{align*}
总之,我们有:
\begin{proposition}[线性时不变系统的等价定义]
    “一个系统是线性时不变系统”等价于“响应是激励与脉冲响应的卷积”。
\end{proposition}

\begin{example}[单位阶跃响应]
求线性时不变系统的单位阶跃响应$g(t)$:\begin{align*}
    g(t)&=H[u(t)]=(u*h)(t)\\
    &=\int_{-\infty}^{\infty}u(t-y)h(y)\,dy\\
    &=\int_{-\infty}^{t}h(y)\,dy
\end{align*}
可以看到,单位阶跃响应就是单位脉冲响应的积分,利用这个性质,可以快速地求出单位阶
跃响应,在使用卷积来描述线性时不变系统时,这个性质很容易得到,但在一些其他的线性
时不变系统中很难想到具有这样的性质,例如下一节将讨论的系统。
\end{example}

\section{微分方程}\label{sec:ODE}

除了上一节中提到的几种系统,还有一种典型的具有因果性的线性时不变系统是用常系数线
性微分方程表示的,在学习数学分析时我们曾遇到过这种微分方程,下面先回忆在数学分析
中对于这种方程的处理方法。\begin{definition}[常系数线性微分方程]
    一个形如
\[a_n y^{(n)}+a_{n-1}y^{(n-1)}+\cdots +a_1 y'+a_0 y=f\]
称为\textbf{n阶常系数线性微分方程},等式左侧的y和f都是t的函数,特别地,如果
$f(t)=0$,称之为\textbf{n阶常系数线性齐次微分方程}。
\end{definition}
不难看出,如果找到了n阶常系数线性齐次微分方程的两个解$y_1,y_2$,则$ay_1+by_2$也是方程的解;另外,如果带
入$y=e^{\lambda t}$并消去这一项,则微分方程化为其\textbf{特征方程}:
\begin{definition}[特征方程]
    \[a_n\lambda^n+a_{n-1}\lambda^{n-1}+\cdots+a_1\lambda+a_0=0\]
\end{definition}

根据代数基本定理,$\lambda$在复数域$\mathbb{C}$中有n个解(称之为\textbf{特征根})
,即微分方程一定有n个形如$e^{\lambda t}$的解。因此,对于常系数线性微分方程,我
们的标准处理方法是:
\begin{enumerate}
    \item 先令$f(t)=0$,得到其对应的\textbf{齐次方程},解其方程的特征方程得到n个\textbf{齐次解}(homogeneous solution),记为$y_h(t)$;
    \item 找到一个\textbf{特解}(particular solution)使得它恰好满足原方程,记为$y_p(t)$;
    \item 将齐次解的线性组合与特解相加,得到\textbf{完全解}:$y(t)=y_h(t)+y_p(t)$。
\end{enumerate}
对于特解,需要一定的配凑技巧,特别地,如果f为一个t的多项式与某个齐次解的乘积,则
需要带入固定形式的特解求解其中的参数(以下默认P、Q为多项式):\begin{itemize}
    \item 如果$f(t)=c$,c为常数,则特解$y_p(t)$也为常数;
    \item 如果$f(t)=e^{\beta t}$,$\beta$不是特征根,则特解$y_p(t)=ce^{\beta t}$,c为常数;
    \item 如果$f(t)$为关于t的n次多项式,则特解$y_p(t)$也为t的n次多项式;
    \item 如果$f(t)=P(t)e^{\lambda t}$,$\lambda$为k重特征根,则特解$y_p(t)=t^k Q(t)e^{\lambda t}$,其中Q是与P次数相同的多项式,即$deg Q=degP$;
    \item 如果$f(t)=(P_1(t)\cos(\omega t)+P_2(t)\sin(\omega t))e^{\lambda t}$,$\lambda\pm i\omega$是k重特征根,则特解$y_p(t)=t^k(Q_1(t)\cos(\omega t)+Q_2(t)\sin(\omega t))e^{\lambda t}$,其中$degQ_1=degQ_2=max\{degP_1,degP_2\}$
\end{itemize}
如果确定了一组初始条件$y(t_0),y'(t_0),\cdots,y^{(n-1)}(t_0)$,则微分方程有唯
一的解,因为完全解中的特解部分是确定的,而齐次解部分的n个解对应着n个系数,只要给
定n个初始条件,就相当于给出了n个方程,可以解出所有的系数。
\begin{example}
    求解微分方程\[y''-3y'+2y=2e^{-x}\cos x+e^{2x}(4x+5)\]
解:特征方程为$\lambda^2-3\lambda+2=0$,有两个特征根$\lambda_1=1,\lambda_2=2$,
齐次解为$y_h(t)=C_1e^{\lambda_1 t}+C_2e^{\lambda_2 t}$。根据解的线性性,可以将方程拆分为:\begin{align*}
    y''-3y'+2y=2e^{-x}\cos x\\
    y''-3y'+2y=e^{2x}(4x+5)
\end{align*}
对于前一个方程,$-1+i$不是特征根,设方程特解为$y_1(t)=e^{-x}(A\cos x+B\sin x)$,带入原方程可得$A=\frac{1}{5},B=-\frac{1}{5}$,即$y_1(t)=\frac{1}{5}e^{-x}(\cos x-\sin x)$。\\
对后一个方程,2是一重特征根,设方程特解为$y_2(t)=e^{2x}x(ax+b)$,带入原方程可得$a=2,b=1$,即$y_2(t)=e^{2x}x(2x+1)$。\\
将齐次解与特解相加,得到完全解:\begin{align*}y(t)&=C_1e^{\lambda_1 t}+C_2e^{\lambda_2 t}+\frac{1}{5}e^{-x}(\cos x-\sin x)+e^{2x}x(2x+1)\\
    &=C_1 e^{x}+\frac{1}{5}e^{-x}(\cos x-\sin x)+e^{2x}(2x^2+x+C_2)
\end{align*}
\end{example}

下面介绍一些信号与系统课程中会用到的术语。
\begin{definition}[自由响应与强迫相应]
    一个用常系数线性微分方程描述的系统,就是指激励为$e(t)=f(t)$,响应为微分方程的完
全解$y(t)$的系统,齐次解$y_h(t)$相当于在激励为0时系统的响应。其中,齐次解又称为
\textbf{自由响应}(natural response),它不依赖于激励的形式,而特解又称为
\textbf{强迫响应}(forced response)。
\end{definition}
注意区分它们与\ref{sec:System}中介绍的
零状态响应、零输入响应,后者对于更一般的线性系统都有定义,对于用微分方程描述的系统,齐
次解和特解、零输入响应和零状态响应也是不同的分类方式。

\begin{example}[两种响应划分方式的区别]
    由微分方程及初始条件
\[y'+2y=10,y(0)=1\]
描述的系统,采用数学的解法,可以求出其特征方程$\lambda+2=0$的解$\lambda=-2$,
从而有齐次解$y_h(t)=Ce^{-2t}$,C为常数,特解$y_p(t)=5$,完全解
$y(t)=y_h(t)+y_p(t)=5+Ce^{-2t}$,带入$y(0)=1$得$5+C=1,C=-4$,因此最终的解为
\[y(t)=5-4e^{-2t}\]
采用系统响应分解,求零输入响应时,应求齐次方程满足初始条件得解,即令$y_h(0)=C=1$,
得到$C=1,y_{z.i.}(t)=e^{-2t}$;而求零状态响应时,则应该求原方程在初始条件为0向
量情况下的解,即在前面得到的完全解中令$y(0)=5+C=0$,得到$C=-5$,
$y_{zs}(t)=5-5e^{-2t}$,将零输入响应、零状态响应相加,又得到了原来的解
$y(t)=5-4e^{-2t}$。

可见,$y_h(t)\neq y_{zi}(t),y_p(t)\neq y_{zs}(t)$。
\end{example}
尽管从数学上我们先得到了齐次解和特解才进一步求出零输入响应、零状态响应,但解的这
两种划分都是有意义的,后者揭示了:响应中有一部分齐次解用来使响应满足初始条件,而
另一部分齐次解会与特解叠加得到无关初始条件、仅依赖于方程形式的解。

由于我们对输入激励建模时,常常认为激励是瞬间施加到系统上的,并且经常认为激励施加
的时间就是0时刻,我们需要区分0的左、右邻域内的系统状态,为此定义:
\begin{definition}[起始状态和初始状态]
    将激励接入之前的瞬间
系统的状态称为\textbf{$0_-$状态}或\textbf{起始状态},记为
\[y_{(k)}(0-)=[y(0-),y'(0-),\cdots,y^{(n-1)}(0-)]\]
将激励接入之后的瞬间系统的状态称为\textbf{$0_+$状态}或\textbf{初始状态},记为
\[y_{(k)}(0+)=[y(0+),y'(0+),\cdots,y^{(n-1)}(0+)]\]
\end{definition}
注意此时我们是允许一些条件发生突变的。
\begin{example}[换路定则]
    RLC振荡电路中,对于电感,
$u_L=Ldi_L/dt$,施加于电感两侧的电压不会无穷大,其电流$i_L(t)$总是连续的,即
$i_L(0_-)=i_L(0_+)=i_L(0)$,而其电压则可以不连续;对于电容,$i_C=Cdu_C/dt$,
通过电容的电流不会无穷大,其电压$u_C(t)$总是连续的,即
$u_C(0_-)=u_C(0_+)=u_C(0)$,而其电流可以不连续。电感电流、电容电压不突变的这个
结果,称为换路定则。
\end{example}

激励$f(t)$不连续,意味着$y(t)$(或它的某阶导数)不连续,这时需要将y视为一个分布来求导,见\ref{sec:distributions}。
例如$f(t)=g(t)u(t)$时,完全解中可能含有$u(t)$及其导数$\delta(t)$。

物理中的受迫振动,以及电路分析中的二阶电路,都是用二阶常系数线性微分方程描述的系
统的例子,从直观上看,如果使激励(施加于系统的力,或者电源提供的电压、电流)经过
一段时间后变为0,则响应一定也会随时间的增大而趋于0。因此,又可以将响应分为\textbf{稳态响应}(steady state response)$y_{ss}(t)$
和\textbf{暂态响应}(transient state)$y_t(t)$:
\begin{definition}[稳态响应和暂态响应]
完全解中,$t\to\infty$时保留下来的分量称为稳态响应,例如一个常数;$t\to\infty$时趋于0的分量为稳态响应,例如$e^{-\lambda t}$。
\end{definition}

\begin{proposition}
    由常系数线性齐次微分方程描述的系统,如果初始状态为0向量,则该系统是因果线性时不变系统。
\end{proposition}
\begin{proof}
    设一个系统描述为
\[a_n r^{(n)}+a_{n-1}r^{(n-1)}+\cdot +a_1 r'+a_0 r=e\]
并且对于给定的激励$e(t)$找到了完全解$r(t)$,那么\begin{align*}
    \tau_b e&=\tau_b(a_n r^{(n)}+a_{n-1}r^{(n-1)}+\cdot +a_1 r'+a_0 r)\\
    &=a_n \tau_b r^{(n)}+a_{n-1}\tau_b r^{(n-1)}+\cdot +a_1 \tau_b r'+a_0 \tau_b r\\
    &=a_n (\tau_b r)^{(n)}+a_{n-1}(\tau_b r)^{(n-1)}+\cdot +a_1 (\tau_b r)'+a_0 (\tau_b r)
\end{align*}
即激励产生延时b时,零状态响应也产生延时b。

线性性是显然的,对此感到疑惑的读者可以自行将激励带入微分方程验证。

对于因果性,见后文中对这种系统的单位脉冲响应的讨论。
\end{proof}
至于零输入响应,它不会由于激励的改变而改变,不仅会破坏系统的时不变性,还会破坏系统的线性性,因为在初始状态非0时,直接
将两个完全解相加,则所得信号的初始条件变为给定初始条件的两倍,但由于零输入相应是
易于研究的,并且只要指出了零输入响应,就可以认为系统的初始状态为n维零向量来进行
研究,最后将所得的零状态响应与零输入响应叠加,因此我们约定:
\begin{definition}
    由常系数线性微分方程描述的系统总是因果线性时不变系统,不论初始状态是否为0。
\end{definition}

以线性时不变系统的视角来看微分方程,自然会想到求这个系统的单位脉冲响应,也就是说,
令方程右侧的激励信号为$\delta$,看所得响应$h(t)$,这样,根据上一小节中的结果,
不论方程右侧的激励$e(t)$变为何种形式,只要它与$h(t)$的卷积是有定义的,就可以直接
求出响应(或者说零状态响应)$r(t)=(h*e)(t)$,只要再叠加上零输入响应,就可以得到
任意给定初始状态下方程的解,这是一个一劳永逸的工作,在实际问题中,我们经常给出系
统的微分方程描述和初始条件,而激励则是任意的,现在就可以避开微分方程,直接计算卷
积来求得系统的响应。历史上,数学家格林在1828年关于电势和磁势的论文中首次提出了这种方法并用它来处理带有边界条件的偏微分方程(如泊松方程)。因此我们说:
\begin{definition}
使用单位脉冲响应处理微分方程的技术称为\textbf{基本解法}或\textbf{格林函数法}(Green's function method),脉冲响应$h(t)$被称为\textbf{基本解}(fundamental solution)或\textbf{格林函数}(Green's function)。
\end{definition}
它在偏微分方程、数学物理和线性系统理论等领域中都有广泛的应用。

那么,如何求这个单位脉冲响应呢?对于微分方程
\[a_n r^{(n)}+a_{n-1}r^{(n-1)}+\cdot +a_1 r'+a_0 r=\delta\]
要使等式右边出现$\delta$,$r(t)$中一定含有$\delta$及其导数、积分,根据\ref{sec:distributions}
中的结果,$u'=\delta$,而$u(t)$的各阶积分都可以用$P(t)u(t)$来表示,其中$P(t)$
是t的多项式,因此可以假定$r(t)=f_1(t)u(t)+f_2(t)\delta(t)+f_3(t)\delta'(t)+\cdots$。
我们知道,$\delta$具有取样性质:$\delta(t)f(t)=\delta(t)f(0)$,类似地,还推导
过$\delta'$与函数的乘积:$g\delta'=g(0)\delta'-g'(0)\delta$,不难想象,$\delta$
的各阶导数与函数的乘积都是类似的(可以通过归纳法验证这一点),于是刚才的$r(t)$简
化为
\[r(t)=f(t)u(t)+a_1\delta+a_2\delta'+\cdots\]
我们知道,分布与函数乘积的求导也满足莱布尼兹法则,运用这一点,将上式带入激励取为
$\delta$的微分方程,可以通过待定系数法求出各个系数,并得到关于$f(t)$的常规的微分
方程。很明显,我们不总是需要$\delta$及其高阶导数项,为此需要研究取到$u(t)$的几阶
导数就足以完成基于待定系数法的求解,我们通过一个简单的例子来说明这个问题。

\begin{example}
    求微分方程
\[r''(t)+3r'(t)+2r(t)=e(t)\]
的单位脉冲响应。令$r''(t)+3r'(t)+2r(t)=\delta$,如果$r(t)$中含有$\delta'$,则
其各阶导数会因此含有$\delta$的更高阶导数,这不是我们所需要的,因此,令
$r(t)=f(t)u(t)+a\delta$,求出其一阶、二阶导数:\begin{align*}
    r'(t)&=f'(t)u(t)+f(t)\delta+a\delta'\\
    &=f'(t)u(t)+f(0)\delta+a\delta'\\
    r''(t)&=f''(t)u(t)+f'(t)\delta+f(0)\delta'+a\delta''\\
    &=f''(t)u(t)+f'(0)\delta+f(0)\delta'+a\delta''
\end{align*}
再带入原方程:\begin{align*}
    &r''(t)+3r'(t)+2r(t)\\
    =&\left[f''(t)u(t)+f'(0)\delta+f(0)\delta'+a\delta''\right]\\
    &+3\left[f'(t)u(t)+f(0)\delta+a\delta'\right]\\
    &+2\left[f(t)u(t)+a\delta\right]\\
    =&\left[f''(t)+3f'(t)+2f(t)\right]u(t)+\left[f'(0)+3f(0)+2a\right]\delta+\left[f(0)+3a\right]\delta'+a\delta''
\end{align*}
对比系数,得到:\begin{align*}\begin{cases}
    &f''(t)+3f'(t)+2f(t)=0\\
    &f'(0)+3f'(0)+2a=1\\
    &f(0)+3a=0\\
    &a=0
\end{cases}
\end{align*}
因此问题转化为求解微分方程
\[f''(t)+3f'(t)+2f(t)=0,f(0)=0,f'(0)=1\]
这个方程正是一开始的微分方程的齐次方程,可以立即得到它的解:
\begin{align*}
    &f(t)=e^{-t}-e^{-2t}\\
    &h(t)=\left(e^{-t}-e^{-2t}\right)u(t)
\end{align*}
总之,只要将齐次解乘以$u(t)$,求出对应阶导数并带回原方程对比系数,即可得到单位脉
冲响应;由于单位阶跃响应形如$h(t)=y_h(t)u(t)$,满足$h(t)=0,t<0$,微分方程描述的
系统还具有因果性。
\end{example}
\begin{definition}[因果信号]
    以上例子中出现了满足$h(t)=y_h(t)u(t)$的信号,将这种信号称为\textbf{因果信号}(causal signal);而如果信号满足$h(t)=y_h(t)u(-t)$,则称之为\textbf{反因果信号}(anti-causal signal)。
\end{definition}

更一般地,由微分方程\[\sum_{n=0}^{N}a_n r^{(n)}(t)=\sum_{m=0}^{M}b_m e^{(m)}(t)\]
描述的系统,其单位脉冲响应满足:
\[\sum_{n=0}^{N}a_n r^{(n)}(t)=\sum_{m=0}^{M}b_m \delta^{(m)}\]
当$r(t)$的最高次导的阶数大于$\delta$的最高阶导阶数,即$N> M$时,解为$y_h(t)u(t)$;当$r(t)$的最高次导的阶数不小于$\delta$的最高阶导阶数,即$N\leq M$时,解为$y_h(t)u(t)+\sum_{k=0}^{M-N}c_k \delta^{(k)}$。换言之,需要利用方程左侧的最高阶导,让方程右侧所需的$\delta$的最高阶导数项出现。

从数学上我们没有必要考虑这种形式的方程,只要把方程右侧看作一个新的非齐次项,它与其他微分方程没有区别,但作为一个线性时不变系统,求解其脉冲响应就显得十分重要了。另一种单位脉冲响应的求解方法是,将系统视作普通的常系数线性微分方程,将基本解与$\sum_{m=0}^{M}b_m e^{(m)}(t)$的卷积作为系统的单位脉冲响应,但这样涉及$\delta$各阶导数的卷积,计算较为复杂,通过待定系数法求解往往更快。

我们在\ref{sec:System}末尾得到:单位脉冲响应是单位阶跃响应的导数,因此可以通过单位脉冲响应求出单位阶跃响应。不过一般不会使用待定系数法求解单位阶跃响应,因为这涉及非齐次项得处理,读者可以在以下例子中,改写对比系数所得得方程组来体会这一点。
\begin{example}
    设系统由微分方程\[r''(t)+3r'(t)+2r(t)=e''(t)+e(t)\]描述,求其单位脉冲响应。\\
    解:齐次方程的特征方程为$\lambda^2+3\lambda+2=0$,$\lambda_1=-1,\lambda_2=-2$,齐次解为$r_h(t)=C_1 e^{-t}+C_2 e^{-2t}$。\\
    将$\delta$带入右式得到$\delta''+\delta$,为了使左式也出现$\delta''$,设单位脉冲响应为$h(t)=r_h(t)u(t)+C\delta$,则$h'(t)=r_h'(t)u(t)+r_h(0)\delta+C\delta',h''(t)=r_h''(t)u(t)+r_h'(0)\delta+r_h(0)\delta'+C\delta''$
    带入原方程得到:\begin{align*}
        LHS=&\left(r_h''(t)u(t)+r_h'(0)\delta+r_h(0)\delta'+C\delta''\right)\\
        &+3\left(r_h'(t)u(t)+r_h(0)\delta+C\delta'\right)\\
        &+2\left(r_h(t)u(t)+C\delta\right)\\
        =&\left(r''_h(t)+3r'_h(t)+2r_h(t)\right)u(t)+\left(r'_h(0)+3r_h(0)+2C\right)\delta+\left(r_h(0)+3C\right)\delta'+C\delta''
    \end{align*}
    对比系数,得到:\begin{align*}
        \begin{cases}
            r'_h(0)+3r_h(0)+2C=1\\
            r_h(0)+3C=0\\
            C=1
        \end{cases}
    \end{align*}
    所以$C=1,r_h(0)=-3,r'_h(0)=8$,注意$u(t)$的系数自然为0,这是由齐次解的性质保证的。满足上述条件的$C_1=2,C_2=-5$,因此单位脉冲响应为$h(t)=\left(2e^{-t}-5e^{-2t}\right)u(t)+\delta$。
\end{example}

\section{系统的频率响应特性}\label{sec:freq_response}

由于这两个小节中,绝大部分文献和教材都采用角频率进行讨论,并且涉及大量的三角函数运算时,角频率也更加方便,本章的这两个小节将直接使用角频率形式的傅里叶变换。

在上一节中我们得到了$r(t)=(h*e)(t)$,看到卷积的结构,自然会想到对其做傅里叶变换,即\[R(\omega)=H(\omega)E(\omega)\]
\begin{definition}[频率响应]
    将$H(\omega)$称为$r(t)$的\textbf{频率响应}。
\end{definition}
一个物理可实现的系统,容易理解它必须满足因果性,即$h(t)=0,t<0$。实际上,这样的系统还必须满足\textbf{佩利-维纳准则}(Paley-Wiener criterion):
\begin{definition}[佩利-维纳准则]
    \begin{align*}
    &\int_{-\infty}^{\infty}\left|H(\omega)\right|^2\,d\omega<\infty\\
    &\int_{-\infty}^{\infty}\frac{\left|\ln|H(\omega)|\right|}{1+\omega^2}\,d\omega<\infty
\end{align*}
\end{definition}

在分析电路时曾见过它的另一种定义方式:
\begin{definition}[频率响应]
    $H(\omega)=Y(\omega)/X(\omega)$,其中$Y(\omega)$是输出端口信号的频域形式,$X(\omega)$是输入端口信号的频域形式
\end{definition}
它与我们用傅里叶变换的理论给出的定义没有区别,但是展示了另一种求频率响应的视角。在学习电路分析时,我们知道线性的电感、电容在相量法下可以与线性电阻等同处理,统一地用\textbf{阻抗}来描述其电压电流关系。感值为L的电感,阻抗为$i\omega L$,容值为C的电容,阻抗为$1/(i\omega C)$,而阻值为R的电阻,阻抗就是R,这种分析方法实际上正是基于傅里叶变换的。
\begin{example}
    描述电容的方程为:
\[i(t)=C\frac{d}{dt}v(t)\]
对其做傅里叶变换,得到
\[I(\omega)=i\omega C V(\omega)\]
因此电容的阻抗为
\[Z_C(\omega)=\frac{V(\omega)}{I(\omega)}=\frac{1}{i\omega C}\]
描述电感的方程为:
\[v(t)=L\frac{d}{dt}i(t)\]
对其做傅里叶变换,得到
\[V(\omega)=i\omega L I(\omega)\]
因此电感的阻抗为
\[Z_L(\omega)=\frac{V(\omega)}{I(\omega)}=i\omega L\]
\end{example}
\begin{example}
    考虑RC电路:
\begin{center}\begin{circuitikz}[american] % 美式电路风格
    % 输入端口
    \draw (0,2) to [short, o-] (1,2) % 输入接线端到电阻
          to [R, l=$R$] (4,2);       % 电阻R
    
    % 电容C与接地
    \draw (4,2) to [C, l_=$C$] (4,0) % 电容C(下标注)
          to [short] (4,-0.5);   % 接地符号延伸
    \draw (4,0) node[ground]{};      % 接地符号
    
    % 输出端口(从电阻与电容连接点引出)
    \draw (4,2) to [short, -o] (6,2);
    
    % 标注输入输出电压
    \draw (0,2) node[left]{$V_{\text{in}}(\omega)$};  % 输入电压
    \draw (6,2) node[right]{$V_{\text{out}}(\omega)$}; % 输出电压
\end{circuitikz}\end{center}
我们直接使用相量法得到其输入输出方程的频域形式,与电阻分压同理:
\[V_{out}(\omega)=\frac{1/(i\omega C)}{R+1/(i\omega C)}V_{in}(\omega)\]
因此频率响应为
\[
    H(\omega)=\frac{V_{out}(\omega)}{V_{in}(\omega)}=\frac{1}{1+i\omega RC}
\]
\end{example}
可见相量法可以省去列写微分方程和求傅里叶变换的过程,直接得到频率响应,这也是前文
中强调定义式$H(\omega)=Y(\omega)/X(\omega)$的原因。

我们曾在\ref{sec:convolution}中提到,滤波就是把信号中的某些成分剔除或大幅减少,相当于在频域乘以一个函数,在时域上的表现则是与这个函数的傅里叶逆变换进行卷积。从频率响应的视角来看,滤波器就是一个系统,理想滤波器的频率响应在某些频率处接近于0,而在另一些频率处接近于1。
\begin{definition}[滤波器的分类]
    低通滤波器是指频率响应$H(\omega)$在低频段($|\omega|<\omega_c$,$\omega_c$为截止频率)接近于1,而在高频段($|\omega|>\omega_c$)接近于0的滤波器;高通滤波器则相反,在高频段接近于1,在低频段接近于0;带通滤波器是指频率响应在某一频率区间内接近于1,而在该区间外接近于0的滤波器;带阻滤波器则相反,在某一频率区间内接近于0,而在该区间外接近于1的滤波器。
\end{definition}
\begin{figure}[H]
    \centering
    \includegraphics[width=0.8\textwidth]{Filter}
\end{figure}
偶函数在傅里叶变换下仍是偶函数,因此以上滤波器的单位脉冲响应也是偶函数,不具有因果性,因此是无法实现的。以它们作为理想滤波器,是因为实信号的幅度谱是偶函数,我们希望实信号经过系统还是实信号。实际设计滤波器时,只能逼近理想滤波器,例如巴特沃斯滤波器、切比雪夫滤波器等。

将频率响应$H(\omega)$看作一个复数,即$H(\omega)=|H(\omega)|e^{i\varphi(\omega)}$,则$|H(\omega)|$表示系统对频率为$\omega$的信号成分的放大倍数,$\varphi(\omega)$表示该频率成分经过系统后相位的变化。我们定义:
\begin{definition}[幅频特性和相频特性]
    $|H(\omega)|-\omega$称为系统的\textbf{幅频特性},$\varphi(\omega)-\omega$称为系统的\textbf{相频特性},这里$\varphi(\omega)$通常取主值,即$(-\pi,\pi]$内的值。
\end{definition}
一般而言,理想滤波器的单位脉冲响应是实函数,从而频率响应满足$H(\omega)=\overline{H(-\omega)}$,即幅频特性是偶函数,相频特性是奇函数。
\begin{example}
将激励信号$V_{in}(t)=\cos(\omega_0 t)$施加于例10中的RC电路,求输出信号$V_{out}(t)$\\
解:激励信号的傅里叶变换为
    \[V_{in}(\omega)=\pi(\delta_{\omega_0}+\delta_{-\omega_0})\]
因此输出信号的傅里叶变换为
\begin{align*}
    V_{out}(\omega)&=H(\omega)V_{in}(\omega)\\
    &=\frac{\pi}{1+i\omega RC}(\delta_{\omega_0}+\delta_{-\omega_0})\\
    &=\frac{\pi}{1+i\omega_0 RC}\delta_{\omega_0}+\frac{\pi}{1+i(-\omega_0) RC}\delta_{-\omega_0}
\end{align*}
反变换得到输出信号为
\begin{align*}
    r(t)&=\frac{1}{2\pi}\int_{-\infty}^{\infty}V_{out}(\omega)e^{i\omega t}\,d\omega\\
    &=\frac{1}{2\pi}\int_{-\infty}^{\infty}\left(\frac{\pi}{1+i\omega_0 RC}\delta_{\omega_0}+\frac{\pi}{1-i\omega_0 RC}\delta_{-\omega_0}\right)e^{i\omega t}\,d\omega\\
    &=\frac{1}{2}\left(\frac{1}{1+i\omega_0 RC}e^{i\omega_0 t}+\frac{1}{1-i\omega_0 RC}e^{-i\omega_0 t}\right)\\
    &=\text{Re}\left[\frac{1}{1+i\omega_0 RC}e^{i\omega_0 t}\right]\\
    &=\frac{1}{\sqrt{1+(\omega_0 RC)^2}}\cos(\omega_0 t-\arctan(\omega_0 RC))
\end{align*}

考虑$\omega_0$处$H(\omega)$的幅度和相位:
\begin{align*}
    &|H(\omega_0)|=\frac{1}{\sqrt{1+(\omega_0 RC)^2}}\\
    &\varphi(\omega_0)=-\arctan(\omega_0 RC)\\
    &r(t)=\left|H(\omega_0)\right|\cos\left(\omega_0 t+\varphi(\omega_0)\right)
\end{align*}
可以看到,$|H(\omega_0)|$为输出信号幅值的放大倍数,$\varphi(\omega_0)$为相位延迟,输出信号$V_{out}(t)$只用到了频域中的两个点$\pm\omega_0$,且由于系统函数偶对称,输出信号只显含$H(\omega_0)$的幅值的相位。这是因为$\cos(\omega_0 t)$在频域中只含有两个对称的频率成分。
\end{example}

将以上过程进一步抽象:
\begin{align*}
    \mathcal{F}\left[H[e^{i\omega_0 t}]\right]&=\mathcal{F} \left[h(t)*e^{i\omega_0 t}\right]\\
    &=H(\omega)\mathcal{F} \left[e^{i\omega_0 t}\right]\\
    &=2\pi H(\omega)\delta_{\omega_0}\\
    &=2\pi H(\omega_0)\delta_{\omega_0}
\end{align*}
因此\[r(t)=H\left[e^{i\omega_0 t}\right]=H(\omega_0)e^{i\omega_0 t}\]
\begin{definition}[特征函数]
    给定$\omega$,将函数$e^{i\omega t}$称为系统H的\textbf{特征函数}(eigenfunction),\textbf{特征值}(eigenvalue)为$H(\omega)$。
\end{definition}

在信号传输的过程中,有时希望信号经过一个系统,只有幅值发生变化,而波形不改变,例如小信号放大电路。我们定义:
\begin{definition}[失真]
波形改变的现象称为\textbf{失真}(distortion)。根据输出信号是否产生新的频率分量,将失真分为\textbf{线性失真}(linear distortion)和\textbf{非线性失真}(nonlinear distortion);线性失真中,根据引起失真的方式,又分为\textbf{幅度失真}(amplitude distortion)和\textbf{相位失真}(phase distortion)。幅度失真是指系统的幅频特性不是常数,即不同频率成分的放大倍数不同;相位失真是指系统的相频特性不是线性的,即不同频率成分的相位延迟不同。
\end{definition}
\begin{definition}[无失真传输系统]
如果一个系统对任意激励信号都不会产生失真,则称之为\textbf{无失真传输系统}(ideal transfer system)
\end{definition}

对于一个理想的\textbf{无失真传输系统},它只能对输入信号进行幅值的统一改变或相位的统一延迟,即\[r(t)=Ke(t-t_0)\]其中K为常数,$t_0$为常数延迟。由傅里叶变换的时移性质可知,无失真传输系统的系统函数为
\[H(\omega) = Ke^{-i\omega t_0}\]
可以看到,$|H(\omega)|=K,t_0=-\varphi(\omega)/\omega$,其幅频特性是常数,图像为一水平线;相频特性是线性函数,图像为过原点的直线,斜率$-t_0$对应系统造成的延时。

由于物理限制,我们不可能实现一个无失真传输系统,但只要所讨论的信号的频谱基本都集中在特定的频段当中,就可以近似认为在这个频段上满足$|H(\omega)|=K,\varphi(\omega)=-\omega t_0$的系统是无失真传输系统。

对于一般的线性时不变系统,可以将时延$t_0=-\varphi(\omega)/\omega$由常数推广为$\omega$的函数$\tau_p(\omega)$。此外,$\varphi(\omega)/\omega$是$\varphi(\omega)-\omega$图像的割线斜率,而$d\varphi(\omega)/d\omega$是该图像在$\omega$处的切线斜率,仿照电路分析中对于动态电阻和静态电阻的处理,我们定义相时延和群时延:
\begin{definition}[相时延和群时延]
\[\tau_p(\omega)=-\frac{\varphi(\omega)}{\omega}\]
描述了系统对不同频率成分造成时延的情况,称之为\textbf{相时延}(phase delay);将
\[\tau_g(\omega)=-\frac{d\varphi(\omega)}{d\omega}\]
称为\textbf{群时延}(group delay)或\textbf{包络延时}(envelope delay)。
\end{definition}

对于无失真传输系统,$\tau_p(\omega)=\tau_g(\omega)=t_0$。根据电路分析的经验,看到$\tau_p(\omega)=-\varphi(\omega)/\omega$而定义$\tau_g(\omega)=-d\varphi(\omega)/d\omega$是自然的,在处理非线性电阻的伏安特性曲线时也曾类似地定义动态电阻和静态电阻,但其物理意义并不十分明显,我们通过两个例子来理解它们的意义。

\begin{example}[相时延]
设输入信号为单频复指数信号$e^{i\omega_0 t}$,求频率响应为$H(\omega)$
的线性时不变系统的输出信号$r(t)$。\\
解:我们已经在前面得到:\[r(t)=H\left[e^{i\omega_0 t}\right]=H(\omega_0)e^{i\omega_0 t}\]
可以看到,输出信号仍然是单频复指数信号,其幅值被放大了$|H(\omega_0)|$倍,相位
延迟了$\varphi(\omega_0)$,折合到时间上,就得到相时延为$\tau_p(\omega_0)=-\varphi(\omega_0)/\omega_0$。
\end{example}

可以想象,能够分解成不同频率成分的信号,其各频率成分一般会产生不同的相时延,从而使得输出信号的波形发生变化,这就是相频特性不是线性函数导致的相位失真。

\begin{example}[群时延]
设输入信号为调幅信号(关于调幅信号,见\ref{sec:modulation}):
\[e(t)=\cos(\omega_m t)\cos(\omega_c t)\]
其中$\omega_c$为载波角频率,$\omega_m$为被调制信号的角频率,且$\omega_c\gg \omega_m$.系统H的单位脉冲响应$h(t)$为实信号,频率响应为$H(\omega)$。求$e(t)$进入系统所得的输出信号$r(t)$。\\
解:
可以看到该信号是一个被低频信号$\pm\cos(\omega_m t)$包络的高频载波信号$\cos(\omega_c t)$:
\begin{figure}[H]
    \centering
    \includegraphics[width=0.6\textwidth]{AM.jpeg}
\end{figure}
将输入信号用积化和差公式展开:
\begin{align*}
    e(t)&=\frac{1}{2}[\cos((\omega_c+\omega_m)t)+\cos((\omega_c-\omega_m)t)]
\end{align*}
输入信号的傅里叶变换为
\begin{align*}
    E(\omega)&=\frac{\pi}{2}\left[\delta_{\omega_c+\omega_m}+\delta_{-(\omega_c+\omega_m)}+\delta_{\omega_c-\omega_m}+\delta_{-(\omega_c-\omega_m)}\right]
\end{align*}
因此输出信号的傅里叶变换为
\begin{align*}
    R(\omega)=&H(\omega)E(\omega)\\
    =&\frac{\pi}{2} \left[H(\omega_c+\omega_m)\delta_{\omega_c+\omega_m}+H(-(\omega_c+\omega_m))\delta_{-(\omega_c+\omega_m)}\right.\\
    &+\left. H(\omega_c-\omega_m)\delta_{\omega_c-\omega_m}+H(-(\omega_c-\omega_m))\delta_{-(\omega_c-\omega_m)}\right]\\
    =&\frac{\pi}{2} \left[H(\omega_c+\omega_m)\delta_{\omega_c+\omega_m}+\overline{H(\omega_c+\omega_m)}\delta_{-(\omega_c+\omega_m)}\right.\\
    &+\left.H(\omega_c-\omega_m)\delta_{\omega_c-\omega_m}+\overline{H(\omega_c-\omega_m)}\delta_{-(\omega_c-\omega_m)}\right]
\end{align*}
反变换得到输出信号为
\begin{align*}
    r(t)=&\frac{1}{4}\left[H(\omega_c+\omega_m)e^{i(\omega_c+\omega_m)t}+\overline{H(\omega_c+\omega_m)}e^{-i(\omega_c+\omega_m)t}\right.\\
    &+\left. H(\omega_c-\omega_m)e^{i(\omega_c-\omega_m)t}+\overline{H(\omega_c-\omega_m)}e^{-i(\omega_c-\omega_m)t}\right]\\
    =&\frac{1}{2}\text{Re}\ \left[H(\omega_c+\omega_m)e^{i(\omega_c+\omega_m)t}+H(\omega_c-\omega_m)e^{i(\omega_c-\omega_m)t}\right]\\
    =&\frac{1}{2}|H(\omega_c+\omega_m)|\cos((\omega_c+\omega_m)t+\varphi(\omega_c+\omega_m))\\
    &+\frac{1}{2}\left|H(\omega_c-\omega_m)\right|\cos\left((\omega_c-\omega_m)t+\varphi(\omega_c-\omega_m)\right)
\end{align*}
由于$\omega_c\gg \omega_m$,可以对$|H(\omega)|$和$\varphi(\omega)$在$\omega_c$处做泰勒展开:
\begin{align*}
    |H(\omega_c+\omega)|&\approx |H(\omega_c)|+\left.\frac{d|H(\omega)|}{d\omega}\right|_{\omega=\omega_c}\omega\\
    \varphi(\omega_c+ \omega)&\approx \varphi(\omega_c)+\left.\frac{d\varphi(\omega)}{d\omega}\right|_{\omega=\omega_c}\omega
\end{align*}
将其带入$r(t)$中,得到
\begin{align*}
    r(t)=&\frac{1}{2}\left[|H(\omega_c)|+\left.\frac{d|H(\omega)|}{d\omega}\right|_{\omega=\omega_c}\omega_m\right]\cos\left((\omega_c+\omega_m)t+\varphi(\omega_c)+\left.\frac{d\varphi(\omega)}{d\omega}\right|_{\omega=\omega_c}\omega_m\right)\\
    &+\frac{1}{2}\left[|H(\omega_c)|-\left.\frac{d|H(\omega)|}{d\omega}\right|_{\omega=\omega_c}\omega_m\right]\cos\left((\omega_c-\omega_m)t+\varphi(\omega_c)-\left.\frac{d\varphi(\omega)}{d\omega}\right|_{\omega=\omega_c}\omega_m\right)\\
    \approx&|H(\omega_c)|\left[\cos\left((\omega_c+\omega_m)t+\varphi(\omega_c)+\left.\frac{d\varphi(\omega)}{d\omega}\right|_{\omega=\omega_c}\omega_m\right)\right.\\
    &\qquad\qquad\left.+\cos\left((\omega_c-\omega_m)t+\varphi(\omega_c)-\left.\frac{d\varphi(\omega)}{d\omega}\right|_{\omega=\omega_c}\omega_m\right)\right]\\
    =&|H(\omega_c)|\cos(\omega_c t+\varphi(\omega_c))\cos\left(\omega_m\left[t+\left.\frac{d\varphi(\omega)}{d\omega}\right|_{\omega=\omega_c}\right]\right)
\end{align*}
对比输入信号\[e(t)=\cos(\omega_m t)\cos(\omega_c t)\]
可以看到,输出信号的幅度大约变为$|H(\omega_c)|$倍,载波信号的相位延迟了$\varphi(\omega_c)$,
调制信号的相位延迟了$\omega_m\left.\frac{d\varphi(\omega)}{d\omega}\right|_{\omega=\omega_c}$,
折合到时间上,就是群时延$\tau_g(\omega_c)=-\left.\frac{d\varphi(\omega)}{d\omega}\right|_{\omega=\omega_c}$。
因此,群时延描述了系统对调制信号的包络造成的延时。从信号的图像上来看,高频的载波
信号的延时是不会造成很明显的变化的,而低频的调制信号的时延则会明显地反映在图像上
,就像一个“信号群”在传播过程中整体产生了一段时延,群时延的名称就由此而来。
\end{example}
\begin{definition}[*色散介质]
    物理上,研究波在介质中的传播时,将波速与频率无关的介质称为\textbf{无色散介质}(non-dispersive medium),将波速与频率有关的介质称为\textbf{色散介质}(dispersive medium)。在无色散介质中,波的相时延和群时延相等,即无失真传输;在色散介质中,波的相时延和群时延不相等,这种波速差往往会导致波形展宽,发生相位失真。
\end{definition}
典型的色散现象如一束白光通过三棱镜时,不同频率的光波由于波速不同而发生空间分离,形成彩虹色带。在无线电通信中,电离层对不同频率的电磁波有不同的折射率,导致信号传播速度不同,从而引起信号失真。

为了讨论方便,我们假设波动是一维的,$x$处的质点在时刻$t$的位置为$y(x,t)$,波速为$u$,$y(x,t)$满足的动力学方程称为\textbf{波动方程},具有如下统一形式:
\[\frac{\partial^2 y}{\partial x^2}=\frac{1}{u^2}\frac{\partial^2 y}{\partial t^2}\]
波动方程的解具有形式$y(x,t)=f(x-ut)+g(x+ut)$,其中$f$和$g$分别表示向x轴正方向和负方向传播的波。
\begin{figure}[H]
    \centering
    \includegraphics[width=0.8\textwidth]{sesan}
    \caption{波通过色散介质时波形展宽示意图}
\end{figure}
特别地,对于简谐波,有
\[y(x,t)=A\cos\omega\left(t-\frac{x}{u}\right)=A\cos\left(\frac{2\pi t}{T}-\frac{2\pi x}{\lambda}\right)\]
引入\textbf{波数}$k=\frac{2\pi}{\lambda}$,则简谐波可以表示为
\[y(x,t)=A\cos(\omega t - k x)\]
这里波数并不是指波的数目,而是一个将空间距离折算为相位延迟的比例系数,也可以将它理解为波的空间角频率:对简谐波$y(x,t)=A\cos(\omega t-kx)$的空间变量做傅里叶变换,将得到空间角频率仅有$\pm k$的成分,对三维空间中的平面波也有类似的结论,见\ref{sec:Multi_Fourier}。我们将宗量$\omega t-kx$称为波的\textbf{相位}(phase),考察相位的传播速度,即令$\omega t-kx=const$(例如,振幅最大值处恒有$\omega t-kx=0$),得到
\[\frac{dx}{dt}=\frac{\omega}{k}=u\]
这个速度称为\textbf{相速度}(phase velocity),它描述了波的相位传播的速度,可以看到相速度正是波速$u$。

借助傅里叶级数或傅里叶变换的理论,可以认为任意的波形都是若干不同频率的简谐波的叠加。

下面,我们将长度为$L$的均匀介质看作一个系统,设波$y(x,t)$沿$x$轴正向传播,将$x=0$处的波形作为输入信号$e(t)=y(0,t)$,$x=L$处的波形作为输出信号$r(t)=y(L,t)$。对于无色散介质,当输入为角频率为$\omega$的正弦波$\cos(\omega t)$时,有
\[r(t)=y(L,t)=e\left(t - \frac{L}{u}\right)=\cos\left(\omega \left(t-\frac{L}{u}\right)\right)\]
因此可以对给定的$\omega$求出频率响应为
\[H(\omega)=e^{-i\omega \frac{L}{u}}=e^{-ikL},k=\frac{\omega}{u}\]
可以看出,$k$实际上是关于$\omega$的函数$k(\omega)=\omega/u$,但在研究单频正弦波时看不出这一点。
\begin{definition}[*色散关系]
    介质中波数$k$与角频率$\omega$之间的函数关系$k(\omega)$称为\textbf{色散关系}(dispersion relation)。
\end{definition}

一般地,设初始波形为$f(x)$,波速为$u$,可以确定$x=0$处的激励信号$e(t)=f(-ut)$。它的不同频率成分通过均匀色散介质构成的系统时,只产生相位失真,因此频率响应可以表示为
\[H(\omega)=e^{-ik(\omega)L}\]
这样就得到一般情况下的波速的定义,从而得到相速度的定义,还可以仿照动态电阻和群时延,定义\textbf{群速度}(group velocity):
\begin{definition}[*相速度和群速度]
    \quad\\
    相速度为\[v_p(\omega)=\frac{\omega}{k(\omega)}\]
    群速度为\[v_g(\omega)=\frac{d\omega}{dk}=\frac{1}{k'(\omega)}\]
    $v_p>v_g$时称为\textbf{正常色散}(normal dispersion),$v_p<v_g$时称为\textbf{反常色散}(anomalous dispersion)。
\end{definition}
另一方面,介质作为一个系统,其相时延和群时延分别为
\begin{align*}
    \tau_p(\omega)&=-\frac{\varphi(\omega)}{\omega}=\frac{k(\omega)L}{\omega}=\frac{L}{v_p(\omega)}\\
    \tau_g(\omega)&=-\frac{d\varphi(\omega)}{d\omega}=\frac{dk(\omega)}{d\omega}L=\frac{L}{v_g(\omega)}
\end{align*}
即\[\tau_p(\omega)v_p(\omega)=\tau_g(\omega)v_g(\omega)=L\]

\begin{definition}*
    波包即脉冲波形,\textbf{波包中心}是指脉冲波形中振幅最大的位置。经验上,用波包中心移动的速度来表征波包的移动速度,并称之为\textbf{波包速度},这也是能量和信息在介质中传播的速度。
\end{definition}
\begin{proposition}*
    对窄带光脉冲,其波包速度等于群速度。
\end{proposition}
\begin{proof}
设窄带光脉冲的中心频率为$\omega_0$,即其振幅谱$\hat{A}(\omega)$集中在$\omega_0$附近,做傅里叶反变换,得到的时域信号为$x=0$处经过的波形$A(t)$。对各频率成分引入相位因子$e^{-ik(\omega)x}$来表征波的传输,得到$x$处的波形为
\[\psi(x,t)=\frac{1}{2\pi}\int_{-\infty}^{\infty}\hat{A}(\omega)e^{i(\omega t-k(\omega)x)}\,d\omega\]
对$k(\omega)$在$\omega_0$处做泰勒展开:
\[k(\omega)\approx k(\omega_0)+k'(\omega_0)(\omega-\omega_0)\]
将其代入$\psi(x,t)$中,得到
\begin{align*}
    \psi(x,t)&=\frac{1}{2\pi}\int_{-\infty}^{\infty}\hat{A}(\omega)e^{i(\omega t-k(\omega_0)x-k'(\omega_0)(\omega-\omega_0)x)}\,d\omega\\
    &=\frac{1}{2\pi}e^{i(\omega_0 t-k(\omega_0)x)}\int_{-\infty}^{\infty}\hat{A}(\omega)e^{i(\omega - \omega_0)(t - k'(\omega_0)x)}\,d\omega\\
    &=\frac{1}{2\pi}e^{i(\omega_0 t-k(\omega_0)x)}\int_{-\infty}^{\infty}\hat{A}(\omega+\omega_0)e^{i\omega(t - k'(\omega_0)x)}\,d\omega&(\omega\to\omega-\omega_0)\\
    &=e^{i(\omega_0 t-k(\omega_0)x)}\mathcal{F}^{-1}[\hat{A}(\omega+\omega_0)]\left(t - k'(\omega_0)x\right)\\
    &=e^{i(\omega_0 t-k(\omega_0)x)}A\left(t - k'(\omega_0)x\right)e^{-i\omega_0 (t- k'(\omega_0)x)}\\
    &=A\left(t - k'(\omega_0)x\right)e^{i(\omega_0 k'(\omega_0)-k(\omega_0))x}
\end{align*}
可以看到,波形的包络$A\left(t - k'(\omega_0)x\right)$以速度$v_g(\omega_0)=\frac{1}{k'(\omega_0)}$传播,因此波包速度等于群速度。
\end{proof}

\section{调制、解调与复用传输}\label{sec:modulation}

现代的通信方式有两种,\textbf{有线通信}(wired communication)和\textbf{无线通信}(wireless communication)。无线通信指用电磁波作为载体的\textbf{射频通信}(radio frequency communication),有线通信则是通过导线传输电信号,例如电话线、光纤等。尽管无线通信的信息传输效率比有线通信低一些,但无线通信配置灵活、建设速度快、通信可靠、维护方便且易于跨越复杂地形,应用广泛。

电磁波是通过天线发射和接受的。在电路分析的课程中,我们学过:当实际电路的几何尺寸远小于使用时最高工作频率所对应的波长时,可以忽略电磁场相互作用,由此抽象出的理想化原件模型被称为\textbf{集总参数元件}(lumped element);当实际电路的几何尺寸与使用时最高工作频率所对应的波长相当时,必须考虑电磁场相互作用,由此抽象出的理想化原件模型被称为\textbf{分布参数元件}(distributed element)。无线通信中的天线就是一个典型的分布参数元件,它将受金属束缚的电磁场(电压、电流)转化为辐射的电磁场,这个转化过程要求天线的尺寸必须与波长相当。以四分之一波长导体线为例,要发射$1kHz$的电磁波,则需要$75km$长的天线,这是难以实现的,因此无线通信通常使用$MHz$甚至$GHz$频率的电磁波作为载体。

将低频基带信号装载到高频信号上的过程,称为\textbf{调制}(modulation),此时高频信号被称为\textbf{载波}(carrier),例如用正弦信号作为载波,可以将基带信号装载到载波的幅度、频率、相位上,对应的调制方式称为\textbf{调幅}(AM, amplitude modulation)、\textbf{调频}(FM, frequency modulation)和\textbf{调相}(PM, phase modulation)。相应地,在信号接收端,从高频的已调信号还原出低频的基带信号的过程,称为\textbf{解调}(demodulation)。
\begin{definition}[调频、调幅、调相]
    设基带信号为$m(t)$,载波信号为\[c(t)=A_c \cos(\omega_c t+\varphi_c)\]
    其中$A_c,\omega_c,\varphi_c$分别为载波的幅值、角频率和初相位,则:
    \begin{itemize}
        \item 调幅信号为\[v_{AM}(t)=[1+k_{AM} m(t)]A_c \cos(\omega_c t+\varphi_c)\]其中$k_{AM}$为调幅灵敏度;
        \item 调频信号为\[v_{FM}(t)=A_c \cos\left(\omega_c t+k_{FM}\int_{0}^{t}m(\tau)\,d\tau+\varphi_c\right)\]其中$k_{FM}$为调频灵敏度;
        \item 调相信号为\[v_{PM}(t)=A_c \cos\left(\omega_c t+k_{PM} m(t)+\varphi_c\right)\]其中$k_{PM}$为调相灵敏度。
    \end{itemize}
\end{definition}
也可以在载波的幅度和相位上同时调制,例如\textbf{正交振幅调制}(QAM, quadrature amplitude modulation)。

下面来解释为什么这三种调制方式的数学表示形式如上。\\
1.\textbf{幅度调制}:已调信号的幅度为\[1+k_{AM}m(t)\]
我们曾用这种调制方式作为\ref{sec:freq_response}中的例子来解释群时延的概念,事实上,当时所使用的是\textbf{双边带调幅}(DSB),一般而言不能略去1倍的载波信号这一项,即
\[v_{AM}(t)\neq k_{AM} m(t)A_c \cos(\omega_c t+\varphi_c)=v_{DSB}(t)\]
这是因为调幅信号通常通过包络检测器直接恢复,虽然有部分功率用于传输载波,但这简化了解调过程。此外,一般被调信号$m(t)$会预先经过衰减器以确保$|k_{AM}m(t)|\leq 1$,这样,$1+km(t)\geq 0$,包络不出现反转,即不会出现过度调制问题,包络检测器才能正常解调。\\
2.\textbf{频率调制}:$v_{FM}(t)$的傅里叶变换较为复杂,但其直观易于理解,这种调制方式下的\textbf{瞬时角频率}为:\begin{align*}
    \frac{d}{dt}\left(\omega_c t+k_{FM}\int_{0}^{t}m(\tau)\,d\tau+\varphi_c\right)=\omega_c+k_{FM}m(t)
\end{align*}
3.\textbf{相位调制}:\textbf{瞬时相位}为\[\omega_c t+k_{PM} m(t)+\varphi_c\]

我们以双边带调制为例给出一个简单的调制解调过程:
\begin{example}
设基带信号为$m(t)=\cos(\omega_m t)$,载波信号为$c(t)=\cos(\omega_c t)$,其中$\omega_c\gg \omega_m$。设计一个简单的解调器以还原出$m(t)$。\\
解:双边带调幅信号为\[v_{DSB}(t)=m(t)c(t)=\cos(\omega_m t)\cos(\omega_c t)\]
我们已经在上一节对它做过频域分析,得到其频谱为
\[V_{DSB}(\omega)=\frac{\pi}{2}\left[\delta_{\omega_c+\omega_m}+\delta_{-(\omega_c+\omega_m)}+\delta_{\omega_c-\omega_m}+\delta_{-(\omega_c-\omega_m)}\right]\]
假设传输过程没有发生失真,接收端得到的信号仍为$v_{DSB}(t)$。为了还原出$m(t)$,我们可以将$v_{DSB}(t)$与载波$c(t)$相乘,得到
\begin{align*}
    y(t)&=v_{DSB}(t)c(t)\\
    &=\cos(\omega_m t)\cos^2(\omega_c t)\\
    &=\frac{1}{2}\cos(\omega_m t)(1+\cos(2\omega_c t))\\
    &=\frac{1}{2}\cos(\omega_m t)+\frac{1}{2}\cos(\omega_m t)\cos(2\omega_c t)
\end{align*}
其傅里叶变换为
\begin{align*}
    Y(\omega)&=\frac{\pi}{2}[\delta_{\omega_m}+\delta_{-\omega_m}]\\
    &+\frac{\pi}{4}[\delta_{2\omega_c+\omega_m}+\delta_{-(2\omega_c+\omega_m)}+\delta_{2\omega_c-\omega_m}+\delta_{-(2\omega_c-\omega_m)}]
\end{align*}
只要再通过一个截止频率在$(\omega_m,\ 2\omega_c-\omega_m)$之间的低通滤波器,就可以滤除高频成分,得到$\frac{1}{2}\cos(\omega_m t)$,从而还原出基带信号$m(t)$。
\end{example}
时域上,这个过程相当于\begin{align*}
    y(t)&=v_{DSB}(t)c(t)\\
    &=m(t)c^2(t)\\
    &=m(t)\frac{1+\cos(2\omega_c t)}{2}\\
    &=\frac{m(t)}{2}+\frac{m(t)}{2}\cos(2\omega_c t)
\end{align*}
而$\frac{m(t)}{2}\cos(2\omega_c t)$被低通滤波器滤除。

事实上,对于带限信号(尤其是窄带信号),我们都可以采用这种调制解调的方法进行传输,这是由于三角函数具有\textbf{频谱搬移}(spectrum shifting)的性质:\begin{align*}
    \mathcal{F} \left[\cos(\omega_c t)\right]&=\pi\left[\delta_{\omega_c}+\delta_{-\omega_c}\right]\\
    \mathcal{F}\left[x(t)\cos(\omega_c t)\right]&=\frac{1}{2\pi}X(\omega)*\mathcal{F}\left[\cos(\omega_c t)\right]\\
    &=\frac{1}{2}\left[X(\omega-\omega_c)+X(\omega+\omega_c)\right]
\end{align*}
经过两次与载波相乘,一半的信号回到原来的位置,另一半被搬移到$2\omega_c$处,可见只要信号的带宽小于$\omega_c$,就可以通过选择合适的低通滤波器还原出基带信号。

利用这种频谱搬移特性,还可以实现信号的\textbf{频分复用}(FDM, frequency division multiplexing):将多个基带信号分别调制到不同的载波频率上进行传输,在接收端先使用不同的带通滤波器分别解调,再与载波信号相乘,最后通过低通滤波器,即可还原出各个基带信号。

可以看到,复用传输能够有效提高信道的利用率。另一种常用的复用传输方式是\textbf{时分复用}(TDM, time division multiplexing):将多个基带信号在时间上交替传输,在接收端根据时间顺序还原出各个基带信号。这种方式适用于数字信号传输。此外,复用方式还包括\textbf{码分复用}(CDM, code division multiplexing):为每个基带信号分配一个独特的编码序列,所有信号同时传输,在接收端通过相关解码技术还原出各个基带信号。这种方式常用于无线通信系统,如CDMA技术;\textbf{波分复用}(WDM, wavelength division multiplexing):主要用于光纤通信中,通过不同波长的光信号同时传输多个基带信号,在接收端通过光学滤波器分离出各个波长的信号进行解调,实质上还是频分复用,只是频率较高;\textbf{空分复用}(SDM, space division multiplexing):利用空间上的不同路径同时传输多个基带信号,在接收端通过空间滤波技术还原出各个基带信号,常用于多输入多输出(MIMO)系统中;\textbf{偏振复用}(PDM, polarization division multiplexing):利用光信号的不同偏振状态同时传输多个基带信号,在接收端通过偏振分离技术还原出各个基带信号,常用于光纤通信中以提高传输容量。

\chapter{拉普拉斯变换与系统的复频域分析}

\section{复变函数概要}\label{sec:Complex}

本小节将串写拉普拉斯变换所需的复变函数论中的一些基本概念和重要结果。实际上,只需要复变函数中一些最基础的概念即可建立起拉普拉斯变换理论的大部分内容。因此,即便是未学过复变函数的读者,也可通过本小节快速把握拉普拉斯变换中所需的复变函数的内容。

我们总是用$z$表示复数$x+iy$。考虑复系数双变元多项式集$\mathbb{C}[x,y]$,通过做替换$x=\frac{z+\overline{z}}{2},y=\frac{z-\overline{z}}{2i}$,可以将$\mathbb{C}[x,y]$中的多项式$f(x,y)$表示为$g(z,\overline{z})$。在一些情况下,$g(z,\overline{z})$不显含$\overline{z}$,例如$f(x,y)=x+iy$,$g(z,\overline{z})=z$。对于$f(x,y)$,我们可以用导数$\frac{\partial f}{\partial x},\frac{\partial f}{\partial y}$来描述它对$x,y$的依赖关系,同理,为了表述多项式对$z,\overline{z}$的依赖关系,我们定义:\begin{definition}[形式导数]
    设$f=u+iv\in C^1(\mathbb{C})$,$u,v$为复数域上的实值函数,则
    \begin{align*}
        \frac{\partial f}{\partial z}=\frac{1}{2}\left(\frac{\partial}{\partial x}-i\frac{\partial}{\partial y}\right)f=\frac{1}{2}\left(\frac{\partial u}{\partial x}+\frac{\partial v}{\partial y}\right)+\frac{i}{2}\left(\frac{\partial v}{\partial x}-\frac{\partial u}{\partial y}\right)\\
        \frac{\partial f}{\partial \overline{z}}=\frac{1}{2}\left(\frac{\partial}{\partial x}+i\frac{\partial}{\partial y}\right)f=\frac{1}{2}\left(\frac{\partial u}{\partial x}-\frac{\partial v}{\partial y}\right)-\frac{i}{2}\left(\frac{\partial v}{\partial x}+\frac{\partial u}{\partial y}\right)
    \end{align*}
\end{definition}
注意$f$未必是多项式,多项式只是为了给我们提供一个直观的动机和理解方式。

这种定义可以视作复合函数求导的结果,即\[\frac{\partial f}{\partial z}=\frac{\partial f}{\partial x}\frac{\partial x}{\partial z}+\frac{\partial f}{\partial y}\frac{\partial y}{\partial z},\frac{\partial f}{\partial \overline{z}}=\frac{\partial f}{\partial x}\frac{\partial x}{\partial \overline{z}}+\frac{\partial f}{\partial y}\frac{\partial y}{\partial \overline{z}}\]

读者可以自行验证,在这样的定义下有:\begin{claim}
    \begin{align*}
        \frac{\partial z}{\partial z}=\frac{\partial \overline{z}}{\partial \overline{z}}=1,\\
        \frac{\partial z}{\partial \overline{z}}=\frac{\partial \overline{z}}{\partial z}=0
    \end{align*}
\end{claim}

这种形式导数在其他方面也很像普通的导数,甚至可以说$z,\overline{z}$构成了一组新的坐标。例如:
\begin{claim}[莱布尼兹法则]
    \[\frac{\partial}{\partial z}(fg)=f\frac{\partial}{\partial z}g+g\frac{\partial}{\partial z}f\]
    \[\frac{\partial}{\partial \overline{z}}(fg)=f\frac{\partial}{\partial \overline{z}}g+g\frac{\partial}{\partial \overline{z}}f\]
\end{claim}
\begin{claim}[链式法则]
    设$f,g\in C^1$,$f\circ g$有定义并记$w=g(z)$,则
    \begin{align*}
        \frac{\partial}{\partial z}(f\circ g)=\frac{\partial f}{\partial w}\frac{\partial g}{\partial z}+\frac{\partial f}{\partial \overline{w}}\frac{\partial\overline{g}}{\partial z}\\
        \frac{\partial}{\partial \overline{z}}(f\circ g)=\frac{\partial f}{\partial w}\frac{\partial g}{\partial \overline{z}}+\frac{\partial f}{\partial \overline{w}}\frac{\partial\overline{g}}{\partial \overline{z}}
    \end{align*}
\end{claim}
以上结论的证明略去,它们只是多元微分学的小练习。
\begin{remark}*
    对于了解微分形式的语言的读者,可以更好地理解$z,\overline{z}$作为坐标的意义以及形式导数的意义。定义1-形式
    $$dz=dx+idy,d\overline{z}=dx-idy$$
    则$dz\wedge d\overline{z}=-2i dx\wedge dy$。定义算子
    $$\partial f=\frac{\partial f}{\partial z}dz,\overline{\partial} f=\frac{\partial f}{\partial \overline{z}}d\overline{z}$$
    作用于1-形式时,有
    $$\partial(f(z,\overline(z))dz+g(z,\overline{z})d\overline{z})=\frac{\partial g}{\partial z}dz\wedge d\overline{z},\overline{\partial}(f(z,\overline{z})dz+g(z,\overline{z})d\overline{z})=\frac{\partial f}{\partial \overline{z}}d\overline{z}\wedge dz$$
    可以验证:$$d=\partial+\overline{\partial},\partial^2=\overline{\partial}^2=0,\partial\overline{\partial}+\overline{\partial}\partial=0$$
\end{remark}
\begin{definition}[全纯函数]
    设开集$U\in\mathbb{C}$,如果$f=u+iv\in C^1(U)$满足$\frac{\partial f}{\partial \overline{z}}=0,\forall z\in U$,则称$f$为$U$上的\textbf{全纯函数}(holomorphic function)。这等价于$u,v$在$U$上满足\textbf{柯西-黎曼方程组}(Cauchy-Riemann equations):
    \begin{align*}
        \begin{cases}
        \frac{\partial u}{\partial x}=\frac{\partial v}{\partial y}\\
        \frac{\partial v}{\partial x}=-\frac{\partial u}{\partial y}
        \end{cases},\text{或}\frac{\partial f}{\partial y}= i\frac{\partial f}{\partial x}
    \end{align*}
\end{definition}
\begin{definition}[整函数]
    如果$f$在整个复平面$\mathbb{C}$上全纯,则称$f$为\textbf{整函数}(entire function)。
\end{definition}

全纯等价于极限\[\lim_{\zeta\to z}\frac{f(\zeta)-f(z)}{\zeta-z}\]
存在,即全纯等价于复可导性,并且由此定义的导数$f'(z)=\frac{\partial f}{\partial z}$;全纯函数还是无限阶可导的,其各阶导数也是全纯函数,并且全纯性等价于解析性,即其泰勒级数在收敛域内一定收敛于这个函数本身:
\begin{definition}[复变函数的泰勒级数展开]
    设$f$为开集$U\in\mathbb{C}$上的全纯函数,$D(a,R)\in U$,则$\forall z\in D(a,R)$,有
    \[f(z)=\sum_{n=0}^{\infty}\frac{f^{(n)}(a)}{n!}(z-a)^n\]
    其中$f^{(n)}(a)$表示$f$在$a$处的第n阶导数。
\end{definition}

以上定义其实隐含了一个很强的结论:只要$D(a,r)\in U$,则$f$在$a$处的泰勒级数在$D(a,r)$上收敛于$f$,即泰勒级数的收敛半径恰为$f$在$a$处到$U$的边界的距离。
\begin{example}
    函数$f(z)=\frac{1}{1-z}$在$D(0,1)$上全纯,可以直接判断它在$z=0$处的泰勒级数$\sum_{n=0}^{\infty}z^n$的收敛半径为1,恰好是0到不全纯的点1的距离。
\end{example}

对实变函数,即便无限阶可导,泰勒级数也未必收敛于函数本身,即未必解析:
\begin{example}
    设$f(x)=\begin{cases}
    e^{-1/x}&\text{,if\ }x> 0\\
    0&\text{,if\ }x\leq 0
    \end{cases}$,$f(x)$在$x=0$处各阶导数都是0,泰勒展开式也为0,收敛域为$\mathbb{R}$,但$f(x)\neq 0,x>0$,它不是解析函数。
\end{example}

数学分析中,我们学过幂级数收敛半径的求法,复变函数的泰勒级数也有类似的结论:
\begin{theorem}[柯西-阿达马公式]
    复幂级数\[\sum_{n=0}^{\infty}c_n(z-a)^n\]
    的\textbf{收敛域}(region of convergence,ROC)为$D(a,R)$,其中收敛半径$R$由下式给出:
    \[R=\left(\varlimsup_{n\to\infty}|c_n|^{1/n}\right)^{-1}\]
    这里的极限是广义极限,即允许$R=0$(仅在中心点收敛)或$R=\infty$(级数在整个复平面上收敛)。
\end{theorem}
在复幂级数的收敛域上,级数是绝对收敛且内闭一致收敛的。其本质还是与几何级数做比较,收敛半径与实幂级数的收敛半径求法一致,实幂级数的收敛半径可看作是复幂级数的收敛半径在实轴上的投影。使用柯西-阿达马公式求出的收敛半径与前面提到的“到不全纯区域的距离”总是一致的,读者可以对例5.1.1.自行验算。

在数学分析中,我们曾学过曲线积分,它对于复平面上的复变函数也有意义:
\begin{definition}[复线积分]
    设开集$U\in\mathbb{C},f\in C^1(U),\gamma:[a,b]\to U$为$\mathbb{C}$中的$C^1$曲线,则
    \begin{align*}
        \int_{\gamma}f(z)\,dz=\int_a^bf(\gamma(t))\gamma'(t)\,dt
    \end{align*}
\end{definition}

与实变函数的曲线积分一致,可以验证此复线积分不依赖于参数$t$的选取,从而是良定义的,并且有:\begin{claim}
    \[f(\gamma(b))-f(\gamma(a))=\int_{a}^{b}(f\circ \gamma)'(t)\,dt=\int_{\gamma}\frac{\partial f}{\partial z}\,dz\]
\end{claim}
\begin{remark}
    前一个等号直接得自微积分基本定理,后一个等号才是需要证明的,读者可以带入形式导数的定义验算。
\end{remark}
实际上还能将曲线推广至分段一阶可导,即$\gamma\in PS[a,b]$,只要分区间使用微积分基本定理,就能得到与上面一致的结论,但为了叙述的简洁性,下面不再提及这一点。

后续内容中,我们用符号$D(z_0,r)$表示复平面上以$z_0=x_0+iy_0$为圆心,$r$为半径的开圆盘,即$D(z_0,r)=\{x+iy\in\mathbb{C}\big| (x-x_0)^2+(y-y_0)^2<r\}$;用符号$\mathcal{A} _a(r,R)=\{z\in\mathbb{C}|r<|z-a|<R\}$表示以$a$为圆心,内半径$r$,外半径$R$的开圆环。
\begin{theorem}[柯西积分公式]
    设开集$U\in\mathbb{C}$,$f\in C^1(U),D(z_0,r)\in U,\gamma:[a,b]\to \mathbb{C}$为$D(z_0,r)$中的$C^1$闭曲线,则
    \[\forall z\in D(z_0,r),f(z)=\frac{1}{2\pi i}\oint_{\gamma}\frac{f(\zeta)}{\zeta-z}\,d\zeta\]
\end{theorem}
\begin{theorem}[柯西积分定理]
    设$f$是单连通开集$U\in\mathbb{C}$上的全纯函数,$\gamma(t)\in U$,则
    \[\oint_{\gamma}f(z)\,dz=0\]
\end{theorem}
\begin{remark}
    \textbf{单连通}(single connected)是一个拓扑学概念,直观上讲,它表示$U$道路连通且“没有洞”,例如单位圆是单连通的,而圆环$\mathcal{A}_0(1,2)$不是单连通的。
\end{remark}
\begin{example}
设$f(z)=1/z$,则$f$在$\mathbb{C}\backslash\{0\}$上全纯,但$\mathbb{C}\backslash\{0\}$不是单连通的,它在单位圆$\gamma(t)=e^{it},t\in[0,2\pi]$上积分为\begin{align*}
    \oint_{\gamma}f(z)\,dz&=\int_0^{2\pi}f(\gamma(t))\gamma'(t)\,dt\\
    &=\int_0^{2\pi}\frac{1}{e^{it}}ie^{it}\,dt\\
    &=\int_0^{2\pi}i\,dt=2\pi i\neq 0
\end{align*}
\end{example}

\begin{remark}*
    根据Morera定理,单连通开集上的任意复线积分为0,则函数也是全纯函数,因此全纯函数与单连通开集上的复线积分为0是等价的。这种等价性表明可以用$\mathbb{R}^2$上的势场来理解全纯函数,其曲线积分具有路径无关性。根据数学分析中处理路径无关性的经验,我们可以分段调整积分路径,避开不全纯的区域。例如对$f(z)=1/z$在曲线$\gamma(t)=e^{it},t\in[0,2\pi]$上积分,尽管$f$在包含原点的区域上不全纯,但可以将$\gamma$分成两段,分别在不包含原点的半圆上积分并调整积分路径(例如调整到$\gamma_2(t)=2e^{it},t\in[0,2\pi]$)。理解这一点将有利于理解后面的留数定理中对积分曲线的要求。
\end{remark}

\begin{remark}*
    对于了解微分形式的语言的读者,可以从柯西-黎曼方程看出全纯函数$f=u+iv$相当于闭形式$u(x,y)dx+v(x,y)dy$。在单连通开集上,则闭形式也是恰当形式,1-形式$udx+vdy$相当于无旋无源场,且$udx+vdy$是某实函数$\varphi$的外导数,即$d\varphi=\frac{\partial \varphi}{\partial x}dx+\frac{\partial \varphi}{\partial y}dy=udx+vdy$;另外,根据另一柯西-黎曼方程$-v(x,y)dx+u(x,y)dy$也是闭形式、恰当形式,是某实函数$\psi$的外导数,即$d\psi=-v\,dx+u\,dy$。由于$\varphi,\psi$满足柯西-黎曼方程,可以定义全纯函数$g=\varphi+i\psi$,将$g$称为向量场$f$的\textbf{复势函数}(complex potential function),$\frac{\partial g}{\partial z}=u-iv$与$f$互为共轭复变函数。如果将$d\psi$取为$v\,dx-u\,dy$,$g=\varphi+i\psi$,则$\frac{\partial g}{\partial z}=u+iv=f$,即单连通开集上的全纯函数$f$总是某个全纯函数$g$的导数。
\end{remark}

对于$f(z)=1/z$这样的函数,它已经是很简洁的分式形式了,但它在$z=0$处并不是全纯函数,因此不能在包含0的开集上展开成泰勒级数。我们将破坏了全纯性的这种点称为\textbf{孤立奇点}(isolated singular point),孤立表示它不构成奇点集的极限点。为了处理这种情况,我们引入\textbf{洛朗级数}:

\begin{definition}[洛朗级数]
    设$a\in\mathbb{C}$,形如
    \[\sum_{n=-\infty}^{\infty}c_n(z-a)^n\]
    的级数称为以$a$为中心的\textbf{洛朗级数},其收敛域记为ROC。

    将它的负幂次项部分视为$w=1/(z-a)$下的幂级数,则可用柯西-阿达马公式确定它的收敛域为圆环$\mathcal{A} _a(r,R)$,其中$r,R$分别称为洛朗级数的\textbf{外半径}(outer radius)和\textbf{内半径}(inner radius),满足:
    \[r=\varlimsup_{n\to\infty}|c_{-n}|^{1/n},R=\left(\varlimsup_{n\to\infty}|c_n|^{1/n}\right)^{-1}\]
    在$\text{ROC}=\mathcal{A} _a(r,R)$上,洛朗级数是绝对收敛且内闭一致收敛的。
\end{definition}
事实上,对任意的全纯函数$f:U\to\mathbb{C}$,及孤立奇点$a\in U$,$f$在$a$处存在且有唯一的洛朗级数展开式。

借助洛朗级数的一致收敛性,可以建立一个对柯西积分公式的直观认识:记$f_n(z)=1/z^n,n\in\mathbb{Z}$,则在逆时针绕0一周的闭曲线$\gamma$上,有
\[\oint_{\gamma}f_n(z)\,dz=\begin{cases}0&n\neq 1\\2\pi i&n=1\end{cases}\]
我们已经在前面说明了路径无关性,下面设$\gamma(t)=e^{it},t\in[0,2\pi]$,则
\begin{align*}
    \oint_{\gamma}f_n(z)\,dz&=\int_0^{2\pi}f_n(\gamma(t))\gamma'(t)\,dt\\
    &=\int_0^{2\pi}\frac{1}{(e^{it})^n} ie^{it}\,dt\\
    &=i\int_0^{2\pi}e^{i(1-n)t}\,dt\\
    &=\begin{cases}
        i\int_0^{2\pi}\,dt=2\pi i&n=1\\
        i\evalat{\left[\frac{e^{i(1-n)t}}{i(1-n)}\right]}{0}{2\pi}=0&n\neq 1
    \end{cases}
\end{align*}

为了验证柯西积分公式
\[f(z)=\frac{1}{2\pi i}\oint_{\gamma}\frac{f(\zeta)}{\zeta-z}\,d\zeta,\gamma:t\mapsto z+re^{it},t\in[0,2\pi]\]将$f$展开为$z$处的洛朗级数$f(\zeta)=\sum_{n=0}^{\infty}c_n (\zeta-z)^n$,由于洛朗级数在闭曲线$\gamma$上一致收敛,可以交换求和与积分的次序,得到
\begin{align*}
    \oint_{\gamma}\frac{f(\zeta)}{\zeta-z}\,dz&=\oint_{\gamma}\sum_{n=0}^{\infty}c_n (\zeta-z)^{n-1}\,d\zeta\\
    &=\sum_{n=0}^{\infty}c_n\oint_{\gamma}(\zeta-z)^{n-1}\,d\zeta\\
    &=2\pi i c_0=2\pi i f(z)
\end{align*}

多数情况下,洛朗级数的负幂次项并不真正构成级数,此时内半径必为0,即只要外半径$R>0$,洛朗级数就在展开点的去心邻域内收敛。例如$f(z)=1/z$在0处的展开式就是它本身。根据负幂次项是否存在、是否构成级数,我们可以将孤立奇点分为三类:
\begin{definition}[孤立奇点的分类]
    设$f$在开集$U\in\mathbb{C}$上全纯,$a$为$f$的孤立奇点,$f$在$a$处的洛朗级数展开式为
    \[f(z)=\sum_{n=-\infty}^{\infty}c_n(z-a)^n\]
    \begin{enumerate}
        \item 如果$c_n=0,\forall n<0$,则称$a$为$f$的\textbf{可去奇点}(removable singularity);
        \item 如果存在$m\in\mathbb{N}_+$使得$c_{-1}=c_{-2}=\cdots=c_{-m}=0,c_{-(m+1)}\neq 0$,则称$a$为$f$的\textbf{m阶极点}(pole of order m),$m$阶极点意味着$f(z)$在$a$处趋于无穷的速度与$1/(z-a)^m$同阶。
        其中$m=1$时,称$a$为$f$的\textbf{简单极点}(simple pole);
        \item 如果$c_n\neq 0$对无穷多个负整数$n$成立,则称$a$为$f$的\textbf{本性奇点}(essential singularity)。
    \end{enumerate}
\end{definition}
\begin{example}
$f(z)=e^{1/z}$在0处有本性奇点,$f(z)=1/\sin(\pi z)$在$z=n,n\in\mathbb{Z}$处有简单极点,$f(z)=\frac{\sin z}{z}$在0处有可去奇点。
\end{example}

类似地也可以根据泰勒级数或等价无穷小量定义全纯函数的若干阶的零点,不过这不属于我们讨论的范围。

柯西积分公式中,$\frac{f(\zeta)}{\zeta-z}$这一结构相当于为$f(\zeta)$添加了一个简单极点。\textbf{留数定理}(residue theorem)就是描述复线积分的值与被积函数在闭曲线内部的“$1/z$”系数关系的定理:
\begin{theorem}[留数定理]
    设$f$为开集$U\in\mathbb{C}$上的全纯函数,$a_1,a_2,\cdots,a_n$为$f$在$U$内的极点,$\gamma:[a,b]\to U$为包含所有$a_k$的$C^1$闭曲线,则
    \[\oint_{\gamma}f(z)\,dz=2\pi i\sum_{k=1}^{n}\text{Res}(f,a_k)\text{Ind}_{\gamma}(a_k)\]
    其中$\text{Res}(f,a_k)$称为$f$在$a_k$处的\textbf{留数}(residue),它等于$f$在$a_k$处洛朗级数展开式中$(z-a_k)^{-1}$项的系数;$\text{Ind}_{\gamma}(a_k)$称为$\gamma$绕点$a_k$的\textbf{环绕数}(winding number),它表示$\gamma$绕$a_k$转了多少圈。
\end{theorem}
注意计算环绕数时,以逆时针为正向,例如$\gamma(t)=e^{it},t\in[0,2\pi]$绕原点的环绕数为1,而$\gamma(t)=e^{-it},t\in[0,2\pi]$绕原点的环绕数为-1。希望了解环绕数严格定义的读者可以参考复变函数或拓扑学的教材。在留数定理的语境下又将积分路径称为\textbf{围道}(contour)。

要找到$f$在极点$a$处的留数的计算公式,可以将$f$视同其在$a$处的洛朗级数展开式:
\[f(z)=\sum_{n=-k}^{\infty}c_n(z-a)^n\]
这里假设$a$为$f$的k阶极点。在处理泰勒级数时,我们在$f(x)=\sum_{n=0}^{\infty}a_n x^n$中两边同时对$x$求导再令$x=0$,得到$f^{(m)}(0)=m!a_m$。类似地,在处理洛朗级数时,我们可以将其乘以$(z-a)^k$从而化归到泰勒级数:
\begin{align*}
    (z-a)^k f(z)=\sum_{n=0}^{\infty}c_{n-k}(z-a)^n
\end{align*}
\begin{align*}
    \frac{d^{k-1}}{dz^{k-1}}((z-a)^k f(z))=\sum_{n=0}^{\infty}c_{n-k}\frac{d^{(k-1)}}{dz^{(k-1)}}(z-a)^n=\sum_{n=k-1}^{\infty}c_{n-k}\frac{n!}{(n-k+1)!}(z-a)^{n-k+1}
\end{align*}
令$z\to a$,就得到留数的计算公式:\begin{proposition}
    设$f$在$a$处有k阶极点,则
    \[\text{Res}(f,a)=\frac{1}{(k-1)!}\lim_{z\to a}\frac{d^{(k-1)}}{dz^{(k-1)}}((z-a)^k f(z))\]
    特别地,对于简单极点,有
    \[\text{Res}(f,a)=\lim_{z\to a}(z-a)f(z)\]
\end{proposition}

我们使用留数定理来计算一个傅里叶变换作为例子。
\begin{example}
    计算$f(t)=\frac{1}{a^2+t^2}$的傅里叶变换$\mathcal{F} f(\omega)$。\\
    解:根据定义,$\mathcal{F} f(\omega)=\int_{-\infty}^{\infty}\frac{1}{a^2+t^2}e^{-i\omega t}\,dt$。为了使用留数定理计算该积分,我们将其推广为复变量上的积分,注意在$\mathbb{R}$上$z=\text{Re}\,z=t$:
    \[\mathcal{F} f(\omega)=\int_{-\infty}^{\infty}\frac{1}{a^2+z^2}e^{-i\omega z}\,dz\]
    设$f(z)=\frac{e^{-i\omega z}}{a^2+z^2}$,围道为:\begin{align*}
        &\gamma_1:t\mapsto t,t\in[-R,R]\\
        &\gamma_2:t\mapsto Re^{-it},t\in[0,\pi]
    \end{align*}
    则$\int_{\gamma_1}f(z)\,dz=\int_{-R}^{R}\frac{1}{a^2+t^2}e^{-i\omega t}\,dt$,当$R\to\infty$时,$\int_{\gamma_1}f(z)\,dz\to \mathcal{F} f(\omega)$。经验上,另一条趋向无穷远处的围道上的积分将变成0,我们来验证这一点:\begin{align*}
        \left|\int_{\gamma_2}f(z)\,dz\right|&=\left|-\int_0^{\pi}\frac{1}{a^2+R^2e^{-2it}}e^{i\omega Re^{it}}iRe^{-it}\,dt\right|\\
        &\leq \sup_{t\in[0,\pi]}\left|\frac{1}{a^2+R^2e^{2it}}e^{i\omega Re^{it}}iRe^{it}\right|\mu(\gamma_2)\\
        &\leq \frac{\pi Re^{-\omega R\sin t}}{R^2+a^2}\to 0&(R\to\infty)
    \end{align*}
    这里的$\mu(\gamma_2)$表示$\gamma_2$的长度。围道内仅有一个简单极点$-ai$,环绕数为-1,根据留数定理,有
    \begin{align*}
        \oint_{\gamma}f(z)\,dz&=-2\pi i\text{Res}(f,-ai)\\
        &=-2\pi i\lim_{z\to -ai}(z+ai)\frac{e^{-i\omega z}}{a^2+z^2}\\
        &=\frac{\pi}{a}e^{a\omega}
    \end{align*}
\end{example}

最后,我们引入\textbf{唯一延拓定理}(uniqueness theorem),它是以下定理的推论:
\begin{theorem}*
    设$U\in\mathbb{C}$为连通开集,$f$为$U$上的全纯函数,如果$f$的\textbf{零集}(zero set)$Z(f)=\{z\in U|f(z)=0\}$有$U$中的极限点,则$f(z)=0,\forall z\in U$。
\end{theorem}
\begin{corollary}*
    设$f,g$为连通开集$U\in\mathbb{C}$上的全纯函数,$S\subset U$,且$S$包含$U$中的极限点,如果$f(z)=g(z),\forall z\in S$,则$f(z)=g(z),\forall z\in U$。
\end{corollary}
这表明连通开集上的全纯函数环是整环。
\begin{corollary}[*唯一延拓定理]
    整函数可以由$\mathbb{R}$上的值唯一确定,即如果整函数$f,g$满足$f(x)=g(x),\forall x\in\mathbb{R}$,则$f(z)=g(z),\forall z\in\mathbb{C}$。
\end{corollary}

借助唯一延拓定理,可以将实变量上的函数延拓为复变量上的全纯函数,例如$\sin x,\cos x,e^x$都可以延拓为复变量上的全纯函数,它们的复幂级数与实幂级数一致,并且运算公式(如和差角公式、欧拉公式等)均可以推广到复变量上,因为在实轴上恒为0的函数在复平面上也恒为0。

\begin{example}[三角函数]
    在实变量的情形下,熟知\begin{align*}
        \sin^2 x+\cos^2 x&=1\\
        \sin(x+y)&=\sin x\cos y+\cos x\sin y\\
        \cos(x+y)&=\cos x\cos y-\sin x\sin y
    \end{align*}
    我们定义:\begin{align*}
        f(z)&=\sin^2 z+\cos^2 z-1\\
        g(z)&=\sin(z+y)-\sin z\cos y-\cos z\sin y\\
        h(z)&=\cos(z+y)-\cos z\cos y+\sin z\sin y
    \end{align*}
    则$f,g,h$均为复变量上的全纯函数,并且在实轴上恒为0,根据唯一延拓定理,$f(z)=g(z)=h(z)=0,\forall z\in\mathbb{C}$,即三角函数的这些性质可以推广到复变量上。
\end{example}
\begin{example}[双曲函数]
    双曲正弦、双曲余弦、双曲正切函数分别定义为
    \begin{align*}
        \sinh z&=\frac{e^z-e^{-z}}{2}\\
        \cosh z&=\frac{e^z+e^{-z}}{2}\\
        \tanh z&=\frac{\sinh z}{\cosh z}=\frac{e^z-e^{-z}}{e^z+e^{-z}}
    \end{align*}
    代入$z=ix$可得
    \begin{align*}
        \sinh (ix)=i\sin x,\cosh (ix)=\cos x
    \end{align*}
    同理有\begin{align*}
        \sin z=-i\sinh (iz),\cos z=\cosh (iz)
    \end{align*}
    因此双曲函数和三角函数互为全纯延拓。由唯一延拓定理,双曲函数的性质可以由三角函数的性质推导而来,例如
    \begin{align*}
        \cosh^2 z-\sinh^2 z&=\cos^2 (iz)+\sin^2 (iz)=1\\
        \sinh(z+w)&=-i\sin(iz+iw)=-i\sin(iz)\cos(iz)-i\cos(iz)\sin(iz)\\
        &=\sinh z\cosh w+\cosh z\sinh w\\
        \cosh(z+w)&=\cos(iz+iw)=\cos(iz)\cos(iz)-\sin(iz)\sin(iz)\\
        &=\cosh z\cosh w+\sinh z\sinh w
    \end{align*}
    由此可得\begin{align*}
        \tanh(z+w)=\frac{\tanh z+\tanh w}{1+\tanh z\cdot\tanh w}
    \end{align*}
\end{example}

\section{拉普拉斯变换}\label{sec:Laplace_Transform}

十八世纪末,法国数学家皮埃尔-西蒙·拉普拉斯(Pierre-Simon Laplace,1749-1827)在研究天体力学问题时,提出了一种积分变换来处理微分方程,这种变换后来被称为\textbf{拉普拉斯变换}(Laplace transform)。拉普拉斯变换在工程学、物理学和控制理论等领域有着广泛的应用,特别是在信号处理和系统分析中。

鉴于拉普拉斯变换的应用场景中使用角频率明显多于使用频率,我们在本书中采用角频率$\omega$作为拉普拉斯变换的变量;另外,本章默认$u(0)=1$。

在\ref{sec:Fourier}中,我们介绍了傅里叶变换:\begin{align*}
    \mathcal{F} f(\omega)=\int_{-\infty}^{\infty}f(t)e^{-i\omega t}\,dt
\end{align*}

在不涉及收敛性的问题时,可以将$\omega$推广为复数$s=\sigma+i\omega$,根据含参变量积分的求导法则,这样的积分变换可以对$s$求导,从而是全纯函数。不过,添加$\sigma$会使得$t\to -\infty$时$|e^{-st}|=e^{-\sigma t}$快速增长,从而对所变换的函数提出很高的要。我们对这个定义稍加修改:
\begin{definition}[拉普拉斯变换]
    记\[\mathcal{E}=\left\{f:[0,\infty)\to\mathbb{C}\left|f\in PC[0,\infty),\exists C>0,a\in\mathbb{R},|f(t)|<Ce^{at},\forall t\geq 0\right.\right\}\]
    其中$PC[0,\infty)$表示$[0,\infty)$上的分段连续函数。
    对$f\in\mathcal{E}$,其拉普拉斯变换定义为
    \[\mathcal{L} f(s)=\int_{0}^{\infty}f(t)e^{-st}\,dt,\quad s=\sigma+i\omega\]
    其中$s$的实部$\sigma$必须满足$\sigma>a$,将它称为拉普拉斯变换的\textbf{收敛域},直线$\sigma=a$称为\textbf{收敛轴}。\\
    如果$f(t)$是时域信号,$\mathcal{L} f(s)$称为\textbf{复频域信号}。
    我们仍记$\mathcal{L} f(s)$为$F(s)$或$\hat{f}(s)$(在涉及傅里叶变换时,我们会避免使用它们或加以说明),并记$f\overset{\mathcal{L} }{\longleftrightarrow} F(s)$。
\end{definition}
在$\mathcal{E} $的定义中我们对$f$的定义域做了要求,但实际上只要$f$是因果信号(见\ref{sec:ODE},$f(t)=f(t)u(t)$),就可以认为$f\in\mathcal{E}$,后文中不会特意区分$f$的定义域\footnote{实际上,在积分存在的前提下,$\int_{-\infty}^{\infty}f(t)e^{-st}\,dt$称为\textbf{双边拉普拉斯变换},对应地,将前面给出的变换称为\textbf{单边拉普拉斯变换}。}。

在$\mathcal{E} $上定义拉普拉斯变换,是因为对于$f\in\mathcal{E} $,当$\sigma>a$时,给出拉普拉斯变换的积分是绝对收敛的:
\begin{align*}
    \int_0^{\infty}|f(t)e^{-st}|\,dt\leq \int_0^{\infty}Ce^{at}e^{-\sigma t}\,dt=\frac{C}{\sigma-a}<\infty
\end{align*}
因此在导数存在的情况下可以交换积分与求导的次序,从而$\mathcal{L} f(s)$是$\sigma>a$上的全纯函数:
\begin{align*}
    \frac{d}{ds}\mathcal{L} f(s)&=\frac{d}{ds}\int_0^{\infty}f(t)e^{-st}\,dt\\
    &=\int_0^{\infty}f(t)\frac{\partial}{\partial s}(e^{-st})\,dt\\
    &=-\int_0^{\infty}tf(t)e^{-st}\,dt\\
    &=-\mathcal{L} [tf(t)](s)
\end{align*}
这正是拉普拉斯变换的复频域微分性质。

根据全纯函数的唯一延拓性,如果$\mathcal{L} f(s)$可以延拓至$\sigma\leq a$的一些区域,则延拓是唯一的,尽管这时积分不一定存在,我们也认为延拓所得的定义域更大的函数是$f$的拉普拉斯变换。
\begin{example}[多项式的拉普拉斯变换]
    设$f(t)=t^\nu(\nu\geq 0)$,则$f\in\mathcal{E} $,且
    \begin{align*}
        \mathcal{L} f(s)&=\int_0^{\infty}t^\nu e^{-st}\,dt\\
        &=\frac{1}{s^{\nu+1}}\int_0^{\infty}x^\nu e^{-x}\,dx&(x=st)\\
        &=\frac{\Gamma(\nu+1)}{s^{\nu+1}},\quad \sigma>0
    \end{align*}
尽管积分只在$\sigma>0$时收敛,但$\mathcal{L} f(s)$可以延拓为$\mathbb{C}\backslash\{0\}$上的全纯函数。
\begin{remark}
    其实,这个公式对实部大于$-1$的$\nu$都成立。尽管$-1<\text{Re}\ \nu<0$时,$f(t)=t^\nu$在$t=0$处无界,但由于
\[\left|\int_{0}^{1}t^{\nu}e^{-st}\,dt\right|\leq \int_0^1 t^\nu\,dt=\frac{1}{\nu+1},\sigma>0\]
\[\left|\int_{1}^{\infty}t^{\nu}e^{-st}\,dt\right|\leq \int_1^\infty e^{-\sigma t}\,dt=\frac{e^{-\sigma}}{\sigma},\sigma>0\]
其拉普拉斯变换是存在的。因此,我们可以认为$t^\nu\in\mathcal{E},\text{Re}\ \nu>-1$.
\end{remark}
\end{example}

在$\mathcal{E} $中,卷积总是良定义的:
\begin{proposition}[$\mathcal{E} $中的卷积]
    设$f,g\in\mathcal{E} $,则它们的卷积为
    \[(f*g)(t)=\begin{cases}
    0                   & \text{if } t<0 \\
    \int_0^t f(x)g(t-x)\,dx & \text{if } t\geq 0
    \end{cases}\]
\end{proposition}
这是因为$x\geq 0,t-x\geq 0$时,$f(x),g(t-x)$才不恒为0,这相当于积分区间为$[0,t]$。

和研究傅里叶变换一样,我们来考虑它的运算性质、卷积定理和逆变换,再给出一些常见函数的拉普拉斯变换。

\begin{proposition}[拉普拉斯变换的性质]
    设$f,g\in\mathcal{E} $,$a,b\in\mathbb{C}$,则:\begin{itemize}
        \item 线性:\[\mathcal{L} [af(t)+bg(t)](s)=a\mathcal{L} f(s)+b\mathcal{L} g(s)\]
        \item 时移:\[\mathcal{L} [f(t-t_0)u(t-t_0)](s)=e^{-s t_0}\mathcal{L} f(s)\]
        \item 频移:\[\mathcal{L} [e^{a t}f(t)](s)=\mathcal{L} f(s-a)\]
        \item 伸缩:设$a>0$,则\[\mathcal{L} [f(at)](s)=\frac{1}{a}\mathcal{L} f\left(\frac{s}{a}\right)\]
        \item 时域微分:设$f\in PS[0,\infty),f'\in\mathcal{E} $,则\[\mathcal{L} [f'(t)](s)=s\mathcal{L} f(s)-f(0+)\]
        $f'(t)$在0处的定义为$f$在0处的右导数$f(0+)$。
        \item 复频域微分:\[\mathcal{L} [t f(t)](s)=-\frac{d}{ds}\mathcal{L} f(s)\]
        \item 时域积分:\[\mathcal{L} \left[\int_0^t f(x)\,dx\right](s)=\frac{1}{s}\mathcal{L} f(s)\]
        \item 复频域积分:设$f(t)/t\in\mathcal{E} $,则\[\mathcal{L} \left[\frac{f(t)}{t}\right](s)=\int_s^{\infty}\mathcal{L}f(x)\,dx\]
        此处积分路径为复平面上满足$Im(s)=\omega$有界的任意从$s$到无穷远处的路径
    \end{itemize}
\end{proposition}
\begin{remark}
    很明显,时域微分性质和复频域微分性质都可以对更高阶的导数进行推广,其中,时域微分性质推广为\begin{align*}
\mathcal{L} [f^{(n)}(t)](s)&=s^n\mathcal{L} f(s)-\sum_{k=0}^{n-1}s^{n-1-k}f^{(k)}(0+)\\
&=s^n\mathcal{L} f(s)-s^{n-1}f(0+)-s^{n-2}f'(0+)-\cdots -f^{(n-1)}(0+)
\end{align*}
\end{remark}
\begin{proof}
    1.线性性:直接得自积分的线性性。\\
    2.时移:\begin{align*}
        \mathcal{L} [f(t-t_0)u(t-t_0)](s)&=\int_0^{\infty}f(t-t_0)u(t-t_0)e^{-st}\,dt\\
        &=\int_{t_0}^{\infty}f(t-t_0)e^{-st}\,dt\\
        &=e^{-s t_0}\int_0^{\infty}f(x)e^{-s x}\,dx&( x=t-t_0)\\
        &=e^{-s t_0}\mathcal{L} f(s)
    \end{align*}
    3.频移:\begin{align*}
        \mathcal{L} [e^{a t}f(t)](s)=\int_0^{\infty}e^{a t}f(t)e^{-st}\,dt=\int_0^{\infty}f(t)e^{-(s-a)t}\,dt=\mathcal{L} f(s-a)
    \end{align*}
    4.伸缩:\begin{align*}
        \mathcal{L} [f(at)](s)&=\int_0^{\infty}f(at)e^{-st}\,dt\\
        &=\int_0^{\infty}f(x)e^{-\frac{s}{a}x}\frac{1}{a}\,dx&(x=at)\\
        &=\frac{1}{a}\mathcal{L} f\left(\frac{s}{a}\right)
    \end{align*}
    5.时域微分:\begin{align*}
        \mathcal{L} [f'(t)](s)&=\int_0^{\infty}f'(t)e^{-st}\,dt\\
        &=\evalat{\left(f(t)e^{-st}\right)}{0}{\infty}+\int_0^{\infty}sf(t)e^{-st}\,dt\\
        &=s\mathcal{L} f(s)-f(0+)
    \end{align*}
    这里$f(0+)$表示$f$在0处的右极限。$f$在0处不连续时,微积分基本定理的确会给出$f(0+)$而非$f(0)$。\\
    6.复频域微分:已经在本节前面证明过。\\
    7.时域积分:利用时域微分性质,有
    \begin{align*}
        \mathcal{L} \left[\int_0^t f(x)\,dx\right](s)&=\frac{1}{s}\mathcal{L} \left[\frac{d}{dt}\left(\int_0^t f(x)\,dx\right)\right](s)\\
        &=\frac{1}{s}\mathcal{L} f(s)
    \end{align*}
    8.复频域积分:利用复频域的微分性质,有
    \begin{align*}
        \frac{d}{ds}\mathcal{L} \left[\frac{f(t)}{t}\right](s)&=-\mathcal{L} f(s)\\
        \mathcal{L} \left[\frac{f(t)}{t}\right](s)&=\int_s^{\infty}\mathcal{L}f(x)\,dx
    \end{align*}
\end{proof}

\begin{theorem}[卷积定理]
    设$f,g\in\mathcal{E} $,则
    \[\mathcal{L} [(f*g)(t)](s)=\mathcal{L} f(s)\cdot \mathcal{L} g(s)\]
\end{theorem}
\begin{proof}
与前面傅里叶变换的卷积定理证明类似:
    \begin{align*}
        \mathcal{L} [(f*g)(t)](s)&=\int_0^{\infty}(f*g)(t)e^{-st}\,dt\\
        &=\int_0^{\infty}\left(\int_0^t f(x)g(t-x)\,dx\right)e^{-st}\,dt\\
        &=\int_0^{\infty}\int_0^{\infty}f(x)g(t-x)u(t-x)e^{-st}\,dx\,dt\\
        &=\int_0^{\infty}f(x)\left(\int_x^{\infty}g(t-x)e^{-st}\,dt\right)\,dx\\
        &=\int_0^{\infty}f(x)e^{-sx}\left(\int_0^{\infty}g(y)e^{-sy}\,dy\right)\,dx&(y=t-x)\\
        &=\mathcal{L} f(s)\cdot \mathcal{L} g(s)
    \end{align*}
\end{proof}

以上所有性质的证明都与傅里叶变换类似,实际上我们不需要做这种重复性的证明工作。给定$s=\sigma+i\omega$,令$\tilde{f}(t)=f(t)e^{-\sigma t}$,则$\mathcal{L} f(s)=\mathcal{F} \tilde{f}(\omega)$,只要$\tilde{f}\in L^1(\mathbb{R} )$,就可以将傅里叶变换的性质直接应用到拉普拉斯变换上。由于我们在$\mathcal{E} $中要求了$f(t)$的分段连续性,在收敛域$\sigma>a$内,$\tilde{f}(t)$总是属于$L^1(\mathbb{R})$的。下面给出两个例子:
\begin{example}[时移性质]
    设$f\in\mathcal{E} ,t_0>0$,令$\tilde{f}(t)=f(t)e^{-\sigma t}$,则
    \begin{align*}
        \mathcal{L}[f(t-t_0)u(t-t_0)](s)&=\mathcal{F} \left[\tilde{f}(t-t_0)u(t-t_0)\right](\omega)\\
        &=e^{-i\omega t_0}\mathcal{F} \tilde{f}(\omega)\\
        &=e^{-s t_0}\mathcal{L} f(s)
    \end{align*}
\end{example}
\begin{example}[卷积定理]
设$f,g\in\mathcal{E} $,令$\tilde{f}(t)=f(t)e^{-\sigma t}, \tilde{g}(t)=g(t)e^{-\sigma t}$,注意到\begin{align*}
    (\tilde{f}*\tilde{g})(t)&=\int_{-\infty}^{\infty}\tilde{f}(x)\tilde{g}(t-x)\,dx\\
    &=\int_0^{t}f(x)e^{-\sigma x}g(t-x)e^{-\sigma (t-x)}\,dx\\
    &=e^{-\sigma t}\int_0^{t}f(x)g(t-x)\,dx\\
    &=e^{-\sigma t}(f*g)(t)
\end{align*}
有
    \begin{align*}
        \mathcal{L} \left[f*g\right](s)&=\mathcal{F} \left[f*g(t)e^{-\sigma t}\right](\omega)\\
        &=\mathcal{F} \left[\tilde{f}*\tilde{g}\right](\omega)\\
        &=\mathcal{F} \tilde{f}(\omega)\cdot \mathcal{F} \tilde{g}(\omega)\\
        &=\mathcal{L} f(s)\cdot \mathcal{L} g(s)
    \end{align*}
\end{example}
其余的证明,请读者自行完成。

下面给出一些常用函数的拉普拉斯变换计算作为例子。

\begin{example}[指数多项式]
    设$f(t)=t^\nu e^{a t}u(t),\text{Re}\nu>-1$,利用拉普拉斯变换的频移性质,有
    \begin{align*}
        \mathcal{L} f(s)&=\mathcal{L}\left[t^\nu e^{a t}u(t)\right](s)\\
        &=\mathcal{L} \left[t^\nu u(t)\right](s-a)\\
        &=\frac{\Gamma(\nu+1)}{(s-a)^{\nu+1}},\sigma>a
    \end{align*}
    根据有理函数的理论,任意真分式函数都可以表示为若干个此类函数的线性组合,因此可以说$f$是指数多项式等价于$\mathcal{L} f$是真分式函数
\end{example}

\begin{example}[正弦、余弦函数]
    首先,$\cos(\omega_0 t),\sin(\omega_0 t)\in\mathcal{E}$,因为$$\forall a>0,|\cos(\omega_0 t)|,|\sin(\omega_0 t)|<e^{a t}$$
    下面给出两种求解方法。\\
    方法一:利用欧拉公式化归为指数函数:
    \begin{align*}
        \mathcal{L}[\cos(\omega_0 t)](s)&=\mathcal{L}[\cos(\omega_0 t)u(t)](s)\\
        &=\frac{1}{2}\mathcal{L} [e^{i\omega_0 t}u(t)](s)+\frac{1}{2}\mathcal{L} [e^{-i\omega_0 t}u(t)](s)\\
        &=\frac{1}{2}\left(\frac{1}{s-i\omega_0}+\frac{1}{s+i\omega_0}\right)\\
        &=\frac{s}{s^2+\omega_0^2},\sigma>0
    \end{align*}
    \begin{align*}
        \mathcal{L}[\sin(\omega_0 t)](s)&=\mathcal{L} [\sin(\omega_0 t)u(t)](s)\\
        &=\frac{1}{2i}\mathcal{L} [e^{i\omega_0 t}u(t)](s)-\frac{1}{2i}\mathcal{L} [e^{-i\omega_0 t}u(t)](s)\\
        &=\frac{1}{2i}\left(\frac{1}{s-i\omega_0}-\frac{1}{s+i\omega_0}\right)\\
        &=\frac{\omega_0}{s^2+\omega_0^2},\sigma>0
    \end{align*}
    方法二:利用微分性质:
    \begin{align*}
        -a^2\mathcal{L}[\cos(\omega_0 t)](s)&=\mathcal{L} [\cos''(\omega_0 t)](s)\\
        &=s^2\mathcal{L} [\cos(\omega_0 t)](s)-s\cos(0+)-\cos'(0+)\\
        &=s^2\mathcal{L} [\cos(\omega_0 t)](s)-s
    \end{align*}
    因此\[\mathcal{L} [\cos(\omega_0 t)](s)=\frac{s}{s^2+\omega_0^2}\]
    同理,有\[\mathcal{L} [\sin(\omega_0 t)](s)=\frac{\omega_0}{s^2+\omega_0^2}\]
\end{example}

\begin{example}[双曲函数]
    因为
    \[\sinh(\omega_0 t)=\frac{e^{\omega_0 t}-e^{-\omega_0 t}}{2},\cosh(\omega_0 t)=\frac{e^{\omega_0 t}+e^{-\omega_0 t}}{2}\]
    所以
    \begin{align*}
        \mathcal{L} \left[\sinh(\omega_0 t)\right](s)&=\frac{1}{2}\mathcal{L} \left[e^{\omega_0 t}\right](s)-\frac{1}{2}\mathcal{L} \left[e^{-\omega_0 t}\right](s)\\
        &=\frac{1}{2}\left(\frac{1}{s-\omega_0}-\frac{1}{s+\omega_0}\right)\\
        &=\frac{\omega_0}{s^2-\omega_0^2},\sigma>|\omega_0|
    \end{align*}
    \begin{align*}
        \mathcal{L} \left[\cosh(\omega_0 t)\right](s)&=\frac{1}{2}\mathcal{L} \left[e^{\omega_0 t}\right](s)+\frac{1}{2}\mathcal{L} \left[e^{-\omega_0 t}\right](s)\\
        &=\frac{1}{2}\left(\frac{1}{s-\omega_0}+\frac{1}{s+\omega_0}\right)\\
        &=\frac{s}{s^2-\omega_0^2},\sigma>|\omega_0|
    \end{align*}
\end{example}

\begin{example}[取样函数]
    首先,$\frac{\sin(\omega_0 t)}{t}\in\mathcal{E}$,因为它是有界函数。
    结合正弦函数的拉普拉斯变换和拉普拉斯变换的复频域积分性质,有
    \begin{align*}
        \mathcal{L} \left[\frac{\sin(\omega_0 t)}{t}\right](s)&=\int_s^{\infty}\mathcal{L}[\sin(\omega_0 t)](x)\,dx\\
        &=\int_s^{\infty}\frac{\omega_0}{x^2+\omega_0^2}\,dx\\
        &=\evalat{\arctan\frac{x}{\omega_0}}{s}{\infty}\\
        &=\frac{\pi}{2}-\arctan\frac{s}{\omega_0}\\
        &=\arctan\left(\frac{\omega_0}{s}\right),\sigma>0
    \end{align*}
\end{example}

\begin{example}[正弦积分函数]
    $$\frac{\sin(\omega_0 t)}{t}\leq Ce^{a t},a>0\implies \text{Si}(\omega_0 t)\leq \frac{C}{a}e^{a t}\implies \text{Si}(\omega_0 t)\in\mathcal{E}$$
    利用正弦函数的拉普拉斯变换和拉普拉斯变换的时域积分性质,有
    \begin{align*}
        \mathcal{L} [\text{Si}(\omega_0 t)](s)&=\mathcal{L} \left[\int_0^{\omega_0 t} \frac{\sin(x)}{x}\,dx\right](s)\\
        &=\mathcal{L} \left[\int_0^{t} \frac{\sin(\omega_0 x)}{x}\,dx\right](s)\\
        &=\frac{1}{s}\mathcal{L} \left[\frac{\sin(\omega_0 t)}{t}\right](s)\\
        &=\frac{1}{s}\arctan\left(\frac{\omega_0}{s}\right),\sigma>0
    \end{align*}    
\end{example}

由于$f\in\mathcal{E}$对函数增长速度的要求是相当低的(作为反例,$e^{t^2}\notin\mathcal{E}$),我们只需要单独处理$\delta$分布,而不需要重新构建一整套分布的理论。仍考虑将拉普拉斯变换化归为傅里叶变换,即
\[\mathcal{L} f(s)=\mathcal{F} [f(t)e^{-\sigma t}](\omega),s=\sigma+i\omega\]
这里$\mathcal{F} [f(t)e^{-\sigma t}](\omega)$实际上依赖于两个变量$\sigma,\omega$,但我们只写出$\omega$,并将$\sigma$视为一个参数。这样一来,我们就可以将拉普拉斯变换推广到$\delta$分布上(为了避免变量符号的问题,我们将变量统一地用$x$表示):
\begin{align*}
    \left\langle \mathcal{L}\delta,\phi\right\rangle=\left\langle e^{-\sigma x}\delta,\mathcal{F} \varphi\right\rangle=\left\langle \delta,e^{-\sigma x}\mathcal{F} \varphi\right\rangle=\mathcal{F}\varphi(0)=\int_{-\infty}^{\infty}\varphi(t)\,dt=\left\langle \mathds{1},\varphi\right\rangle
\end{align*}
其中$\mathds{1}$表示恒为1的函数。我们知道,多项式的拉普拉斯变换是真分式函数,而如果将$\delta$分布也纳入考虑,则可以得到任意有理函数。

用同样的方法,我们还可以得到$\delta$的平移和导数的拉普拉斯变换:
\begin{align*}
    \left\langle \mathcal{L}\delta_a,\varphi\right\rangle&=\left\langle e^{-\sigma x}\delta_a,\mathcal{F} \varphi\right\rangle\\
    &=e^{-\sigma a}\mathcal{F}\varphi(a)\\
    &=\int_{-\infty}^{\infty}\varphi(t)e^{-s a}\,dt\\
    &=\left\langle e^{-s a},\varphi\right\rangle\\
    \left\langle \mathcal{L}\left[\delta'\right],\varphi\right\rangle&=\left\langle e^{-\sigma x}\delta',\mathcal{F} \varphi\right\rangle\\
    &=-\left\langle \delta,\left(e^{-\sigma x}\mathcal{F} \varphi\right)'\right\rangle\\
    &=-\left\langle \delta,-\sigma e^{-\sigma x}\mathcal{F} \varphi + e^{-\sigma x}(\mathcal{F} \varphi)'\right\rangle\\
    &=\left\langle \delta,\sigma e^{-\sigma x}\mathcal{F} \varphi + i\omega e^{-\sigma x}\mathcal{F} \varphi\right\rangle\\
    &=(\sigma + i\omega)\mathcal{F} \varphi(0)\\
    &=\left\langle s,\varphi\right\rangle
\end{align*}
因此我们得到:
\begin{proposition}
    设$a\in\mathbb{R}$,则$\mathcal{L} \delta_a=e^{-s a}$;$\mathcal{L} \left[\delta'\right]=s$,进一步可得$\mathcal{L} \left[\delta^{(n)}\right]=s^n$。
\end{proposition}

看上去,$\delta_a$在$a<0$时非因果,但它是作为一个分布定义的。

此外我们来介绍一种拉普拉斯变换的级数求法。我们知道,多项式的拉普拉斯变换都可以通过$\mathcal{L} [t^n](s)=\frac{\Gamma(n+1)}{s^{n+1}}$来计算,那么对于一些可以展开为幂级数的函数,不难想到逐项求拉普拉斯变换,下面讨论这种操作的适用范围。在此之前,我们先介绍一个引理:
\begin{lemma}
    设$f\in\mathcal{E} ,|f(t)|\leq Ce^{at}$,则\begin{itemize}
        \item $\forall x>a,\mathcal{L} f(x+iy)\to 0,|y|\to\infty$
        \item $\forall y,\mathcal{L} f(x+iy)\to 0,x\to\infty$
    \end{itemize}
\end{lemma}
\begin{proof}
    1.令$g(t)=f(t)e^{-x t}$,则$|g(t)|\leq Ce^{-(x-a)t}$,因此$g\in L^1(\mathbb{R} )$,根据Riemann-Lebesgue引理,有
    \[\mathcal{L} f(x+iy)=\mathcal{F} g(y)\to 0,|y|\to\infty\]
    2.利用海涅归并原理,任取实部趋于无穷的序列$\{z_n\}$,满足$\mathcal{L} f(z_n)$存在(这要求$Re z>a$),我们要证明序列$\mathcal{L} f(z_n)\to 0$。由控制收敛定理,有
    \begin{align*}
        \int_{0}^{\infty}|f(t)e^{-z_n t}|\,dt&\leq \int_0^{\infty}Ce^{at}e^{-Re(z_n) t}\,dt\\
        &=C\int_0^{\infty}e^{-(Re(z_n)-a)t}\,dt=\frac{C}{Re(z_n)-a}<\infty\\
        \lim_{n\to\infty}\mathcal{L} f(z_n)&=\lim_{n\to\infty}\int_0^{\infty}f(t)e^{-z_n t}\,dt\\
        &=\int_0^{\infty}\lim_{n\to\infty}f(t)e^{-z_n t}\,dt\\
        &=0
    \end{align*}
\end{proof}

\begin{theorem}[拉普拉斯变换的级数求法]
    设$f\in\mathcal{E} $,且在$[0,\infty)$上有幂级数展开
    \[f(t)=\sum_{n=0}^{\infty}a_n t^n\]
    如果存在$\omega_0>0$,使得级数
    \[\sum_{n=0}^{\infty}|a_n|\frac{\Gamma(n+1)}{\omega_0^{n}}\]
    收敛,则在$\sigma>\omega_0$时,级数
    \[\sum_{n=0}^{\infty}a_n \frac{\Gamma(n+1)}{s^{n+1}}\]
    收敛且等于$\mathcal{L} f(s)$,即$f$的拉普拉斯变换可以通过对其幂级数求拉普拉斯变换得到。
\end{theorem}
\begin{proof}
    由条件知级数$\sum_{n=0}^{\infty}a_n \frac{\Gamma(n+1)}{s^{n+1}}$在$\sigma>\omega_0$时绝对收敛、内闭一致收敛,因此只需证明其和等于$\mathcal{L} f(s)$。拉普拉斯变换是一种积分变换,要将求和与积分换序,可以使用控制收敛定理:
    \begin{align*}
        &|a_n \Gamma(n+1)\omega_0^{-n}|\to 0, n\to\infty\\
        \implies &|a_n|\leq C\frac{\omega_0^{n}}{\Gamma(n+1)}\\
        \implies &|f(t)|\leq \sum_{n=0}^{\infty}|a_n|t^n\leq C\sum_{n=0}^{\infty}\frac{(\omega_0 t)^{n}}{\Gamma(n+1)}=Ce^{\omega_0 t}\\
        \implies &\forall s,\text{Re}\ s=\omega_0+\phi>\omega_0,|f(t)e^{-s t}|\leq Ce^{-\phi t}
    \end{align*}
    因此控制收敛定理的条件成立,换序即证。
\end{proof}

更一般地,对于幂次不是整数的情况,可以推广以上结论:
\begin{corollary}*
    设$f\in\mathcal{E} $,且在$[0,\infty)$上有广义幂级数展开
    \[f(t)=\sum_{n=0}^{\infty}a_n t^{n+\alpha}\]
    其中$0\leq\alpha<1$,如果存在$\omega_0>0$,使得级数
    \[\sum_{n=0}^{\infty}|a_n|\frac{\Gamma(n+\alpha+1)}{\omega_0^{n}}\]
    收敛,则在$\sigma>\omega_0$时,级数
    \[\sum_{n=0}^{\infty}a_n \frac{\Gamma(n+\alpha+1)}{s^{n+\alpha+1}}\]
    收敛且等于$\mathcal{L} f(s)$,即$f$的拉普拉斯变换可以通过对其广义幂级数求拉普拉斯变换得到。
\end{corollary}

\section{拉普拉斯逆变换}\label{sec:Laplace_inverse}

我们仍然令$g(t)=f(t)e^{-\sigma t}$,从而将傅里叶反演公式推广到拉普拉斯反演公式:
\begin{theorem}[拉普拉斯反演公式]
    设$f\in\mathcal{E} $,则
    \[f(t)=\frac{1}{2\pi i}\lim_{r\to\infty}\int_{b-ir}^{b +ir}\mathcal{L} f(s)e^{st}\,ds\]
    其中$b>a$,$a$为$f$的拉普拉斯变换的收敛域下界。
\end{theorem}
\begin{remark}
    严格来说,应该要求$f\in PS[0,\infty)$,并且最终的结论是:
    \[\frac{f(t-)+f(t+)}{2}=\frac{1}{2\pi i}\lim_{r\to\infty}\int_{b-ir}^{b +ir}\mathcal{L} f(s)e^{st}\,ds\]
    只要令$g(t)=f(t)e^{-\sigma t}$,并利用附录\ref{sec:approach}中傅里叶逆变换的结论,即可得到拉普拉斯逆变换。不过,我们仍不关心不连续点处的值,因此直接使用上面的结论。
\end{remark}
\begin{remark}
    这里的积分\[\lim_{r\to\infty}\int_{b-ir}^{b +ir}\mathcal{L} f(s)e^{st}\,ds\]
    应该理解为沿着复平面上的直线$Re z=b$从$b-i\infty$到$b+i\infty$的积分。由于全纯函数的路径无关性,我们也可以选择其他路径,甚至可以添加一些路径,再使用留数定理计算拉普拉斯逆变换,见本节最后的讨论。
\end{remark}
\begin{proof}
    \begin{align*}
        \frac{1}{2\pi i}\lim_{r\to\infty}\int_{b-ir}^{b +ir}\mathcal{L} f(s)e^{st}\,ds&=\frac{1}{2\pi i}\lim_{r\to\infty}\int_{-ir}^{ir}\mathcal{L} f(b +i\omega)e^{(b +i\omega)t}i\,d\omega\\
        &=\frac{e^{b t}}{2\pi}\lim_{r\to\infty}\int_{-r}^{r}\mathcal{F} \tilde{f}(\omega)e^{i\omega t}\,d\omega\\
        &=e^{b t}\tilde{f}(t)=f(t)
    \end{align*}
\end{proof}

通过这个定理可以看出,拉普拉斯逆变换一定能还原出$f(t)$(在不考虑间断点函数值的意义下),因此拉普拉斯变换是双射。

但是,以上证明已经假定了$F(s)=\mathcal{L} f(s)$是某个函数$f\in\mathcal{E} $的拉普拉斯变换。从另一个角度出发,假设有一个在右半平面$\text{Re}\ s>a$上全纯的函数$F(s)$,就需要考虑它是否是某个函数的拉普拉斯变换,以及如何求出这个函数。为此,我们给出以下定理:
\begin{theorem}[*拉普拉斯反演公式]
    设$F(s)$在$\text{Re}\ s>a$上全纯,且满足:
    \[|F(s)|\leq C|s|^{-\alpha},\alpha>1,\forall \text{Re}\ s>a\]
    则$\forall b>a$,函数
    \[f(t)=\frac{1}{2\pi i}\lim_{r\to\infty}\int_{b-ir}^{b +ir}F(s)e^{st}\,ds\]
    属于$\mathcal{E} $,且$F(s)=\mathcal{L} f(s),\forall \text{Re}\ s>a$。
\end{theorem}
\begin{remark}
    这里的条件$\alpha>1$保证了$F(b+i\omega)\in L^1$(作为$\omega$的函数),尽管$b=0$时$\frac{1}{|b+i\omega|^{\alpha}}\notin L^1$,这是因为$F$是全纯函数,在$\omega=0$处不会出现奇点。事实上,$\alpha>\frac{1}{2}$即可保证定理成立,但这样证明会非常复杂,本书不再讨论。
\end{remark}
\begin{proof}
    首先,我们证明由积分$\frac{1}{2\pi i}\lim_{r\to\infty}\int_{b-ir}^{b +ir}F(s)e^{st}\,ds$定义的函数$f(t)$是良定义的。一方面,由于$|F(s)|\leq C|s|^{-\alpha}$,有
    \[\left|\lim_{r\to\infty}\int_{b -ir}^{b +ir}F(s)e^{st}\,ds\right|=\left|\int_{-\infty}^{\infty}F(b+i\omega)e^{(b+i\omega)t}i\,d\omega\right|\leq e^{b t}\int_{-\infty}^{\infty}|F(b+i\omega)|\,d\omega<\infty\]

    另一方面,我们验证$f_b(t)=\frac{1}{2\pi i}\lim_{r\to\infty}\int_{b-ir}^{b +ir}F(s)e^{st}\,ds$与$b$无关。设$b_1,b_2>a$,选择以$b_1\pm ir,b_2\pm ir$为顶点的矩形围道$C_r$,则由于$|F(s)|\leq C|s|^{-\alpha}$,矩形顶边、底边的积分趋于0,而$F(s)$在$\text{Re}\ s>a$上全纯,由柯西定理知$\int_{C_r}F(s)e^{st}\,ds=0$,因此$f_{b_1}(t)=f_{b_2}(t)$。

    注意到以上证明并不依赖于$t$的取值,极限过程$\lim_{r\to\infty}$对$t\in\mathbb{R}$是内闭一致收敛的,因此可以由被积函数$F(s)e^{st}$的连续性推出$f(t)$的连续性。

    接下来,我们证明$f(t)\in\mathcal{E} $。一方面,$f$是因果的:$\forall t<0$,取$b>0$,并取围道如下:
    \begin{figure}[H]
        \centering
        \includegraphics[width=0.6\textwidth]{contour2}
    \end{figure}
    其中$\gamma_1$为$b+ir$到$b -ir$的直线段,$\gamma_2$为以原点为圆心,$b -ir$到$b+ir$的圆弧,半径$R=\sqrt{b^2+r^2}$。由柯西积分定理,有
    \[\int_{\gamma_1+\gamma_2}F(s)e^{st}\,ds=0\]
    对$\gamma_2$,有
    \[\left|\int_{\gamma_2}F(s)e^{st}\,ds\right|\leq \int_{\gamma_2}|F(s)||e^{st}|\,|ds|\leq C\int_{\gamma_2}|s|^{-\alpha}e^{b t}\,|ds|\leq C|R|^{-\alpha}\cdot \pi R\to 0,r\to\infty\]
    因此$f(t)=\frac{1}{2\pi i}\lim_{r\to\infty}\int_{\gamma_1}F(s)e^{st}\,ds=0,t<0$。

    另一方面,取$b>a$,有
    \begin{align*}
        |f(t)|=\left|\frac{1}{2\pi i}\lim_{r\to\infty}\int_{b -ir}^{b +ir}F(s)e^{st}\,ds\right|\leq \frac{e^{b t}}{2\pi}\int_{-\infty}^{\infty}|F(b+i\omega)|\,d\omega=k e^{b t},k\text{为常数}
    \end{align*}

    又因为我们已经证明了$f$的连续性,所以$f\in\mathcal{E} $。

    最后,我们证明$F(s)=\mathcal{L} f(s),\forall \text{Re}\ s>a$。上一步中,$\forall b>a$,我们都能找到常数$k$使得$|f(t)|\leq k e^{b t}$,因此可以取$b'\in(a,b)$,使得$|f(t)e^{-bt}|\leq k'e^{(b'-b)t}$,从而$f(t)e^{-bt}\in L^1(\mathbb{R})$。根据$f$的定义,有
    \begin{align*}
        f(t)e^{-b t}&=\frac{1}{2\pi}\lim_{r\to\infty}\int_{-r}^{r}F(b+i\omega)e^{i\omega t}\,d\omega
        &=\mathcal{F}^{-1} [F(b+i\omega)](t)
    \end{align*}
    我们已经证明过$f$连续和$F(b+i\omega)\in L^1(\mathbb{R})$,因此可以使用傅里叶反演公式(见附录\ref{sec:approach}的定理A.2.5.),得到
    \begin{align*}
        F(b+i\omega)&=\mathcal{F} [f(t)e^{-b t}](\omega)\\
        &=\int_{-\infty}^{\infty}f(t)e^{-b t}e^{-i\omega t}\,dt\\
        &=\int_{0}^{\infty}f(t)e^{-(b+i\omega) t}\,dt\\
        &=\mathcal{L} f(b+i\omega)
    \end{align*}
\end{proof}

\begin{example}[真分式函数]
    我们知道,真分式函数的拉普拉斯逆变换是指数多项式的线性组合。这里给出一个例子:设$F(s)=\frac{2s+5}{s^2+3s+2}$,则
    \begin{align*}
        F(s)&=\frac{2s+5}{(s+1)(s+2)}\\
        &=\frac{3}{s+1}-\frac{1}{s+2}
    \end{align*}
    因此
    \[\mathcal{L}^{-1} F(s)=3e^{-t}-e^{-2t}\]
\end{example}

一般地,真分式函数可以通过部分分式展开法来计算拉普拉斯逆变换。根据代数基本定理,任意多项式在复数域上必有根,并且实系数多项式的非实根成共轭复数对出现,因此多项式可以分解为若干个一次因式的乘积,如果在实数范围内做因式分解,则共轭复根对应的一次因式可以合并为一个二次因式。进一步,任意真分式函数都可以表示为若干个部分分式的和:
\begin{theorem}[部分分式展开定理]
    设\begin{align*}
    F(s)=\frac{P(s)}{Q(s)}=\frac{P(s)}{(s-r_1)^{m_1}\cdots (s-r_k)^{m_k}(s^2+a_1 s+b+1)^{n_1}\cdots (s^2+a_l s+b_l f)^{n_l}}
    \end{align*}
其中$degP<degQ$,$r_i,i=1,2,\cdots,k$为$Q(s)$的实根,$s^2+a_1 s+b_1,\cdots,s^2+a_l s+b_l$为$Q(s)$的不可约二次因式,则$F(s)$可以展开为部分分式的和:
\begin{align*}
    F(s)&=\frac{A_{11}}{s-r_1}+\frac{A_{12}}{(s-r_1)^2}+\cdots +\frac{A_{1m_1}}{(s-r_1)^{m_1}}\\
    &+\frac{B_{11}s+C_{11}}{s^2+as+b}+\frac{B_{12}s+C_{12}}{(s^2+as+b)^2}+\cdots +\frac{B_{1n_1}s+C_{1n_1}}{(s^2+as+b)^{n_1}}\\
    &+\cdots \\
    &+\frac{B_{l1}s+C_{l1}}{s^2+es+f}+\frac{B_{l2}s+C_{l2}}{(s^2+es+f)^2}+\cdots +\frac{B_{ln_l}s+C_{ln_l}}{(s^2+es+f)^{n_l}}
\end{align*}
\end{theorem}
这个命题表明,分母多项式中,一次因式的若干次幂在分解式中对应其各幂次的线性组合,而二次因式在分解式中其分子为一次多项式。

下面介绍两种求解部分分式展开的系数的方法。

\noindent \textbf{法一:待定系数法}

按照以上命题给出的展开式形式设未知数,将展开式通分,然后与原式的分子比较系数,得到一个线性方程组,解出各未知数即可。
\begin{example}
    设\[F(s)=\frac{2s+5}{s^2+3s+2}\]
    则
    \begin{align*}
        F(s)&=\frac{A}{s+1}+\frac{B}{s+2}\\
        &=\frac{A(s+2)+B(s+1)}{(s+1)(s+2)}\\
        &=\frac{(A+B)s+(2A+B)}{(s+1)(s+2)}
    \end{align*}
    比较系数,有
    \[\begin{cases}
        A+B=2\\
        2A+B=5
    \end{cases}\implies \begin{cases}
        A=3\\
        B=-1
    \end{cases}\]
    因此\[F(s)=\frac{3}{s+1}-\frac{1}{s+2}\]
\end{example}

\noindent \textbf{法二:取极限法}(又称为赫维赛德掩盖法)

按照以上命题给出的展开式形式设未知数,\begin{itemize}
    \item 对于一次因式最高次幂$(s-r)^m$的系数,将等式两边同时乘以$(s-r)^m$,然后令$s\to r$,即可解出最高次幂的系数;
    \item 对于一次因式次高次幂的系数,将等式两边同时乘以$(s-r)^{m-1}$,然后对$s$求导,再令$s\to r$,即可解出次高次幂的系数;也可以先约去已经求出的最高次幂项,再乘以$(s-r)^{m-1}$并令$s\to r$,依此类推
    \item 二次因式的处理类似,因为可以在复数范围内做因式分解(如果你愿意,可以对比正余弦函数的拉普拉斯变换,配平方得到二次不可约因式的拉普拉斯逆变换,但相信你看懂例5.3.4.后就再也不会这样做了)
\end{itemize}

\begin{example}
    设\[F(s)=\frac{1}{(s+1)^2(s+2)(s^2+1)}\]
    则
    \begin{align*}
        F(s)&=\frac{A_1}{s+1}+\frac{A_2}{(s+1)^2}+\frac{B}{s+2}+\frac{C s+D}{s^2+1}
    \end{align*}
    两边同时乘以$(s+1)^2$,令$s=-1$,有
    \[A_2=\frac{1}{(-1+2)(-1^2+1)}=1\]
    同理$B=\frac{1}{3}$。约去$1/(s+1)^2$项后,两边同时乘以$(s+1)$,令$s=-1$,有
    \[A_1=\lim_{s\to -1}(s+1)\left[F(s)-\frac{1}{(s+1)^2}\right]=\lim_{s\to -1}\frac{(s+1)^2}{(s+2)(s^2+1)}=\frac{1}{2}\]
    两边同时乘以$(s^2+1)$,令$s=i$,有
    \[C i+D=\lim_{s\to i}(s-i)F(s)=\lim_{s\to i}\frac{(s-i)}{(s+1)^2(s+2)(s^2+1)}=-\frac{i}{5}\]
    令$s=-i$,有
    \[-C i+D=\lim_{s\to -i}(s+i)F(s)=\lim_{s\to -i}\frac{(s+i)}{(s+1)^2(s+2)(s^2+1)}=\frac{i}{5}\]
    解得$C=-\frac{1}{5},D=0$,因此
    \[F(s)=\frac{1/2}{s+1}+\frac{1}{(s+1)^2}+\frac{1/3}{s+2}+\frac{-\frac{1}{5}s}{s^2+1}\]
\end{example}

我们也可以用留数定理计算拉普拉斯逆变换,通常选择由$b\pm ir,-r\pm ir$组成的矩形围道,其中$b>0$需保证$F(s)$在$\text{Re}\ s>b$上全纯(因此在$r\to\infty$时极点全部位于围道内),见左图:
\begin{figure}[H]
    \centering
    \includegraphics[width=0.8\textwidth]{contour}
\end{figure}
\begin{example}[真分式函数,留数定理方法]
    设\[F(s)=\frac{2s+5}{s^2+3s+2}\]
    $\Gamma_r$为上面左图所示的矩形围道,$b>0$,我们来论证当$r\to\infty$时,另三条线段上的积分趋于0,从而
    \begin{align*}
        f(t)&=\mathcal{L}^{-1} F(s)\\
        &=\frac{1}{2\pi i}\lim_{r\to\infty}\int_{b -ir}^{b +ir}F(s)e^{st}\,ds\\
        &=\frac{1}{2\pi i}\lim_{r\to\infty}\oint_{\Gamma_r}F(s)e^{st}\,ds
    \end{align*}

    当$t>0$时,对于矩形的上边(下边同理),有
    \begin{align*}
        \left|\int_{top}F(s)e^{st}i\,ds\right|&\leq \sup_{x\leq b}|F(x+ir)e^{(x+ir)t}|\int_{-\infty}^{b}e^{xt}\,dx\\
        &\leq \frac{e^{bt}}{t}\sup_{x\leq b}|F(x+ir)|
    \end{align*}
    在前面的引理中,我们已经证明了$|F(x+ir)|\to 0,|r|\to\infty$,因此上边的积分趋于0。对于矩形的左边,$|e^{st}|=e^{-rt}$,以指数函数的速度衰减至0,积分当然趋于0。

    当$t<0$时,$e^{st}$在$\text{Re}\ s$充分大时有较小的模值,因此我们可以使用上面右图中的围道来论证$f(t)=0$。而$t=0$的值则不需要考虑。
    
    根据留数定理,有
    \begin{align*}
        f(t)&=\sum \text{Res}[F(s)e^{st},s_k]\\
        &=\text{Res}\left[\frac{2s+5}{(s+1)(s+2)}e^{st},-1\right]+\text{Res}\left[\frac{2s+5}{(s+1)(s+2)}e^{st},-2\right]\\
        &=\lim_{s\to -1}(s+1)\frac{2s+5}{(s+1)(s+2)}e^{st}+\lim_{s\to -2}(s+2)\frac{2s+5}{(s+1)(s+2)}e^{st}\\
        &=3e^{-t}-e^{-2t}
    \end{align*}
\end{example}

下面再给出一个计算有无穷多极点的函数拉普拉斯逆变换的例子。
\begin{example}*
    设\[F(s)=\frac{1}{s^2}-\frac{\pi}{s\sinh(\pi s)}\]
    则$F(s)$在$s=ni,n\in\mathbb{Z}\backslash\{0\}$处有简单极点,而0是一个可去奇点:\[F(s)=\frac{\sinh(\pi s)-\pi s}{s^2 \sinh(\pi s)}=\frac{[(\pi s)+(\pi^3s^3/3!)+\cdots]-\pi s}{s^2 \cdot(\pi s)}=\frac{\pi^2}{3},s\to 0\]

    我们使用与上面左图类似的围道$\Gamma_r$($b>0$)来计算拉普拉斯逆变换,但这里需要选取$r=N+0.5,N\in\mathbb{N} $以保证所有围道上被积函数的全纯性。

    我们有\[\sinh\pi\left[x\pm i\left(N+\frac{1}{2}\right)\right]=\pm(-1)^N i\cosh\pi x,\left|F\left(x\pm i\left(N+\frac{1}{2}\right)\right)\right|\leq \frac{1}{N^2}+\frac{\pi}{N\cosh x}\]
    因此可以用类似上例的方法论证当$N\to\infty$时,围道的其余三条边上的积分趋于0。

    在$in(n\in\mathbb{Z}\backslash\{0\})$处,留数为
    \begin{align*}
        \text{Res}[F(s)e^{st},in]&=\lim_{s\to in}(s-in)F(s)e^{st}\\
        &=\lim_{s\to in}\frac{e^{st}}{s}\cdot\frac{-\pi(s-in)}{\sinh(\pi s)}\\
        &=\lim_{s\to in}\frac{e^{int}}{in}\cdot\frac{-1}{\cosh(\pi in)}\\
        &=\frac{(-1)^{n+1}e^{int}}{in}
    \end{align*}
    其中$\cosh(\pi in)=\cos(n\pi)=(-1)^n$。

    由留数定理,有
    \begin{align*}
        f(t)&=\mathcal{L}^{-1} F(s)\\
        &=\frac{1}{2\pi i}\lim_{N\to\infty}\oint_{\Gamma_{N+0.5}}F(s)e^{st}\,ds\\
        &=\sum_{n\in\mathbb{Z}\backslash\{0\}}\text{Res}[F(s)e^{st},in]\\
        &=\sum_{n\in\mathbb{Z}\backslash\{0\}}\frac{(-1)^{n+1}e^{int}}{in}\\
        &=2\sum_{n=1}^{\infty}\frac{(-1)^{n+1}}{n}\sin(nt)
    \end{align*}
    它是锯齿波信号$f(t)=-t,t\in(-\pi,\pi)$的周期化延拓的傅里叶级数展开。
\end{example}

\section{拉普拉斯变换的收敛域}\label{convergence}

我们在定义拉普拉斯变换时,曾要求$|f(t)|\leq C e^{at}$,这样拉普拉斯积分$\mathcal{L} f(s)=\int_{0}^{\infty}f(t)e^{-st}\,dt$在\\$\text{Re}\ s>a$上收敛,但是我们不知道这个收敛域是否是最精确的。本节就来讨论拉普拉斯变换收敛域的问题,并由此给出一些由拉普拉斯变换分析时域性质的方法。

\begin{definition}[收敛横坐标]
    \quad\\
    \textbf{收敛横坐标}为使得拉普拉斯积分$\mathcal{L} f(s)=\int_{0}^{\infty}f(t)e^{-st}\,dt$最大收敛域的收敛轴实部(横坐标)$\sigma_c$,使得$\mathcal{L} f(s)$恰好在$\text{Re}\ s>\sigma_c$上收敛,而在$\text{Re}\ s<\sigma_c$上发散,即
    \[\sigma_c=\inf\{a\in\mathbb{R}|\mathcal{L} f(s)\text{在}\text{Re}\ s>a\text{上收敛}\}\]
\end{definition}

为了寻找收敛横坐标,我们有两种思路:
\begin{itemize}
    \item 从时域出发,分析$|f(t)|$的增长速率;
    \item 从复频域出发,已知$F(s)$在$\text{Re}\ s>a$上是$f$的拉普拉斯变换,讨论它在更大的区域上是否也是$f$的拉普拉斯变换。
\end{itemize}

首先来看思路一。由于$f\in\mathcal{E} $,在有限区间上的表现不会影响拉普拉斯积分的收敛性,因此我们只需考虑$t\to\infty$时$f$的增长速率。仿照等价无穷大量,我们用$|f(t)|\sim C e^{a t}$表示$|f(t)|$与$C e^{a t}$在$t\to\infty$时具有相同增长速率,即
\[|f(t)|\sim C e^{a t}\iff \lim_{t\to\infty}\frac{|f(t)|}{C e^{a t}}=1\]
则\begin{align*}
    |f(t)|\sim C e^{a t}\iff ln|f(t)|\sim ln C +a t\iff\frac{ln|f(t)|}{t}\sim a +\frac{ln C}{t}=a
\end{align*}

但是,我们很容易构造这样一个反例来说明这个方法并不总是有效。考虑$f(t)=e^{t\sin t}$,则极限
$$\lim_{t\to\infty}\frac{ln|f(t)|}{t}=\lim_{t\to\infty}\sin t$$
不存在,但很明显$\mathcal{L} f(s)$在$\text{Re}\ s>1$上收敛。对照这个反例改进这一方法,得到:
\begin{theorem}
    设$f\in\mathcal{E} $,则收敛横坐标可由以下公式得到:
    \[\sigma_c=\varlimsup_{t\to\infty}\frac{ln|f(t)|}{t}\]
\end{theorem}
\begin{proof}
    \[\sigma_c=\varlimsup_{t\to\infty}\frac{ln|f(t)|}{t}\iff \forall \epsilon>0,\exists T>0,\text{s.t.}\forall t>T,\frac{ln|f(t)|}{t}<\sigma_c+\epsilon\]
    因此$t$充分大时,$|f(t)|<e^{(\sigma_c+\epsilon)t}$,从而$\mathcal{L} f(s)$在$\text{Re}\ s>\sigma_c+\epsilon$上收敛,$\epsilon$任意,故$\mathcal{L} f(s)$在$\text{Re}\ s>\sigma_c$上收敛。

    另一方面,如果$\exists a' <\sigma_c$,使得$\mathcal{L} f(s)$在$\text{Re}\ s>a'$上收敛,则$\mathcal{L} f(s)$在$\frac{a'+\sigma_c}{2}$上收敛,因此
    \[\int_{0}^{\infty}|f(t)|e^{-\frac{a'+\sigma_c}{2}t}\,dt<\infty\]
    由此可知,$|f(t)|e^{-\frac{a'+\sigma_c}{2}t}\to 0,t\to\infty$,即$t$充分大时,$|f(t)|<e^{\frac{a'+\sigma_c}{2}t}$,从而
    \[\varlimsup_{t\to\infty}\frac{ln|f(t)|}{t}\leq \frac{a'+\sigma_c}{2}<\sigma_c\]
\end{proof}

其实,以上的定理和证明对$\varlimsup_{t\to\infty}\frac{ln|f(t)|}{t}=\pm\infty$的情形也是成立的,它们分别对应$\mathcal{L} f(s)$无定义和$\mathcal{L} f(s)$为整函数的情形。

下面来看思路二。设$\mathcal{L} f(s)$在$\text{Re}\ s>a$上良定义且等于$F(s)$,我们讨论$F(s)$在更大的区域$\text{Re}\ s>a'(a'<a)$上是否也是$f$的拉普拉斯变换。我们在上一节的定理5.3.2.中处理过类似的情况,因此我们先来看$F(s)$满足这一定理的条件时的情形。

\begin{theorem}
    设$F(s)$是某个函数$f\in\mathcal{E} $的拉普拉斯变换,且
    $$|F(s)|\leq C|s|^{-\alpha},\alpha>1,\forall \text{Re}\ s>a'$$
    令
    \[\sigma_s=\sup\left\{\text{Re}\ s_j\in\mathbb{C}\left|s_j\text{为}F(s)\text{的奇点}\right.\right\}\]
    则有$\sigma_c=\sigma_s$。
\end{theorem}
\begin{proof}
    一方面,$\sigma_c\geq\sigma_s$。$\mathcal{L} f(s)$在$\text{Re}\ s<\sigma_s$上当然是发散的,因为$F(s)$在$\text{Re}\ s=\sigma_s$上有奇点。

    另一方面,$\sigma_c\leq\sigma_s$。用反证法。假设$\sigma_c>\sigma_s$,则在带状区域$\sigma_s<\text{Re}\ s\leq\sigma_c$上,$F(s)$满足定理5.3.2.的条件,它可以做拉普拉斯逆变换,得到唯一的时域函数,由于拉普拉斯逆变换
    $$\frac{1}{2\pi i}\lim_{r\to\infty}\int_{b-ir}^{b+ir}F(s)e^{st}\,ds$$
    不依赖于$b$的选取,该时域函数必然就是$f(t)$,从而$\mathcal{L} f(s)=F(s)$在$\text{Re}\ s>\sigma_s$上成立,矛盾。
\end{proof}

这个定理有一定的缺陷,例如即便是最基本的一类函数$f(t)=e^{at},a\in\mathbb{R} $,它的拉普拉斯变换$F(s)=\frac{1}{s-a}$也不满足定理的条件(尽管对这类函数我们可以直接验证$\sigma_c=\sigma_s=a$)。下面从另一个角度出发,给出一个类似的定理:
\begin{theorem}
    设$F(s)$是某个函数$f\in\mathcal{E} $的拉普拉斯变换,且$f(t)$单调,则有$\sigma_c=\sigma_s$($\sigma_s$定义同上)。
\end{theorem}
\begin{proof}
    一方面,我们已经说明$\sigma_c\geq\sigma_s$。

    另一方面,$\sigma_c\leq\sigma_s$。用反证法。假设$\sigma_c>\sigma_s$,取$b>\sigma_c$,由于$F(s)$在$\text{Re}\ s>\sigma_s$上全纯,可以将它展开为$b$处的泰勒级数:
    \[F(s)=\sum_{n=0}^{\infty}\frac{F^{(n)}(b)}{n!}(s-b)^n,\text{Re}\ s>\sigma_s\]
    并且该级数的收敛半径至少为$b-\sigma_s$,在$D(b,b-\sigma_s)=\{s\in\mathbb{C}||s-b|<b-\sigma_s\}$上,该级数绝对收敛、内闭一致收敛。

    容易验证$f(t)(-t)^n e^{-b t}$在$[0,\infty)$上的积分一致收敛,根据含参变量反常积分积分的求导法则,有
    \[F^{(n)}(b)=\int_{0}^{\infty}f(t)(-t)^n e^{-b t}\,dt\]
    \begin{figure}[H]
        \centering
        \includegraphics[width=0.4\textwidth]{sigma_c}
    \end{figure}
    考虑截断积分
    \begin{align*}
        \int_{0}^{R}f(t)e^{-a t}\,dt&=\int_{0}^{R}f(t)e^{-b t}\sum_{n=0}^{\infty}\frac{(b-a)^n}{n!}t^n\,dt
    \end{align*}
    由$\sum_{n=0}^{\infty}\frac{(b-a)^n}{n!}t^n\,dt$在$[0,R]$上一致收敛,可交换求和与积分次序,得到
    \begin{align*}
        \int_{0}^{R}f(t)e^{-a t}\,dt&=\sum_{n=0}^{\infty}\frac{(b-a)^n}{n!}\int_{0}^{R}f(t)t^n e^{-b t}\,dt
    \end{align*}
    由于$f(t)$单调,不妨设$f(t)\geq 0$(可以考虑$g(t)=-f(t)$,$f(t)$单调递增,剩余的工作只是处理$u(t)$),而$b>a$,级数的每一项都是非负的。由单调收敛定理,有
    \begin{align*}
        \lim_{R\to\infty}\int_{0}^{R}f(t)e^{-a t}\,dt&=\lim_{R\to\infty}\sum_{n=0}^{\infty}\frac{(b-a)^n}{n!}\int_{0}^{R}f(t)t^n e^{-b t}\,dt\\
        &=\sum_{n=0}^{\infty}\frac{(b-a)^n}{n!}\int_{0}^{\infty}f(t)t^n e^{-b t}\,dt\\
        &=\sum_{n=0}^{\infty}\frac{(b-a)^n}{n!}F^{(n)}(b)\\
        &=F(a)
    \end{align*}
    即$\mathcal{L} f(a)=F(a)$收敛,从而在$\text{Re}\ s=a$上$\mathcal{L} f$也收敛:\begin{align*}
        |\mathcal{L} f(a+i\omega)|&=\left|\int_{0}^{\infty}f(t)e^{-(a+i\omega) t}\,dt\right|\\
        &\leq \int_{0}^{\infty}\left|f(t)e^{-a t}\right|\,dt\\
        &=\int_{0}^{\infty}f(t)e^{-a t}\,dt&\left(f(t)\geq 0\right)\\
        &=\mathcal{L}f(a)<\infty
    \end{align*}
    由唯一延拓定理,$\mathcal{L} f(s)=F(s)$在$\text{Re}\ s>a$上成立。

    由$a$的任意性,我们证明了$\mathcal{L} f(s)=F(s)$在$\text{Re}\ s>\sigma_s$上成立,矛盾。
\end{proof}

为了说明定理的条件不能被去掉,我们给出以下反例:
\begin{example}
    考虑函数
    \[f(t)=\frac{\sin(e^t)}{e^t}\]
    带入前文中的公式可得$\sigma_c=-1$。但是它的拉普拉斯变换为
    \begin{align*}
        F(s)&=\int_{0}^{\infty}\frac{\sin(e^t)}{e^t}e^{-s t}\,dt\\
        &=\int_{1}^{\infty}\frac{\sin u}{u^{s+1}}\,du&(u=e^t)\\
        &=-\evalat{\frac{\cos(u)}{u^{s+1}}}{1}{\infty}+\int_{1}^{\infty}\cos(u)\,d\frac{1}{u^{s+1}}\\
        &=\cos(1)-(s+1)\int_{1}^{\infty}\frac{\cos(u)}{u^{s+2}}\,du
    \end{align*}
    由A-D判别法可知,最后一行的积分在$\text{Re}\ s>-2$上收敛,因此$F(s)$可全纯延拓至$\text{Re}\ s>-2$,进一步,多次分部积分后,可以发现$F(s)$能够延拓为整函数。
\end{example}

总之,对于相当大范围的函数,我们可以\textbf{通过分析其拉普拉斯变换的奇点来确定收敛横坐标}$\sigma_c$,从而确定拉普拉斯变换的最大收敛域。在前面我们已经证明:$$f\in\mathcal{E}\implies \sigma_c=\varlimsup_{t\to\infty}\frac{ln|f(t)|}{t}$$
因此我们可以\textbf{通过分析拉普拉斯变换的奇点来确定$f(t)$的增长速率},也可以\textbf{通过分析$f(t)$的增长速率来确定拉普拉斯变换的奇点位置}。

基于以上结论,我们可以使用拉普拉斯变换分析系统的\textbf{稳定性}。在\ref{sec:System}中曾介绍,系统的稳定性是指:
\begin{definition}
    如果一个系统在激励信号有界时,响应也是有界的,则称该系统是稳定系统。
\end{definition}

对于线性时不变系统,稳定性有充要条件:
\begin{proposition}[稳定性的时域判据]
    设线性时不变系统的冲激响应为$h(t)$,则系统稳定的充分必要条件是$\int_{-\infty}^{\infty}|h(t)|\,dt<\infty$。
\end{proposition}
\begin{proof}
    一方面,对于有界输入$|e(t)|<M$,有
    \begin{align*}
        |r(t)|&=\left|\int_{-\infty}^{\infty}h(\tau)e(t-\tau)\,d\tau\right|\\
        &\leq \int_{-\infty}^{\infty}|h(\tau)||e(t-\tau)|\,d\tau\\
        &<M\int_{-\infty}^{\infty}|h(\tau)|\,d\tau<\infty
    \end{align*}
    因此输出有界。

    另一方面,设$h(t)$不绝对可积,取有界激励信号$e(t)=\text{sgn}(h(-t))$,则
    \begin{align*}
        |r(0)|&=\left|\int_{-\infty}^{\infty}h(\tau)e(-\tau)\,d\tau\right|\\
        &=\left|\int_{-\infty}^{\infty}h(\tau)\text{sgn}(h(\tau))\,d\tau\right|\\
        &=\int_{-\infty}^{\infty}|h(\tau)|\,d\tau=\infty
    \end{align*}
    输出无界。
\end{proof}

下面使用拉普拉斯变换来分析系统的稳定性。对因果的线性时不变系统,我们定义
\begin{definition}[系统函数]
    设$h\in\mathcal{E} $是线性时不变系统的冲激响应,则系统函数定义为
    $$H(s)=\frac{R(s)}{E(s)}=\mathcal{L} h(s)$$
    如果虚轴$\text{Re}\ s=0$在$H(s)$的收敛域内,则$H(i\omega)$与第四章中定义的频率响应一致。
\end{definition}
对$r(t)=h(t)*e(t)$两边同时进行拉普拉斯变换,有
\[\mathcal{L} r(s)=\mathcal{L} h(s)\mathcal{L} e(s),\text{即}R(s)=H(s)E(s)\]

\begin{proposition}[稳定性的复频域判据]
    设因果线性时不变系统的冲激响应为$h(t)\in\mathcal{E}$,系统函数为$H(s)=\mathcal{L} h(s)$,则稳定性的充分必要条件是$H(s)$的所有奇点均位于复平面的左半平面$\text{Re}\ s<0$上。
\end{proposition}
\begin{proof}
    奇点在左半平面意味着单位冲激响应的增长速度至多相当于$e^{-\alpha t}( \alpha>0)$,从而单位冲激响应绝对可积,系统稳定;反之,奇点在右半平面或虚轴上意味着单位冲激响应增长速度至少相当于$e^{\alpha t}(\alpha\geq0)$,从而单位冲激响应不绝对可积,系统不稳定。
\end{proof}
\begin{definition}
    根据单位冲激响应的拉普拉斯变换的奇点位置,将因果线性时不变系统分为三类:
    \begin{itemize}
        \item \textbf{稳定系统}:所有奇点均位于左半平面上;
        \item \textbf{临界稳定系统}:至少有一个奇点位于虚轴上,且其余奇点均位于左半平面上;
        \item \textbf{不稳定系统}:至少有一个奇点位于右半平面上。
    \end{itemize}
\end{definition}

下面,我们介绍\textbf{初值定理}和\textbf{终值定理},它们可以通过拉普拉斯变换直接得到函数在$t=0$和$t\to\infty$时的极限值,而不需要求出函数的显式表达式。
\begin{theorem}[初值定理]
    设$f\in PS[0,\infty),f,f'\in\mathcal{E},F(s)=\mathcal{L} f(s)$,则
    \[f(0+)=\lim_{\text{Re}\ s\to\infty}sF(s)\]
\end{theorem}
这表明系统的高频响应对应时域的瞬时的初始行为。
\begin{proof}
    由拉普拉斯变换的时域微分性质,有
    \[sF(s)=s\mathcal{L} f(s)=\mathcal{L} f'(s)+f(0+)\]
    取极限$\text{Re}\ s\to\infty$,有
    \[\lim_{\text{Re}\ s\to\infty}sF(s)=\lim_{\text{Re}\ s\to\infty}\mathcal{L} f'(s)+f(0+)=\lim_{\text{Re}\ s\to\infty}\int_{0}^{\infty}f'(t)e^{-st}\,dt+f(0+)\]
    由于$f'\in\mathcal{E} $,在$\text{Re}\ s$充分大时时,$f'(t)e^{-st}\leq C e^{-\epsilon t},\epsilon$为一充分小正数。因此由控制收敛定理,有
    \[\lim_{\text{Re}\ s\to\infty}\int_{0}^{\infty}f'(t)e^{-st}\,dt=\int_{0}^{\infty}\lim_{\text{Re}\ s\to\infty}f'(t)e^{-st}\,dt=0\]
    从而\[f(0+)=\lim_{\text{Re}\ s\to\infty}sF(s)\]
\end{proof}

\begin{theorem}[终值定理]
    设$f\in\mathcal{E} $,$F(s)=\mathcal{L} f(s)$在$s=0$处有简单极点,且其余奇点均位于左半平面$\text{Re}\ s<0$上,则
    \[\lim_{t\to\infty}f(t)=\lim_{s\to 0}sF(s)\]
\end{theorem}
这表明系统的低频响应对应时域的稳态行为。
\begin{remark}
多数文献中,这个定理的允许$s=0$不是奇点,但由于$\sigma_s<0$时,多数情况下有$\sigma_c<0$,从而$f(t)\to 0$,因此这里我们只考虑非平凡情形。此外高阶极点对应的$\lim_{t\to\infty}f(t)=\infty$同样可以用以下方法证明。
\end{remark}
\begin{proof}
    考虑$F(s)$在$s=0$处的洛朗级数展开:
    \[F(s)=\sum_{n=-1}^{\infty}a_n s^n=\frac{a_{-1}}{s}+G(s),a_{-m}\neq 0\]
    其中$G(s)$在$s=0$的邻域内全纯,因此也在包含虚轴的某个右半平面$\text{Re}\ s>-\epsilon$上全纯,其拉普拉斯逆变换$g(t)$满足$g(t)\to 0,t\to\infty$,因此有\begin{align*}
        \lim_{s\to 0}sF(s)&=\lim_{s\to 0}\left(a_{-1}+sG(s)\right)=a_{-1}
    \end{align*}
    时域上,\begin{align*}
        f(t)&=\mathcal{L}^{-1} F(s)=\mathcal{L}^{-1}\left(\frac{a_{-1}}{s}\right)+\mathcal{L}^{-1} G(s)=a_{-1}u(t)+g(t)\to a_{-1},t\to\infty
    \end{align*}
    于是$\lim_{t\to\infty}f(t)=a_{-1}=\lim_{s\to 0}sF(s)$。
\end{proof}

最后,我们对\ref{sec:Laplace_Transform}中提到的双边拉普拉斯变换的收敛域进行简单讨论,以便后面与z变换进行比较。双边拉普拉斯变换具有一些与单边拉普拉斯变换不同的性质,但这里不再讨论。
\begin{definition}[双边拉普拉斯变换]
    设函数$f$局部可积,$\exists a,b\in\mathbb{R},a<b$(我们允许$a=-\infty$和$b=\infty$),使得$f(t)e^{-at}$和$f(t)e^{-bt}$均绝对可积,则对$\text{Re}\ s\in[a,b]$,$f$的双边拉普拉斯变换定义为
    \[\mathcal{L}f(s)=\int_{-\infty}^{\infty}f(t)e^{-st}\,dt\]
\end{definition}
“局部可积”即$f$在任意有限区间上可积,这个条件是为了保证在添加适当的收敛因子$e^{-st}$后,拉普拉斯积分收敛,从而拉普拉斯变换有定义。

为了确定$a,b$的取值范围,我们可以拆分双边拉普拉斯积分,并仿照单边拉普拉斯变换的收敛域分析方法,分别讨论$t\to\infty$和$t\to-\infty$时$f(t)$的增长速率:
\[\mathcal{L}f(s)=\int_{-\infty}^{0}f(t)e^{-st}\,dt+\int_{0}^{\infty}f(t)e^{-st}\,dt\]
\begin{definition}[收敛横坐标]
    \quad\\
    \textbf{收敛横坐标}为使得拉普拉斯积分$\mathcal{L}f(s)=\int_{-\infty}^{\infty}f(t)e^{-st}\,dt$最大收敛域的收敛轴实部(横坐标)$\sigma_{c}^-,\sigma_{c}^+$,使得$\mathcal{L}f(s)$恰好在$\text{Re}\ s\in(\sigma_{c}^-,\sigma_{c}^+)$上收敛,而在$\text{Re}\ s<\sigma_{c}^-$和$\text{Re}\ s>\sigma_{c}^+$上发散,即
    \[\sigma_{c}^-=\sup\{a\in\mathbb{R}|\mathcal{L} _b f(s)\text{在}\text{Re}\ s\in(a,a+\epsilon)\text{上收敛}\}\]
    \[\sigma_c^+=\inf\{b\in\mathbb{R}|\mathcal{L} _b f(s)\text{在}\text{Re}\ s\in(b-\epsilon,b)\text{上收敛}\}\]
    这里要求$\text{Re}\ s\in(a,a+\epsilon)$和$\text{Re}\ s\in(b-\epsilon,b)$是为了处理收敛域为带状域的情况。
\end{definition}
要使$\int_{0}^{\infty}f(t)e^{-st}\,dt$收敛,自然要求
\[\text{Re}\ s>\varlimsup_{t\to\infty}\frac{ln|f(t)|}{t}\]
因此可以看出$\sigma_c^-=\varlimsup_{t\to\infty}\frac{ln|f(t)|}{t}$。对于$\int_{-\infty}^{0}f(t)e^{-st}\,dt$,记
\[|f(t)|\sim C e^{b t}\iff \lim_{t\to -\infty}\frac{|f(t)|}{C e^{bt}}=1\]
则有
\[|f(t)|\sim C e^{b t}\iff ln|f(t)|\sim b t + ln C\iff \frac{|f(t)|}{t}\sim b+\frac{ln C}{t}=b\]
这时有
$$\text{Re}\ s=b-\epsilon,\int_{-\infty}^{0}f(t)e^{-st}\,dt\leq\int_{-\infty}^{0}C e^{\epsilon t}\,dt<\infty$$
因此可以看出$\sigma_c^+=\varliminf_{t\to -\infty}\frac{ln|f(t)|}{t}$(可以带入$f(t)=e^{-t\sin t}$验证这里必须取下极限)。

综上所述,我们得到:\begin{theorem}
    设函数$f$局部可积,则其双边拉普拉斯变换的收敛横坐标可由以下公式得到:
    \[\sigma_{c}^-=\varlimsup_{t\to\infty}\frac{ln|f(t)|}{t},\sigma_{c}^+=\varliminf_{t\to -\infty}\frac{ln|f(t)|}{t}\]
\end{theorem}

与单边拉普拉斯变换一样,我们允许$\sigma_{c}^-,\sigma_{c}^+=\pm\infty$。如果$\sigma_{c}^-=-\infty,\sigma_c^+=\infty$,则$\mathcal{L} _b f(s)$为整函数;如果$\sigma_{c}^-\geq\sigma_c^+$,则$\mathcal{L} _b f(s)$无定义。

特别地,我们指出:\begin{itemize}
    \item 如果$f(t)=0,t<0$,则$\sigma_{c}^-=-\infty$,双边拉普拉斯变换退化为单边拉普拉斯变换;
    \item 如果$f(t)=0,t>0$,则$\sigma_{c}^+=\infty$;
    \item 如果$f$紧支,则$\sigma_{c}^-=-\infty,\sigma_c^+=\infty$,双边拉普拉斯变换为整函数。
    \item 如果$f(t)$在$t\to\pm\infty$时均与$e^{a t}$表现类似,其双边拉普拉斯变换的收敛域为空集。除了前面三条中讨论的情况,典型的具有非空收敛域的例子如$f(t)=e^{-|t|}$,其双边拉普拉斯变换的收敛域为$\text{Re}\ s\in(-1,1)$。
\end{itemize}

不难发现,双边拉普拉斯变换的收敛性分析与洛朗级数有着诸多相似之处,我们将在\ref{sec:z_Transform}中讨论这一点。

%初值定理和终值定理

%\section{分布及其拉普拉斯变换}\label{sec:Laplace_Distribution}

%直观上,很容易理解$\mathcal{L} \delta(s)=1$,但要严格地证明它,我们需要引入分布的拉普拉斯变换的概念。仿照\ref{sec:distributions}中的做法,我们希望$\langle \mathcal{L} T,\phi\rangle=\langle T,\mathcal{L} \phi\rangle$,然而$\mathcal{L} T$定义在复频域上,我们需要明确$\langle,\rangle$在什么范围上进行。

%一种解决办法是,沿用前两个小节的思路,将拉普拉斯变换看作傅里叶变换的推广,即
%\[\mathcal{L} f(s)=\mathcal{F} [f(t)e^{-\sigma t}](\omega),s=\sigma +i\omega\]
%这里$\mathcal{F} [f(t)e^{-\sigma t}](\omega)$实际上依赖于两个变量$\sigma,\omega$,但我们只写出$\omega$,并将$\sigma$视为一个参数。这样一来,我们就可以将拉普拉斯变换推广到分布上:
%\begin{align*}
%    \langle \mathcal{L} T,\phi\rangle=\langle e^{-\sigma t}T,\mathcal{F} \phi\rangle
%\end{align*}
%我们从形式上得到了分布的拉普拉斯变换,下面逐步建立它的严格定义,并推导类似函数的拉普拉斯变换的一些性质。

%\begin{definition}[施瓦兹函数类及缓增分布]
%   在$[0,\infty)$上,定义施瓦兹函数类为
%    \[\mathcal{S} ([0,\infty))=\{\phi\in C^{\infty} ([0,\infty))|\lim_{x\to\infty}|x|^m \varphi^{(n)}(x)=0,\forall m,n\in\mathbb{N}\}\]
%    其对偶空间$\mathcal{T} ([0,\infty))$称为缓增分布空间,其元素$T$称为缓增分布,将$\varphi\in\mathcal{S} ([0,\infty))$连续、线性地映射到一个复数
%    $T(\varphi)=\langle T,\varphi\rangle\in\mathbb{C}$。
%\end{definition}
%我们同样不区分函数是否定义在$[0,\infty)$上还是$\mathbb{R} $上,只要$\varphi(t)=\varphi(t)u(t)$。正则分布$T_f,f=uf\in\mathcal{E}$的定义可以写作
%\[T_f(\varphi)=\langle T_f,\varphi\rangle=\int_{-\infty}^{\infty}f(t)\varphi(t)\,dt=\int_0^{\infty}f(t)\varphi(t)\,dt,\forall \varphi\in\mathcal{S} ([0,\infty))\]

%\begin{definition}[分布的拉普拉斯变换]
%    对分布$T\in\mathcal{T} ([0,\infty))$,如果存在实数$a$,使得$\forall\sigma>a$,$e^{-\sigma t}T\in\mathcal{T} ([0,\infty))$,则称$T$在右半平面$\text{Re}\ s>a$上可拉普拉斯变换。其拉普拉斯变换定义为:
%    \[\langle \mathcal{L} T,\phi\rangle=\langle e^{-\sigma t}T,\mathcal{F} \phi\rangle,\forall \phi\in\mathcal{S} ([0,\infty))\]
%\end{definition}

%此外,\ref{sec:distributions}中分布的导数、分布与函数的乘法、分布的共轭等概念很容易推广到$\mathcal{T} ([0,\infty))$上:
%\begin{definition}
%    设$T\in\mathcal{T} ([0,\infty))$,则其导数$T'$定义为
%    \[\langle T',\varphi\rangle=-\langle T,\varphi'\rangle\]
%    设$f\in C^{\infty} ([0,\infty))$,则分布与函数的乘法定义为
%    \[\langle f T,\varphi\rangle=\langle T,f \varphi\rangle\]
%    分布的共轭定义为
%    \[\langle \overline{T},\varphi\rangle=\overline{\langle T,\overline{\varphi}\rangle}\]
%    分布的时延定义为
%    \[\langle \tau_b T,\varphi\rangle=\langle T,u\tau_{-b}\varphi\rangle\]
%    分布的伸缩定义为
%    \[\langle \sigma_a T,\varphi\rangle=\langle T,\frac{1}{a}\sigma_{1/a}\varphi\rangle\]
%    分布的卷积定义为
%    \[\langle g*T,\varphi\rangle=\langle T,g^- *\varphi\rangle\]
%\end{definition}

\section{系统的复频域分析}\label{sec:Complex_Freq_Analysis}
%说明单边的改进形式0-,然后用它讨论微分方程和动态电路;微分性质要仔细考虑一下
与第四章的思路类似,我们可以通过系统函数来分析微分方程、动态电路,,还将介绍系统流图与梅森增益公式,下面一一进行介绍。

拉普拉斯变换是处理常系数线性微分方程的有力工具。根据拉普拉斯变换的时域微分性质:
\[\mathcal{L}\left[f^{(n)}(t)\right](s)=s^n\mathcal{L} f(s)-s^{n-1}f(0+)-s^{n-2}f'(0+)-\cdots -f^{(n-1)}(0+)\]
可以将常系数线性微分方程转化为代数方程,还能够一并处理初始条件。我们先提出一个一般的解法,再通过一个具体的例子来展示拉普拉斯变换的威力。

考虑常系数线性微分方程
\[a_n r^{(n)}(t)+a_{n-1}r^{(n-1)}(t)+\cdots +a_1 r'(t)+a_0 r(t)=e(t)\]
其中$f,r\in\mathcal{E} $,并且给定初始条件(注意区分它与\ref{sec:ODE}中初始状态的区别):
\[r(0)=c_0,r'(0)=c_1,\cdots,r^{(n-1)}(0)=c_{n-1}\]
记$\mathcal{L} r(s)=R(s),\mathcal{L} e(s)=E(s)$,则对方程两边进行拉普拉斯变换,有
\begin{align*}
    &a_n\left(s^n R(s)-s^{n-1}c_0 -s^{n-2}c_1 -\cdots -c_{n-1}\right)\\
    &+a_{n-1}\left(s^{n-1} R(s)-s^{n-2}c_0 -s^{n-3}c_1 -\cdots -c_{n-2}\right)\\
    &+\cdots +a_1\left(s R(s)-c_0\right)+a_0 R(s)=E(s)
\end{align*}
整理得
\[P(s)R(s)=E(s)+Q(s)\]
其中\begin{align*}
P(s)&=a_n s^n +a_{n-1} s^{n-1}+\cdots +a_1 s+a_0\\
Q(s)&=a_n\left(s^{n-1}c_0 +s^{n-2}c_1 +\cdots +c_{n-1}\right)+a_{n-1}\left(s^{n-2}c_0 +s^{n-3}c_1 +\cdots +c_{n-2}\right)+\cdots +a_1 c_0
\end{align*}
因此\[R(s)=\frac{E(s)}{P(s)}+\frac{Q(s)}{P(s)}=R_{zs}(s)+R_{zi}(s)\]
现在,微分方程问题就转化为求$R(s)$的拉普拉斯逆变换的问题,并且可以求出系统函数
\[H(s)=\frac{R_{zs}(s)}{E(s)}=\frac{1}{P(s)}\]

\begin{example}[常系数线性微分方程]
    设系统由如下微分方程描述:
    \[r''(t)+3r'(t)+2r(t)=e(t)\]
    初始条件为$r(0)=1,r'(0)=2$。求系统函数$H(s)$及单位阶跃响应$g(t)$。\\
    解:对方程两边进行拉普拉斯变换,有
    \begin{align*}
        \left(s^2 R(s)-s-2\right)+3\left(s R(s)-1\right)+2 R(s)=E(s)
    \end{align*}
    整理得
    \[R(s)=\frac{E(s)+s+5}{s^2+3s+2}\]
    因此系统函数为
    \[H(s)=\frac{R(s)}{E(s)}=\frac{1}{s^2+3s+2}=\frac{1}{s+1}-\frac{1}{s+2}\]
    做拉普拉斯反变换得到(零状态)单位脉冲响应:
    \begin{align*}
        h(t)&=\mathcal{L}^{-1} \left[\frac{1}{s+1}-\frac{1}{s+2}\right](t)\\
        &=(e^{-t}-e^{-2t})u(t)
    \end{align*}
    积分得到零状态响应:
    \begin{align*}
    r_{zs}(t)=\int_0^{t}h(x)\,dx=(0.5+e^{-t}-0.5e^{-2t})u(t)
    \end{align*}
    对于零输入响应,有
    \begin{align*}
        R_{zi}(s)&=\frac{s+5}{s^2+3s+2}\\
        &=\frac{1.5}{s+1}-\frac{0.5}{s+2}
    \end{align*}
    反变换即得$r_{zi}(t)=(1.5e^{-t}-0.5e^{-2t})u(t)$,因此全响应为$r(t)=(1.5+e^{-t}-1.5e^{-2t})u(t)$。

    此外,也可以用$\mathcal{L} u(s)=1/s$直接计算单位阶跃响应得全响应:
    \begin{align*}
        R(s)&=\frac{1}{s(s^2+3s+2)}+\frac{s+5}{s^2+3s+2}\\
        &=\frac{1.5}{s}+\frac{1}{s+1}-\frac{1.5}{s+2}
    \end{align*}
    反变换即得$r(t)=(1.5+e^{-t}-1.5e^{-2t})u(t)$。
\end{example}

下面介绍动态电路的复频域分析方法。与\ref{sec:freq_response}节类似,对描述线性元件的微分方程做拉普拉斯变换,即可得到它们的复频域模型:
\begin{example}
    描述电容的方程为:
    \[i(t)=C\frac{dv(t)}{dt}\]
    对其做拉普拉斯变换,得到
    \[I(s)=C\left(sV(s)-v(0)\right)\]
    由于$v(0)$未必为0,不能直接用阻抗$1/sC$来描述电容,需要引入电容的初始电压源模型:
    \[V(s)=\frac{1}{sC}I(s)+\frac{v(0)}{s}\]
    \begin{center}
        \begin{circuitikz}[american]
            \draw (0,0) to[short, o-] (1,0)
            to[C=$1/sC$] (3.5,0)
            to[V=$v(0)/s$] (6,0)
            to[short, -o] (7,0);
        \end{circuitikz}
    \end{center}
    描述电感的方程为:
    \[v(t)=L\frac{di(t)}{dt}\]
    对其做拉普拉斯变换,得到
    \[V(s)=L\left(sI(s)-i(0)\right)\]
    引入电感的初始电流源模型:
    \begin{center}
        \begin{circuitikz}[american]
            \draw (0,0) to[short, o-] (1,0)
            to[L=$sL$] (3.5,0)
            to[I=$Li(0)$] (6,0)
            to[short, -o] (7,0);
        \end{circuitikz}
    \end{center}
    在零状态条件下,初始电压源/电流源均为0,可以直接使用阻抗模型。
    \begin{remark}
        一些文献中,使用$\mathcal{L} f(s)=\int_{0-}^{\infty}f(t)e^{-st}\,dt$作为拉普拉斯变换的定义,这样得到的初始电压源/电流源模型的值为$v(0-)/s$和$i(0-)/s$,但由于电容电压连续、电感电流连续,两种模型实质上是一样的。
    \end{remark}
\end{example}

作为一个例子,我们使用复频域分析的方法来处理一个二阶动态电路:
\begin{example}[二阶动态电路]
    设电路如下图所示,起始状态为0,$t=0$时刻开关闭合,求电流$i(t)$。
    \begin{center}
        \begin{circuitikz}[american]
            \draw (0,0) to[V=$E$] (0,3)
            to[closing switch] (2.5,3)
            to[L=$L$, i=$i(t)$] (5,3)
            to[C=$C$] (5,0)
            to[R=$R$] (0,0);
        \end{circuitikz}
    \end{center}
    解:采用复频域模型:
    \begin{center}
        \begin{circuitikz}[american]
            \draw (0,0) to[V=$E/s$] (0,3)
            to[short] (2.5,3)
            to[L=$sL$, i=$I(s)$] (5,3)
            to[C=$1/sC$] (5,0)
            to[R=$R$] (0,0);
        \end{circuitikz}
    \end{center}
    由基尔霍夫电压定律,有
    \[\frac{E}{s}=I(s)\left(R+sL+\frac{1}{sC}\right)\]
    因此
    \[I(s)=\frac{E}{s\left(R+sL+\frac{1}{sC}\right)}=\frac{E}{Ls^2+Rs+\frac{1}{C}}\]
    引入参数
    \[\alpha=\frac{R}{2L},\omega_0=\frac{1}{\sqrt{LC}}\]
    则
    \[I(s)=\frac{E/L}{s^2+2\alpha s+\omega_0^2}=\frac{E/L}{(s+\alpha)^2+\omega_0^2-\alpha^2}\]
    \ding{172}设$\alpha>\omega_0$,即$R^2-4L/C>0$,则
    \[I(s)=\frac{E/L}{(s-r_1)(s-r_2)}\]
    其中$$r_1=\frac{-R+\sqrt{R^2-4L/C}}{2L}=-\alpha+\beta,r_2=\frac{-R-\sqrt{R^2-4L/C}}{2L}=-\alpha-\beta,\beta=\frac{\sqrt{R^2-4L/C}}{2L}$$
    做拉普拉斯逆变换,得到
    \[i(t)=\frac{E}{L(r_1-r_2)}\left(e^{r_1 t}-e^{r_2 t}\right)u(t)=\frac{E}{2L\beta}\left[e^{(-\alpha+\beta) t}-e^{(-\alpha-\beta) t}\right]u(t)\]
    系统处于过阻尼状态,电流为两项指数衰减的叠加,缓慢无振荡地回到平衡位置。\\
    \ding{173}设$\alpha=\omega_0$,即$R^2-4L/C=0$,则
    \[I(s)=\frac{E}{L}\cdot\frac{1}{(s+\alpha)^2}\]
    做拉普拉斯逆变换,得到
    \[i(t)=\frac{E}{L}te^{-\alpha t}u(t)\]
    系统处于临界阻尼状态,电流快速无振荡地回到平衡位置。\\
    \ding{174}设$\alpha<\omega_0$,即$R^2-4L/C<0$,则
    \[I(s)=\frac{E}{L}\cdot\frac{1}{(s+\alpha)^2+\beta^2}\]
    做拉普拉斯逆变换,得到
    \[i(t)=\frac{E}{L\beta}e^{-\alpha t}\sin(\beta t)u(t)\]
    系统处于欠阻尼状态,电流以衰减振荡的形式缓慢回到平衡位置。
\end{example}
\begin{figure}[H]
    \centering
    \includegraphics[width=0.8\textwidth]{damp}
\end{figure}
%零极点特性与高通低通
%系统流图与梅森增益公式

\chapter{离散时间变换与离散系统分析}

\section{z变换}\label{sec:z_Transform}

中学数学中处理递推数列的一种经典方法是使用生成函数,将数列转化为形式幂级数,从而利用级数的性质来分析数列。z变换可以看作生成函数的推广,它将数列推广为离散时间信号,将形式幂级数推广为洛朗级数,从而可以利用复变函数的理论来分析离散时间信号与离散系统。
\begin{definition}[z变换]
    离散时间信号$x[n]$的z变换定义为
    \[X(z)=\mathcal{Z} x[n]=\sum_{n=-\infty}^{\infty}x[n]z^{-n},z\in\mathbb{C}\]
    我们记$x[n]\overset{\mathcal{Z} }{\longleftrightarrow}X(z)$。
\end{definition}

\begin{example}[单位脉冲序列的z变换]
    设$x[n]=\delta[n-k],k\in\mathbb{Z}$,则
    \[X(z)=\sum_{n=-\infty}^{\infty}\delta[n-k]z^{-n}=z^{-k},z\in\mathbb{C}\backslash\{0\}\]
\end{example}
\begin{example}[单位阶跃序列的z变换]
    设$x[n]=u[n]$,则
    \[X(z)=\sum_{n=0}^{\infty}z^{-n}=\frac{1}{1-z^{-1}}=\frac{z}{z-1},|z|>1\]
\end{example}
\begin{example}[指数序列的z变换]
    设$x[n]=a^n u[n],a\in\mathbb{C}$,则
    \[X(z)=\sum_{n=0}^{\infty}a^n z^{-n}=\frac{1}{1-a z^{-1}}=\frac{z}{z-a},|z|>|a|\]
\end{example}
\begin{example}[单边余弦序列的z变换]
    \begin{align*}
        \mathcal{Z} \left[\cos(\Omega n)u[n]\right]&=\frac{1}{2}\left(\mathcal{Z} \left[e^{i\Omega n}u[n]\right]+\mathcal{Z} \left[e^{-i\Omega n}u[n]\right]\right)\\
        &=\frac{1}{2}\left(\frac{z}{z-e^{i\Omega}}+\frac{z}{z-e^{-i\Omega}}\right)\\
        &=\frac{z(z-\cos\Omega)}{z^2-2z\cos\Omega+1},|z|>1
    \end{align*}
\end{example}
\begin{example}[单边正弦序列的z变换]
    \begin{align*}
        \mathcal{Z} \left[\sin(\Omega n)u[n]\right]&=\frac{1}{2i}\left(\mathcal{Z} \left[e^{i\Omega n}u[n]\right]-\mathcal{Z} \left[e^{-i\Omega n}u[n]\right]\right)\\
        &=\frac{1}{2i}\left(\frac{z}{z-e^{i\Omega}}-\frac{z}{z-e^{-i\Omega}}\right)\\
        &=\frac{z\sin\Omega}{z^2-2z\cos\Omega+1},|z|>1
    \end{align*}
\end{example}

根据\ref{sec:Complex}中介绍的洛朗级数理论,z变换的收敛域为一个环状区域:
\begin{definition}[z变换的收敛域]
    设离散时间信号$x[n]$的z变换为$X(z)=\sum_{n=-\infty}^{\infty}x[n]z^{-n}$,则其\textbf{收敛域}(region of convergence, ROC)定义为
    \[\text{ROC}=\left\{z\in\mathbb{C} \left|r<|z|<R\right.\right\}\]
    其中
    \[r=\varlimsup_{n\to\infty}|x[-n]|^{\frac{1}{n}},R=\left(\varlimsup_{n\to\infty}|x[n]|^{\frac{1}{n}}\right)^{-1}\]
\end{definition}
以上定义对于$r=0$或$R=\infty$或$r\geq R$的退化情况同样适用。特别地,我们指出:
\begin{itemize}
    \item 若 \( x[n] = 0,\ n < 0 \),则 \( r = 0 \),双边 z 变换退化为单边 z 变换;
    \item 若 \( x[n] = 0,\ n > 0 \),则 \( R = \infty \);
    \item 若 \( x[n] \) 是有限长序列,则 \( r = 0,\ R = \infty \),z 变换在整个复平面(除 \( z = 0 \) 和 \( z = \infty \) 外)解析;
    \item 若 \( |x[n]| \) 在 \( n \to \pm \infty \) 时分别以 \( |a|^n \) 的速率增长,则其双边 z 变换的收敛域可能为空。除了前面三种情况外,典型的非空收敛域例子如 \( x[n] = \alpha^{|n|} \)(\( |\alpha| < 1 \)),其收敛域为 \( |\alpha| < |z| < \frac{1}{|\alpha|} \)。
\end{itemize}

和洛朗级数一样,更常用的确定收敛域的方法是分析$X(z)$的奇点位置,并结合信号特征确定收敛域的范围。
\begin{definition}
    \textbf{左边信号(反因果信号)}:$x[n]=0,n\geq 0$,或$x[n]=x[n]u[-n-1]$;\\
    \textbf{右边信号(因果信号)}:$x[n]=0,n<0$,或$x[n]=x[n]u[n]$;\\
    \textbf{双边信号}:既不是左边信号,也不是右边信号。
\end{definition}
\begin{example}[收敛域的确定]
    设$$X(z)=\frac{1}{1-z}+\frac{1}{2z-1}$$
    在离散信号$x[n]$分别是左边、右边和双边信号时,求$x[n]$及其收敛域。\\
    解:$X(z)$的奇点为$z=1,1/2$。\\
    \ding{172}设$x[n]$为左边信号,$X(z)$只有正幂次项,则收敛域为$|z|<1/2$。
    \begin{align*}
        X(z)=&\frac{1}{1-z}+\frac{1}{2z-1}\\
        =&\frac{1}{1-z}-\frac{1}{1-2z}\\
        =&\sum_{n=0}^{\infty}z^{n}-\sum_{n=0}^{\infty}\left(2z\right)^{n}\\
        =&\sum_{n=0}^{\infty}\left(1-2^{n}\right)z^{n}
    \end{align*}
    因此$x[n]=(1-2^{n})u[-n-1]$。\\
    \ding{173}设$x[n]$为右边信号,$X(z)$只有负幂次项,则收敛域为$|z|>1$。
    \begin{align*}
        X(z)=&-\frac{1/z}{1-\frac{1}{z}}+\frac{1/2z}{1-\frac{1}{2z}}\\
        =&-\sum_{n=1}^{\infty}z^{-n}+\sum_{n=1}^{\infty}\left(2z\right)^{-n}\\
        =&\sum_{n=1}^{\infty}\left(2^{-n}-1\right)z^{-n}
    \end{align*}
    因此$x[n]=(2^{-n}-1)u[n]$。\\
    \ding{174}设$x[n]$为双边信号,则收敛域为$1/2<|z|<1$。 
    \begin{align*}
        X(z)=&\frac{1}{1-z}+\frac{1/2z}{1-\frac{1}{2z}}\\
        =&\sum_{n=0}^{\infty}z^{n}-\sum_{n=1}^{\infty}\left(2z\right)^{-n}\\
        =&\sum_{n=0}^{\infty}z^{n}+\sum_{n=-\infty}^{-1}\left(2^{-n}-1\right)z^{n}
    \end{align*}
    因此$x[n]=u[n]+(2^{-n}-1)u[-n-1]$。

    可以看到,同一个复变函数在不同的域上展开,对应不同的表达式,并且在不同类型的收敛域上对应不同类型的信号,其共性是可以用奇点位置判断收敛域。这种凑$\frac{1}{1-z}$的方法属于常用技术,需要掌握。
\end{example}

下面我们来推导z变换的一些运算性质。
\begin{proposition}[z变换的运算性质]
    设$x[n],y[n]$的z变换分别为$X(z),Y(z)$,则有:
    \begin{itemize}
        \item 线性:在$X(z)$与$Y(z)$共同的收敛域内,$a x[n]+b y[n]\xleftrightarrow{\mathcal{Z}} a X(z)+b Y(z)$
        \item 移位:$x[n-n_0]\xleftrightarrow{\mathcal{Z}} z^{-n_0}X(z)$
        \item 共轭:$\overline{x[n]}\xleftrightarrow{\mathcal{Z}} \overline{X(\overline{z})}$
        \item 指数加权:$a^{n} x[n]\xleftrightarrow{\mathcal{Z}} X\left(\frac{z}{a}\right)(z\neq 0)$,收敛域变为$|a|r<|z|<|a|R$
        \item 反转:$x[-n]\xleftrightarrow{\mathcal{Z}} X\left(\frac{1}{z}\right)$
        \item 升采样:$x[n/L]\xleftrightarrow{\mathcal{Z}} X\left(z^{L}\right)$,收敛域变为$r^{1/L}<|z|<R^{1/L}$
        \item 降采样:$x[nM]\xleftrightarrow{\mathcal{Z}} \frac{1}{M}\sum_{k=0}^{M-1}X\left(z^{\frac{1}{M}}e^{-i\frac{2\pi}{M}k}\right)$,收敛域为$r^{M}<|z|<R^{M}$
        \item z域微分:$n x[n]\xleftrightarrow{\mathcal{Z}} -z\frac{dX(z)}{dz}$
    \end{itemize}
\end{proposition}
\begin{proof}
    我们仅给出后两条性质的证明。\\
    1.降采样:
    \begin{align*}
        \mathcal{Z}\left[x[nM]\right]=\sum_{n=-\infty}^{\infty}x[nM]z^{-n}=\sum_{m=nM}x[m]z^{-\frac{m}{M}}\quad(m=nM)
    \end{align*}
    为了处理$m$只能取$n$的整数倍的情况,我们引入\textbf{选择函数}:
    \[\shah_M[n]=\frac{1}{M}\sum_{k=0}^{M-1}e^{i\frac{2\pi}{M}kn}=\begin{cases}
        1 & \text{if }m=nM\\
        0 & \text{otherwise}
    \end{cases}\]
    可以利用等比数列求和公式验证上述结论。带入上式并交换求和次序,得到
    \begin{align*}
        \mathcal{Z}\left[x[nM]\right]&=\sum_{m=-\infty}^{\infty}x[m]z^{-\frac{m}{M}}\shah_M[m]\\
        &=\frac{1}{M}\sum_{k=0}^{M-1}\sum_{m=-\infty}^{\infty}x[m]\left(z^{\frac{1}{M}}e^{-i\frac{2\pi}{M}k}\right)^{-m}\\
        &=\frac{1}{M}\sum_{k=0}^{M-1}X\left(z^{\frac{1}{M}}e^{-i\frac{2\pi}{M}k}\right)
    \end{align*}
    对每个$X\left(z^{\frac{1}{M}}e^{-i\frac{2\pi}{M}k}\right)$,收敛域为
    \[r<\left|z^{\frac{1}{M}}e^{-i\frac{2\pi}{M}k}\right|< R\iff r^{M}<|z|<R^{M}\]
    2.z域微分:在洛朗级数的收敛域内,级数时内闭一致收敛的,因此可以逐项求导,即
    \begin{align*}
        -z\frac{dX(z)}{dz}&=-z\frac{d}{dz}\left(\sum_{n=-\infty}^{\infty}x[n]z^{-n}\right)\\
        &=-z\sum_{n=-\infty}^{\infty}x[n](-n)z^{-n-1}\\
        &=\sum_{n=-\infty}^{\infty}n x[n]z^{-n}
    \end{align*}
\end{proof}

此外,对于离散信号,定义:
\begin{definition}[卷积和]
    设$x[n],y[n]$为离散信号,则它们的\textbf{卷积和}定义为
    \[(x*y)[n]=\sum_{k=-\infty}^{\infty}x[k]y[n-k]\]
\end{definition}
类似连续信号的卷积定理,有:
\begin{proposition}[卷积和定理]
    在$X(z)$与$Y(z)$共同的收敛域内,有
    \[x[n]*y[n]\xleftrightarrow{\mathcal{Z}} X(z)Y(z)\]
\end{proposition}
%z域卷积定理
\begin{proof}
    在$X(z)$与$Y(z)$共同的收敛域内,有
    \begin{align*}
        \mathcal{Z} \left[x[n]*y[n]\right]&=\sum_{n=-\infty}^{\infty}\left(\sum_{k=-\infty}^{\infty}x[k]y[n-k]\right)z^{-n}\\
        &=\sum_{k=-\infty}^{\infty}x[k]z^{-k}\sum_{m=-\infty}^{\infty}y[m]z^{-m}\quad(m=n-k)\\
        &=X(z)Y(z)
    \end{align*}
\end{proof}

一般而言,离散信号在做线性组合或卷积和时,其z变换的收敛域为各信号z变换收敛域的交集,但如果涉及到零极点的抵消,则收敛域可能会扩大。
\begin{example}[收敛域的扩张]
    设$x_1[n]=a^{n}u[n],x_2[n]=\delta[n]-a^{n}u[n]$,则
    \begin{align*}
        X_1(z)&=\sum_{n=0}^{\infty}a^{n}z^{-n}=\frac{z}{z-a},|z|>|a|\\
        X_2(z)&=1-\sum_{n=0}^{\infty}a^{n}z^{-n}=1-\frac{z}{z-a}=\frac{a}{a-z},|z|>|a|
    \end{align*}
    但是$x[n]=x_1[n]+x_2[n]=\delta[n]$,其z变换为$X(z)=1$,收敛域为$\mathbb{C}$。\\
    设$y_1[n]=\delta[n]-a\delta[n-1],y_2[n]=a^{n}u[n-1]$,则
    \begin{align*}
        Y_1(z)&=1-a z^{-1},z\in\mathbb{C}\backslash\{0\}\\
        Y_2(z)&=\sum_{n=1}^{\infty}a^{n}z^{-n}=\frac{1}{1-a z^{-1}},|z|>|a|
    \end{align*}
    但是\begin{align*}
        y[n]=\left(y_1*y_2\right)[n]=\sum_{k=-\infty}^{\infty}y_1[k]y_2[n-k]=y_2[n]-a y_2[n-1]=a^{n}u[n-1]-a^{n}u[n-2]=\delta[n]
    \end{align*}
    其z变换为$Y(z)=1$,收敛域为$\mathbb{C}$。
\end{example}

\section{亚纯函数与逆z变换}\label{sec:Inverse_z_Transform}

逆z变换即由$X(z)$求$x[n]$,根据洛朗级数展开的唯一性,容易想到有两种常用的方法:
\begin{itemize}
    \item 将$X(z)$展开为洛朗级数,直接读取系数;
    \item 利用留数的计算公式,$z^{n-1} X(z)=\sum_{m=-\infty}^{\infty} x[m] z^{n-m-1}$在$z=0$处的留数即为$x[n]$
\end{itemize}

在前面的例子6.1.6.中,我们已经展示了前一种方法,但是在$X(z)$收敛域不同的情况下展开,我们得到了三种洛朗级数,这似乎与洛朗级数的唯一性矛盾;在$x[n]$为右边信号的情况下,展开式的负幂次项构成级数,$z=0$为本性奇点,这与原信号在$z=0$处全纯的结果矛盾。实际上,这是因为在收敛域不是$|z|<1/2$的情况下,我们将分式$1/(2z-1)$展开为负幂次项级数时,是在做$z=\infty$处的洛朗展开,这样展开得到的洛朗级数也不能用来判定$z=0$处的奇点类型。

为了说明这个问题,我们需要引入亚纯函数的概念:
\begin{definition}[亚纯函数]
    设$U\subset\mathbb{C}$为开集,$S\subset U$没有极限点,函数$f:U\backslash S\to\mathbb{C}$称为\textbf{亚纯函数},如果$f$在$U\backslash S$内全纯,且在$S$中没有本性奇点。
\end{definition}

亚纯函数是全纯函数的推广,它允许孤立极点和可去奇点的存在。如果将开集$U\subset \mathbb{C}$上的全纯函数集视为环$\mathcal{O} (U)$,则亚纯函数集构成$\mathcal{O} (U)$的分式域,记为$\mathcal{M} (U)$。

在扩展复平面$\hat{\mathbb{C}}=\mathbb{C}\cup\{\infty\}$上,我们定义:
\begin{definition}[无穷远点的邻域]
    无穷远点$\infty$的邻域定义为
    \[U(\infty)=\left\{\left.z\in\hat{\mathbb{C}} \right||z|>R\right\}\cup\{\infty\},R>0\]
    $\infty$是集合$S\cup\{\infty\}$的极限点当且仅当$S$中存在点列$\{z_n\}$,使得$\lim_{n\to\infty}|z_n|=\infty$。
\end{definition}

\begin{definition}[无穷远点的奇点类型]
    定义$g(z)=f(1/z)$,则$f$在$\infty$处的奇点类型与$g$在$0$处的奇点类型相同,即:
    \begin{itemize}
        \item 若$f(1/w)$在$w=0$处有可去奇点,则称$f$在$\infty$处有可去奇点;
        \item 若$f(1/w)$在$w=0$处有极点,则称$f$在$\infty$处有极点;
        \item 若$f(1/w)$在$w=0$处有本性奇点,则称$f$在$\infty$处有本性奇点。
    \end{itemize}
\end{definition}

我们称$g(z)$在$0$处的洛朗展开为$f(z)$在$\infty$处的洛朗展开,并根据洛朗级数的类型,判定奇点类型:
\begin{proposition}
    设$g(z)$在$z=0$处的洛朗展开为
    \[g(z)=\sum_{n=-\infty}^{\infty}c_n z^{n}\]
    则$g(z)$在$z=0$处的奇点类型定义与\ref{sec:Complex}中一致。代回$f(z)=g(1/z)$,则$f(z)$在$\infty$处的洛朗展开为
    \[f(z)=\sum_{n=-\infty}^{\infty}c_n z^{-n}\]
    因此,$f(z)$在$\infty$处的奇点类型为:
    \begin{itemize}
        \item 如果$c_n=0,\forall n<0$,,即级数无正幂次项,则$f(z)$在$\infty$处有可去奇点;
        \item 如果存在$m\in\mathbb{N}_+$,使得$c_{-1}=c_{-2}=\cdots =c_{-m}=0,c_{-(m+1)}\neq 0$,即级数正幂次项有限,则$f(z)$在$\infty$处有极点;
        \item 如果$c_n\neq 0$对无穷多个负整数$n$成立,即级数正幂次项无限,则$f(z)$在$\infty$处有本性奇点。
    \end{itemize}
\end{proposition}

可以看出,如果$f(z)$在$\infty$处的洛朗级数收敛域必为$|z|>R$的形式,它与$z=0$处的洛朗级数展开式一般不一致,并且不能用于判定$z=0$处的奇点类型。基于以上定义,我们可以给出扩展复平面上的亚纯函数定义:
\begin{definition}[扩展复平面上的亚纯函数]
    设$U\subset\hat{\mathbb{C}}$为开集,$S\subset U$没有极限点,函数$f:U\backslash S\to\hat{\mathbb{C}}$称为\textbf{亚纯函数}(meromorphic functions),如果$f$在$U\backslash S$内全纯,且在$S$中没有本性奇点。
\end{definition}
\begin{example}
    对于有理函数$f(z)=\dfrac{P(z)}{Q(z)}$,其中$P(z),Q(z)$为多项式,$f(z)$在$\hat{\mathbb{C}}$上亚纯。其中,如果$\text{deg}P\leq\text{deg}Q$,则$f(z)$在$\infty$处有可去奇点;如果$\text{deg}P>\text{deg}Q$,则$f(z)$在$\infty$处有$\text{deg}P-\text{deg}Q$阶极点。

    事实上,扩充复平面上“亚纯函数”等价于“有理函数”。
\end{example}

现在也就不难理解为什么上一节的例子6.1.6中,在不同收敛域上展开的洛朗级数会有不同的形式,并且在某些收敛域上无法判定$z=0$处的奇点类型了——在$|z|>1/2$的收敛域上,我们其实是将$1/(2z-1)$展开为$z=\infty$处的洛朗级数,这个级数只能说明$1/(2z-1)$在$\infty$处为可去奇点,而不能用于判断$z=0$处的奇点类型。

经验上,在做逆z变换时,我们更多地使用洛朗级数展开,直接对比系数的方法,这种方法可以处理本性奇点的情况。
\begin{example}
    设$X(z)=e^{1/z}$,求$x[n]$。\\
    解:$X(z)$在$z=0$处有本性奇点,因此不能使用留数计算(注意我们获得留数计算公式的方法,是将洛朗级数乘以$z^{n-1}$,从而化为泰勒级数)。但它的展开式很容易获得:
    \[X(z)=\sum_{k=0}^{\infty}\frac{1}{k!}z^{-k}\]
    因此$x[n]=\frac{1}{n!}u[-n-1]$。可以验证其收敛域为$\hat{\mathbb{C}}\backslash\{0\}$,并且在$\infty$处展开与上式一致(在没有$0$以外的奇点时,这是必然的,因为同一收敛域上的洛朗级数展开唯一)。
\end{example}

如果$X(z)$是亚纯函数,计算留数仍然是一个有效的方法,因为要将有理函数展开为部分分式的计算量可能比较大(见\ref{sec:Laplace_inverse})。通过变量代换或直接对比洛朗级数,我们可以得到无穷远点的留数计算公式(注意无穷远点的留数指展开式中$1/z$的系数,只有它会在围道积分之后留下来):

\begin{claim}[无穷远点的留数计算公式]
    设$f(z)$在$\infty$处有孤立奇点,则$f(z)$在$\infty$处的留数为
    \[\text{Res}(f,\infty)=-\text{Res}\left(\frac{f(1/z)}{z^2},0\right)\]
\end{claim}
在$X(z)z^{n-1}$中,对比$1/z$项系数,就得到使用该公式计算右边信号的逆z变换公式:
\begin{claim}
    设右边离散信号$x[n]$的z变换为$X(z)$,则有
    \[x[n]=-\text{Res}\left(X(z)z^{n-1},\infty\right)=\text{Res}\left(\frac{X(1/z)}{z^{n+1}},0\right)\]
\end{claim}

要使用$\text{Res}\left(\dfrac{X(1/z)}{z^{n+1}}\right)$,逐阶计算是非常繁琐的,需要求高阶极点的留数,实际上我们一般会使用留数定理将问题转化为其他极点的留数计算。注意我们并不打算算出积分,留数定理只是从一些极点到另一些极点的桥梁.
\begin{example}
    设右边信号$x[n]$的z变换为
    \[X(z)=\frac{z^2}{(z-1)^3},\text{ROC}:|z|>1\]
    求$x[n]$。\\
    解:$X(z)$在$\infty$处有极点,因此可以使用留数定理:
    \[x[n]=-\text{Res}\left(X(z)z^{n-1},\infty\right)=-\frac{1}{2\pi i}\oint_{C^-_R}X(z)z^{n-1}\,dz\]
    其中$C^-_R$为以原点为中心、半径$R$充分大的\textbf{顺时针}圆周,使得$X(z)$的所有极点都在$C^-_R$内。顺时针是为了使它绕$\infty$的环绕数$\text{Ind}_{C^-_R}(\infty)=1$。由于$C^-_R$绕$z=1$的环绕数为$-1$,根据留数定理,有
    \[-\oint_{C^-_R}X(z)z^{n-1}\,dz=\oint_{C_R}X(z)z^{n-1}\,dz=2\pi i \text{Res}\left(X(z)z^{n-1},1\right)\]
    现在问题转化为一个三阶极点的留数求解,而且只需要计算一次:
    \begin{align*}
        \text{Res}\left(X(z)z^{n-1},1\right)&=\lim_{z\to 1}\frac{1}{2!}\frac{d^2}{dz^2}\left[(z-1)^3\frac{z^{n+1}}{(z-1)^3}\right]\\
        &=\lim_{z\to 1}\frac{1}{2}\frac{d^2}{dz^2}z^{n+1}\\
        &=\frac{(n+1)n}{2}
    \end{align*}
    因此$$x[n]=-\text{Res}\left(X(z)z^{n-1},1\right)=-\frac{(n+1)n}{2}u[n]$$    
\end{example}

左边信号对应经典的泰勒级数情况,我们已经在前面说明,$x[n]=\text{Res}\left(X(z)z^{n-1},0\right)$。不过实际计算时,使用留数定理将高阶极点的留数转化为其他低阶极点的留数会更加方便:
\begin{example}
    设左边信号$x[n]$的z变换为
    \[X(z)=\frac{z^2}{(z-1)^3},\text{ROC}:|z|<1\]
    求$x[n]$。\\
    解:由留数定理:
    \[x[n]=\text{Res}\left(X(z)z^{n-1},0\right)=\frac{1}{2\pi i}\oint_{C_r}X(z)z^{n-1}\,dz\]
    其中$C_r$为以原点为中心、半径$r$充分小的逆时针圆周,使得$X(z)$的其他极点都在$C_r$外。在$n\geq -1$时,$X(z)z^{n-1}$在$z=0$处为可去奇点,留数为0;在$n<-1$时,$X(z)z^{n-1}$在$z=0$处有$-(n-1)$阶极点,我们将其转化为围道外的极点留数计算。由于$\text{Ind}_{C_r}(1)=\text{Ind}_{C_r}(\infty)=-1$,根据留数定理,有
    \[\oint_{C_r}X(z)z^{n-1}\,dz=-2\pi i \left[\text{Res}\left(X(z)z^{n-1},1\right)+\text{Res}\left(X(z)z^{n-1},\infty\right)\right]\]
    现在问题转化为两个极点的留数求解。前者与上例相同,后者可以使用无穷远点的留数计算公式:
    \begin{align*}
        \text{Res}\left(X(z)z^{n-1},\infty\right)&=-\text{Res}\left(\frac{X(1/z)}{z^{n+1}},0\right)=-\text{Res}\left(\frac{1}{(1-z)^3 z^{n-1}},0\right)
    \end{align*}
    在$n<-1$时,$\dfrac{1}{(1-z)^3 z^{n-1}}$在$z=0$处全纯,留数为0。因此,
    \begin{align*}
        x[n]&=-\left[\text{Res}\left(X(z)z^{n-1},1\right)+\text{Res}\left(X(z)z^{n-1},\infty\right)\right]=-\frac{(n+1)n}{2}u[-n-1]
    \end{align*}
    其实,由于$\text{deg}\ z^{n+1}\leq \text{deg}\ (z-1)^3$,$X(z)z^{n-1}$在$\infty$处为可去奇点,这种情况下可以直接断定无穷远点的留数为0。
\end{example}

总之,设离散信号$x[n]$的z变换为$X(z)$,则有:
\begin{itemize}
    \item 如果$x[n]$为左边信号,则
    \[x[n]=\text{Res}\left(X(z)z^{n-1},0\right)=\oint_{C_r}X(z)z^{n-1}\,dz\]
    \item 如果$x[n]$为右边信号,则
    \[x[n]=-\text{Res}\left(X(z)z^{n-1},\infty\right)=-\oint_{C^-_R}X(z)z^{n-1}\,dz=\oint_{C_R}X(z)z^{n-1}\,dz\]
\end{itemize}

根据我们先前对双边信号的处理经验,可以想到,如果我们能够将$X(z)$拆成左边信号和右边信号的z变换之和,那么就可以分别计算它们的逆z变换,再将结果相加。对于我们最常遇到的有理函数,这总是可行的,因为理论上可以对部分分式展开式\textbf{分组通分}。
\begin{example}
    设双边信号的z变换为
    \[X(z)=\frac{1}{(z-1)^3(z-2)},\text{ROC}:1<|z|<2\]
    求$x[n]$。\\
    解:考虑将$X(z)$拆成左边信号和右边信号的z变换之和:
    \begin{align*}
        X(z)&=\frac{1}{(z-1)^3(z-2)}=\frac{A}{z-2}+\frac{B}{z-1}+\frac{C}{(z-1)^2}+\frac{D}{(z-1)^3}\\
        &=\frac{A}{z-2}+\frac{B(z-1)^2+C(z-1)+D}{(z-1)^3}\\
        &=\frac{A}{z-2}+\frac{B'z^2+C'Z+D'}{(z-1)^3}
    \end{align*}
    容易求得$A=1$,做减法并约去公因式$(z-2)$,即得分解式的另一部分:
    \[X(z)=\frac{1}{z-2}-\frac{z^2-z+1}{(z-1)^3}\]
    很明显,第一项对应左边信号,第二项对应右边信号。分别计算它们的逆z变换:
    \begin{align*}
        x_{left}[n]&=\text{Res}\left(\frac{z^{n-1}}{z-2},0\right)\\
        &=\frac{1}{2\pi i}\oint_{C_r}\frac{z^{n-1}}{z-2}\,dz\\
        &=-\text{Res}\left(\frac{z^{n-1}}{z-2},2\right)\\
        &=-2^{n-1}u[-n]\\
        x_{right}[n]&=-\text{Res}\left(-\frac{z^2-z+1}{(z-1)^3}z^{n-1},\infty\right)\\
        &=-\frac{1}{2\pi i}\oint_{C_R}\frac{z^2-z+1}{(z-1)^3}z^{n-1}\,dz\\
        &=-\left[\text{Res}\left(\frac{z^2-z+1}{(z-1)^3}z^{n-1},1\right)+\text{Res}\left(\frac{z^2-z+1}{(z-1)^3}z^{n-1},0\right)\right]\\
        &=-\lim_{z\to 1}\frac{1}{2}\frac{d^2}{dz^2}\left(z^{n+1}-z^{n}+z^{n-1}\right)-\delta[n]\\
        &=-\frac{n^2-n+2}{2}u[n]-\delta[n]\\
        &=-\frac{n^2-n+2}{2}u[n-1]
    \end{align*}
    因此
    \[x[n]=x_{left}[n]+x_{right}[n]=-2^{n-1}u[-n]-\frac{n^2-n+2}{2}u[n-1]\]
\end{example}

注意到上个例子中我们可以选取两条一样的围道,只要它处于收敛域内。由此我们可以总结出逆z变换的一般方法:
\begin{theorem}[逆z变换,留数定理方法]
    设离散信号$x[n]$的z变换为$X(z)\in\mathcal{M} \left(\hat{\mathbb{C}}\right)$,其收敛域为$r<|z|<R$,任取围道$C\subset\mathcal{A} _0(r,R)$,则有
    \[x[n]=\frac{1}{2\pi i}\oint_{C}X(z)z^{n-1}\,dz\]
\end{theorem}

我们在前面提到:
\begin{claim}
    扩充复平面$\hat{\mathbb{C}}$上,“亚纯函数”等价于“有理函数”。
\end{claim}
对于有理函数,使用留数定理求逆z变换的方法就很好懂了,它的证明需要用到部分分式展开定理,形式较为复杂,这里不再赘述。围道的任意性是由柯西积分定理保证的,因为$X(z)$在$\mathcal{A} _0(r,R)$内全纯。

使用逆z变换公式,我们还可以得到\textbf{z域卷积定理}:
\begin{theorem}[z域卷积定理]
    设离散信号$x[n],y[n]$的z变换分别为$X(z),Y(z)$,收敛域分别为$\mathcal{A} _0(r_1,R_1),\mathcal{A} _0(r_2,R_2)$,则在$\mathcal{A} _0(r_1 r_2,R_1 R_2)$内取围道$C$,有
    \[x[n]y[n]\xleftrightarrow{\mathcal{Z}} \frac{1}{2\pi i}\oint_{C}X(\nu)Y\left(\frac{z}{\nu}\right)\nu^{-1}\,d\nu\]
\end{theorem}
\begin{proof}
    对于收敛域,可以用z变换的收敛域公式验证:
    \[r=\varlimsup_{n\to\infty}|x[-n]y[-n]|^{\frac{1}{n}}=\varlimsup_{n\to\infty}|x[-n]|^{\frac{1}{n}}\cdot\varlimsup_{n\to\infty}|y[-n]|^{\frac{1}{n}}=r_1 r_2\]
    \[R=\left(\varlimsup_{n\to\infty}|x[n]y[n]|^{\frac{1}{n}}\right)^{-1}=\left(\varlimsup_{n\to\infty}|x[n]|^{\frac{1}{n}}\right)^{-1}\cdot\left(\varlimsup_{n\to\infty}|y[n]|^{\frac{1}{n}}\right)^{-1}=R_1 R_2\]
    在$\mathcal{A} _0(r_1 r_2,R_1 R_2)$内取围道$C$,则对于任意$\nu\in C$,都有$\nu\in\mathcal{A} _0(r_1,R_1)$,且$z/\nu\in\mathcal{A} _0(r_2,R_2)$,因此$X(\nu),Y(z/\nu)$均全纯。由逆z变换公式,有
    \begin{align*}
        \oint_{C}X(\nu)Y\left(\frac{z}{\nu}\right)\nu^{-1}\,d\nu&=\oint_{C}\left(\sum_{m=-\infty}^{\infty}x[m]\nu^{-m}\right)\left(\sum_{k=-\infty}^{\infty}y[k]\left(\frac{z}{\nu}\right)^{-k}\right)\nu^{-1}\,d\nu\\
        &=\sum_{m=-\infty}^{\infty}\sum_{k=-\infty}^{\infty}x[m]y[k]z^{-k}\oint_{C}\nu^{-(m-k+1)}\,d\nu\\
        &=2\pi i \sum_{n=-\infty}^{\infty}x[n]y[n]z^{-n}
    \end{align*}
    积分与求和的交换是由收敛域内洛朗级数的内闭一致收敛性保证的。因此,
    \[x[n]y[n]\xleftrightarrow{\mathcal{Z}} \frac{1}{2\pi i}\oint_{C}X(\nu)Y\left(\frac{z}{\nu}\right)\nu^{-1}\,d\nu\]
\end{proof}

最后,我们来介绍另一种求逆z变换的方法——\textbf{长除法}。中学中我们仅使用它来做因式分解。它适用于$X(z)$为有理函数,且只需要求有限项的情况,我们通过一个例子来说明它的使用方法:
\begin{example}
    设右边信号$x[n]$的z变换为
    \[X(z)=\frac{z}{z^2-2z+1},\text{ROC}:|z|>1\]
    求$x[n]$的前六项$x[0],x[1], x[2], x[3], x[4], x[5]$。\\
    解:将$X(z)$写成多项式除法的形式:
    \[
    \renewcommand\arraystretch{1.2}
    \setlength{\arraycolsep}{1pt}
    \begin{array}{rc rrrrrrrrr}
    & & & & z^{-1} & +2z^{-2} & +3z^{-3} & +4z^{-4} & +5z^{-5} & \cdots \\
    \cline{3-10}
    z^2 - 2z + 1 & \big) & z & & & & & & & \\
    & & z & -2 & +z^{-1} & & & & & \\
    \cline{3-5}
    & & & 2 & -z^{-1} & & & & & \\
    & & & 2 & -4z^{-1} & +2z^{-2} & & & & \\
    \cline{4-6}
    & & & & 3z^{-1} & -2z^{-2} & & & & \\
    & & & & 3z^{-1} & -6z^{-2} & +3z^{-3} & & & \\
    \cline{5-7}
    & & & & & 4z^{-2} & -3z^{-3} & & & \\
    & & & & & 4z^{-2} & -8z^{-3} & +4z^{-4} & & \\
    \cline{6-8}
    & & & & & & 5z^{-3} & -4z^{-4} & & \\
    & & & & & & 5z^{-3} & -10z^{-4} & +5z^{-5} & \\
    \cline{7-9}
    & & & & & & & 6z^{-4} & -5z^{-5} & \cdots
    \end{array}
    \]
    对照z变换的定义$X(z)=\sum_{n=-\infty}^{\infty}x[n]z^{-n}$,可得
    \[x[0]=0,x[1]=1,x[2]=2,x[3]=3,x[4]=4,x[5]=5,\cdots\]
    可以猜想,$x[n]=n u[n]$。
\end{example}

\section{离散系统的z域分析}\label{sec:Discrete_System_z_Analysis}

\section{离散时间傅里叶变换}\label{sec:DTFT}

\section{离散系统的频率响应特性}\label{sec:Discrete_System_Freq_Response}

\chapter{离散傅里叶变换}

\section{离散傅里叶变换}\label{sec:DFT}

\section{快速傅里叶变换}\label{sec:FFT}

%\chapter{其他积分变换}

%\section{梅林变换}\label{sec:Mellin_Transform}

%\section{拉东变换}\label{sec:Radon_Transform}

%\section{离散余弦变换}\label{sec:DCT}

%\section{小波变换}\label{sec:Wavelet_Transform}
\appendix

\chapter{傅里叶级数的渐进特性,吉布斯现象}\label{sec:Asymptotic_Behaviour}

在用计算机模拟函数的傅里叶级数展开时,只能取有限项,自然要问计算到多少项时误差
足够小,为此,我们不加证明地给出以下定理:
\begin{theorem}[收敛速度的估计]
    \begin{align*}
     & \text{设}f\in C^p(\mathbb{R} )(p\geq1)\text{是周期函数,则部分和}
    S_N^f(t)=\sum_{-N}^{N}c_k e^{ik\omega t}
    \text{在}\mathbb{R} \text{上逐点收敛、}                                   \\
     & \text{内闭一致收敛,且} \max\left|f(t)-S_N^f(t)\right|<\frac{1}{N^{p-\frac{1}{2}}}
\end{align*}
其中$C^p(\mathbb{R} )$表示p次连续可导的函数集。
\end{theorem}

当$f(t)$不连续时,傅里叶级数的会在间断点处产生\textbf{吉布斯现象} (Gibbs' Phenomenon):
\begin{definition}[吉布斯现象]
    部分和$S_N^f(t)$在间断点处总会\textbf{过冲}(在间断点两侧出现超过原函数的峰值)
,过冲幅度约为跳变幅度的9\%,并且$S_N^f(t)$会在间断点附近高频振荡
\end{definition}
\begin{figure}[H]
    \centering
    \includegraphics[width=0.5\textwidth]{gibbs}
    \caption{吉布斯现象示意图}
\end{figure}
\begin{example}
    对于跳变幅度为2、周期为$2\pi$的周期矩形脉冲信号
\[R(x) =
    \begin{cases}
        1  & \text{if } 0<x<\pi  \\
        -1 & \text{if } -\pi<x<0
    \end{cases}\]
其傅里叶级数的跳变值为1,
$\varlimsup_{N \to \infty}S_N^R(t)=1.089490 \dots$。
这是因为光滑的基函数很难逼近这种剧烈的局部变化,不得不用高频分量来补偿,高频分量带来了剧烈震动。
\end{example}
$\varlimsup_{N \to \infty}S_N^R(t)>1$并不意味着狄利克雷定理失效,因为定理给出的是逐点收敛:
\[\varlimsup_{N \to \infty}S_N^R(t)=\lim_{N \to \infty}\sup_{t\in \mathbb{R} }S_N^R(t)\neq \sup_{t\in\mathbb{R}}\lim_{N\to\infty}S_N^R(t)\]
使极限号与取上界号交换的一个充分条件是一致收敛,吉布斯现象表明,傅里叶级数在包含间断点的区间上不可能一致收敛。

为了直观地理解它,我们来看一个经典的例子:
\begin{example}
    \begin{align*}
    f_n(x)=
    \begin{cases}
        nx   & \text{if } 0<x\leq 1/n        \\
        2-nx & \text{if } 1/n<x<2/n \\
        0    & \text{otherwise}
    \end{cases}
\end{align*}
随n增大,$f(x)$逐点趋于0,因为对每一点$2/n$总能取到更小的值;但$f(x)$
的最大值永远是1。
\end{example}

下面来研究傅里叶级数的逐点收敛性。直接由狄利克雷条件证明收敛性是很困难的,并且需要更加专业的分析学工具,我们将给出更强的条件下的证明,并从这个更强的条件获得一些额外的性质。

研究傅里叶级数的渐进特性时,一个非常好用的工具是\textbf{狄利克雷核} (Dirichlet kernel):
\begin{definition}[狄利克雷核]
    \[D_N(t)=\sum_{k=-N}^{N}e^{ik\omega t}=1+\sum_{k=1}^{N}\left(e^{ik\omega t}+e^{-ik\omega t}\right)=1+2\sum_{k=1}^{N}\cos(k\omega t)\]
    它是依赖于所研究函数的周期T的,但简便起见,在符号$D_N(t)$中不体现这一点。
\end{definition}

我们可以用等比数列求和或积化和差裂项的方法化简$D_N(t)$:
\begin{align*}
    D_N(t) & =\sum_{k=-N}^{N}e^{ik\omega t}=e^{-iN\omega t}\frac{1-e^{i(2N+1)\omega t}}{1-e^{i\omega t}}                  \\
           & =\frac{e^{i(N+1)\omega t}-e^{-iN\omega t}}{e^{i\omega t}-1}                                   \\
    D_N(t) & =1+\sum_{k=1}^{N}\left(e^{ik\omega t}+e^{-ik\omega t}\right)=1+2\sum_{k=1}^{N}\cos(k\omega t)           \\
           & =1+\frac{2}{\sin(\frac{\omega t}{2})}\sum_{k=1}^{N}\cos(k\omega t)\sin\left(\frac{\omega t}{2}\right)                   \\
           & =1+\frac{1}{\sin(\frac{\omega t}{2})}\sum_{k=1}^{N}\left(\sin\left((k+\frac{1}{2})\omega t\right)-\sin\left((k-\frac{1}{2})\omega t\right)\right) \\
           & =1+\frac{\sin\left((N+\frac{1}{2})\omega t\right)-\sin(\frac{\omega t}{2})}{\sin(\frac{\omega t}{2})}                     \\
           & =\frac{\sin\left((N+\frac{1}{2})\omega t\right)}{\sin(\frac{\omega t}{2})}
\end{align*}
这两种结果是相符的,读者可自行验证,并且可以从后一结果想象出狄利克雷核的函数图像,它被$\pm 1/\sin(\frac{\omega t}{2})$包络并高速振荡。函数图像如下。
\begin{figure}[H]
    \centering
    \includegraphics[width=0.7\textwidth]{Figure_3}
\end{figure}

引入狄利克雷核后,就可以用以下恒等式研究傅里叶级数的部分和:
\begin{proposition}
    \[S_N^f(t)=\frac{1}{T}\int_{T}f(\tau)D_N(t-\tau)\,d\tau\]
\end{proposition}
\begin{proof}
    \begin{align*}
    S_N^f(t) & =\sum_{-N}^{N}c_k e^{ik\omega t}                                                                                \\
             & =\sum_{-N}^{N}\left(\frac{1}{T}\int_{T}f(\tau)e^{-k\omega \tau}\,d\tau\right) e^{ik\omega t}                    \\
             & =\frac{1}{T}\int_{T}f(\tau)\sum_{k=-N}^{N}e^{ik\omega (t-\tau)}\,d\tau                                          \\
             & =\frac{1}{T}\int_{T}f(\tau)D_N(t-\tau)\,d\tau                                                                   \\
             & =\frac{1}{T}\int_{T}f(t-\tau)D_N(\tau)\,d\tau                                                & (\tau\to t-\tau) \\
             & =\frac{1}{T}\int_{T}f(t+\tau)D_N(\tau)\,d\tau                                                & (\tau\to t+\tau)
\end{align*}
\end{proof}

在讨论傅里叶级数的收敛性前,先给出两个引理。第一个引理表明狄利克雷核在半周期上
积分值为$\frac{T}{2}$,在证明傅里叶级数的逐点收敛性时将用到它。
\begin{lemma}
    $\int_{-\frac{T}{2}}^{0}D_N(t)\,dt=\int_{0}^{\frac{T}{2}}D_N(t)\,dt=\frac{T}{2}$
\end{lemma}
\begin{proof}
    \begin{align*}
    D_N(t)&=1+2\sum_{k=1}^{N}\cos(k\omega t)\\
    \int_{0}^{\frac{T}{2}}D_N(t)\,dt & =\int_{0}^{\frac{T}{2}}\left(1+\sum_{k=1}^{N}\cos(k\omega t)\right)\,dt                              \\
                                     & =\frac{T}{2}+\sum_{k=1}^{N}\left.\frac{\sin(k\omega t)}{k\omega}\right|_{0}^{\frac{T}{2}} \\
                                     & =\frac{T}{2}+\frac{1}{\omega}\sum_{k=1}^{N}\frac{\sin(k\pi)}{k}=\frac{T}{2}               
\end{align*}
$D_N(t)$是偶函数,得证。
\end{proof}

\begin{lemma}[贝塞尔不等式]
设$ f\in L^2([0,T]),c_n=\frac{1}{T}\int_{T}f(t)e^{-ik\omega t}\,dt$,则
\begin{align*}
    \sum_{-\infty}^{\infty}|c_n|^2\leq \frac{1}{T}\int_{T}|f(t)|^2\,dt=\frac{1}{T}\|f\|_2^2
\end{align*}
\end{lemma}
它给出了傅里叶系数平方和的上界的估计。收敛级数的通项必收敛,所以由此可以看出$c_n\to 0,n\to\infty$.\\
\begin{proof}
    \begin{align*}
    \left|f(t)-\sum_{n=-N}^{N}c_n e^{in\omega t}\right|^2 & =\left(f(t)-\sum_{n=-N}^{N}c_n e^{in\omega t}\right)\left(f(t)-\sum_{n=-N}^{N}c_n e^{in\omega t}\right)^*                                       \\
                                               & =\left(f(t)-\sum_{n=-N}^{N}c_n e^{in\omega t}\right)\left(f^*(t)-\sum_{n=-N}^{N}c_n e^{-in\omega t}\right)                                      \\
                                               & =|f(t)|^2-\sum_{n=-N}^{N}\left(c_n^*f(t)e^{in\omega t}+c_n f^*(t)e^{-in\omega t}\right)+\sum_{m,n=-N}^{N}c_m c_n^*e^{i(m-n)\omega t}
\end{align*}
将上式在一个周期上积分,我们知道
\[\int_{T}f(t)e^{in\omega t}\,dt=Tc_n,\int_{T}e^{i(m-n)\omega t}\,dt=\begin{cases}
        0 & \text{if }m\neq n \\
        T & \text{if }m=n
    \end{cases}\]
故\begin{align*}
      & \int_{T}|f(t)|^2\,dt-\sum_{n=-N}^{N}\left(c_n^*\int_{T}f(t)e^{in\omega t}\,dt+c_n \int_{T}f^*(t)e^{-in\omega t}\,dt\right)+\sum_{m,n=-N}^{N}c_m c_n^*\int_{T}e^{i(m-n)\omega t}\,dt \\
    = & \int_{T}|f(t)|^2\,dt-T\sum_{n=-N}^{N}(c_n^* c_n+c_n c_n^*)+T\sum_{n=-N}^{N}c_n^* c_n                                                                                     \\
    = & \int_{T}|f(t)|^2\,dt-T\sum_{n=-N}^{N}|c_n|^2
\end{align*}
这是非负函数的积分,积分值非负,即\begin{align*}
    \sum_{-\infty}^{\infty}|c_n|^2\leq \frac{1}{T}\int_{T}|f(t)|^2\,dt<\infty
\end{align*}
\end{proof}

直接由狄利克雷条件证明逐点收敛性需要很专业的分析学工具,但我们可以适当地加强狄
利克雷条件。
\begin{definition}[分段光滑]
$f\in PS([0,T])\iff$除有限个点外f均可导,并且这些点是f的第一类间断点
\end{definition}
我们研究的多数函数是满足这样的性质的,并且我们将看到满足此条件会带来一些额外的
性质。此时就可以相对简单地证明逐点收敛性:
\begin{theorem}[傅里叶级数的逐点收敛性]
    \[\lim_{N\to\infty}S_N^f(t_0)=\frac{f(t_0+)+f(t_0-)}{2}\]
\end{theorem}
\begin{proof}
\begin{align*}
    S_N^f(t_0)-\frac{f(t_0^+)+f(t_0^-)}{2} =&\frac{1}{T}\left(\int_{T}f(t_0-\tau)D_N(\tau)\,d\tau\right.\\
    &\left.-\int_{0}^{\frac{T}{2}}f(t_0^+)D_N(\tau)\,d\tau-\int_{-\frac{T}{2}}^{0}f(t_0^-)D_N(\tau)\,d\tau\right) \\
                                                                     =&\frac{1}{T}\left(\int_{0}^{\frac{T}{2}}(f(t_0-\tau)-f(t_0^+))D_N(\tau)\,d\tau\right.\\
                                                                     &+\left.\int_{-\frac{T}{2}}^{0}(f(t_0-\tau)-f(t_0^-))D_N(\tau)\,d\tau\right)         \\
    S_N^f(t_0)-\frac{f(t_0^+)+f(t_0^-)}{2}  =&\frac{1}{T}\int_{T}g(t)\left(e^{i(N+1)\omega t}-e^{iN\omega t}\right)\,dt                                                                                             
\end{align*}
其中\begin{align*}
g(t):=\begin{cases}
                                                                             \frac{f(t_0+t)-f(t_0^-)}{e^{i\omega t}-1} & \text{if }-\frac{T}{2}<t_0<0 \\
                                                                             \frac{f(t_0+t)-f(t_0^+)}{e^{i\omega t}-1} & \text{if }0<t_0<\frac{T}{2}
                                                                         \end{cases}
\end{align*}                        
由洛必达法则,$t\to 0$时,\[\lim_{t\to 0^+}g(t)=\lim_{t\to 0^+}\frac{f(t_0+t)-f(t_0^+)}{e^{i\omega t}}=\lim_{t\to 0^+}\frac{f'(t_0+t)}{ie^{i\omega t}}=\lim_{t\to 0^+}\frac{f'(t_0^+)}{i}\]
$t\to 0^-$时同理。故g分段连续,当然是平方可积的,由贝塞尔不等式,
$g(t)$的傅里叶系数平方和收敛,通项趋于0,$S_N^f(t_0)-\left(f(t_0^+)+f(t_0^-)\right)/2=C_{-(N+1)}-C_N\to 0$
,得证。
\end{proof}

在分段光滑的条件下,容易得到$f'(t)$的傅里叶系数,注意微积分基本定理可以分区间使用:
\begin{proposition}[导函数的傅里叶系数]
    \begin{equation*}
    a'_n=n\omega b_n,b'_n=-n\omega a_n,c'_n=in\omega c_n
\end{equation*}
\end{proposition}
\begin{proof}
    以$c_n$为例:
\begin{align*}
    c'_n & =\frac{1}{T}\int_{T}f'(t)e^{-in\omega t}\,dt                                                         \\
         & =\frac{1}{T}\evalat{f(t)e^{-in\omega t}}{0}{T}+in\omega\int_{T}f(t)e^{-i\omega t}\,dt=in\omega c_n
\end{align*}
\end{proof}

f的原函数F的傅里叶系数同理,并且只要\textbf{分段连续}(见\ref{sec:Fourier_Series})
即可保证f可积,但是我们必须保证F是周期函数,这要求f的直流分量为0:
\[F(t+T)-F(t)=\int_{T}f(t)dt=Tc_0=0,c_0=0\]
此时,用刚刚得到的公式(2.16)就可直接得到F的傅里叶系数:
\begin{proposition}[原函数的傅里叶系数]
    \begin{equation*}
    A_n=\frac{a_n}{n\omega},B_n=\frac{b_n}{n\omega},C_n=\frac{c_n}{in\omega}
\end{equation*}
\end{proposition}

分段光滑还能够推出f的傅里叶级数\textbf{一致收敛}于f,从而可以逐项积分、逐项求
导。回顾数学分析中的魏尔斯特拉斯M判别法:
\begin{theorem}[M判别法]
    对于函数项级数
$\sum_{n=1}^{\infty}f_n(x)$,如果存在正项级数$\sum_{n=1}^{\infty}M_n<\infty$
使得在区间E上$|f_n(x)|<M_n$,则$\sum_{n=1}^{\infty}f_n(x)$在E上绝对收敛且
一致收敛。
\end{theorem}
对于上述命题,只需证明$\sum_{n=1}^{\infty}|c_n|<\infty$.直接应用
贝塞尔不等式是无效的,但可以通过一个小技巧完成证明:\begin{proof}
    记$c'_n$为$f'$的傅里叶系数,$c'_n =in\omega c_n$,
    \begin{align*}
    \sum_{n=-\infty}^{\infty}|c_n|=   |c_0|+\sum_{n\neq 0}\left| \frac{c'_n}{n}\right|\leq & |c_0|+\left(\sum_{n\neq 0}\frac{1}{n^2}\right)^{1/2}\left(\sum_{n\neq 0}|c'_n|^2\right)^{1/2}<\infty
\end{align*}
最后一步使用了柯西-施瓦兹不等式。
\end{proof}

请读者思考:我们探究了指数形式傅里叶级数收敛的条件,对于三角函数形式的傅里叶级
数应该怎么办?

\chapter{分布的逼近,傅里叶反演公式}\label{sec:approach}

在\ref{sec:Dirac_Comb}中,我们对带限函数提出了公式:如果$\text{supp}\ \mathcal{F} f\subset[-\frac{T}{2},\frac{T}{2}]$,则有
\lr{
    f(t)=Tsinc(Tt)*f(t)
}{
    f(t)=\frac{T}{2\pi}Sa(Tt)*f(t)
}

如果希望对更一般的函数$f(x)$,找到一个函数$g(x)$,使得$f(x)=g*f(x)$,自然会想到狄拉克$\delta$分布,然而它是一个奇异分布,我们只能从形式上定义一个$\delta$函数。事实上,不存在$g,f*g=f,\forall f$,但很容易构造一个函数列$g_n$,使得$f*g_n\to f$。回忆我们在\ref{sec:convolution}中提到的“平均化”卷积,设$g_a(x)=\frac{1}{2a}\chi_{[-a,a]}$,则$g_a*f$是$f$在区间$[x-a,x+a]$上的平均值,即$\frac{1}{2a}f*g_a(t)=\int_{x-a}^{x+a}f(x)\,dx$,若$f\in C(\mathbb{R})$,则$g_a*f\to f,a\to 0$。

一般地,我们考虑函数的缩放,设$g\in L^1(\mathbb{R})$,我们记
\[g_\lambda(x)=\frac{1}{\lambda}g\left(\frac{x}{\lambda}\right)\]
这类似于时域的伸缩在傅里叶变换下的表现,它是一种“保面积”的缩放,即:
\[\int_{-\infty}^{\infty}g_\lambda(x)\,dt=\int_{-\infty}^{\infty}g\left(\frac{y}{\lambda}\right)\,d\frac{y}{\lambda}=\int_{-\infty}^{\infty}g(y)\,dy\]
更一般地,我们给出接下来要频繁使用的引理:\begin{lemma}
\[\int_{a}^{b}g_\lambda(x)\,dx=\int_{a/\lambda}^{b/\lambda}g(y)\,dy\]
\end{lemma}
下面我们给出一个表述和证明有些类似于傅里叶级数收敛性的定理:
\begin{theorem}
    设$g\in L^1(\mathbb{R}),\int_{-\infty}^{\infty}g(x)\,dx=1$,并记$\alpha=\int_{-\infty}^{0}g(x)\,dx,\beta=\int_{0}^{\infty}g(x)\,dx$。假设$f\in PC(\mathbb{R})$,则当$g(x)$时限或$f(x)$有界时,$f*g(x)$良定义,并且
    \[\lim_{\lambda\to 0}f*g_\lambda(x)=\alpha f(x+)+\beta f(x-)\]
    特别地,如果$f(x)$在点$x$连续,则有$\lim_{\lambda\to 0}f*g_\lambda(x)=f(x)$;如果$f(x)$在某个闭区间$[a,b]$上连续,则以上极限是一致收敛于$f(x)$的。
\end{theorem}
\begin{proof}
以下证明中,我们需要使用$\epsilon-\delta$语言,暂时不用$\delta$表示狄拉克分布。
\begin{align*}
f*g_\lambda(x)-\alpha f(x+)-\beta f(x-)=\int_{-\infty}^{0}[f(x-y)-f(x+)]g_\lambda(y)\,dy+\int_{0}^{\infty}[f(x-y)-f(x-)]g_\lambda(y)\,dy
\end{align*}
我们希望证明在$\lambda$充分小时,以上两个积分也充分小。以$\int_{0}^{\infty}$为例,由于$f\in PC(\mathbb{R})$,存在$\delta>0$,使得当$|y|<\delta$时,$|f(x-y)-f(x-)|<\epsilon$。于是
\begin{align*}
    \left|\int_{0}^{\delta}[f(x-y)-f(x-)]g_\lambda(y)\,dy\right|
    <&\epsilon\left|\int_{0}^{\delta}g_\lambda(y)\,dy\right|\\
    =&\epsilon\left|\int_{0}^{\delta/\lambda}g(y)\,dy\right| \\
    \leq &\epsilon\int_{0}^{\infty}|g(y)|\,dy
\end{align*}
我们用剩下的条件估计$\int_{\delta}^{\infty}[f(x-y)-f(x-)]g_\lambda(y)\,dy$。如果$g(x)$时限,设$supp\,g\subset[-T,T]$,则当$\lambda<\delta/T$时,$$y>\delta/\lambda>T\implies g(y)=0\implies\int_{\delta}^{\infty}[f(x-y)-f(x-)]g_\lambda(y)\,dy=\int_{\delta/\lambda}^{\infty}[f(x-\lambda y)-f(x-)]g(y)\,dy=0$$

如果$f(x)$有界,设$|f(x)|<M$,则
\begin{align*}
    \left|\int_{\delta}^{\infty}[f(x-y)-f(x-)]g_\lambda(y)\,dy\right|&\leq \int_{\delta}^{\infty}|f(x-y)-f(x-)||g_\lambda(y)|\,dy\\
    &\leq 2M\int_{\delta}^{\infty}|g_\lambda(y)|\,dy\\
    &=2M\int_{\delta/\lambda}^{\infty}|g(y)|\,dy
\end{align*}
$\int_{-\infty}^{\infty}|g(y)|\,dy<\infty$的必要条件是$\int_{\delta/\lambda}^{\infty}|g(y)|\,dy\to 0,\delta/\lambda\to\infty$,因此上式随$\lambda\to 0$趋于0。

最后,如果$f(x)$在闭区间$[a,b]$上连续,则它在该区间上一致连续,以上证明中的$\delta$与$x$无关,极限是一致收敛的。
\end{proof}
一种常用的函数列是高斯函数列,这里取一个积分值为1的特例:
\[G(x)=\frac{1}{\sqrt{\pi}}e^{-x^2},G_\lambda(x)=\frac{1}{\lambda\sqrt{\pi}}e^{-x^2/\lambda^2}\]
尽管用很多其他函数都可以逼近$\delta$分布,但高斯函数列有一系列非常好的性质:\begin{itemize}
    \item 高斯函数是偶函数,$\alpha=\beta=\frac{1}{2}$;
    \item 高斯函数是无限阶可导的。我们知道,卷积能够继承函数的可导性:$f*g'=f'*g=(f*g)'$,使用高斯函数做卷积,所得的函数光滑
    \item 高斯函数是施瓦兹函数,即$G_\lambda\in \mathcal{S}$
    \item 高斯函数的傅里叶变换仍是高斯函数,这使得傅里叶变换更容易计算
    \item 高斯函数$G_\lambda\in L^1(\mathbb{R})$并且是有界的,与任意$f\in PC(\mathbb{R})$卷积时,卷积积分良定义且趋向于$f$
\end{itemize}

作为以上定理的推论,可以得到一个非常好用的定理:\begin{corollary}[*魏尔斯特拉斯逼近定理]
    设$f\in C([a,b]),-\infty<a<b<\infty$,则$f$是某个多项式序列在$[a,b]$上的一致极限,即\[\forall \epsilon>0,\exists \text{多项式}P,\sup_{a\leq x\leq b}|f(x)-P(x)|<\epsilon\]
\end{corollary}
尽管它与傅里叶分析本身关系不大,但在处理分析问题时,将连续函数用多项式一致逼近是一个很强大的工具。
\begin{proof}
    将$f$做连续延拓,使得$f\in C(\mathbb{R}),f(x)=0\ for\ x\in (-\infty,a-1]\cup[b+1,\infty)$,则$\lim_{\lambda\to 0}f*G_\lambda \rightrightarrows f,x\in[a,b]$,即$f*G_\lambda$在区间$[a,b]$上一致收敛到$f$。具体来说,
    \[\forall \epsilon>0,\exists \lambda>0,\sup_{a\leq x\leq b}\left|f(x)-\frac{1}{\lambda\sqrt{\pi}}\int_{a-1}^{b+1}e^{-(x-y)^2/\lambda^2}f(y)\,dy\right|<\frac{\epsilon}{2}\]
    考虑使用剩下的半个$\epsilon$,将右边的积分用多项式近似。我们知道,$e^{-t^2}$可以展开为泰勒级数,并且在$\mathbb{C}$上内闭一致收敛,即
    \[\sum_{n=0}^{N}\frac{t^n}{n!}\rightrightarrows e^{-t^2},N\to\infty\]
    具体来说,取$|f(x)|$的上界$M$(紧支连续函数一致连续,必有上界,见数学分析教材),存在$N$,使得
    \[\sup_{t\in[a-1,b+1]}\left|e^{-t^2}-\sum_{n=0}^{N}\frac{(-1)^n t^{2n}}{n!}\right|<\frac{\epsilon\sqrt{\pi}}{2M(b-a+2)}\]
    于是
    \[P(x)=\frac{1}{\lambda\sqrt{\pi}}\sum_{0}^{N}\int_{a-1}^{b+1}\frac{(-1)^n (x-y)^{2n}}{\lambda^{2n}n!}f(y)\,dy,\sup_{a\leq x\leq b}\left|f(x)-P(x)\right|<\epsilon\]
    我们还可以求出$P$的显式表达式:
    \[P(x)=\sum_{n=0}^{N}\sum_{k=0}^{2n}c_{kn}x^k,c_{kn}=\frac{(-1)^{k-n}C_{2n}^{k}}{\lambda^{2n+1}n!\sqrt{\pi}}\int_{a-1}^{b+1}y^{2n-k}f(y)\,dy\]
    使用其他光滑函数列逼近$\delta$分布也可以得到不同的多项式逼近,不过,绝大多数情况下我们并不需要计算出这个表达式,而是直接用多项式的一致逼近来获取函数的性质。
\end{proof}

下面,我们来补充\ref{sec:Fourier}中提到的\textbf{黎曼-勒贝格引理}的证明。
\begin{theorem}[黎曼-勒贝格引理]
    设$f\in L^1(\mathbb{R})$,则其傅里叶变换$\mathcal{F} f(x)\to 0,x\to\infty$。
\end{theorem}
我们给出一个基于阶梯函数逼近的证明,尽管现在不需要掌握它,但可以从这个证明体会到分析学中逼近技术的强大威力。
\begin{proof}
    首先假设$f$是阶梯函数,即$f(t)=\sum_{k=1}^{n}c_k \chi_{[a_k,b_k]}(t)$,我们知道:\lr{
        \mathcal{F} \chi_{[a,b]}(\xi)=\int_{a}^{b}e^{-2\pi i\xi t}\,dt=\frac{e^{-2\pi i\xi a}-e^{-2\pi i\xi b}}{2\pi i\xi}
    }{
        \mathcal{F} \chi_{[a,b]}(\omega)=\int_{a}^{b}e^{-i\omega t}\,dt=\frac{e^{-i\omega a}-e^{-i\omega b}}{i\omega}
    }
    分子是两个模值为1的数相减,分母趋于无穷,故$\mathcal{F} \chi_{[a,b]}(x)\to 0,x\to\infty$。由线性性,$\mathcal{F} f(x)\to 0,x\to\infty$。\\
    设$f\in L^1(\mathbb{R})$,则根据实变函数的理论,存在阶梯函数列$f_n$,使得$\|f-f_n\|_1\to 0,n\to\infty$。我们知道,
    \[|\mathcal{F} g(x)|\leq \int_{-\infty}^{\infty}|g(t)|\,dt=\|g\|_1<\infty,g\in L^1(\mathbb{R})\]
    因此,
    \[\sup_{x\in\mathbb{R}}|\mathcal{F} f(x)-\mathcal{F} f_n(x)|\leq \|f-f_n\|_1\to 0,n\to\infty\]
    即$\mathcal{F} f_n\rightrightarrows\mathcal{F} f,\forall x\in\mathbb{R}$,于是$\mathcal{F} f(x)\to 0,x\to\infty$,得证。
\end{proof}

最后,我们给出傅里叶反演公式及其证明。
\begin{theorem}[傅里叶反演公式]
    设$f\in L^1(\mathbb{R})\cap PC(\mathbb{R})$,则对于任意点$x$,都有
    \lr{
        &\frac{f(x+)+f(x-)}{2}\\
        =&\lim_{\lambda\to 0}\int_{-\infty}^{\infty}e^{2\pi i\xi x}\mathcal{F} f(\xi)e^{-\pi^2 \lambda^2 \xi^2}\,d\xi
    }{
        &\frac{f(x+)+f(x-)}{2}\\
        =&\lim_{\lambda\to 0}\frac{1}{2\pi}\int_{-\infty}^{\infty}e^{i\omega x}\mathcal{F} f(\omega)e^{-\lambda^2\omega^2/4}\,d\omega
    }
    如果还有$\mathcal{F} f(x)\in L^1(\mathbb{R})$,则有
    \lr{
        \frac{f(x+)+f(x-)}{2}=\int_{-\infty}^{\infty}e^{2\pi i\xi x}\mathcal{F} f(\xi)\,d\xi
    }{
        \frac{f(x+)+f(x-)}{2}=\frac{1}{2\pi}\int_{-\infty}^{\infty}e^{i\omega x}\mathcal{F} f(\omega)\,d\omega
    }
\end{theorem}
我们指出,$\mathcal{S} \subset L^1(\mathbb{R})\cap PC(\mathbb{R})$。

要说明傅里叶逆变换的确能够还原出时域函数,一种直接的想法是积分换序,即\lr{
    &\int_{-\infty}^{\infty}e^{2\pi i\xi x}\mathcal{F} f(\xi)\,d\xi\\
    =&\int_{-\infty}^{\infty}\left(\int_{-\infty}^{\infty}f(x)e^{-2\pi i\xi x}\,dx\right)e^{2\pi i\xi x}\,d\xi
}{
    &\frac{1}{2\pi}\int_{-\infty}^{\infty}e^{i\omega x}\mathcal{F} f(\omega)\,d\omega\\
    =&\frac{1}{2\pi}\int_{-\infty}^{\infty}\left(\int_{-\infty}^{\infty}f(x)e^{-i\omega x}\,dx\right)e^{i\omega x}\,d\omega
}
然而,$\int_{-\infty}^{\infty}e^{2\pi i\xi (x-y)}\,d\xi$和$\int_{-\infty}^{\infty}e^{i\omega (x-y)}\,d\omega$并不收敛,无法直接使用积分换序定理。我们可以使用“收敛因子”来处理这个问题,从信号处理的角度,这就是\textbf{加窗}(windowing)的手法,典型的\textbf{窗函数}(windowing function)如矩形窗(对应短时傅里叶变换)、三角窗和高斯窗(对应加博变换),更一般地还有\textbf{小波变换}(wavelet transform,WT),见\ref{sec:Finite_Interval}。这里我们用高斯窗,仍记$G_\lambda(\xi)=\frac{1}{\lambda\sqrt{\pi}}e^{-\xi^2/\lambda^2}$,有\lr{
    \mathcal{F} G_\lambda(\xi)=e^{-\pi^2 \lambda^2 \xi^2}
}{
    \mathcal{F} G_\lambda(\omega)=e^{-\lambda^2 \omega^2/4}
}
下面的证明中将用到高斯函数的傅里叶逆变换,但我们可以不依赖于傅里叶反演公式而预先求出它,具体的做法可以参考\ref{sec:Fourier}中求傅里叶正变换的过程。
\begin{proof}
    考虑
\lr{
    &\int_{-\infty}^{\infty}e^{2\pi i\xi x}\mathcal{F} f(\xi)e^{-\pi^2\lambda^2 \xi^2}\,d\xi\\
    =&\int_{-\infty}^{\infty}\left(\int_{-\infty}^{\infty}f(x)e^{-2\pi i\xi x}\,dx\right)e^{2\pi i\xi x}e^{-\pi^2 \lambda^2 \xi^2}\,d\xi\\
    =&\int_{-\infty}^{\infty}f(x)\,dx\int_{-\infty}^{\infty}e^{2\pi i\xi (x-y)}e^{-\pi^2 \lambda^2 \xi^2}\,d\xi
}{
    &\frac{1}{2\pi}\int_{-\infty}^{\infty}e^{i\omega x}\mathcal{F} f(\omega)e^{-\lambda^2\omega^2/4}\,d\omega\\
    =&\frac{1}{2\pi}\int_{-\infty}^{\infty}\left(\int_{-\infty}^{\infty}f(x)e^{-i\omega x}\,dx\right)e^{i\omega x}e^{-\lambda^2\omega^2/4}\,d\omega\\
    =&\frac{1}{2\pi}\int_{-\infty}^{\infty}f(x)\,dx\int_{-\infty}^{\infty}e^{i\omega (x-y)}e^{-\lambda^2\omega^2/4}\,d\omega
}

以上过程利用了高斯函数的速降性质,保证了积分的绝对收敛性,从而可以进行积分换序。内层的积分恰为傅里叶变换:
\lr{
    &\int_{-\infty}^{\infty}e^{2\pi i\xi (x-y)}e^{-\pi^2 \lambda^2 \xi^2}\,d\xi\\
    =&\mathcal{F}^{-1}[e^{-\pi^2 \lambda^2 \xi^2}](x-y)\\
    =&\frac{1}{\sqrt{\pi}\lambda}e^{-(x-y)^2/\lambda^2}\\
    =&G_{\lambda}(x-y)
}{
    &\frac{1}{2\pi}\int_{-\infty}^{\infty}e^{i\omega (x-y)}e^{-\lambda^2\omega^2/4}\,d\omega\\
    =&\mathcal{F}^{-1}[e^{-\lambda^2\omega^2/4}](x-y)\\
    =&\frac{1}{\sqrt{\pi}\lambda}e^{-(x-y)^2/\lambda^2}\\
    =&G_{\lambda}(x-y)
}
于是我们得到
\lr{
    &\int_{-\infty}^{\infty}e^{2\pi i\xi x}\mathcal{F} f(\xi)e^{-\pi^2 \lambda^2 \xi^2}\,d\xi\\
    =&\int_{-\infty}^{\infty}f(y)G_{\lambda}(x-y)\,dy\\
    =&f*G_{\lambda}(x)
}{
    &\frac{1}{2\pi}\int_{-\infty}^{\infty}e^{i\omega x}\mathcal{F} f(\omega)e^{-\lambda^2\omega^2/4}\,d\omega\\
    =&\int_{-\infty}^{\infty}f(y)G_{\lambda}(x-y)\,dy\\
    =&f*G_{\lambda}(x)
}
我们知道,$\lim_{\lambda\to 0}f*G_{\lambda}(x)=\frac{f(x+)+f(x-)}{2}$,因此
\lr{
    &\lim_{\lambda\to 0}\int_{-\infty}^{\infty}e^{2\pi i\xi x}\mathcal{F} f(\xi)e^{-\pi^2 \lambda^2 \xi^2}\,d\xi=&\frac{f(x+)+f(x-)}{2}
}{
    &\lim_{\lambda\to 0}\frac{1}{2\pi}\int_{-\infty}^{\infty}e^{i\omega x}\mathcal{F} f(\omega)e^{-\lambda^2\omega^2/4}\,d\omega=&\frac{f(x+)+f(x-)}{2}
}
现在,只需要证明$\lim_{\lambda\to 0}$与积分号可以交换顺序。我们有:\lr{
    \left|e^{2\pi i\xi x}\mathcal{F} f(\xi)e^{-\pi^2 \lambda^2 \xi^2}\right|\leq & |\mathcal{F} f(\xi)|\in L^1(\mathbb{R})
}{
    \left|\frac{1}{2\pi}e^{i\omega x}\mathcal{F} f(\omega)e^{-\lambda^2\omega^2/4}\right|\leq & \frac{1}{2\pi}|\mathcal{F} f(\omega)|\in L^1(\mathbb{R})
}   
因此可以使用控制收敛定理,得到
\lr{
    &\frac{f(x+)+f(x-)}{2}\\
    =&\lim_{\lambda\to 0}\int_{-\infty}^{\infty}e^{2\pi i\xi x}\mathcal{F} f(\xi)e^{-\pi^2 \lambda^2 \xi^2}\,d\xi\\
    =&\int_{-\infty}^{\infty}e^{2\pi i\xi x}\mathcal{F} f(\xi)\,d\xi
}{
    &\frac{f(x+)+f(x-)}{2}\\
    =&\lim_{\lambda\to 0}\frac{1}{2\pi}\int_{-\infty}^{\infty}e^{i\omega x}\mathcal{F} f(\omega)e^{-\lambda^2\omega^2/4}\,d\omega\\
    =&\frac{1}{2\pi}\int_{-\infty}^{\infty}e^{i\omega x}\mathcal{F} f(\omega)\,d\omega
}
\end{proof}
傅里叶反演公式还有一些其他的表述。例如,如果我们将窗函数换为矩形窗,则可得到:
\begin{theorem}[傅里叶反演公式]
    设$f\in L^1(\mathbb{R})\cap PS(\mathbb{R})$,则对于任意点$x$,都有
    \lr{
        &\frac{f(x+)+f(x-)}{2}\\
        =&\lim_{R\to \infty}\int_{-R}^{R}e^{2\pi i\xi x}\mathcal{F} f(\xi)\,d\xi
    }{
        &\frac{f(x+)+f(x-)}{2}\\
        =&\lim_{R\to \infty}\frac{1}{2\pi}\int_{-R}^{R}e^{i\omega x}\mathcal{F} f(\omega)\,d\omega
    }
\end{theorem}
等式右侧的积分与反常积分是有区别的,无穷限的反常积分要求积分上下限以二重极限的方式趋于无穷,而这里给出的是\textbf{主值意义下的积分},即
\[\int_{-\infty}^{\infty}f(x)\,dx=\lim_{a\to-\infty,b\to\infty}\int_{a}^{b}f(x)\,dx\neq \lim_{R\to\infty}\int_{-R}^{R}f(x)\,dx=\text{p.v.}\int_{-\infty}^{\infty}f(x)\,dx\]
\begin{example}
    设$f(x)=x$,则$\int_{-\infty}^{\infty}f(x)\,dx$发散,但$\text{p.v.}\int_{-\infty}^{\infty}f(x)\,dx=0$
\end{example}
\begin{proof}
    我们有\lr{
        \int_{-R}^{R}e^{2\pi i\xi x}\,d\xi=&\frac{e^{2\pi i R x}-e^{-2\pi i R x}}{2\pi i x}\\
        =&\frac{\sin(2\pi R x)}{\pi x}
    }{
        \frac{1}{2\pi}\int_{-R}^{R}e^{i\omega x}\,d\omega=&\frac{e^{i R x}-e^{-i R x}}{i x}\\
        =&\frac{\sin(R x)}{\pi x}
    }
    以及\[\int_{-\infty}^{0}\frac{\sin(ax)}{\pi x}\,dx=\int_{0}^{\infty}\frac{\sin(ax)}{\pi x}\,dx=\frac{1}{2}\]
    因此
\lr{
    &\int_{-R}^{R}e^{2\pi i\xi x}\mathcal{F} f(\xi)\,d\xi\\
    =&\int_{-\infty}^{\infty}f(y)\,dy\int_{-R}^{R}e^{2\pi i\xi (x-y)}\,d\xi\\
    =&\int_{-\infty}^{\infty}f(y)\frac{\sin(2\pi R (x-y))}{\pi (x-y)}\,dy\\
    =&f*\left(\frac{\sin(2\pi R x)}{\pi x}\right)
}{
    &\frac{1}{2\pi}\int_{-R}^{R}e^{i\omega x}\mathcal{F} f(\omega)\,d\omega\\
    =&\int_{-\infty}^{\infty}f(y)\,dy\frac{1}{2\pi}\int_{-R}^{R}e^{i\omega (x-y)}\,d\omega\\
    =&\int_{-\infty}^{\infty}f(y)\frac{\sin(R (x-y))}{\pi (x-y)}\,dy\\
    =&f*\left(\frac{\sin(R x)}{\pi x}\right)
}
但是,由于$sinc\notin L^1(\mathbb{R})$,无法保证$\lim_{R\to \infty} f(x)*sinc(R x)=f(x)$,因此需要条件$f\in L^1(\mathbb{R})\cup PS(\mathbb{R})$,做更细致的估计:
\lr{
    &\int_{-R}^{R}e^{2\pi i\xi x}\mathcal{F} f(\xi)\,d\xi-\frac{f(x+)+f(x-)}{2}\\
    =&\int_{-\infty}^{\infty}f(y)\frac{\sin(2\pi R (x-y))}{\pi (x-y)}\,dy-\frac{f(x+)+f(x-)}{2}\\
    =&\int_{-\infty}^{\infty}f(x-y)\frac{\sin(2\pi R y)}{\pi y}\,dy-\frac{f(x+)+f(x-)}{2}\\
    =&\int_{-\infty}^{0}[f(x-y)-f(x+)]\frac{\sin(2\pi R y)}{\pi y}\,dy\\
    &+\int_{0}^{\infty}[f(x-y)-f(x-)]\frac{\sin(2\pi R y)}{\pi y}\,dy
}{
    &\frac{1}{2\pi}\int_{-R}^{R}e^{i\omega x}\mathcal{F} f(\omega)\,d\omega-\frac{f(x+)+f(x-)}{2}\\
    =&\int_{-\infty}^{\infty}f(y)\frac{\sin(R (x-y))}{\pi (x-y)}\,dy-\frac{f(x+)+f(x-)}{2}\\
    =&\int_{-\infty}^{\infty}f(x-y)\frac{\sin(R y)}{\pi y}\,dy-\frac{f(x+)+f(x-)}{2}\\
    =&\int_{-\infty}^{0}[f(x-y)-f(x+)]\frac{\sin(R y)}{\pi y}\,dy\\
    &+\int_{0}^{\infty}[f(x-y)-f(x-)]\frac{\sin(R y)}{\pi y}\,dy
}
我们需要证明当$R\to\infty$时,以上两个积分都趋于0。以$\int_{0}^{\infty}$为例,仍然考虑分段估计:$\int_{0}^{\infty}=\int_{0}^{K}+\int_{K}^{\infty}$,取$K\geq 1$,有
\lr{
    &\left|\int_{K}^{\infty}\frac{\sin(2\pi Ry)}{\pi y}f(x-y)\,dy\right|\\
    \leq & \int_{K}^{\infty}\left|\frac{\sin(2\pi Ry)}{\pi y}\right| |f(x-y)|\,dy\\
    \leq & \int_{K}^{\infty}|f(x-y)|\,dy\\
    &\int_{K}^{\infty}\frac{\sin(2\pi Ry)}{\pi y}f(x-)\,dy\\
    = & f(x-)\int_{RK}^{\infty}\frac{\sin(2\pi Ry)}{\pi y}\,dy
}{
    &\left|\int_{K}^{\infty}\frac{\sin(R y)}{\pi y}f(x-y)\,dy\right|\\
    \leq & \int_{K}^{\infty}\left|\frac{\sin(R y)}{\pi y}\right| |f(x-y)|\,dy\\
    \leq & \int_{K}^{\infty}|f(x-y)|\,dy\\
    &\int_{K}^{\infty}\frac{\sin(R y)}{\pi y}f(x-)\,dy\\
    = & f(x-)\int_{RK}^{\infty}\frac{\sin(R y)}{\pi y}\,dy
}
取$K$充分大,则以上两积分都是收敛的无穷限积分的尾部,从而是趋于0的。

对于$\int_{0}^{K}$部分,定义\begin{align*}
    g(y)=\begin{cases}
    \frac{f(x-y)-f(x-)}{\pi y},&y\in [0,K]\\
    0,&\text{otherwise}
    \end{cases}
\end{align*}
则\lr{
    &\int_{0}^{K}[f(x-y)-f(x-)]\frac{\sin(2\pi R y)}{\pi y}\,dy\\
    &=\int_{0}^{K}g(y)\frac{e^{2\pi i Ry}-e^{-2\pi iR y}}{2i}\,dy\\
    &=\frac{1}{2i}[\mathcal{F} g(-R)-\mathcal{F} g(R)]
}{
    &\int_{0}^{K}[f(x-y)-f(x-)]\frac{\sin(R y)}{\pi y}\,dy\\
    &=\int_{0}^{K}g(y)\frac{e^{i R y}-e^{-i R y}}{2i}\,dy\\
    &=\frac{1}{2i}[\mathcal{F} g(-R)-\mathcal{F} g(R)]
}
由于$f\in PS(\mathbb{R}),g\in PC(\mathbb{R}^*),g(0+)=f'(x-)$,有$g\in L^1(\mathbb{R})$,根据黎曼-勒贝格引理,$\mathcal{F} g(\pm R)\to 0,R\to\infty$,故$\int_{0}^{K}[f(x-y)-f(x-)]\frac{\sin(R y)}{\pi y}\,dy\to 0,R\to\infty$。得证。
\end{proof}

此外,还有一种类似的的傅里叶反演公式,我们仅仅给出其表述:
\begin{theorem}[傅里叶反演公式]
    设$f\in L^1(\mathbb{R})$,则
    \lr{
        &\frac{f(x+)+f(x-)}{2}\\
        =&\lim_{\lambda\to 0}\int_{-\infty}^{\infty}e^{2\pi i\xi x}\mathcal{F} f(\xi)e^{-\pi^2 \lambda^2 \xi^2}\,d\xi,a.e.
    }{
        &\frac{f(x+)+f(x-)}{2}\\
        =&\lim_{\lambda\to 0}\frac{1}{2\pi}\int_{-\infty}^{\infty}e^{i\omega x}\mathcal{F} f(\omega)e^{-\lambda^2\omega^2/4}\,d\omega,a.e.
    }
    a.e.表示几乎处处成立,即在勒贝格测度为0的集合以外,上式都是成立的。
\end{theorem}

\chapter{施瓦兹函数类,时限与带限}\label{sec:Schwartz_Functions}

%傅里叶下的封闭性
在\ref{sec:distributions}中,我们介绍了施瓦兹函数类及其对偶空间分布,并直接假定了施瓦兹函数类在傅里叶变换下的封闭性。这一小节将填补这个逻辑漏洞,并介绍施瓦兹函数类的一些其他好处。由于不涉及具体的计算,简洁起见,我们将在本节使用角频率$\omega$。

首先,可以根据施瓦兹函数类的定义\[\mathcal{S} =\left\{\varphi\in C^{\infty}(\mathbb{R} )\left|\lim_{|x|\to\infty}|x|^m\varphi^{(n)}(x)=0,\forall m,n\in\mathbb{N} \right.\right\}\]给出它们在$|x|\to\infty$时的估计,从而施瓦兹函数$\varphi$必满足$\int_{-\infty}^{\infty}|\varphi^{(n)}(x)|\,dx<\infty,n\in\mathbb{N}$,即施瓦兹函数类是$L^1(\mathbb{R} )$的子集,于是\begin{align*}
    \int_{-\infty}^{\infty}\varphi(t)e^{-i\omega t}\,dt&\leq \int_{-\infty}^{\infty}|\varphi(t)||e^{-i\omega t}|\,dt\\
    &=\int_{-\infty}^{\infty}|\varphi(t)|\,dt<\infty
\end{align*}
即施瓦兹函数的傅里叶变换是良定义的,傅里叶逆变换同理。

显然,施瓦兹函数在求导运算下是封闭的,即$\varphi\in\mathcal{S}\implies \varphi'\in\mathcal{S}$,并且乘以$x$的运算也是封闭的,即$\varphi\in\mathcal{S}\implies x\varphi\in\mathcal{S}$。利用这两个性质,可以证明施瓦兹函数类在傅里叶变换下的封闭性:
\begin{theorem}[施瓦兹函数类在傅里叶变换下的封闭性]
    设$\varphi\in\mathcal{S}$,则$\mathcal{F} \varphi,\mathcal{F} ^{-1} \varphi\in\mathcal{S}$。
\end{theorem}
\begin{proof}
    只需证明$\lim_{|\omega|\to\infty}|\omega|^m \mathcal{F} \varphi^{(n)}(\omega)=0,\forall m,n\in\mathbb{N}$。由傅里叶变换的微分性质(见\ref{sec:Fourier}),
    \begin{align*}
    \mathcal{F} [\varphi'](\omega)&=i\omega \mathcal{F} \varphi(\omega)\\
    \mathcal{F} [\varphi^{(n)}](\omega)&=(i\omega)^n \mathcal{F} \varphi(\omega)
\end{align*}
根据黎曼-勒贝格引理(见\ref{sec:approach}),\[\varphi^{(n)}\in L^1(\mathbb{R})\implies\lim_{|\omega|\to\infty}\mathcal{F}\left[ \varphi^{(n)}(\omega)\right]=0\]
即\[\lim_{|\omega|\to\infty}(i\omega)^n \mathcal{F} \varphi(\omega)\to 0\]
故$\mathcal{F} \varphi(\omega)\in\mathcal{S} $,同理可证明$\mathcal{F}^{-1} \varphi(\omega)\in\mathcal{S} $。
\end{proof}

%紧支函数的傅里叶变换不再是紧支函数%时域频域不同时紧支(见ee261,248)
我们来验证,紧支函数的傅里叶变换不再是紧支函数,从而一开始介绍的$\mathcal{C} $在傅里叶变换下不封闭。事实上,非零函数不可能既时限,又带限,即:\begin{theorem}
设$f\in L^1(\mathbb{R}),f\neq 0$,如果$f$在时域上紧支,则其傅里叶变换$\mathcal{F} f(\omega)$不可能在频域上紧支,反之亦然。
    
\end{theorem}

\begin{proof}
我们假设$\text{supp}\ f\in[-T/2,T/2],\text{supp}\ \mathcal{F} f\in[-\Omega/2,\Omega/2]$,证明$f=0$。

\textbf{法一}:
\begin{align*}
    f(t)&=\frac{1}{2\pi}\int_{-\infty}^{\infty}\mathcal{F} f(\omega)e^{i\omega t}\,d\omega\\
    &=\frac{1}{2\pi}\int_{-\Omega/2}^{\Omega/2}\mathcal{F} f(\omega)e^{i\omega t}\,d\omega
\end{align*}
由于$\text{supp}\ f\in[-T/2,T/2]$,所以$f(t)$在$|t|>T/2$时恒为0,故\[\int_{-\Omega/2}^{\Omega/2}\mathcal{F} f(\omega)e^{i\omega t}\,d\omega=0,\forall |t|>T/2\]
对以上恒等式两边同时求n阶导数,得到\[\int_{-\Omega/2}^{\Omega/2}(i\omega)^n \mathcal{F} f(\omega)e^{i\omega t}\,d\omega=0,\forall |t|>T/2\]
取定$t_0>T/2$,有\begin{align*}
    f(t)=&\frac{1}{2\pi}\int_{-\Omega/2}^{\Omega/2}\mathcal{F} f(\omega)e^{i\omega t_0}e^{i\omega (t-t_0)}\,d\omega
\end{align*}
将$e^{i\omega (t-t_0)}$展开成幂级数:\begin{align*}
    e^{i\omega (t-t_0)}=&\sum_{n=0}^{\infty}\frac{(i\omega (t-t_0))^n}{n!}
\end{align*}
则该幂级数在$\mathbb{R}$上收敛且内闭一致收敛,特别地,在有限区间$[-\Omega/2,\Omega/2]$上一致收敛,因此可以将求和与积分换序,得到\begin{align*}
    f(t)=&\frac{1}{2\pi}\int_{-\Omega/2}^{\Omega/2}\mathcal{F} f(\omega)e^{i\omega t_0}\sum_{n=0}^{\infty}\frac{(i\omega (t-t_0))^n}{n!}\,d\omega\\
    =&\frac{1}{2\pi}\sum_{n=0}^{\infty}\frac{((t-t_0))^n}{n!}\int_{-\Omega/2}^{\Omega/2}(i\omega)^n \mathcal{F} f(\omega)e^{i\omega t_0}\,d\omega=0
\end{align*}
这表明只有恒零函数才能做到时域、频域均紧支。

\textbf{法二}:利用解析延拓的唯一性。设$\text{supp}\ f\in[-T/2,T/2]$,则\begin{align*}
    \mathcal{F} f(\omega)&=\int_{-\infty}^{\infty}f(t)e^{-i\omega t}\,dt\\
    &=\int_{-T/2}^{T/2}f(t)e^{-i\omega t}\,dt
\end{align*}
这是一个可导的实变函数,可以将其做全纯延拓。我们在\ref{sec:Laplace_Transform}中已经介绍过拉普拉斯变换,不过那时使用的是单边拉普拉斯变换,这里为了涵盖$\text{supp}\ f\in[-T/2,T/2]$的情形,我们使用双边拉普拉斯变换:
\begin{align*}
    F(s)=\int_{-\infty}^{\infty}f(t)e^{-st}\,dt=\int_{-T/2}^{T/2}f(t)e^{-st}\,dt,F(i\omega)=\mathcal{F} f(\omega)
\end{align*}
它同样具备全纯性。由于积分区间有限,$f\in\mathcal{C} $光滑,以上积分总是存在的,并且可以对$s$求导,从而也是全纯的:\[F'(s)=-\int_{-T/2}^{T/2}tf(t)e^{-st}\,dt\]
即$F(s)$是整函数。但是,$F(s)=0,\forall s=i\omega,|\omega|> \Omega/2$,由唯一延拓定理,$F(s)=0,\forall s\in\mathbb{C}$,因此$f(t)=0$。
\end{proof}

%时域频域“带宽”积,海森堡不确定性原理
必须指出,实际应用中我们总是考虑时域、频域都近似紧支的函数,指出时限、带限不共存,只是理论需要。事实上,时域上的“宽度”与频域上的“宽度”之积存在一个下界,这一点从\ref{sec:Fourier}中提出的伸缩定理就可以看出:
\lr{
    \mathcal{F} [f(at)](\xi)=\frac{1}{|a|}\mathcal{F} f\left(\frac{\xi}{a}\right)
}{
    \mathcal{F} [f(at)](\omega)=\frac{1}{|a|}\mathcal{F} f\left(\frac{\omega}{a}\right)
}
随着$a$的增大,时域函数$f(at)$变得越来越集中,而频域函数$\mathcal{F} f\left(\frac{x}{a}\right)$则变得越来越分散,反之亦然。
我们可以用方差来定量描述这一关系。
\begin{theorem}
    时域信号、频域信号的方差之积满足不等式
    \[\sigma(f)\sigma(F)\geq \frac{1}{4\pi}\]
    其中$f\in L^1(\mathbb{R} )\cap L^2(\mathbb{R} )$,$F(\xi)=\int_{-\infty}^{\infty}f(t)e^{-2\pi i\xi t}\,dt$,$\sigma(f),\sigma(F)$分别为$f,F$的方差。
\end{theorem}
物理规律中许多现象都可以归结为这一数学事实,例如著名的\textbf{海森堡不确定性原理} (Heisenberg's uncertainty principle)\footnote{尽管广义的不确定性原理应该归结为算符或矩阵的不对易,但由于动量空间波函数是位置空间波函数的傅里叶变换,以上结果的确能够证明$\sigma_x \sigma_p\geq\frac{\hbar}{2}$}。
\begin{proof}
    我们首先验证,$|f|^2$与$|F|^2$作为某个随机变量的概率密度函数是良定义的,从而其方差是良定义的。根据普朗歇尔定理(见\ref{sec:Fourier}),\lr{
    \int_{-\infty}^{\infty}|f(t)|^2\,dt=\int_{-\infty}^{\infty}|\mathcal{F} f(\xi)|^2\,d\xi
}{
    \int_{-\infty}^{\infty}|f(t)|^2\,dt=\frac{1}{2\pi}\int_{-\infty}^{\infty}|\mathcal{F} f(\omega)|^2\,d\omega
}
    要使它们都是概率密度函数,我们就需要把$|f|^2,|F|^2$归一化,在频率形式的傅里叶变换下,这是自动成立的,而如果使用角频率来做傅里叶变换,则需要引入额外的归一化系数,因此下面仅使用频率。

    我们知道,如果随机变量$X$的概率密度函数为$f(x)$,并且其方差有定义(二阶中心矩存在),则方差为$\sigma^2=\int_{-\infty}^{\infty}(x-\mu)^2 f(x)\,dx$,其中$\mu=\int_{-\infty}^{\infty}x f(x)\,dx$为数学期望。
    
    我们假设$f$和$F$的数学期望为0,否则可以通过时移、频移定理将它调整为0,而不影响另一个域上信号的模值;又假设\[\int_{-\infty}^{\infty}t^2|f(t)|^2\,dt<\infty,\int_{-\infty}^{\infty}\xi^2|F(\xi)|^2\,d\xi<\infty\]
    即方差存在,则\begin{align*}
    4\pi^2 \sigma^2(f)\sigma^2(F)&=4\pi^2 \int_{-\infty}^{\infty}t^2|f(t)|^2\,dt\int_{-\infty}^{\infty}\xi^2|F(\xi)|^2\,d\xi\\
    &=\int_{-\infty}^{\infty}t^2|f(t)|^2\,dt\int_{-\infty}^{\infty}|2\pi i\xi F(\xi)|^2\,d\xi\\
    &=\int_{-\infty}^{\infty}t^2|f(t)|^2\,dt\int_{-\infty}^{\infty}\left| \mathcal{F} [f'(t)](\xi)\right|^2\,d\xi&\text{(时域微分性质)}\\
    &=\int_{-\infty}^{\infty}t^2|f(t)|^2\,dt\int_{-\infty}^{\infty}| f'(t)|^2\,dt&\text{(普朗歇尔定理)}\\
    &\geq \left(\int_{-\infty}^{\infty}|tf(t)||f'(t)|\,dt\right)^2&\text{(柯西-施瓦兹不等式)}\\
    &=\frac{1}{4}\int_{-\infty}^{\infty}|f(t)|^2\,dt^2&\text{(分部积分)}\\
    &=\frac{1}{4}&\text{(归一化)}
    \end{align*}
    因此,$\sigma(f)\sigma(F)\geq \frac{1}{4\pi}$,得证。
\end{proof}

读者可以自行验证,以上不等式取等于高斯函数。

使用方差的定义可以轻松地得到一个有用的不等式,它给出了一种基于方差估计函数集中程度的方法,这就是\textbf{切比雪夫不等式} (Chebyshev's inequality):
\begin{theorem}[切比雪夫不等式]
    设$f\in L^2(\mathbb{R} )$,则对于任意$a>0$,都有
    \[\int_{|t|\geq a}|f(t)|^2\,dt\leq \frac{1}{a^2}\int_{-\infty}^{\infty}t^2|f(t)|^2\,dt\]
    在概率论中它的表述是:设随机变量$X$的数学期望为$\mu$,方差为$\sigma^2$,则对于任意$k>0$,都有\[P(|X-\mu|\geq a)\leq \frac{\sigma^2}{a^2}\]
\end{theorem}
\begin{proof}
    由于$a>0$,有
    \begin{align*}
    \int_{|t|\geq a}|f(t)|^2\,dt&=\int_{|t|\geq a}\frac{t^2}{t^2}|f(t)|^2\,dt\\
    &\leq \frac{1}{a^2}\int_{|t|\geq a}t^2|f(t)|^2\,dt\\
    &\leq \frac{1}{a^2}\int_{-\infty}^{\infty}t^2|f(t)|^2\,dt
\end{align*}
\end{proof}

\begin{example}
    取$a=2\sigma(f)$,则有\[\int_{|t|\geq 2\sigma(f)}|f(t)|^2\,dt\leq \frac{1}{4}\]
即$f$在区间$[-2\sigma(f),2\sigma(f)]$之外的能量不超过总能量的$1/4$。
\end{example}
可见,前面证明的$\sigma_f \sigma_F\geq 1/4\pi$又可以表述为时域上的“宽度”与频域上的“宽度”之积有下界,由此可以定义信号的时宽、带宽积:
\begin{definition}[时宽-带宽积]
    设$f\in L^1(\mathbb{R} )\cap L^2(\mathbb{R} )$,$F(\xi)=\int_{-\infty}^{\infty}f(t)e^{-2\pi i\xi t}\,dt$,且它们的方差存在,则$f$的\textbf{时宽}(time width)和\textbf{带宽}(bandwidth)定义分别为
    \[T=2\sigma(f),B=2\sigma(F)\]
    它们的乘积称为\textbf{时宽-带宽积} (time-bandwidth product):
    \[TB(f)=4\sigma(f)\sigma(F)\]
\end{definition}

我们也可以根据时域、频域函数的半高宽或$1/20$高宽(1dB带宽)来估计它们的集中程度。

施瓦兹函数类的另一个好处在于,它不仅保证了傅里叶变换下的封闭性,还使得傅里叶变换$\mathcal{F} :\mathcal{S} \to \mathcal{S} $是一个\textbf{线性拓扑自同构},具体来说,这要求\begin{itemize}
    \item $\mathcal{F} $是线性的
    \item $\mathcal{F} $是双射(我们已经找到逆映射$\mathcal{F} ^{-1} $)
    \item $\mathcal{F} $及其逆映射$\mathcal{F} ^{-1} $均为连续映射。
\end{itemize}
我们已经证明过$\mathcal{F}$是线性的双射,下面来说明连续性。为此,我们需要给施瓦兹函数类赋予一个拓扑结构。尽管$\mathcal{S}\in L^1(\mathbb{R} )$,但$L^p$空间的范数并不能很好地刻画施瓦兹函数类的性质,因此我们需要引入一族\textbf{半范数} (seminorm):
\begin{definition}[*施瓦兹函数类的半范数]
    对于$m,n\in\mathbb{N} $,定义施瓦兹函数类上的半范数
    \[\left\|\varphi\right\|_{m,n}=\sup_{x\in\mathbb{R} }|x|^m\left|\varphi^{(n)}(x)\right|\]
\end{definition}
在这个范数下$\mathcal{S} $构成\textbf{弗雷歇空间} (Fréchet space)。利用这些半范数,我们可以定义施瓦兹函数类中的收敛:
\begin{definition}[*施瓦兹函数类中的收敛]
    设$\{\varphi_k\}\subset\mathcal{S} $,如果对于任意$m,n\in\mathbb{N} $,都有
    \[\lim_{k\to\infty}\left\|\varphi_k-\varphi\right\|_{m,n}=0\]
    则称$\varphi_k$在施瓦兹函数类中收敛于$\varphi$,记为$\varphi_k\to \varphi$。
\end{definition}
有了这些定义,我们就可以说明傅里叶变换及其逆映射的连续性:
\begin{theorem}[*$\mathcal{S} $上傅里叶变换的连续性]
    设$\varphi_k\to \varphi$,则$\mathcal{F} \varphi_k\to \mathcal{F} \varphi$。
\end{theorem}
\begin{proof}
    不妨设$\varphi=\mathcal{F} \varphi=0$。由傅里叶变换的微分性质(见\ref{sec:Fourier}),
    \begin{align*}
    \mathcal{F}\left[\left((-it)^n\varphi_k\right)^{(m)}\right](\omega)&=(i\omega)^m \mathcal{F}\left[(-it)^n \varphi_k\right](\omega)\\
    &=(i\omega)^m (\mathcal{F} \varphi_k)^{(n)}(\omega)\\
    \|\mathcal{F} \varphi_k\|_{m,n}&=\sup_{\omega\in\mathbb{R} }|\omega|^m\left|(\mathcal{F} \varphi_k)^{(n)}(\omega)\right|\\
    &=\sup_{\omega\in\mathbb{R} }\left|\mathcal{F}\left[((-it)^n \varphi_k)^{(m)}\right](\omega)\right|
\end{align*}
令$\phi_k=\left((-it)^n \varphi_k\right)^{(m)}\in\mathcal{S} $,则
\begin{align*}
    \|\mathcal{F} \varphi_k\|_{m,n}&=\sup_{\omega\in\mathbb{R} }\left|\mathcal{F} \phi_k(\omega)\right|\leq \int_{-\infty}^{\infty}|\phi_k(t)|\,dt
\end{align*}
问题转化为证明\[\int_{-\infty}^{\infty}|\phi_k(t)|\,dt\to 0\]
考虑利用$\varphi_k\in\mathcal{S} ,\|\varphi_k\|_{m,n}\to 0,\forall m,n\in\mathbb{N}$给出$|\phi_k(t)|$的估计。利用莱布尼兹求导法则,有:
\begin{align*}
    |\phi_k(t)|&=\left|\left((-it)^n \varphi_k\right)^{(m)}(t)\right|\\
&=\left|\sum_{j=0}^{m}\binom{m}{j}(t^n)^{(j)}\varphi_k^{(m-j)}(t)\right|\\
    &\leq \sum_{j=0}^{\min(m,n)}\binom{m}{j}\frac{n!}{(n-j)!}|t|^{n-j}\left|\varphi_k^{(m-j)}(t)\right|
\end{align*}
由于$\|\varphi_k\|_{n-j+2,m-j}\to 0$,有:$\forall \epsilon>0,\exists N\in\mathbb{N},k>N\implies \|\varphi_k\|_{n-j+2,m-j}<\epsilon$,于是
\begin{align*}
    |\phi_k(t)|&\leq \sum_{j=0}^{\min(m,n)}\binom{m}{j}\frac{n!}{(n-j)!}\frac{\|\varphi_k\|_{n-j+2,m-j}}{|t|^2}\\
    &<\frac{C\epsilon}{|t|^2},C=\sum_{j=0}^{\min(m,n)}\binom{m}{j}\frac{n!}{(n-j)!}
\end{align*}
因此有\begin{align*}
    \int_{|t|\geq 1}|\phi_k(t)|\,dt&\leq \int_{|t|\geq 1}\frac{C\epsilon}{|t|^2}\,dt\\
    &=2C\epsilon\int_{1}^{\infty}\frac{1}{t^2}\,dt=2C\epsilon
\end{align*}
而在$|t|<1$时,\begin{align*}
    \sum_{j=0}^{\min(m,n)}\binom{m}{j}\frac{n!}{(n-j)!}|t|^{n-j}\left|\varphi_k^{(m-j)}(t)\right|&\leq \sum_{j=0}^{\min(m,n)}\binom{m}{j}\frac{n!}{(n-j)!}\|\varphi_k\|_{0,m-j}\\
    &\leq C'\max_{0\leq j\leq \min(m,n)}\|\varphi_k\|_{0,m-j}<\epsilon
\end{align*}
即$\phi_k$一致收敛于0,因此有
\[\int_{|t|<1}|\phi_k(t)|\,dt\to 0\]
综上,\[0\leq\|\mathcal{F} \varphi_k\|_{m,n}\leq\int_{-\infty}^{\infty}|\phi_k(t)|\,dt\to 0\]
根据夹逼定理,$\|\mathcal{F} \varphi_k\|_{m,n}\to 0$,得证。傅里叶逆变换同理。
\end{proof}

其实,如果我们考虑$L^1(\mathbb{R})\cap L^2(\mathbb{R})$中傅里叶变换的连续性,可以得到类似的结论,但证明要简单得多,因为$L^1(\mathbb{R})\cap L^2(\mathbb{R})$中有普朗歇尔定理(见\ref{sec:Fourier})。
\begin{theorem}[$L^1(\mathbb{R})\cap L^2(\mathbb{R})$上傅里叶变换的连续性]
    设$f_k$在$L^2$范数下收敛到$f$,则$\mathcal{F} f_k$在$L^2$范数下收敛到$\mathcal{F} f$。
\end{theorem}
\begin{remark}
    尽管$L^1(\mathbb{R})\cap L^2(\mathbb{R})$是比$\mathcal{S} $更大的空间,但这个定理也有一定的局限性:其一,依$L^2$范数收敛不能保证依半范数收敛,即证得的收敛性会减弱;其二,$L^1(\mathbb{R})\cap L^2(\mathbb{R})$并不是傅里叶变换下的封闭空间,因此我们无法保证傅里叶逆变换的连续性。
\end{remark}
\begin{proof}
    由普朗歇尔定理,有
    \lr{
    \left\|\mathcal{F} f_k-\mathcal{F} f\right\|_2^2&=\int_{-\infty}^{\infty}\left|\mathcal{F} f_k(\xi)-\mathcal{F} f(\xi)\right|^2\,d\xi\\
    &=\int_{-\infty}^{\infty}\left|f_k(t)-f(t)\right|^2\,dt\\
    &=\left\|f_k-f\right\|_2^2
    }{
    \left\|\mathcal{F} f_k-\mathcal{F} f\right\|_2^2&=\int_{-\infty}^{\infty}\left|\mathcal{F} f_k(\omega)-\mathcal{F} f(\omega)\right|^2\,d\omega\\
    &=2\pi\int_{-\infty}^{\infty}\left|f_k(t)-f(t)\right|^2\,dt\\
    &=2\pi\left\|f_k-f\right\|_2^2
    }
    因此,$\left\|f_k-f\right\|_2\to 0\Rightarrow \left\|\mathcal{F} f_k-\mathcal{F} f\right\|_2\to 0$,得证。
\end{proof}

另外,在弱收敛的意义下,傅里叶变换同样是$\mathcal{T} =\mathcal{S} ' $上的线性拓扑自同构。在\ref{sec:distributions}中,我们已经证明过分布的傅里叶变换的连续性:\begin{theorem}[$\mathcal{T}$上傅里叶变换的连续性]
    设$T_k\to T$,则$\mathcal{F} T_k\to \mathcal{F} T$。
\end{theorem}

\end{document}